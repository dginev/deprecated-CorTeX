\documentclass{article}
\usepackage{planetmath-specials}
\usepackage{pmath}
% this is the default PlanetMath preamble.  as your knowledge
% of TeX increases, you will probably want to edit this, but
% it should be fine as is for beginners.

% almost certainly you want these
\usepackage{amssymb}
\usepackage{amsmath}
\usepackage{amsfonts}

% used for TeXing text within eps files
%\usepackage{psfrag}
% need this for including graphics (\includegraphics)
%\usepackage{graphicx}
% for neatly defining theorems and propositions
%\usepackage{amsthm}
% making logically defined graphics
%\usepackage{xypic}

% there are many more packages, add them here as you need them

% define commands here

\begin{document}
The Chernoff-Cram\`{e}r inequality is a very general and powerful way of bounding random variables. Compared with the famous Chebyshev bound, which implies inverse polynomial decay inequalities, the Chernoff-Cram\`{e}r method yields exponential decay inequalities, at the cost of  needing a few more
hypotheses on random variables' \PMlinkescapetext{behaviour}.


Theorem: (Chernoff-Cram\`{e}r inequality)

Let $\{X_{i}\}_{i=1}^{n}$ be a collection of independent random
variables such that $E[\exp \left( tX_{i}\right) ]<+\infty $ \ $\forall i$ in a right neighborhood of $t=0$, i.e. for any $t\in(0,c)$ (Cram\`{e}r condition).

Then a zero-valued for $x=0$, positive, strictly increasing, strictly convex function $\Psi (x):[0,\infty )\mapsto R^{+}$ exists
such that: \\
\[
\Pr\left\{ \sum_{i=1}^{n}\left( X_{i}-E[X_{i}]\right) >\varepsilon \right\}
\leq \exp \left( -\Psi (\varepsilon )\right) \ \forall \varepsilon \geq 0
\]

Namely, one has:
\begin{eqnarray*}
\Psi (x) &=&\sup_{0 < t < c}\left( tx -\psi (t)\right)  \\
\psi (t) &=&\sum_{i=1}^{n}\ln E\left[ e^{t\left( X_{i}-EX_{i}\right) }\right]
=\sum_{i=1}^{n}\left(\ln E\left[ e^{tX_{i}}\right] -tE\left[ X_{i}\right]\right)
\end{eqnarray*}

that is, $\Psi (x)$ is the Legendre \PMlinkescapetext{transform} of the cumulant generating function of the $\sum_{i=1}^{n}\left( X_{i}-E[X_{i}]\right)$ random variable.

Remarks:

1) Besides its importance for theoretical questions, the Chernoff-Cram\'{e}r bound is also the starting point to derive many deviation
or concentration inequalities, among which \PMlinkname{Bernstein}{BernsteinInequality}, Kolmogorov, \PMlinkname{Prohorov}{ProhorovInequality}, \PMlinkname{Bennett}{BennettInequality},
Hoeffding and Chernoff ones are worth mentioning. All of these
inequalities are obtained imposing various further conditions on $X_{i}$ random variables, which turn out to affect the general form of the cumulant generating function $\psi(t)$. \\
2) Sometimes, instead of bounding the sum of n independent random variables, one needs to estimate they '\PMlinkescapetext{mean}' i.e. the quantity $\frac{1}{n}\sum_{i=1}^{n}X_i$; in \PMlinkescapetext{order} to reuse Chernoff-Cram\'{e}r bound, it's enough to note that
\[
\Pr\left\{\frac{1}{n}\sum_{i=1}^{n}\left( X_{i}-E[X_{i}]\right) >\varepsilon^{\prime}\right\}=\Pr\left\{\sum_{i=1}^{n}\left( X_{i}-E[X_{i}]\right) >n\varepsilon^{\prime}\right\}
\]
so that one has only to replace, in the above stated inequality, $\varepsilon$ with $n\varepsilon^{\prime}$.
\\
3) It turns out that the Chernoff-Cramer bound is asymptotically sharp, as Cramèr \PMlinkescapetext{limit} theorem shows.
\end{document}
