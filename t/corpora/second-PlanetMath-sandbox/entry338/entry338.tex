\documentclass{article}
\usepackage{planetmath-specials}
\usepackage{pmath}
\usepackage{amssymb}
\usepackage{amsmath}
\usepackage{amsfonts}

\newcommand{\R}[0]{\mathbb{R}}
\newcommand{\C}[0]{\mathbb{C}}
\newcommand{\N}[0]{\mathbb{N}}
\newcommand{\Z}[0]{\mathbb{Z}}
\begin{document}
\PMlinkescapeword{words}

{\bf Definition} A set is \emph{uncountable} if it is not countable.
In other words, a set $S$ is uncountable, if
there is no subset of $\N$ (the set of natural numbers) with the same cardinality as $S$.

\begin{enumerate}
\item All uncountable sets are infinite. However, the converse is not
true, as $\N$ is both infinite and countable.
\item The real numbers form an uncountable set. The famous proof
of this result is based on Cantor's diagonal argument.
\end{enumerate}
\end{document}
