\documentclass{article}
\usepackage{planetmath-specials}
\usepackage{pmath}
\usepackage{amssymb}
\usepackage{amsmath}
\usepackage{amsfonts}
\usepackage{graphicx}
\usepackage{xypic}
\newcommand{\pt}{\mathbf}
\begin{document}
\PMlinkescapeword{representations}

This is the list of known standard representations and their nuances.\\
\\
$\frac{du}{dv}, \frac{df}{dx}, \frac{dy}{dx}-$ The most common notation, this is read as the derivative of $u$ with respect to $v$. Exponents relate which derivative, for example, $\frac{d^2y}{dx^2}$ is the second derivative of $y$ with respect to $x$.\\
\hrule
$f'(x)\; ,\vec{f}'(\pt{x})\;, y''-$ This is read as $f$ prime of $x$. The number of primes tells the derivative, ie. $f'''(x)$ is the third derivative of $f(x)$ with respect to $x$. Note that in higher dimensions, this may be a tensor of a rank equal to the derivative.\\
\hrule
$D_xf(\pt{x}), F_y(\pt{x}), f_{xy}(\pt{x})-$ These notations are rather arcane, and should not be used generally, as they have other meanings. For example $F_y$ can easily by the $y$ component of a vector-valued function. The subscript in this case means ``with respect to'', so $F_{yy}$ would be the second derivative of $F$ with respect to $y$.\\
\hrule
$D_1f(\pt{x}), F_2(\pt{x}), f_{12}(\pt{x})-$ The subscripts in these cases refer to the derivative with respect to the nth variable. For example, $F_2(x,y,z)$ would be the derivative of $F$ with respect to $y$. They can easily represent higher derivatives, ie. $D_{21}f(\pt{x})$ is the derivative with respect to the first variable of the derivative with respect to the second variable.\\
\hrule
$\frac{\partial u}{\partial v}\; ,\frac{\partial f}{\partial x}-$ The partial derivative of $u$ with respect to $v$. This symbol can be manipulated as in $\frac{du}{dv}$ for higher partials.\\
\hrule
$\frac{d}{dv}\;,\frac{\partial}{\partial v}-$ This is the operator version of the derivative. Usually you will see it acting on something such as $\frac{d}{dv}(v^2+3u) = 2v$.\\
\hrule
$[\mathbf{Jf}(\pt{x})]\:,[\mathbf{Df}(\pt{x})]-$ The first of these represents the \PMlinkid{Jacobian}{842} of $\mathbf{f}$, which is a matrix of partial derivatives such that
$$[\mathbf{Jf}(\pt{x})] = \left[\begin{array}{ccc}
D_1f_1(\pt{x}) & \dots & D_nf_1(\pt{x})\\
\vdots & \ddots & \vdots\\
D_1f_m(\pt{x}) & \dots & D_nf_m(\pt{x})\\
\end{array}\right]$$
where $f_n$ represents the nth function of a vector valued function. The second of these notations represents the derivative matrix, which in most cases is the Jacobian, but in some cases, does not exist, even though the Jacobian exists. Note that the directional derivative in the direction $\vec{v}$ is simply $[\mathbf{Jf}(\pt{x})]\vec{v}$.\\
\end{document}
