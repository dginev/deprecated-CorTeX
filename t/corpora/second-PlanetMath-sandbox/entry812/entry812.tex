\documentclass{article}
\usepackage{planetmath-specials}
\usepackage{pmath}
% this is the default PlanetMath preamble.  as your knowledge
% of TeX increases, you will probably want to edit this, but
% it should be fine as is for beginners.

% almost certainly you want these
\usepackage{amssymb}
\usepackage{amsmath}
\usepackage{amsfonts}

% used for TeXing text within eps files
%\usepackage{psfrag}
% need this for including graphics (\includegraphics)
%\usepackage{graphicx}
% for neatly defining theorems and propositions
%\usepackage{amsthm}
% making logically defined graphics
%\usepackage{xypic}

% there are many more packages, add them here as you need them

% define commands here
\begin{document}
Given an integer $n > 0$,

$$M(n) = \sum_{i = 1}^n \mu(i)$$

where $\mu(x)$ is the M\"obius function.

There is no $x$ such that $M(x) > x$. Franz Mertens conjectured that there is no $x$ such that $M(x) > \sqrt{x}$, but this was proven false later.

\begin{thebibliography}{6}
\bibitem{fm} F. Mertens, ``Ãber eine zahlentheoretische Funktion" {\it Akademie Wissenschaftlicher Wien Mathematik-Naturlich Kleine Sitzungsber}, IIa 106, 761-830 (1897)
\bibitem{ao} A. M. Odlzyko and te Riele, ``Disproof of the Mertens Conjecture" {\it Journal für die reine und angewandte Mathematik}, 357, 138-160 (1985)
\end{thebibliography}

\subsection{External link}

\PMlinkexternal{M\"obius and Mertens Values For n = 1 to 2500}{http://primefan.tripod.com/Mertens2500.html}
\end{document}
