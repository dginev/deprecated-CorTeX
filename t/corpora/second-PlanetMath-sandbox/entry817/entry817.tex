\documentclass{article}
\usepackage{planetmath-specials}
\usepackage{pmath}
\usepackage{amsmath}
\usepackage{amsfonts}
\usepackage{amssymb}

\newcommand{\reals}{\mathbb{R}}
\newcommand{\natnums}{\mathbb{N}}
\newcommand{\cnums}{\mathbb{C}}

\newcommand{\lp}{\left(}
\newcommand{\rp}{\right)}
\newcommand{\lb}{\left[}
\newcommand{\rb}{\right]}

\newcommand{\supth}{^{\text{th}}}

\newtheorem{proposition}{Proposition}
\begin{document}
Let $S\subset\reals$ be a set of real numbers.  Recall that a limit
point of $S$ is a real number $x\in\reals$ such that for all
$\epsilon>0$ there exist infinitely many $y\in S$ such that
$$\vert x-y\vert <\epsilon.$$
We define $\liminf S$, pronounced the
{\em limit inferior} of $S$, to be the infimum of all the limit
points of $S$.  If there are no limit points, we define the limit
inferior to be $+\infty$.

The two most common notations for the limit inferior are 
$$\liminf S$$ and
$$\underline{\lim}\, S\,.$$

An alternative, but equivalent, definition is available in the case of
an infinite sequence of real numbers $x_0, x_1, x_2, ,\ldots$.  For
each $k\in\natnums$, let $y_k$ be the infimum of the $k\supth$ tail,
$$y_k = \inf_{j\geq k} x_j .$$
This construction produces a
non-decreasing sequence
$$y_0 \leq y_1 \leq y_2 \leq \ldots,$$
which either converges to its supremum, or diverges to $+\infty$.
We define the limit inferior of the original sequence to be this limit;
$$\liminf_{k} x_k = \lim_k y_k.$$
\end{document}
