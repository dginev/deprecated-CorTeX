\documentclass{article}
\usepackage{ids}
\usepackage{planetmath-specials}
\usepackage{pmath}
\usepackage{amssymb}
\usepackage{amsmath}
\usepackage{amsfonts}
\usepackage{graphicx}
\usepackage{xypic}
\begin{document}
The rational numbers $\mathbb{Q}$ are the fraction field of the ring $\mathbb{Z}$ of integers. In more elementary terms, a rational number is a quotient $a/b$ of two integers $a$ and $b$, where $b$ is nonzero. Two fractions $a/b$ and $c/d$ are equivalent if the product of the cross terms is equal:
$$
\frac{a}{b} = \frac{c}{d} \iff ad = bc
$$
Addition and multiplication of fractions are given by the formulae
\begin{eqnarray*}
\frac{a}{b} + \frac{c}{d} & = & \frac{ad + bc}{bd} \\
\frac{a}{b} \cdot \frac{c}{d} & = & \frac{ac}{bd}
\end{eqnarray*}

The field of rational numbers is an ordered field, under the ordering relation $\leq$ defined as follows: $a/b \leq c/d$ if
\begin{enumerate}
\item the inequality $a\cdot d \leq b \cdot c$ holds in the integers, and $b$ has the same sign as $d$, or
\item the inequality $a\cdot d \geq b \cdot c$ holds in the integers, and $b$ has the opposite sign as $d$.
\end{enumerate}
Under this ordering relation, the rational numbers form a topological space under the order topology. The set of rational numbers is dense when considered as a subset of the real numbers.
\end{document}
