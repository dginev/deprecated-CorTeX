\documentclass{article}
\usepackage{planetmath-specials}
\usepackage{pmath}
\usepackage{amssymb}
\usepackage{amsmath}
\usepackage{amsfonts}
\usepackage{graphicx}
\usepackage{xypic}
\begin{document}
Stirling's formula gives an approximation for $n!$, the factorial \PMlinkescapetext{function}.  It is 

$$ n! \approx \sqrt{2n\pi} n^n e^{-n} $$

We can derive this from the gamma function.  Note that for large $x$, 

\begin{equation}
 \label{eq:glx}
 \Gamma(x) = \sqrt{2\pi}x^{x-\frac{1}{2}} e^{-x + \mu(x)}
\end{equation}

where

$$ \mu(x) = \sum_{n=0}^\infty \left(x+ n +\frac{1}{2}\right) \ln \left(1+ \frac{1}{x+n} \right) -1 = \frac{\theta}{12x} $$

with $0 < \theta < 1$.  Taking $x=n$ and multiplying by $n$, we have

\begin{equation}
\label{eq:penult}
 n! = \sqrt{2\pi} n^{n+\frac{1}{2}} e^{-n + \frac{\theta}{12n}}
\end{equation}

Taking the approximation for large $n$ gives us Stirling's formula.

There is also a big-O notation version of Stirling's approximation:

\begin{equation}
\label{eq:bigover}
n! = \left(\sqrt{2\pi n}\right)\left(\frac{n}{e}\right)^n \left(1+\mathcal{O}\left(\frac{1}{n}\right)\right)
\end{equation}

We can prove this equality starting from~\eqref{eq:penult}.  It is clear that the big-O portion of~\eqref{eq:bigover} must come from $e^{\frac{\theta}{12n}}$, so we must consider the asymptotic behavior of $e$.  

First we observe that the Taylor series for $e^x$ is 

$$ e^x = 1 + \frac{x}{1} + \frac{x^2}{2!} + \frac {x^3}{3!} + \cdots $$

But in our case we have $e$ to a vanishing exponent.   Note that if we vary $x$ as $\frac{1}{n}$, we have as $ n \longrightarrow \infty $

$$ e^x = 1 + \mathcal{O}\left(\frac{1}{n}\right) $$

We can then (almost) directly plug this in to~\eqref{eq:penult} to get~\eqref{eq:bigover} (note that the factor of 12 gets absorbed by the big-O notation.)
\end{document}
