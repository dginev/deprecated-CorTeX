\documentclass{article}
\usepackage{ids}
\usepackage{planetmath-specials}
\usepackage{pmath}
\usepackage{amssymb}
\usepackage{amsmath}
\usepackage{amsfonts}
\usepackage{graphicx}
\usepackage{psfrag}
\begin{document}
The ``Golden Ratio'', or $\phi$, has the value 

$$ 1.61803398874989484820\ldots $$

This number gets its rather illustrious name from the fact that the Greeks thought that a rectangle with ratio of side lengths equal to $\phi$ was the most pleasing to the eye, and much of classical Greek architecture is based on this premise.  In \PMlinkescapetext{addition}, an aesthetically pleasing aspect of a rectangle with this ratio, from a mathematical viewpoint, is that if we embed and remove a $w\times w$ square in the below diagram, the remaining rectangle also has a width-to-length ratio of $\phi$.

\begin{center}

\psfrag{l}{$\;\;\;\;\;\;\;\;l$}
\psfrag{w}{$\!\!\!w$}
\includegraphics[scale=.6]{golden.eps}

{\tiny Above: The golden rectangle; $l/w = \phi$. }
\end{center}

$\phi$ has plenty of interesting mathematical \PMlinkescapetext{properties}, however.  Its value is exactly

$$ \frac{1+\sqrt{5}}{2} $$

The value 
$$ \frac{1-\sqrt{5}}{2} $$

is often called $\phi'$.  $\phi$ and $\phi'$ are the two roots of the recurrence relation given by the Fibonacci sequence.  The following \PMlinkescapetext{identities} hold for $\phi$ and $\phi'$ :

\begin{itemize}
\item $\frac{1}{\phi} = - \phi' $
\item $1-\phi = \phi'$
\item $\frac{1}{\phi'} = - \phi $
\item $1-\phi' = \phi $
\end{itemize}

and so on.  These give us

$$ \phi^{-1} + \phi^0 = \phi^{1} $$

which implies 

$$ \phi^{n-1} + \phi^n = \phi^{n+1} $$
\end{document}
