\documentclass{article}
\usepackage{planetmath-specials}
\usepackage{pmath}
\usepackage{amssymb}
\usepackage{amsmath}
\usepackage{amsfonts}
\usepackage{graphicx}
\usepackage{xypic}
\begin{document}
An integer $p \in \mathbb{Z}$ is prime if it has exactly two positive divisors. The first few positive prime numbers are $2, 3, 5, 7, 11, \dots$.

A prime number is often (but not always) required to be positive.

Prime numbers are very important objects in many fields of mathematics.  The notion of prime number has been generalized in a number of ways.  

In class field theory, one defines a prime to be a family of equivalent valuations; using this definition, the primes of the rational numbers are given by the positive prime numbers (for a prime $p$, the corresponding valuation is the $p$-adic absolute value) and one extra prime usually denoted $\infty$ (corresponding to the usual absolute value on $\mathbb{Q}$). 

In ring theory, one defines the notion of a prime ideal, and also a notion of \PMlinkname{prime element}{PrimeElement}.  However, the notion of prime ideal is more natural in this context than the notion of prime element.  For number fields, for example, one has unique factorization of ideals into prime ideals but one does not always have unique factorization of elements into prime elements.  The prime ideals of $\mathbb{Z}$ are the ideals generated by the prime numbers, as well as the zero ideal.  Note that the ideal generated by $p$ is the same as the ideal generated by $-p$, so we can consider negative primes equivalent to positive primes from this viewpoint as well.
\end{document}
