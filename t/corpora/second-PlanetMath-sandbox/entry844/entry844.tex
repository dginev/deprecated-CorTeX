\documentclass{article}
\usepackage{planetmath-specials}
\usepackage{pmath}
% this is the default PlanetMath preamble.  as your knowledge
% of TeX increases, you will probably want to edit this, but
% it should be fine as is for beginners.

% almost certainly you want these
\usepackage{amssymb}
\usepackage{amsmath}
\usepackage{amsfonts}

% used for TeXing text within eps files
%\usepackage{psfrag}
% need this for including graphics (\includegraphics)
%\usepackage{graphicx}
% for neatly defining theorems and propositions
%\usepackage{amsthm}
% making logically defined graphics
%\usepackage{xypic}

% there are many more packages, add them here as you need them

% define commands here
\begin{document}
A \emph{M\"obius transformation} is a bijection on the extended complex plane $\mathbb{C} \cup \{\infty\}$ given by

$$f(z) =
\begin{cases}
\frac{a}{c} &\text{if } z = \infty \\
\infty &\text{if } z = -{d \over c}\\
\frac{az+b}{cz+d} &\text{otherwise}
\end{cases}
$$

where $a,b,c,d \in \mathbb{C}$ and $ad - bc \ne 0$

It can be shown that the inverse, and composition of two M\"obius transformations are similarly defined, and so the M\"obius transformations form a group under composition.

The geometric interpretation of the M\"obius group is that it is the group of automorphisms of the Riemann sphere.

Any M\"obius map can be composed from the elementary transformations - dilations, translations and inversions. If we define a line to be a circle passing through $\infty$ then it can be shown that a M\"obius transformation maps circles to circles, by looking at each elementary transformation.
\end{document}
