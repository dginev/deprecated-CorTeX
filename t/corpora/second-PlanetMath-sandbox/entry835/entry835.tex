\documentclass{article}
\usepackage{ids}
\usepackage{planetmath-specials}
\usepackage{pmath}
% this is the default PlanetMath preamble.  as your knowledge
% of TeX increases, you will probably want to edit this, but
% it should be fine as is for beginners.

% almost certainly you want these
\usepackage{amssymb}
\usepackage{amsmath}
\usepackage{amsfonts}

% used for TeXing text within eps files
%\usepackage{psfrag}
% need this for including graphics (\includegraphics)
%\usepackage{graphicx}
% for neatly defining theorems and propositions
%\usepackage{amsthm}
% making logically defined graphics
%\usepackage{xypic}

% there are many more packages, add them here as you need them

% define commands here
\begin{document}
Let $F$ be a field of characteristic $p>0$. Then for any $a, b \in F$, 
\begin{eqnarray*}
(a + b)^p &=& a^p + b^p,
\\ (ab)^p &=& a^p b^p.
\end{eqnarray*}

Thus the map
$$
\begin{matrix}\phi: F &\to& F 
\\ a &\mapsto& a^p
\end{matrix}
$$
is a field homomorphism, called the \emph{Frobenius homomorphism}, or simply the \emph{Frobenius map} on $F$.
If it is surjective then it is an automorphism, and is called the \emph{Frobenius automorphism}.

Note: This morphism is sometimes also called the ``small Frobenius'' to distinguish it from the map $a \mapsto a^q$, with $q=p^n$. This map is then also referred to as the ``big Frobenius'' or the ``power Frobenius map''.
\end{document}
