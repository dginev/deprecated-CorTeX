\documentclass{article}
\usepackage{planetmath-specials}
\usepackage{pmath}
% this is the default PlanetMath preamble.  as your knowledge
% of TeX increases, you will probably want to edit this, but
% it should be fine as is for beginners.

% almost certainly you want these
\usepackage{amssymb}
\usepackage{amsmath}
\usepackage{amsfonts}

% used for TeXing text within eps files
%\usepackage{psfrag}
% need this for including graphics (\includegraphics)
%\usepackage{graphicx}
% for neatly defining theorems and propositions
 \usepackage{amsthm}
% making logically defined graphics
%\usepackage{xypic}
\usepackage{pstricks}
\usepackage{pst-plot}

% there are many more packages, add them here as you need them

% define commands here

\theoremstyle{definition}
\newtheorem*{thmplain}{Theorem}

\begin{document}
The \emph{floor} of a real number is the greatest integer less than or equal to the number. The floor of $x$ is usually denoted by $\lfloor x\rfloor$.

\begin{center}
\begin{pspicture}(-5.5,-4.5)(5.5,4.3)
\psaxes[Dx=1,Dy=1]{->}(0,0)(-4.5,-3.5)(4.5,3.5)
\rput(0.3,3.55){$y$}
\rput(4.6,0.3){$x$}
\psdots[linecolor=blue](-3,-3)(-2,-2)(-1,-1)(1,1)(2,2)(3,3)
\psdot[linecolor=blue,linewidth=0.03](0,0)
\psline[linecolor=blue,linewidth=0.06](-3,-3)(-2,-3)
\psline[linecolor=blue,linewidth=0.06](-2,-2)(-1,-2)
\psline[linecolor=blue,linewidth=0.06](-1,-1)(0,-1)
\psline[linecolor=blue,linewidth=0.07](0,0)(1,0)
\psline[linecolor=blue,linewidth=0.06](1,1)(2,1)
\psline[linecolor=blue,linewidth=0.06](2,2)(3,2)
\psline[linecolor=blue,linewidth=0.06](3,3)(4,3)
\rput(-2.5,2.5){$\mbox{Graph\; } y = \lfloor{x}\rfloor$}
\end{pspicture}
\end{center}

The real function \,$x \mapsto \lfloor{x}\rfloor$\, is monotonically nondecreasing and satisfies
$$x-1 < \lfloor{x}\rfloor \leqq x$$
for all $x$.\, The function is continuous everywhere except in the integer points \,$0,\,\pm1,\,\pm2,\,\ldots$\, where it is only continuous from the right.\, One has
$$\lfloor\lfloor{x}\rfloor\rfloor \;=\; \lfloor{x}\rfloor,$$
i.e. the function is idempotent.\\


\textbf{Some examples}:
\begin{itemize}
\item $\lfloor 6.2\rfloor=6$,
\item $\lfloor 0.4\rfloor=0$,
\item $\lfloor 7\rfloor=7$,
\item $\lfloor -5.1\rfloor=-6$,
\item $\lfloor \pi\rfloor=3$,
\item $\lfloor -4\rfloor=-4$.
\end{itemize}

Note that this function is not the integer part ($[x]$), since 
$\lfloor -3.5\rfloor = -4$ and $[ -3.5]=-3$.
However, both functions agree for non-negative numbers.

The notation for floor and ceiling was introduced by Iverson in 1962\cite{Higham}.
In some texts however, the bracket notation is used to denote the floor function (although they actually work with integer part) so it is sometimes also called the \emph{bracket function}.
 
\begin{thebibliography}{9}
\bibitem{Higham} N. Higham, Handbook of writing for the mathematical sciences, Society for Industrial and Applied Mathematics, 1998.
\end{thebibliography}
\end{document}
