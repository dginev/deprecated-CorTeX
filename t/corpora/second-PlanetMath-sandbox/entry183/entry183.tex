\documentclass{article}
\usepackage{ids}
\usepackage{planetmath-specials}
\usepackage{pmath}
\usepackage{amssymb}
\usepackage{amsmath}
\usepackage{amsfonts}
\usepackage{graphicx}
\usepackage{xypic}
\begin{document}
Let $X$ and $Y$ be sets. A function $f\colon X\to Y$ that is one-to-one and onto is called a \emph{bijection} or \emph{bijective function} from $X$ to $Y$.

When $X=Y$, $f$ is also called a \emph{permutation} of $X$.

An important consequence of the bijectivity of a function $f$ is the existence of an inverse function $f^{-1}$.  Specifically, a function is invertible if and only if it is bijective.  Thus if $f:X\rightarrow Y$ is a bijection, then for any $A\subset X$ and $B\subset Y$ we have 

\begin{align*}
f\circ f^{-1}(B)&=B\\
f^{-1}\circ f(A)&=A\\
\end{align*}

It easy to see the inverse of a bijection is a bijection, and that a composition of bijections is again bijective.
\end{document}
