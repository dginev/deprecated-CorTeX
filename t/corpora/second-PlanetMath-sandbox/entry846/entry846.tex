\documentclass{article}
\usepackage{planetmath-specials}
\usepackage{pmath}
\usepackage{amssymb}
\usepackage{amsmath}
\usepackage{amsfonts}

\def\N{\mathbb{N}}
\def\Z{\mathbb{Z}}
\begin{document}
\PMlinkescapeword{finite}
\PMlinkescapeword{quotient}
\PMlinkescapeword{quotients}

A group is said to be \emph{cyclic} if it is generated by a single element.

Suppose $G$ is a cyclic group generated by $x\in G$.
Then every element of $G$ is equal to $x^k$ for some $k\in \Z$.
If $G$ is infinite, then these $x^k$ are all distinct,
and $G$ is isomorphic to the group $\Z$.
If $G$ has \PMlinkname{finite order}{OrderGroup} $n$,
then every element of $G$ can be expressed as $x^k$ with $k\in\{0,\dots,n-1\}$,
and $G$ is isomorphic to the quotient group $\Z/n\Z$.

Note that the isomorphisms mentioned in the previous paragraph
imply that all cyclic groups of the same order are isomorphic to one another.
The infinite cyclic group is sometimes written $C_\infty$,
and the finite cyclic group of order $n$ is sometimes written $C_n$.
However, when the cyclic groups are written additively,
they are commonly represented by $\Z$ and $\Z/n\Z$.

While a cyclic group can, by definition, be generated by a single element,
there are often a number of different elements that can be used as the generator: an infinite cyclic group has $2$ generators,
and a finite cyclic group of order $n$ has $\phi(n)$ generators,
where $\phi$ is the Euler totient function.

Some basic facts about cyclic groups:
\begin{itemize}
\item Every cyclic group is abelian.
\item Every subgroup of a cyclic group is cyclic.
\item Every quotient of a cyclic group is cyclic.
\item Every group of prime order is cyclic. (This follows immediately from Lagrange's Theorem.)
\item Every finite subgroup of the multiplicative group of a field is cyclic.
\end{itemize}
\end{document}
