\documentclass{article}
\usepackage{ids}
\usepackage{planetmath-specials}
\usepackage{pmath}
% this is the default PlanetMath preamble.  as your knowledge
% of TeX increases, you will probably want to edit this, but
% it should be fine as is for beginners.

% almost certainly you want these
\usepackage{amssymb}
\usepackage{amsmath}
\usepackage{amsfonts}

% used for TeXing text within eps files
%\usepackage{psfrag}
% need this for including graphics (\includegraphics)
%\usepackage{graphicx}
% for neatly defining theorems and propositions
%\usepackage{amsthm}
% making logically defined graphics
%\usepackage{xypic} 

% there are many more packages, add them here as you need them

% define commands here
\renewcommand{\v}{\mathbf{v}}
\newcommand{\w}{\mathbf{w}}

\begin{document}
A \emph{sesquilinear form} over a pair of complex vector spaces $(V,W)$ is a function $B\colon V \times W \to \mathbb{C}$ satisfying the following properties:
\begin{enumerate}
\item $B(\v_1+\v_2,\w) = B(\v_1,\w) + B(\v_2,\w)$
\item $B(\v,\w_1+\w_2) = B(\v,\w_1) + B(\v,\w_2)$
\item $B(c\v, d\w) = c B(\v,\w) \overline{d}$
\end{enumerate}
for all $\v,\v_1,\v_2 \in V$, $\w, \w_1, \w_2 \in W$, and $c,d \in \mathbb{C}$.  The vector spaces $V$ and $W$ are often identical, although the definition does not require them to be the same vector space.

A sesquilinear form $B\colon V \times V \to \mathbb{C}$ over a single vector space $V$ is called a \emph{Hermitian form} if it is complex conjugate symmetric: namely, if $B(\v_1,\v_2) = \overline{B(\v_2,\v_1)}$.

An inner product over a complex vector space is a positive definite Hermitian form.
\end{document}
