\documentclass{article}
\usepackage{planetmath-specials}
\usepackage{pmath}
\usepackage{amssymb}
\usepackage{amsmath}
\usepackage{amsfonts}
\usepackage{graphicx}
\usepackage{xypic}
\begin{document}
\PMlinkescapeword{covering}
Let $X$ and $E$ be topological spaces and suppose there is a surjective continuous map 
$p\colon E \rightarrow X$ which satisfies the following condition: for each $x \in X$, there is an open neighborhood $U$ of $x$ such that 
\begin{itemize}
\item $p^{-1}(U)$ is a disjoint union of open sets $E_{i} \subset E$, and 
\item each $E_{i}$ is mapped homeomorphically onto $U$ via $p$. 
\end{itemize}
Then $E$ is called a \emph{covering space}, $p$ is called a \emph{covering map}, the $E_{i}$'s are \emph{sheets} of the covering of $U$ and for each $x \in X$, $p^{-1}(x)$ is the \emph{fiber} of $p$ above $x$. The open set $U$ is said to be \emph{evenly covered}.
If $E$ is simply connected, it is called the \emph{universal covering space}.

From this we can derive that $p$ is a local homeomorphism, so that any local property $E$ 
has is inherited by $X$ (local connectedness, local path connectedness etc.). Covering spaces
are foundational in the study of the fundamental group of a topological space; in particular, there is a \PMlinkname{correspondence}{ClassificationOfCoveringSpaces} between connected coverings of $X$ and subgroups of the group of deck transformations of its universal covering space which is exactly analogous to the fundamental theorem of Galois theory.

Covering maps are especially important in the study of Riemann surfaces; in this context, one sometimes discusses a generalized notion of covering map called a ``ramified covering''; this allows one to replace a discrete set of the local homeomorphisms by maps that locally look like $z\mapsto z^n$ in the complex plane near zero. Covering maps are also generalized in algebraic geometry; there the corresponding notion is that of \'etale morphism.


Note that this is a completely separate usage of the word ``cover'' than we encounter in ``open cover''; confusion usually does not arise.
\end{document}
