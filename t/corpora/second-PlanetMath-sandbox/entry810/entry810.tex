\documentclass{article}
\usepackage{planetmath-specials}
\usepackage{pmath}
\usepackage{amssymb}
\usepackage{amsmath}
\usepackage{amsfonts}
\usepackage{amsthm}

%\usepackage{psfrag}
%\usepackage{graphicx}
%\usepackage{xypic}
\theoremstyle{definition}
\newtheorem*{defn}{Multiset}
\begin{document}
A \emph{multiset} is a set for which repeated elements are considered.

Note that the standard definition of a set also allows repeated elements, but these are not \emph{treated} as repeated elements.  For example, $\{1,1,3\}$ as a \emph{set} is actually equal to $\{1,3\}$. However, as a \emph{multiset}, $\{1,1,3\}$ is not simplifiable further.

A definition that makes clear the distinction between set and multiset follows:

\begin{defn}
A multiset is a pair $(X, f)$, where $X$ is a set, and $f$ is a function mapping $X$ to the cardinal numbers greater than zero.  $X$ is called the \emph{underlying set} of the multiset, and for any $x \in X$, $f(x)$ is the \emph{multiplicity} of $x$.
\end{defn}

Using this definition and expressing $f$ as a set of ordered pairs, we see that the multiset $\{1,3\}$ has $X = \{1,3\}$ and $f = \{(1,1),(3,1)\}$.  By contrast, the multiset $\{1,1,3\}$  has $X = \{1,3\}$  and $f = \{(1,2),(3,1)\}$.

Generally, a multiplicity of zero is not allowed, but a few mathematicians do allow for it, such as Bogart and Stanley. It is far more common to disallow infinite multiplicity, which greatly complicates the definition of operations such as unions, intersections, complements, etc.

\begin{thebibliography}{3}
\bibitem{kb} Kenneth P. Bogart, {\it Introductory Combinatorics}. Florence, Kentucky: Cengage Learning (2000): 93
\bibitem{jh} John L. Hickman, ``A note on the concept of multiset'' {\it Bulletin of the Australian Mathematical Society}  {\bf 22} (1980): 211 - 217
\bibitem{rs} Richard P. Stanley, {\it Enumerative Combinatorics} Vol 1. Cambridge: Cambridge University Press (1997): 15
\end{thebibliography}
\end{document}
