\documentclass{article}
\usepackage{ids}
\usepackage{planetmath-specials}
\usepackage{pmath}
% this is the default PlanetMath preamble.  as your knowledge
% of TeX increases, you will probably want to edit this, but
% it should be fine as is for beginners.

% almost certainly you want these
\usepackage{amssymb}
\usepackage{amsmath}
\usepackage{amsfonts}

% used for TeXing text within eps files
%\usepackage{psfrag}
% need this for including graphics (\includegraphics)
%\usepackage{graphicx}
% for neatly defining theorems and propositions
%\usepackage{amsthm}
% making logically defined graphics
%\usepackage{xypic} 

% there are many more packages, add them here as you need them

% define commands here
\begin{document}
Let $X$ be a topological space and let $x\in X$.  $X$ is said to be \emph{\PMlinkescapetext{first countable} at $x$} if there is a sequence $(B_n)_{n\in\mathbb{N}}$ of open sets such that whenever $U$ is an open set containing $x$, there is $n\in\mathbb{N}$ such that $x\in B_n\subseteq U$.

The space $X$ is said to be \emph{\PMlinkescapetext{first countable}} if for every $x\in X$, $X$ is first countable at $x$.

\textbf{Remark}.  Equivalently, one can take each $B_n$ in the sequence to be open neighborhood of $x$.
\end{document}
