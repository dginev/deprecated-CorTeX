\documentclass{article}
\usepackage{ids}
\usepackage{planetmath-specials}
\usepackage{pmath}
\usepackage{amssymb}
\usepackage{amsmath}
\usepackage{amsfonts}
\usepackage{graphicx}
\usepackage{xypic}
\begin{document}
We prove the theorem for a ring. We do not assume a unit for the ring.
We do not need commutativity of the ring, but only that $a$ and $b$ commute.

When $n=1$, the result is clear.

For the inductive step, assume it holds for $m$. Then for $n = m+1$,
\begin{eqnarray*}
(a+b)^{m+1} & = & (a+b)(a+b)^m  \\
& = & (a+b)(a^m + b^m+ \sum_{k=1}^{m-1} \binom{m}{k} a^{m-k} b^k )\text{ by the inductive hypothesis} \\
& = & a^{m+1} + b^{m+1} + ab^m + ba^m + \sum_{k=1}^{m-1} \binom{m}{k} a^{m-k+1} b^k + \sum_{k=1}^{m-1} \binom{m}{k} a^{m-k} b^{k+1} \\ 
& = & a^{m+1} + b^{m+1} + \sum_{k=1}^m \binom{m}{k} a^{m-k+1} b^k + \sum_{k=0}^{m-1} \binom{m}{k} a^{m-k} b^{k+1} \text{ by combining terms} \\
& = & a^{m+1} + b^{m+1} + \sum_{k=1}^m \binom{m}{k} a^{m-k+1} b^k + \sum_{j=1}^m \binom{m}{j-1} a^{m+1-j} b^j \text{ let j=k+1 in second sum} \\
& = & a^{m+1} + b^{m+1} + \sum_{k=1}^m \left[ \binom{m}{k} + \binom{m}{k-1} \right] a^{m+1-k}b^k \text{ by combining the sums}\\
& = & a^{m+1} + b^{m+1} + \sum_{k=1}^m \binom{m+1}{k} a^{m+1-k}b^k \text{ from Pascal's rule} \\
\end{eqnarray*}
as desired.
\end{document}
