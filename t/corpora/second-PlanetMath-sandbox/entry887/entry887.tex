\documentclass{article}
\usepackage{planetmath-specials}
\usepackage{pmath}
\usepackage{amsmath}
\usepackage{amsfonts}
\usepackage{amssymb}

\newcommand{\reals}{\mathbb{R}}
\newcommand{\natnums}{\mathbb{N}}
\newcommand{\cnums}{\mathbb{C}}

\newcommand{\lp}{\left(}
\newcommand{\rp}{\right)}
\newcommand{\lb}{\left[}
\newcommand{\rb}{\right]}

\newcommand{\supth}{^{\text{th}}}


\newtheorem{proposition}{Proposition}
\begin{document}
\paragraph{Definition.}
Three real numbers $x, y, p$, with $x,y>0$ and $x \neq 1$, are said to obey the
logarithmic relation
$$\log_x(y)= p$$
if they obey the corresponding exponential relation:
$$x^p=y.$$
Note that by the monotonicity and continuity property of the
exponential operation, for given $x$ and $y$ there exists a unique $p$
satisfying the above relation.  We are therefore able to says that
{\em $p$ is the logarithm of $y$ relative to the base $x$.}

\paragraph{Properties.}
There are a number of basic algebraic identities involving logarithms.
\begin{align*}
\log_x(yz) &= \log_x(y) + \log_x(z)\\ 
\log_x \left( \frac{y}{z} \right) &= \log_x(y) - \log_x(z)\\
\log_x(y^z) &= z \log_x(y) \\
\log_x(1) &= 0 \\
\log_x(x) &= 1 \\
\log_x(y) \log_y(x) &= 1\\
\log_y(z) &= \frac{\log_x(z)}{\log_x(y)}
\end{align*}

By the very first identity, any logarithm \PMlinkname{restricted}{RestrictionOfAFunction} to the set of positive integers is an additive function.

{\bf Notes.} In essence, logarithms convert multiplication to
addition, and exponentiation to multiplication. Historically, these
properties of the logarithm made it a useful tool for doing numerical
calculations. Before the advent of electronic calculators and
computers, tables of logarithms and the logarithmic slide rule were
essential computational aids.

Scientific applications predominantly make use of
logarithms whose base is the Eulerian number $e = 2.71828\ldots$.
Such logarithms are called {\em natural logarithms} and are commonly
denoted by the symbol $\ln$, e.g.
$$\ln(e) = 1.$$
Natural logarithms naturally give rise to the natural logarithm function.

A frequent convention, seen in elementary mathematics texts and on
calculators, is that logarithms that do not give a base explicitly are
assumed to be base $10$, e.g.
$$\log(100) = 2.$$
This is far from \PMlinkescapetext{universal}.  In Rudin's ``Real and
Complex analysis'', for example, we see a baseless $\log$ used to
refer to the natural logarithm.  By contrast, computer science and
information theory texts often assume 2 as the default logarithm base.
This is motivated by the fact that $\log_2(N)$ is the approximate
number of bits required to encode $N$ different messages.


The invention of logarithms is commonly credited to John Napier [
\PMlinkexternal{Biography}{http://www-groups.dcs.st-and.ac.uk/~history/Mathematicians/Napier.html}]
\end{document}
