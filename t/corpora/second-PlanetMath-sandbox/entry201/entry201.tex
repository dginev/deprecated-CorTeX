\documentclass{article}
\usepackage{ids}
\usepackage{planetmath-specials}
\usepackage{pmath}
\usepackage{amssymb}
\usepackage{amsmath}
\usepackage{amsfonts}
\usepackage{graphicx}
\usepackage{xypic}
\begin{document}
{\bf Definition. }
Let $R$ be a ring and let $I$ be a \PMlinkname{two-sided ideal}{Ideal} of $R$.
To define the quotient ring $R/I$, let us first
define an equivalence relation in $R$. We say that the elements $a,b\in R$
are equivalent, written as $a\sim b$, if and only if $a-b\in I$.
If $a$ is an element of $R$, we denote the corresponding equivalence
class by $[a]$. Thus $[a]=[b]$ if and only if $a-b\in I$.
The {\em quotient ring} of $R$ modulo $I$ is the set
$R/I=\{[a]\, |\, a\in R\}$, with a ring structure defined as follows.
If $[a],[b]$ are equivalence classes in $R/I$, then
 \begin{itemize}
 \item $[a]+[b] = [a+b]$,
 \item $[a]\cdot [b]=[a\cdot b]$.
 \end{itemize}
Here $a$ and $b$ are some elements in $R$ that represent $[a]$ and $[b]$.
By construction, every element in $R/I$ has such a representative in $R$.
Moreover, since $I$ is closed under addition and multiplication, one can
 verify that the ring structure in $R/I$ is well defined.

A common notation is $a+I=[a]$ which is consistent with the notion of classes $[a]=aH\in G/H$ for a group $G$ and a normal subgroup $H$.

\subsubsection*{Properties}
\begin{enumerate}
\item If $R$ is commutative, then $R/I$ is commutative.
\item The mapping $R\to R/I$, $a\mapsto [a]$ is a homomorphism, and 
is called the \PMlinkname{natural homomorphism}{NaturalHomomorphism}. 
\end{enumerate}

\subsubsection*{Examples}
\begin{enumerate}
\item For a ring $R$, we have $R/R=\{[0]\}$ and $R/\{0\}=R$.
\item Let $R=\mathbb{Z}$, and let $I=2\mathbb{Z}$ be the set of even numbers.
Then $R/I$ contains only two classes; one for  even numbers,
and one for odd numbers. Actually this quotient ring is a field. It is the only field with two elements (up to isomorphy) and is also denoted by $\mathbb{F}_2$.
\item One way to construct complex numbers is to consider the field $\mathbb{R}[T]/(T^2+1)$. This field can viewed as the set of all polynomials of degree $1$ with normal addition and $(a+bT)(c+dT)=ac-bd+(ad+bc)T$, which is like complex multiplication.
\end{enumerate}
\end{document}
