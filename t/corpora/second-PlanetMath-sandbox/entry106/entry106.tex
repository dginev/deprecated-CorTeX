\documentclass{article}
\usepackage{planetmath-specials}
\usepackage{pmath}
\usepackage{amssymb}
\usepackage{amsmath}
\usepackage{amsfonts}
\usepackage{amsthm}
\usepackage[dvips]{graphics}
\begin{document}
\PMlinkescapeword{arrow}
\PMlinkescapeword{equivalence}
\PMlinkescapeword{equivalences}
\PMlinkescapeword{right}
\PMlinkescapeword{scenario}
\newtheorem*{axiom*}{Axiom}

The Zermelo-Fraenkel axioms for set theory are often accepted as a basis for an axiomatic set theory. 
On the other hand, the \emph{axiom of choice} is somewhat controversial, and it is currently segregated from the ZF system of set theory axioms.  When the axiom of choice is combined with the ZF axioms, the whole axiom system is
called ``ZFC'' (for ``Zermelo-Fraenkel with Choice'').

\begin{axiom*}[Axiom of choice]
Let $C$ be a set of nonempty sets. Then there exists a function
\[
f\colon C\to \bigcup_{S\in C} S
\]
such that $f(x) \in x$ for all $x \in C$.
\end{axiom*}
The function $f$ is sometimes called a \emph{choice function} on $C$.

For finite sets $C$, a choice function can be constructed without appealing to the axiom of choice.  In particular, if $C=\varnothing$, then the choice function is clear: it is the empty set!  It is only for infinite (and usually uncountable) sets $C$ that the existence of a choice function becomes an issue. Here one can see why it is not considered ``obvious'' and always taken for an axiom by everyone:
one really cannot imagine any process which makes uncountably many selections without also imagining some rule for making the selections. Given such a rule, the axiom of choice is not needed.  Thus, objects that are proved to exist using the axiom of choice cannot generally be described by any kind of systematic rule, for if they could it would not be necessary to their construction.

Let us consider a couple of examples.  Imagine that there are infinitely many pairs of shoes (each consisting of one left shoe and one right shoe).  Let $\mathcal{P}$ denote the set of all pairs of shoes.  In this scenario, it can be verified that the function $\displaystyle f \colon \mathcal{P} \to \bigcup \mathcal{P}$ defined by $f(P)=\text{ the left shoe of }P$ is a choice function.  Similarly, imagine that there are infinitely many pairs of socks.  Let $\mathcal{S}$ denote the set of all pairs of socks.  In this scenario, one cannot assume that a function $\displaystyle g \colon \mathcal{S} \to \bigcup \mathcal{S}$ exists without appealing to the axiom of choice (or something equivalent to it).  Note that this scenario cannot be resolved in the same manner as the previous scenario because most people do not \PMlinkescapetext{differentiate between} a "left sock" and a "right sock".

Strange objects that can be constructed using the axiom of choice include non-measurable sets (leading to the \PMlinkid{Hausdorff}{7057} and \PMlinkid{Banach-Tarski paradoxa}{4464}), and Hamel bases for any vector space. A Hamel
basis may not seem strange, but try to imagine a set $S$ of continuous functions such that every continuous function can be expressed uniquely as a linear combination of \emph{finitely many} elements of $S$. Since in fact the existence of a basis for every vector space is equivalent to the axiom of
choice, it is almost guaranteed that no such set $S$ can ever be described. It is for this reason that some mathematicians dislike the axiom of choice.

On the other hand, many very useful facts can be proven using the axiom of choice. For example, the fact that every vector space has a basis, every ring with identity element $1\neq 0$ has a maximal ideal and many other algebraic theorems which are difficult or impossible to prove without using the axiom of choice.

In pure set theory, the axiom of choice is only relevant where most people's intuition more or less breaks down, when dealing with hierarchies of uncountable infinities.

The relevance of the axiom of choice to various \PMlinkescapetext{branches} of mathematics has led to a detailed study of its truth. It turns out that if the Zermelo-Fraenkel axioms are consistent, then they remain consistent upon adding the axiom of choice. But they also remain consistent upon adding the negation of the axiom of choice (see~\cite{G} and~\cite{C}).

Some mathematicians have suggested an axiom that would result in all subsets of the real numbers being measurable; this would of course imply the negation of the axiom of choice.

There are many alternative formulations of the axiom of choice, although it is not always trivial to prove \PMlinkescapetext{equivalence}. These include:
\begin{itemize}
\item The generalized Cartesian product of a non-empty family of sets is non-empty iff every set is non-empty.  Due to this equivalence the axiom of choice is sometimes called the \emph{multiplicative axiom} (\PMlinkid{Proof}{11514}).
\item Every relation $R$ is a union of functions with the same domain as $R$ (\PMlinkid{Proof}{8763}).
\item Every ring with identity element $1\neq 0$ has a maximal ideal (\PMlinkid{Proof}{4713}).
\item Every surjection is a \PMlinkname{retraction}{TypesOfMorphisms} (\PMlinkid{Proof}{11517}).
\item Every tree has a branch.
\item Every vector space has a basis (for AC implying existence of vector bases, see \PMlinkid{here}{3494}, and its converse, proved by Andreas Blass in 1984, \PMlinkexternal{here}{http://www.math.lsa.umich.edu/~ablass/bases-AC.pdf}).
\item Given any two sets, there exists an injection from one to another (\PMlinkid{Proof}{11645}).
\item Hausdorff's maximum principle (\PMlinkid{Proof}{3493}).
\item \PMlinkid{K\"onig's theorem}{5598}.
\item Kuratowski's lemma (\PMlinkid{Proof}{11512}).
\item Tukey's lemma (\PMlinkid{Proof}{4672}).
\item Tychonoff's theorem (for AC implying Tychonoff, see \PMlinkid{here}{9797}, and its converse \PMlinkid{here}{11540}).
\item Zermelo's postulate (\PMlinkid{Proof}{8342}).
\item Zermelo's well-ordering theorem (\PMlinkid{Proof}{3493}).
\item Zorn's lemma (\PMlinkid{Proof}{3358}).
\end{itemize}


Figure~\ref{fig:implications} shows how these equivalences are proven on PlanetMath.

\begin{figure}
\label{fig:implications}
\begin{center}
\includegraphics{AxiomOfChoice.1.eps}
%(Something is wrong with the PM file upload mechanism. Please stand by.)
\end{center}
\sf\caption{Structure of the equivalence proofs on PlanetMath. The
abbreviations are explained in table~\ref{tab:abbr}\@. An arrow
$A\rightarrow B$ means that ``$A$ implies $B$'' is proven on
PlanetMath.}
\end{figure}

\begin{table}
\label{tab:abbr}
\begin{center}
\begin{tabular}{ll}
AC&Axiom of Choice\\
Hamel&Every vector space has a basis\\
Hausdorff&Hausdorff's maximum principle \\
K\"onig & K\"onig's theorem \\
Krull & Every ring with identity element $1\neq 0$ has a maximal ideal\\
Tukey & Tukey's lemma\\
Well&\PMlinkname{Zermelo's well-ordering theorem}{ZermelosWellOrderingTheorem}\\
Zorn&Zorn's lemma
\end{tabular}
\end{center}
\sf\caption{Abbreviations}
\end{table}

\begin{thebibliography}{C}

\bibitem[C]{C} \textsc{P.~J.~Cohen}, The independence of the continuum
hypothesis.~I,~II, \emph{Proc.\ Natl.\ Acad.\ Sci.}\ USA 50,
1143--1148 (1963); 51, 105--110 (1964).

\bibitem[G]{G} \textsc{K.~G\"{o}del}, The consistency of the axiom of
choice and of the generalized continuum-hypothesis, \emph{Proc.\
Natl.\ Acad.\ Sci.}\ USA 24, 556--557 (1938).

\end{thebibliography}
\end{document}
