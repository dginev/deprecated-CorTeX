\documentclass{article}
\usepackage{planetmath-specials}
\usepackage{pmath}
% this is the default PlanetMath preamble.  as your knowledge
% of TeX increases, you will probably want to edit this, but
% it should be fine as is for beginners.

% almost certainly you want these
\usepackage{amssymb}
\usepackage{amsmath}
\usepackage{amsfonts}

% used for TeXing text within eps files
%\usepackage{psfrag}
% need this for including graphics (\includegraphics)
%\usepackage{graphicx}
% for neatly defining theorems and propositions
%\usepackage{amsthm}
% making logically defined graphics
%\usepackage{xypic}

% there are many more packages, add them here as you need them

% define commands here
\begin{document}
The \emph{completeness principle} is a property of the real numbers, and is one of the foundations of real analysis. The most common formulation of this principle is that every non-empty set which is bounded from above has a supremum.

This statement can be reformulated in several ways.  Each of the following statements is \PMlinkescapetext{equivalent} to the above definition of the completeness principle:
\begin{enumerate}
\item The limit of every infinite decimal sequence is a real number.
\item Every bounded monotonic sequence is convergent.
\item A sequence is convergent iff it is a Cauchy Sequence.
\end{enumerate}
\end{document}
