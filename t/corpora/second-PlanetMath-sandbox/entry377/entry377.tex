\documentclass{article}
\usepackage{planetmath-specials}
\usepackage{pmath}
\usepackage{amssymb}
\usepackage{amsmath}
\usepackage{amsfonts}
\usepackage{graphicx}
\usepackage{xypic}
\begin{document}
\textbf{Definition:} \newline Let $A$ be an square matrix of
 order $n$ with real entries $(a_{ij})$. 
The matrix $A$ is skew-symmetric if $a_{ij} =
 -a_{ji}$ for all $1 \leq i \leq n, 1 \leq j \leq n$.
 \begin{center}$A =
 \begin{pmatrix}
  a_{11}=0 & \cdots & a_{1n} \\
  \vdots & \ddots & \vdots \\
  a_{n1} & \cdots & a_{nn}=0
 \end{pmatrix}$
 \end{center}
 The main diagonal entries are zero because 
$a_{i,i} = -a_{i,i}$ implies $a_{i,i} = 0$.

One can see skew-symmetric matrices as a
special case of  complex skew-Hermitian matrices. Thus, 
all properties of skew-Hermitian matrices also hold
for skew-symmetric matrices. 

 \textbf{Properties:}
 \begin{enumerate}
 \item The matrix $A$ is skew-symmetric if and only if 
$A^t = -A$, where $A^t$ is the matrix transpose
 \item For the trace operator, we have that 
$\operatorname{tr}(A) = \operatorname{tr}(A^t)$.
Combining this with property (1), it follows
that $ \operatorname{tr}(A)=0$ for a skew-symmetric matrix $A$.
 \item Skew-symmetric matrices form a vector space: If $A$ and $B$
are skew-symmetric and $\alpha, \beta\in \mathbb{R}$, then 
$\alpha A + \beta B$ is also skew-symmetric. 
 \item Suppose $A$ is a skew-symmetric matrix and $B$ is a matrix of
same order as $A$. Then $B^t A B$ is skew-symmetric.
\item All eigenvalues of skew-symmetric matrices are
purely imaginary or zero. This result is proven on the page
for skew-Hermitian matrices. 
\item According to Jacobi's Theorem, the determinant of a
skew-symmetric matrix of odd order is zero. 
 \end{enumerate}
 
\textbf{Examples:}
 \begin{itemize}
 \item $\begin{pmatrix}
  0 & b \\
  -b & 0
  \end{pmatrix}$
 \item $\begin{pmatrix}
  0 & b & c \\
  -b & 0 & e \\
  -c & -e & 0
  \end{pmatrix}$
 \end{itemize}
\end{document}
