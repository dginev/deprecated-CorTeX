\documentclass{article}
\usepackage{planetmath-specials}
\usepackage{pmath}
\usepackage{amssymb}

\def\R{\mathbb{R}}

\begin{document}
A series $\sum a_n$ is said to be \emph{convergent}
if the sequence of partial sums $\sum_{i=1}^n a_i$ is convergent.
A series that is not convergent is said to be \emph{divergent}.

A series $\sum a_n$ is said to be \emph{absolutely convergent}
if $\sum |a_n|$ is convergent.

When the terms of the series live in $\R^n$,
an equivalent condition for absolute convergence of the series
is that all possible series obtained by rearrangements
of the terms are also convergent.
(This is not true in arbitrary metric spaces.)

It can be shown that absolute convergence implies convergence.
A series that converges, but is not absolutely convergent,
is called conditionally convergent.

Let $\sum a_n$ be an absolutely convergent series,
and $\sum b_n$ be a conditionally convergent series.
Then any rearrangement of $\sum a_n$ is convergent to the same sum.
It is a result due to Riemann that $\sum b_n$
can be rearranged to converge to any sum, or not converge at all.

\end{document}
