\documentclass{article}
\usepackage{planetmath-specials}
\usepackage{pmath}
\usepackage{amssymb}
\usepackage{amsmath}
\usepackage{amsfonts}
\usepackage{graphicx}
\usepackage{xypic}
\renewcommand{\u}{\mathbf{u}}
\renewcommand{\v}{\mathbf{v}}
\newcommand{\w}{\mathbf{w}}
\newcommand{\0}{\mathbf{0}}
\begin{document}
Let $F$ be a field (or, more generally, a division ring). A \emph{vector space} $V$ over $F$ is a set with two operations, $+: V \times V \longrightarrow V$ and $\cdot: F \times V \longrightarrow V$, such that
\begin{enumerate}
\item $(\u+\v)+\w = \u+(\v+\w)$ for all $\u,\v,\w \in V$
\item $\u+\v=\v+\u$ for all $\u,\v\in V$
\item There exists an element $\0 \in V$ such that $\u+\0=\u$ for all $\u \in V$
\item For any $\u \in V$, there exists an element $\v \in V$ such that $\u+\v=\0$
\item $a \cdot (b \cdot \u) = (a \cdot b) \cdot \u$ for all $a,b \in F$ and $\u \in V$
\item $1 \cdot \u = \u$ for all $\u \in V$
\item $a \cdot (\u+\v) = (a \cdot \u) + (a \cdot \v)$ for all $a \in F$ and $\u,\v \in V$
\item $(a+b) \cdot \u = (a \cdot \u) + (b \cdot \u)$ for all $a,b \in F$ and $\u \in V$
\end{enumerate}

Equivalently, a vector space is a module $V$ over a ring $F$ which is a field (or, more generally, a division ring).

The elements of $V$ are called \emph{vectors}, and the element $\0 \in V$ is called the \emph{zero vector} of $V$.
\end{document}
