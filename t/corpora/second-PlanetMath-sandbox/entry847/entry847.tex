\documentclass{article}
\usepackage{planetmath-specials}
\usepackage{pmath}
\usepackage{amsfonts}

% The following lines should work as the command
% \renewcommand{\bibname}{References}
% without creating havoc when rendering an entry in
% the page-image mode.
\makeatletter
\@ifundefined{bibname}{}{\renewcommand{\bibname}{References}}
\makeatother

\begin{document}
\PMlinkescapeword{contains}
\PMlinkescapeword{divisible}
\PMlinkescapeword{multiple}

Let $G$ be a finite group, and suppose $p$ is a prime divisor of $|G|$.
Consider the set $X$ of all $p$-tuples $(x_1, \ldots, x_p)$ 
for which $x_1\cdots x_p = 1$.
Note that $|X| = |G|^{p-1}$ is a multiple of $p$.
There is a natural group action of 
the cyclic group $\mathbb{Z}/p\mathbb{Z}$ on $X$
under which $m \in \mathbb{Z}/p\mathbb{Z}$ sends 
the tuple $(x_1, \ldots, x_p)$ 
to $(x_{m+1}, \ldots, x_p, x_1, \ldots, x_m)$.
By the Orbit-Stabilizer Theorem, each orbit contains exactly $1$ or $p$ tuples.
Since $(1,\ldots, 1)$ has an orbit of cardinality $1$, 
and the orbits partition $X$,
the cardinality of which is divisible by $p$,
there must exist at least one other tuple $(x_1,\ldots, x_p)$
which is left fixed by every element of $\mathbb{Z}/p\mathbb{Z}$.
For this tuple we have $x_1 = \ldots = x_p$,
and so $x_1^p=x_1\cdots x_p=1$,
and $x_1$ is therefore an element of order $p$.

\begin{thebibliography}{9}
\bibitem{mckay}
 James H.~McKay.
 {\it Another Proof of Cauchy's Group Theorem},
 American Math.~Monthly, 66 (1959), p119.
\end{thebibliography}

\end{document}
