\documentclass{article}
\usepackage{planetmath-specials}
\usepackage{pmath}
% this is the default PlanetMath preamble.  as your knowledge
% of TeX increases, you will probably want to edit this, but
% it should be fine as is for beginners.

% almost certainly you want these
\usepackage{amssymb}
\usepackage{amsmath}
\usepackage{amsfonts}
\usepackage{amsthm}

\usepackage{mathrsfs}

% used for TeXing text within eps files
%\usepackage{psfrag}
% need this for including graphics (\includegraphics)
%\usepackage{graphicx}
% for neatly defining theorems and propositions
%
% making logically defined graphics
%\usepackage{xypic}

% there are many more packages, add them here as you need them

% define commands here

\newcommand{\sR}[0]{\mathbb{R}}
\newcommand{\sC}[0]{\mathbb{C}}
\newcommand{\sN}[0]{\mathbb{N}}
\newcommand{\sZ}[0]{\mathbb{Z}}

 \usepackage{bbm}
 \newcommand{\Z}{\mathbbmss{Z}}
 \newcommand{\C}{\mathbbmss{C}}
 \newcommand{\R}{\mathbbmss{R}}
 \newcommand{\Q}{\mathbbmss{Q}}



\newcommand*{\norm}[1]{\lVert #1 \rVert}
\newcommand*{\abs}[1]{| #1 |}



\newtheorem{thm}{Theorem}
\newtheorem{defn}{Definition}
\newtheorem{prop}{Proposition}
\newtheorem{lemma}{Lemma}
\newtheorem{cor}{Corollary}
\begin{document}
Let $U$ be an open set in $\R^n$ and $f\colon U\to \C$ is a differentiable
function. If $u\in U$ and $v\in \sR^n$, then the 
\emph{directional derivative} of $f$ in the direction of $v$ is
$$
  (D_v f)(u) = \frac{d}{ds} f(u+sv) \Big|_{s=0}.
$$
In other words, $(D_v f)(u)$ measures how $f$ changes in the direction of $v$
from $u$. 

Alternatively, 
\begin{eqnarray*}
 (D_v f)(u) &=& \lim_{h\to 0} \frac{ f(u+ hv) - f(u)}{h} \\
   &=& Df(u)\cdot v,
\end{eqnarray*}
where $Df$ is the Jacobian matrix of $f$.

\subsubsection*{Properties}
Let $u\in U$. 
\begin{enumerate}
\item $D_v f$ is linear in $v$. If $v, w\in \R^n$ and $\lambda, \mu \in \R$, 
then 
$$
   D_{\lambda v+\mu w}f(u) =    \lambda D_{v}f(u) +\mu D_{w}f(u).
$$
In particular, $D_0 f=0$.
\item If $f$ is twice differentiable and $v,w\in \R^n$, then 
\begin{eqnarray*}
   D_v D_w f(u) &=& \frac{\partial^2}{\partial s\partial t} f(u+sv + tw) \Big|_{s=0}, \\
     &=& v^T\cdot \operatorname{Hess}f(u)\cdot w,
\end{eqnarray*}
where $\operatorname{Hess}$ is the Hessian matrix of $f$. 
\end{enumerate}

\subsubsection*{Example}
For example, if $f\left(\begin{array}{c}x\\y\\z\end{array}\right) = x^2 + 3y^2z$, and we wanted to find the derivative at the point $\mathbf{a}=\left(\begin{array}{c}1\\2\\3\end{array}\right)$ in the direction $\vec{v}=\left[\begin{array}{c}1\\1\\1\end{array}\right]$, our equation would be
\begin{eqnarray*}
\lim_{h\rightarrow 0}\frac{1}{h}\left((1+h)^2 + 3(2+h)^2(3+h) - 37\right)
&=&\lim_{h\rightarrow 0}\frac{1}{h}(3h^3+37h^2+50h)\\
&=&\lim_{h\rightarrow 0}3h^2+37h +50 = 50\end{eqnarray*}
\end{document}
