\documentclass{article}
\usepackage{ids}
\usepackage{planetmath-specials}
\usepackage{pmath}
\usepackage{amssymb}
\usepackage{amsmath}
\usepackage{amsfonts}
\usepackage{graphicx}
\usepackage{xypic}
\begin{document}
An \emph{Euler path} in a graph is a path which traverses each edge of the graph exactly once.  An Euler path which is a cycle is called an \emph{Euler cycle}.  For loopless graphs without isolated vertices, the existence of an Euler path implies the connectedness of the graph, since traversing every edge of such a graph requires visiting each vertex at least once.

If a connected graph has an Euler path, one can be constructed by applying Fleury's algorithm.  A connected graph has an Euler path if it has exactly zero or two vertices of odd degree.  If every vertex has even degree, the graph has an Euler cycle.

\begin{center}
\includegraphics[width=1in,height=1in]{ecircuit}
\end{center}

This graph has an Euler cycle.  All of its vertices are of even degree.

\begin{center}
\includegraphics[width=1in,height=1in]{epath}
\end{center}

This graph has an Euler path which is not a cycle. It has exactly two vertices of odd degree.

\end{document}
