\documentclass{article}
\usepackage{planetmath-specials}
\usepackage{pmath}
% this is the default PlanetMath preamble.  as your knowledge
% of TeX increases, you will probably want to edit this, but
% it should be fine as is for beginners.

% almost certainly you want these
\usepackage{amssymb}
\usepackage{amsmath}
\usepackage{amsfonts}
\usepackage{amsthm}

% used for TeXing text within eps files
%\usepackage{psfrag}
% need this for including graphics (\includegraphics)
%\usepackage{graphicx}
% for neatly defining theorems and propositions
%\usepackage{amsthm}
% making logically defined graphics
%\usepackage{xypic}

% there are many more packages, add them here as you need them

% define commands here

\newcommand{\mc}{\mathcal}
\newcommand{\mb}{\mathbb}
\newcommand{\mf}{\mathfrak}
\newcommand{\ol}{\overline}
\newcommand{\ra}{\rightarrow}
\newcommand{\la}{\leftarrow}
\newcommand{\La}{\Leftarrow}
\newcommand{\Ra}{\Rightarrow}
\newcommand{\nor}{\vartriangleleft}
\newcommand{\Gal}{\text{Gal}}
\newcommand{\GL}{\text{GL}}
\newcommand{\Z}{\mb{Z}}
\newcommand{\R}{\mb{R}}
\newcommand{\Q}{\mb{Q}}
\newcommand{\C}{\mb{C}}
\newcommand{\<}{\langle}
\renewcommand{\>}{\rangle}
\begin{document}
If $(\Omega,\mc{A},P)$ is a probability space, then a \textbf{random variable} on $\Omega$ is a measurable function $X: (\Omega,\mc{A}) \to S$ to a measurable space $S$ (frequently taken to be the real numbers with the standard measure).  The \emph{law} of a random variable is the probability measure $PX^{-1}:S\to \R$ defined by $PX^{-1}(s)=P(X^{-1}(s))$.

A random variable $X$ is said to be \emph{discrete} if the set $ \{X(\omega) : \omega \in \Omega \}$ (i.e. the range of $X$) is finite or countable.  A more general version of this definition is as follows:  A random variable $X$ is discrete if there is a countable subset $B$ of the range of $X$  such that $P(X \in B)=1$ (Note that, as a countable subset of $\mathbb{R}$, $B$ is measurable).

A random variable $Y$ is said to be \emph{\PMlinkescapetext{continuous}} if it has a cumulative distribution function which is \PMlinkname{absolutely continuous}{AbsolutelyContinuousFunction2}.

Example:

Consider the event of throwing a coin. Thus, $\Omega = \{ H, T \}$ where $H$ is the event in which the coin falls head and $T$ the event in which falls tails.
Let $X=$number of tails in the experiment. Then $X$ is a (discrete) random variable.
\end{document}
