\documentclass{article}
\usepackage{planetmath-specials}
\usepackage{pmath}
\usepackage{amssymb}
\usepackage{amsmath}
\usepackage{amsfonts}
\usepackage{graphicx}
\usepackage{xypic}
\begin{document}
A sequence $x_0, x_1, x_2, \dots$ in a metric space $(X,d)$ is a \emph{Cauchy sequence} if, for every real number $\epsilon > 0$, there exists a natural number $N$ such that $d(x_n,x_m) < \epsilon$ whenever $n,m > N$.

Likewise, a sequence $v_0, v_1, v_2, \dots$ in a topological vector space $V$ is a \emph{Cauchy sequence} if and only if for every neighborhood $U$ of $\mathbf{0}$, there exists a natural number $N$ such that $v_n - v_m \in U$ for all $n,m > N$. These two definitions are equivalent when the topology of $V$ is induced by a metric.
\end{document}
