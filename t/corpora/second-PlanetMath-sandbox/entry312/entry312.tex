\documentclass{article}
\usepackage{ids}
\usepackage{planetmath-specials}
\usepackage{pmath}
\usepackage{amssymb}
\usepackage{amsmath}
\usepackage{amsfonts}
\usepackage{graphicx}
\usepackage{xypic}

\begin{document}
A \emph{multigraph} is a graph in which we allow more than one edge to join a pair of vertices. Two or more edges that join a pair of vertices are called \emph{parallel edges}. Every graph, then, is a multigraph, but not all multigraphs are graphs.
Some authors define the concept of a graph by excluding graphs with multiple 
edges or loops. Then if they want to consider more general graphs the \PMlinkescapetext{term}
multigraph is introduced. Usually, such graphs have no loops.
Formally, a multigraph $G=(V,  E)$ is a pair, where $E=(V^{(2)}, f)$
is a multiset for which $f(x,x) = 0$ and $V^{(2)}$ is the set of unordered pairs
of $V$.

A multigraph can be used to \PMlinkescapetext{represent} a matrix whose entries are nonnegative integers. To do this, suppose that $A=(a_{ij})$ is an $m\times n$
matrix of nonnegative integers. 
Let $V=S \cup T$, where $S=\{1, \ldots , m\}$ and 
$T =\{1', \ldots , n'\}$ and connect vertex $i\in S$ to vertex $j'\in T$ with $a_{ij}$
edges. 
 

\end{document}
