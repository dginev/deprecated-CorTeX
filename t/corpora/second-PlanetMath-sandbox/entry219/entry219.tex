\documentclass{article}
\usepackage{ids}
\usepackage{planetmath-specials}
\usepackage{pmath}
\usepackage{amssymb}
\usepackage{amsmath}
\usepackage{amsfonts}
\usepackage{graphicx}
\usepackage{xypic}
\begin{document}
A set $S$ is \emph{finite} if there exists a natural number $n$ and a bijection from $S$ to $n$. Note that we are using the set theoretic definition of natural number, under which the natural number $n$ equals the set $\{0,1,2,\ldots,n-1\}$.  If there exists such an $n$, then it is unique, and we call $n$ the \emph{cardinality} of $S$.

Equivalently, a set $S$ is finite if and only if there is no bijection between $S$ and any proper subset of $S$.

\end{document}
