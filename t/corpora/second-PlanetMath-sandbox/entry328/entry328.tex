\documentclass{article}
\usepackage{planetmath-specials}
\usepackage{pmath}
\usepackage{amssymb}
\usepackage{amsmath}
\usepackage{amsfonts}

\def\R{\mathbb{R}}
\def\N{\mathbb{N}}
\def\Z{\mathbb{Z}}

\begin{document}
\PMlinkescapeword{dimensions}
\PMlinkescapeword{series}

Let $(X,x_{0})$ be a pointed topological space (that is, a topological space with a chosen basepoint $x_{0}$).
Denote by $[(S^1,1),(X,x_{0})]$
the set of homotopy classes of maps $\sigma\colon S^{1} \to X$
such that $\sigma(1)=x_{0}$.
Here, $1$ denotes the basepoint $(1,0) \in S^{1}$.
Define a product
$[(S^1,1),(X,x_{0})] \times [(S^1,1),(X,x_{0})] \to [(S^1,1),(X,x_{0})]$
by $[\sigma][\tau]=[\sigma\tau]$,
where $\sigma\tau$ means ``travel along $\sigma$ and then $\tau$''.
This gives $[(S^1,1),(X,x_{0})]$ a group structure
and we define the \emph{fundamental group} of $(X,x_0)$
to be $\pi_1(X,x_{0})= [(S^1,1),(X,x_{0})]$.

In general, the fundamental group of a topological space
depends upon the choice of basepoint.
However, basepoints in the same path-component of the space
will give isomorphic groups.
In particular, this means that the fundamental group of a (non-empty) path-connected space is well-defined, up to isomorphism,
without the need to specify a basepoint.

Here are some examples of fundamental groups of familiar spaces: 
\begin{itemize}
\item $\pi_1(\R^n)\cong\{0\}$ for each $n\in\N$.
\item $\pi_1(S^1)\cong\mathbb{Z}$.
\item $\pi_1(T)\cong \mathbb{Z}\oplus\mathbb{Z}$, where $T$ is the torus.
\end{itemize}

It can be shown that $\pi_1$ is a functor
from the category of pointed topological spaces to the category of groups.
In particular, the fundamental group is a topological invariant,
in the sense that
if $(X,x_0)$ is homeomorphic to $(Y,y_0)$ via a basepoint-preserving map,
then $\pi_1(X,x_0)$ is isomorphic to $\pi_1(Y,y_{0})$. 

It can also be shown that two homotopically equivalent path-connected spaces
have isomorphic fundamental groups.

Homotopy groups generalize the concept of the fundamental group to higher dimensions.
The fundamental group is the first homotopy group,
which is why the notation $\pi_1$ is used.
\end{document}
