\documentclass{article}
\usepackage{ids}
\usepackage{planetmath-specials}
\usepackage{pmath}
\usepackage{amssymb}
\usepackage{amsmath}
\usepackage{amsfonts}
\usepackage{graphicx}
\usepackage{xypic}
\newcommand{\Complex}{\mathbb{C}}

\begin{document}
\PMlinkescapeword{fields}
\PMlinkescapeword{terms}
{\em Euler's relation} (also known as {\em Euler's formula}) is considered the first \PMlinkescapetext{bridge} between the fields of algebra and geometry, as it relates the exponential function to the trigonometric sine and cosine functions.

Euler's relation states that
\[e^{ix} = \cos{x}+i\sin{x}\]

Start by noting that
\[
i^k=\begin{cases} 
1 & \mbox{if\; } k\equiv 0\!\!\pmod 4\\
i & \mbox{if\; } k\equiv 1\!\!\pmod 4\\
-1 & \mbox{if\; } k\equiv 2\!\!\pmod 4\\
-i & \mbox{if\; } k\equiv 3\!\!\pmod 4
\end{cases}
\]

Using the Taylor series expansions of $e^x$, $\sin x$ and $\cos x$ (see the entries on the complex exponential function and the complex sine and cosine), it follows that
\begin{eqnarray*}
e^{ix} & = \sum_{n=0}^{\infty} \frac{i^n x^n}{n!}\\
& = \sum_{n=0}^{\infty}\left(\frac{x^{4n}}{(4n)!}+
\frac{ix^{4n+1}}{(4n+1)!}
-\frac{x^{4n+2}}{(4n+2)!}-\frac{ix^{4n+3}}{(4n+3)!}\right)
\end{eqnarray*}
Because the series expansion above is absolutely convergent for all $x$,
we can rearrange the terms of the series as
\begin{eqnarray*}
e^{ix} &=  \sum_{n=0}^{\infty} (-1)^n\frac{x^{2n}}{(2n)!}+
i\sum_{n=0}^{\infty} (-1)^n\frac{x^{2n+1}}{(2n+1)!}\\
&= \cos{x}+i\sin{x}
\end{eqnarray*}

As a special case, we get the beautiful and well-known identity, often called \emph{Euler's identity}:
\[e^{i\pi}=-1\]
\end{document}
