\documentclass{article}
\usepackage{planetmath-specials}
\usepackage{pmath}
\usepackage{amssymb}
\usepackage{amsmath}
\usepackage{amsfonts}
%\usepackage{graphicx}
%\usepackage{xypic}
\usepackage{pstricks,pst-plot}
\begin{document}
An Argand diagram is the graphical representation of complex numbers written in polar coordinates.  For example, if $z\in \mathbb{C}$ is a complex number, then $z$ can be written as $re^{i\theta}$, where $r$ is the length of $z$ considered as a  vector $(x,y)\in \mathbb{R}^2$, with $z=x+yi$, and $\theta$ is the value such that $\tan \theta = \frac{y}{x}$, and can be interpreted as the angle $z$ makes with the $x$-axis.  The Argand diagram of $z$ is thus

\begin{center}
\begin{pspicture}(-0.5,-0.5)(4.5,2)
\SpecialCoor
\psline{->}(0,0)(4.5,0)
\rput(4,-0.3){$x$}
\psline{->}(0,0)(3,1.75)
\rput(3.75,1.5){$z=(r,\theta)$}
\psarc{->}(0,0){1}{0}{(3,1.75)}
\rput(1.25,0.3){$\theta$}
\rput(1.5,1.2){$r$}
\end{pspicture}
\end{center}

Argand is the name of Jean-Robert Argand, the Frenchman who is credited with the geometric interpretation of the
complex numbers \PMlinkexternal{[Biography]}{http://www-groups.dcs.st-and.ac.uk/~history/Mathematicians/Argand.html}
\end{document}
