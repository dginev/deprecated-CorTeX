\documentclass{article}
\usepackage{ids}
\usepackage{planetmath-specials}
\usepackage{pmath}
%\documentclass{article}
\usepackage{amssymb, amsmath, amsthm, alltt, setspace}
\newtheorem{thm}{Theorem}

\theoremstyle{definition}
\newtheorem*{defn}{Definition}
\theoremstyle{definition}
\newtheorem*{rem}{Remark}

\theoremstyle{definition}
\newtheorem*{nott}{Notation}

\newtheorem{lemma}{Lemma}
\newtheorem{cor}{Corollary}
\newtheorem*{eg}{Example}
\newtheorem*{ex}{Exercise}
\newtheorem*{prop}{Proposition}


\newcommand{\RR}{\mathbb{R}}
\newcommand{\QQ}{\mathbb{Q}}
\newcommand{\ZZ}{\mathbb{Z}}
\newcommand{\NN}{\mathbb{N}}
\newcommand{\leftbb}{[ \! [}
\newcommand{\rightbb}{] \! ]}
\newcommand{\bt}{\begin{thm}}
\newcommand{\et}{\end{thm}}
\newcommand{\Rel}{\mathbf{R}}
\newcommand{\er}{\thicksim}
\newcommand{\sqle}{\sqsubseteq}
\newcommand{\floor}[1]{\lfloor{#1}\rfloor}
\newcommand{\ceil}[1]{\lceil{#1}\rceil}
\begin{document}
Euclid's algorithm describes a procedure for finding the greatest common divisor of two integers.

Suppose $a,~b \in \ZZ$, and without loss of generality, $b > 0$, because if $b < 0$, 
then $\gcd(a,b) = \gcd (a, |b| )$, and if $b = 0$, then $\gcd (a,b) = |a| $.  
Put $d := \gcd (a,b)$.

By the division algorithm for integers, we may find integers $q$ and $r$ such that
$a = q_0 b + r_0  $  where $ 0 \leq r_0 < b$.
  
Notice that $\gcd(a,b) = \gcd(b, r_0)$, because $d \mid a$ and $d \mid b$, so $d \mid r_0 = a - q_0b$ and 
if $b$ and $r_0$ had a common divisor $d'$ larger  than $d$, then $d'$ would also be a common divisor of 
$a$ and $b$, contradicting $d$'s maximality.  Thus, $d = \gcd(b, r_0)$.

So we may repeat the division, this time with $b$ and $r_0$.  Proceeding recursively, we obtain
\begin{eqnarray*}
  a   &=& q_0b   + r_0 ~\mbox{  with } 0 \leq r_0 < b    \\
  b   &=& q_1r_0 + r_1 \mbox{ with } 0 \leq r_1 < r_0  \\
  r_0 &=& q_2r_1 + r_2 \mbox{ with } 0 \leq r_2 < r_1  \\
  r_1 &=& q_3r_2 + r_3 \mbox{ with } 0 \leq r_3 < r_2  \\
      &\vdots&
\end{eqnarray*}
Thus we obtain a decreasing sequence of nonnegative integers $b > r_0 > r_1 > r_2 > \dots$, which
must eventually reach zero, that is to say,
$r_n = 0$ for some $n$, and the algorithm terminates.  We may easily generalize the previous 
argument to show that $d = \gcd(r_{k-1}, r_k) = \gcd(r_k, r_{k+1})$ for $k = 0, 1, 2, \dots$, 
where $r_{-1} = b$.  Therefore, \[d = \gcd(r_{n-1}, r_n) = \gcd(r_{n-1}, 0) = r_{n-1}.\]  More 
colloquially, the greatest common divisor is the last nonzero remainder in the algorithm.

The algorithm provides a bit more than this.  It also yields a way to express the $d$ as a linear
combination of $a$ and $b$, a fact obscurely known as Bezout's lemma.  For we have that
\begin{eqnarray*}
  a  -q_0b    &=&  r_0    \\
  b  -q_1r_0  &=&   r_1   \\
  r_0 - q_2r_1 &=& r_2    \\
  r_1 - q_3r_2 &=& r_3    \\
      &\vdots&            \\
  r_{n-3} - q_{n-1}r_{n-2} &=& r_{n-1}  \\
  r_{n-2} &=& q_nr_{n-1} \\
\end{eqnarray*}
so substituting each remainder $r_k$ into the next equation we obtain 
\begin{table}[ht]
\[
\begin{array}{ccccc}
  b                     - q_1(a - q_0b)    &=& k_1a + l_1b &=& r_1 \\
  (a - q_0b)            - q_2(k_1a + l_1b) &=& k_2a + l_2b &=& r_2 \\
  (k_1a + l_1b)         - q_3(k_2a + l_2b) &=& k_3a + l_3b &=& r_3  \\
                                       &\vdots &        &\vdots&   \\
  (k_{n-3}a + l_{n-3}b) - q_n(k_{n-2}a + l_{n-2}b) &=& k_{n-1} a + l_{n-1} b &=& r_{n-1} \\
\end{array}
\]
\end{table}

Sometimes, especially for manual computations, it is preferable to write  all the algorithm in a tabular
format.  As an example, let us apply the algorithm to $a=756$ and $b=595$.  The following table details
the procedure. The variables at the top of each column (without subscripts) have the same meaning as above.
That is to say, $r$ is used for the sequence of remainders and $q$ for the corresponding sequence of 
quotients.  The entries in the $k$ and $l$ columns are obtained by multiplying the current values for $k$
and $l$ by the $q$ in this row, and subtracting the results from the $k$ and $l$ in the previous row.

\begin{table}[h]
\[
\begin{array}{c@{~\vline~}c@{~\vline~}c@{~\vline~}c} 
  r   & q  & k  &  l \\
\hline
 756  &    &    1 &    0   \\
 595  & 1  &    0 &    1   \\
 161  & 3  &    1 &   -1   \\
 112  & 1  &   -3 &    4   \\
  49  & 2  &    4 &   -5   \\
  14  & 3  &  -11 &   14   \\
   7  & 2  &   37 &  -47   \\
   0  &    &      &        \\
\end{array}
\]
\end{table}
Thus, $gcd(756, 595) = 7$ and $37 \cdot 756 - 47 \cdot 595 = 7$.

Euclid's algorithm was first described in his classic work {\it Elements} (see propositions VII 1 and VII 2),
which also contained procedures
for geometrical constructions. These are the first known 
formally described algorithms. Prior to this, informally defined algorithms were in common use to 
perform various computations, but {\it Elements} contained the first attempt to rigorously 
describe a procedure and explain why its results are admissible. Euclid's algorithm for greatest 
common divisor is still commonly used today; since {\it Elements} was published in the fourth 
century BC, this algorithm has been in use for nearly 2400 years!
\end{document}
