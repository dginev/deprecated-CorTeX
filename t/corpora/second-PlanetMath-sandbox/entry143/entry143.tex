\documentclass{article}
\usepackage{planetmath-specials}
\usepackage{pmath}
\usepackage{amssymb}
\usepackage{amsmath}
\usepackage{amsfonts}
\usepackage{graphicx}
\begin{document}
\section{Algebraic view}

Let $K$ be a field. The {\em Brauer group} $\operatorname{Br}(K)$ of $K$ is the set of all equivalence classes of central simple algebras over $K$, where two central simple algebras $A$ and $B$ are equivalent if there exists a division ring $D$ over $K$ and natural numbers $n,m$ such that $A$ (resp. $B$) is isomorphic to the ring of $n \times n$ (resp. $m \times m$) matrices with coefficients in $D$.

The group operation in $\operatorname{Br}(K)$ is given by tensor product: for any two central simple algebras $A,B$ over $K$, their
product in $\operatorname{Br}(K)$ is the central simple algebra $A \otimes_K B$. The identity element in $\operatorname{Br}(K)$ is the class of $K$ itself, and the inverse of a central simple algebra $A$ is
the {\em opposite algebra} $A^{\operatorname{opp}}$ defined by reversing
the order of the multiplication operation of $A$.

\section{Cohomological view}

The Brauer group of $K$ is naturally isomorphic to the second Galois cohomology group 
$H^2({\operatorname{Gal}}(K^{\operatorname{sep}}/K),
(K^{\operatorname{sep}})^{\times})$.
See \PMlinkexternal{http://www.math.harvard.edu/~elkies/M250.01/index.html}{http://www.math.harvard.edu/~elkies/M250.01/index.html} Theorem 12 and succeeding remarks.
\end{document}
