\documentclass{article}
\usepackage{ids}
\usepackage{planetmath-specials}
\usepackage{pmath}
\usepackage{amssymb}
\usepackage{amsmath}
\usepackage{amsfonts}

% used for TeXing text within eps files
%\usepackage{psfrag}
% need this for including graphics (\includegraphics)
%\usepackage{graphicx}
% for neatly defining theorems and propositions
%\usepackage{amsthm}
% making logically defined graphics
%\usepackage{xypic} 

% there are many more packages, add them here as you need them

% define commands here
\newcommand{\md}{d}
\newcommand{\mv}[1]{\mathbf{#1}}	% matrix or vector
\newcommand{\mvt}[1]{\mv{#1}^{\mathrm{T}}}
\newcommand{\mvi}[1]{\mv{#1}^{-1}}
\newcommand{\mderiv}[1]{\frac{\md}{\md {#1}}} %d/dx
\newcommand{\mnthderiv}[2]{\frac{\md^{#2}}{\md {#1}^{#2}}} %d^n/dx
\newcommand{\mpderiv}[1]{\frac{\partial}{\partial {#1}}} %partial d^n/dx
\newcommand{\mnthpderiv}[2]{\frac{\partial^{#2}}{\partial {#1}^{#2}}} %partial d^n/dx
\newcommand{\borel}{\mathfrak{B}}
\newcommand{\integers}{\mathbb{Z}}
\newcommand{\rationals}{\mathbb{Q}}
\newcommand{\reals}{\mathbb{R}}
\newcommand{\complexes}{\mathbb{C}}
\newcommand{\naturals}{\mathbb{N}}
\newcommand{\defined}{:=}
\newcommand{\var}{\mathrm{var}}
\newcommand{\cov}{\mathrm{cov}}
\newcommand{\corr}{\mathrm{corr}}
\newcommand{\set}[1]{\left\{#1\right\}}
\newcommand{\powerset}[1]{\mathcal{P}(#1)}
\newcommand{\bra}[1]{\langle#1 \vert}
\newcommand{\ket}[1]{\vert \hspace{1pt}#1\rangle}
\newcommand{\braket}[2]{\langle #1 \ket{#2}}
\newcommand{\abs}[1]{\left|#1\right|}
\newcommand{\norm}[1]{\left|\left|#1\right|\right|}
\newcommand{\esssup}{\mathrm{ess\ sup}}
\newcommand{\Lspace}[1]{L^{#1}}
\newcommand{\Lone}{\Lspace{1}}
\newcommand{\Ltwo}{\Lspace{2}}
\newcommand{\Lp}{\Lspace{p}}
\newcommand{\Lq}{\Lspace{q}}
\newcommand{\Linf}{\Lspace{\infty}}
\begin{document}
\PMlinkescapeword{similarity}
\PMlinkescapeword{canonical}
\PMlinkescapeword{solution}
\PMlinkescapeword{represent}

\paragraph{Definition}
A square matrix $A$ is \emph{similar} (or \emph{conjugate}) to a square matrix $B$ if there exists a nonsingular square matrix $S$ such that

\begin{equation}
A = S^{-1}BS.
\end{equation}

Note that, given $S$ as above, we can define $R=S^{-1}$ and have $A=RBR^{-1}$.  Thus, whether the inverse comes first or last does not matter.

Transformations of the form $S^{-1}BS$ (or $SBS^{-1}$) are called \emph{similarity transformations}.

\paragraph{Discussion}
Similarity is useful for turning recalcitrant matrices into pliant ones.  The canonical example is that a diagonalizable matrix $A$ is similar to the diagonal matrix of its eigenvalues $\Lambda$, with the matrix of its eigenvectors acting as the similarity transformation.  That is,

\begin{align}
A &= T\Lambda T^{-1} \\
&= \left[ \begin{array}{cccc}
v_1 & v_2 & \ldots & v_n
\end{array} \right]
\left[ \begin{array}{ccc}
\lambda_1 & 0 & \ldots \\
0 & \lambda_2 & \ldots \\
\vdots & \vdots & \lambda_n
\end{array} \right]
\left[
\begin{array}{cccc}
v_1 & v_2 & \ldots & v_n
\end{array} \right]^{-1}.
\end{align}

This follows directly from the equation defining eigenvalues and eigenvectors,

\begin{equation}
AT=T\Lambda.
\end{equation}

If $A$ is \PMlinkname{symmetric}{SymmetricMatrix} for example, then through this transformation, we have turned $A$ into the product of two orthogonal matrices and a diagonal matrix.  This can be very useful.  As an application, see the solution for the normalizing constant of a multidimensional Gaussian integral.

\paragraph{Properties of similar matrices}
\begin{enumerate}
\item Similarity is \PMlinkname{reflexive}{Reflexive}: All square matrices $A$ are similar to themselves via the similarity transformation $A=I^{-1}AI$, where $I$ is the identity matrix with the same dimensions as $A$.

\item Similarity is \PMlinkname{symmetric}{Symmetric}: If $A$ is similar to $B$, then $B$ is similar to $A$, as we can define a matrix $R=S^{-1}$ and have

\begin{equation}
B=R^{-1}AR
\end{equation}

\item Similarity is \PMlinkname{transitive}{Transitive3}: If $A$ is similar to $B$, which is similar to $C$, we have

\begin{equation}
A=S^{-1}BS=S^{-1}(R^{-1}CR)S=(S^{-1}R^{-1})C(RS)=(RS)^{-1}C(RS).
\end{equation}

\item Because of 1, 2 and 3, similarity defines an equivalence relation (\PMlinkescapetext{reflexive, symmetric, and transitive}) on square matrices, \PMlinkname{partitioning}{Partition} the space of such matrices into a disjoint set of equivalence classes.

\item If $A$ is similar to $B$, then their determinants are equal; \PMlinkname{i.e.}{Ie}, $\det A=\det B$.  This is easily verified:

\begin{equation}
\det A=\det(S^{-1}BS)=\det(S^{-1})\det B \det S=(\det S)^{-1}\det B \det S=\det B.
\end{equation}

In fact, similar matrices have the same characteristic polynomial, which implies this result directly, the determinant being the constant term of the characteristic polynomial (up to sign).

\item Similar matrices represent the same linear transformation after a change of basis.

\item It can be shown that a matrix $A$ and its transpose $A^T$ are always similar.
\end{enumerate}
\end{document}
