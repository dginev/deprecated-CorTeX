\documentclass{article}
\usepackage{planetmath-specials}
\usepackage{pmath}
\usepackage{amssymb}
\usepackage{amsmath}
\usepackage{amsfonts}
\usepackage{graphicx}
\usepackage{xypic}

\begin{document}
The idea is to prove the cosines law:
\[
a^2 = b^2 + c^2 - 2bc\cos\theta
\]
where the variables are defined by the triangle:
\[
\begin{xy}
,(0,0)
;(40,0)**@{-}
;(60,30)**@{-}
;(0,0)**@{-}
,(20,-3)*{c}
,(7,2)*{\theta}
,(50,12)*{a}
,(30,17)*{b}
\end{xy}
\]

Let's add a couple of lines and two variables, to get
\[
\begin{xy}
,(0,0)
;(40,0)**@{-}
;(60,30)**@{-}
;(0,0)**@{-}
,(20,-3)*{c}
,(7,2)*{\theta}
,(50,12)*{a}
,(30,17)*{b}
,(40,0)
;(60,0)**@{--}
;(60,30)**@{--}
,(50,-3)*{x}
,(63,15)*{y}
\end{xy}
\]

This is all we need.  We can use Pythagoras' theorem to show that
\[
a^2 = x^2 + y^2
\]
and
\[
b^2 = y^2 + \left(c+x\right)^2
\]

So, combining these two we get
\begin{eqnarray*}
a^2 & = & x^2 + b^2 - \left(c+x\right)^2\\
a^2 & = & x^2 + b^2 - c^2 - 2cx - x^2\\
a^2 & = & b^2 - c^2 - 2cx
\end{eqnarray*}

So, all we need now is an expression for $x$.  Well, we can use the
definition of the cosine function to show that
\begin{eqnarray*}
c + x & = & b \cos\theta\\
x & = & b\cos\theta - c
\end{eqnarray*}

With this result in hand, we find that
\begin{eqnarray}
a^2 & = & b^2 - c^2 - 2cx\nonumber\\
a^2 & = & b^2 - c^2 - 2c\left( b\cos\theta - c\right)\nonumber\\
a^2 & = & b^2 - c^2 -2bc\cos\theta + 2c^2\nonumber\\
a^2 & = & b^2 + c^2 - 2bc\cos\theta
\end{eqnarray}
\end{document}
