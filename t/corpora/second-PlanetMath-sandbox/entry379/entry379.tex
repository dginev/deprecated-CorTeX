\documentclass{article}
\usepackage{ids}
\usepackage{planetmath-specials}
\usepackage{pmath}
% this is the default PlanetMath preamble.  as your knowledge
% of TeX increases, you will probably want to edit this, but
% it should be fine as is for beginners.

% almost certainly you want these
\usepackage{amssymb}
\usepackage{amsmath}
\usepackage{amsfonts}

% used for TeXing text within eps files
%\usepackage{psfrag}
% need this for including graphics (\includegraphics)
%\usepackage{graphicx}
% for neatly defining theorems and propositions
\usepackage{amsthm}
% making logically defined graphics
%\usepackage{xypic}

% there are many more packages, add them here as you need them

% define commands here
\newtheorem{thm}{Theorem}
\newtheorem{prop}[thm]{Proposition} 
\newtheorem{lemma}[thm]{Lemma}
\newtheorem{cor}[thm]{Corollary}

\theoremstyle{definition} 
\newtheorem{dfn}[thm]{Definition}

\theoremstyle{remark}
\newtheorem{rmk}[thm]{Remark}
\newtheorem{ex}[thm]{Example} 

\newcommand{\lie}{\mathcal{L}}

\begin{document}
Suppose $M$ is a smooth manifold, and denote by $\Omega(M)$ the algebra of differential forms on $M$.  Then, the \emph{Cartan calculus} consists of the following three types of linear operators on $\Omega(M)$:
\begin{enumerate}
\item the exterior derivative $d$,
\item the space of Lie derivative operators $\lie_X$, where $X$ is a vector field on $M$, and
\item the space of contraction operators $\iota_X$, where $X$ is a vector field on $M$.
\end{enumerate}

The above operators satisfy the following identities for any vector fields $X$ and $Y$ on $M$:
\begin{align}
d^2 &= 0, \label{cartfirst}\\
d \lie_X - \lie_X d &= 0, \\
d \iota_X + \iota_X d &= \lie_X, \label{magic}\\
\lie_X \lie_Y - \lie_Y \lie_X &= \lie_{[X,Y]}, \\
\lie_X \iota_Y - \iota_Y \lie_X &= \iota_{[X,Y]},\\
\iota_X \iota_Y + \iota_Y \iota_X &= 0, \label{cartlast}
\end{align}
where the brackets on the right hand side denote the Lie bracket of vector fields.

The identity (\ref{magic}) is known as \emph{Cartan's magic formula} or \emph{Cartan's identity}

\subsection*{Interpretation as a Lie Superalgebra}

Since $\Omega(M)$ is a graded algebra, there is a natural grading on the space of linear operators on $\Omega(M)$.  Under this grading, the exterior derivative $d$ is degree $1$, the Lie derivative operators $\lie_X$ are degree $0$, and the contraction operators $\iota_X$ are degree $-1$.

The identities (\ref{cartfirst})-(\ref{cartlast}) may each be written in the form
\begin{equation}
AB \pm BA = C,
\end{equation}
where a plus sign is used if $A$ and $B$ are both of odd degree, and a minus sign is used otherwise.  Equations of this form are called \emph{supercommutation relations} and are usually written in the form
\begin{equation}\label{supercom}
[A,B] = C,
\end{equation}
where the bracket in (\ref{supercom}) is a \emph{Lie superbracket}.  A Lie superbracket is a generalization of a Lie bracket.

Since the Cartan Calculus operators are closed under the Lie superbracket, the vector space spanned by the Cartan Calculus operators has the structure of a \emph{Lie superalgebra}.

\subsection*{Graded derivations of $\Omega(M)$}
\begin{dfn}
A degree $k$ linear operator $A$ on $\Omega(M)$ is a \emph{graded derivation} if it satisfies the following property for any $p$-form $\omega$ and any differential form $\eta$:
\begin{equation}
A (\omega \wedge \eta) = A(\omega) \wedge \eta + (-1)^{kp} \omega \wedge A(\eta).
\end{equation}
\end{dfn}

All of the Calculus operators are graded derivations of $\Omega(M)$.

\end{document}
