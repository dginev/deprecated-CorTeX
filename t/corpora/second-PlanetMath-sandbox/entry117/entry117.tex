\documentclass{article}
\usepackage{planetmath-specials}
\usepackage{pmath}
% This is Cosmin's preamble.

% Packages
  \usepackage{amsmath}
  \usepackage{amssymb}
  \usepackage{amsfonts}
  \usepackage{amsthm}
  \usepackage{mathrsfs}
  %\usepackage{graphicx}
  %\usepackage{xypic}
  %\usepackage{babel}

% Theorem Environments
  \newtheorem*{thm}{Theorem}
  \newtheorem{thmn}{Theorem}
  \newtheorem*{lem}{Lemma}
  \newtheorem{lemn}{Lemma}
  \newtheorem*{cor}{Corollary}
  \newtheorem{corn}{Corollary}
  \newtheorem*{prop}{Proposition}
  \newtheorem{propn}{Proposition}


  % Other Commands
    \renewcommand{\geq}{\geqslant}
    \renewcommand{\leq}{\leqslant}
    \newcommand{\vect}[1]{\boldsymbol{#1}}
    \newcommand{\mat}[1]{\mathsf{#1}}
    \renewcommand{\div}{\!\mid\!}
\begin{document}
The \emph{supremum} of a set $X$ having a partial order is the least upper bound of $X$ (if it exists) and is denoted $\sup{X}$.

Let $A$ be a set with a partial order $\leq$, and let $X \subseteq A$. Then $s = \sup X$ if and only if: \begin{enumerate}
\item[\textbf{1.}]{For all $x\in X$, we have $x \leq s$ (i.e. $s$ is an upper bound).}
\item[\textbf{2.}]{If $s^{\prime}$ meets condition \textbf{1}, then $s \leq s^{\prime}$ ($s$ is the \emph{least} upper bound).}
\end{enumerate}

There is another useful definition which works if $A = \mathbb{R}$ with $\leq$ the usual order on $\mathbb{R}$, supposing that s is an upper bound: \[s = \sup X \text{ if and only if } \forall \varepsilon > 0, \exists x\in X : s-\varepsilon < x.\]  

Note that it is not necessarily the case that $\sup X \in X$.  Suppose $X = {]0, 1[}$, then $\sup X = 1$, but $1 \not\in X$.

Note also that a set may not have an upper bound at all.
\end{document}
