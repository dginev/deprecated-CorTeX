\documentclass{article}
\usepackage{ids}
\usepackage{planetmath-specials}
\usepackage{pmath}
\usepackage{amssymb}
\usepackage{amsmath}
\usepackage{amsfonts}
\usepackage{graphicx}
\usepackage{xypic}
\begin{document}
\PMlinkescapeword{term}
Let $R$ be a ring. A \emph{left ideal} (resp., \emph{right ideal}) $I$ of $R$ is a nonempty subset $I \subset R$ such that:
\begin{itemize}
\item $a-b \in I$ for all $a,b \in I$
\item $r \cdot a \in I$ (resp. $a \cdot r \in I$) for all $a \in I$ and $r \in R$
\end{itemize}
A \emph{two-sided ideal} is a left ideal $I$ which is also a right ideal. If $R$ is a commutative ring, then these three notions of ideal are equivalent. Usually, the word ``ideal'' by itself means two-sided ideal.

The name ``ideal'' comes from the study of number theory.  When the failure of unique factorization in number fields was first noticed, one of the solutions was to work with so-called ``ideal numbers'' in which unique factorization did hold.  These ``ideal numbers'' were in fact ideals, and in Dedekind domains, unique factorization of ideals does indeed hold.  The term ``ideal number'' is no longer used; the term ``ideal'' has replaced and generalized it.
\end{document}
