\documentclass{article}
\usepackage{ids}
\usepackage{planetmath-specials}
\usepackage{pmath}
\usepackage{amssymb}
\usepackage{amsmath}
\usepackage{amsfonts}
\usepackage{graphicx}
\usepackage{xypic}
\begin{document}
Let $S$ be a set with a partial ordering $\leq$, and let $T$ be a subset of $S$. A \emph{lowest upper bound}, or \emph{supremum}, of $T$ is an upper bound $x$ of $T$ with the property that $x \leq y$ for every upper bound $y$ of $T$. The lowest upper bound of $T$, when it exists, is denoted $\operatorname{sup}(T)$.

A lowest upper bound of $T$, when it exists, is unique.

Greatest lower bound is defined similarly: a \emph{greatest lower bound}, or \emph{infimum}, of $T$ is a lower bound $x$ of $T$ with the property that $x \geq y$ for every lower bound $y$ of $T$. The greatest lower bound of $T$, when it exists, is denoted $\operatorname{inf}(T)$.

If $A = \{a_1,a_2,\ldots,a_n\}$ is a finite set, then the supremum of $A$ is simply $\max(A)$, and the infimum of $A$ is equal to $\min(A)$.
\end{document}
