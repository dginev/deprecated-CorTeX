\documentclass{article}
\usepackage{planetmath-specials}
\usepackage{pmath}
\usepackage{amssymb}
\usepackage{amsmath}
\usepackage{amsfonts}
\usepackage{graphicx}
\usepackage{xypic}
\begin{document}
The \emph{transpose} of a matrix $A$ is the matrix formed by ``flipping'' $A$ about the diagonal line from the upper left corner.  It is usually denoted $A^t$, although sometimes it is written as $A^T$ or $A'$. So if $A$ is an $m \times n$ matrix and

$$
 A = 
\begin{pmatrix}
 a_{11} & a_{12} & \cdots & a_{1n} \\
 a_{21} & a_{22} & \cdots & a_{2n} \\
 \vdots & \vdots & \ddots & \vdots \\
 a_{m1} & a_{m2} & \cdots & a_{mn} 
\end{pmatrix} 
$$
then

$$ A^t =
\begin{pmatrix}
 a_{11} & a_{21} & \cdots & a_{m1} \\
 a_{12} & a_{22} & \cdots & a_{m2} \\
 \vdots & \vdots & \ddots & \vdots \\
 a_{1n} & a_{2n} & \cdots & a_{nm} 
\end{pmatrix} 
$$

Note that the transpose of an $m \times n$ matrix is a $n \times m$ matrix.

\subsubsection*{Properties}

Let $A$ and $B$ be $m \times m$ matrices, $C$ and $D$ be $m\times n$ matrices, $E$ be an $n\times k$ matrix, and $c$ be a constant.  Let $x$ and $y$ be column vectors with $n$ rows. Then

\begin{enumerate}
\item $(C^t)^t = C$
\item $(C+D)^t = C^t + D^t$
\item $(cD)^t = cD^t$
\item $(DE)^t=E^tD^t.$
\item $(AB)^t = B^t A^t.$
\item If $A$ is invertible , then $(A^t)^{-1} = (A^{-1})^t $
\item If $A$ is real, $\operatorname{trace}(A^tA) \ge 0$ (where $\operatorname{trace}$ is the trace of a matrix).
\item The transpose is a linear mapping from the vector space of matrices to itself.  That is, $(\alpha A + \beta B)^t = \alpha (A)^t + \beta (B)^t$, for same-sized matrices $A$ and $B$ and scalars $\alpha$ and $\beta$.
\end{enumerate}

The familiar vector dot product can also be defined using the matrix transpose.  If $x$ and $y$ are column vectors with $n$ rows each, 

$$ x^t y = x \cdot y $$

which implies

$$ x^t x = x \cdot x = ||x||_2^2 $$

which is another way of defining the square of the vector Euclidean norm.
\end{document}
