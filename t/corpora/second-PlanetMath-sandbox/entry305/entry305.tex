\documentclass{article}
\usepackage{planetmath-specials}
\usepackage{pmath}
\usepackage{graphicx}
%\usepackage{xypic} 
\usepackage{bbm}
\newcommand{\Z}{\mathbbmss{Z}}
\newcommand{\C}{\mathbbmss{C}}
\newcommand{\R}{\mathbbmss{R}}
\newcommand{\Q}{\mathbbmss{Q}}
\newcommand{\mathbb}[1]{\mathbbmss{#1}}
\newcommand{\figura}[1]{\begin{center}\includegraphics{#1}\end{center}}
\newcommand{\figuraex}[2]{\begin{center}\includegraphics[#2]{#1}\end{center}}
\begin{document}
Let $ABC$ a triangle. Let $T$ a point in the circumcircle such that $BT$ is a diameter. 
\figura{sineslawproof}

So $\angle A=\angle CAB$ is equal to $\angle CTB$ (they subtend the same arc). Since $\triangle CBT$ is a right triangle, from the definition of sine we get
$$\sin \angle CTB =\frac{BC}{BT}=\frac{a}{2R}.$$

On the other hand $\angle CAB=\angle CTB$ implies their sines are the same and so
$$\sin \angle CAB=\frac{a}{2R}$$
and therefore
$$\frac{a}{\sin A}=2R.$$

Drawing diameters passing by $C$ and $A$ will let us prove in a similar way the relations
$$\frac{b}{\sin B}=2R\quad\mbox{and}\quad\frac{c}{\sin C}=2R$$
and we conclude that $$\frac{a}{\sin A}=\frac{b}{\sin B}=\frac{c}{\sin C}=2R.$$

Q.E.D.
\end{document}
