\documentclass{article}
\usepackage{planetmath-specials}
\usepackage{pmath}
\usepackage{amssymb}
\usepackage{amsmath}
\usepackage{amsfonts}
\usepackage{graphicx}
\usepackage{xypic}

\begin{document}
This is a geometrical proof of Pythagorean theorem.  We begin with our
triangle:

\[
\begin{xy}
,(0,0)
;(20,0)**@{-}
;(20,10)**@{-}
;(0,0)**@{-}
,(10,-2)*{a}
,(23,6)*{b}
,(10,7)*{c}
\end{xy}
\]

Now we use the hypotenuse as one side of a square:
\[
\begin{xy}
,(0,0)
;(20,0)**@{-}
;(20,10)**@{-}
;(0,0)**@{-}
;(-10,20)**@{-}
;(10,30)**@{-}
;(20,10)**@{-}
,(10,-2)*{a}
,(23,6)*{b}
,(10,7)*{c}
\end{xy}
\]
and draw in four more identical triangles
\[
\begin{xy}
,(0,0)
;(20,0)**@{-}
;(20,10)**@{-}
;(0,0)**@{-}
;(-10,20)**@{-}
;(10,30)**@{-}
;(20,10)**@{-}
;(20,30)**@{-}
;(-10,30)**@{-}
;(-10,0)**@{-}
;(0,0)**@{-}
,(10,-2)*{a}
,(23,6)*{b}
,(10,7)*{c}
\end{xy}
\]

Now for the proof.  We have a large square, with each side of length
$a+b$, which is subdivided into one smaller square and four
triangles.  The area of the large square must be equal to the combined
area of the shapes it is made out of, so we have
\begin{eqnarray}
\left(a+b\right)^2 & = & c^2 + 4\left(\frac{1}{2}ab\right)\nonumber\\
a^2 + b^2 + 2ab & = & c^2 + 2ab \nonumber\\
a^2 + b^2 & = & c^2
\end{eqnarray}
\end{document}
