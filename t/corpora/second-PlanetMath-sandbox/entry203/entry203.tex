\documentclass{article}
\usepackage{planetmath-specials}
\usepackage{pmath}
\usepackage{amssymb}
\usepackage{amsmath}
\usepackage{amsfonts}
\usepackage{graphicx}
\usepackage{xypic}
\begin{document}
Given two sets $A$ and $B$, we say that $A$ is a subset of $B$ (which we denote as $A\subseteq B$ or simply $A\subset B$) if every element of $A$ is also in $B$. That is, the following implication holds:
$$x\in A\Rightarrow x\in B.$$
The relation between $A$ and $B$ is then called {\em set inclusion}.

Some examples:

The set $A=\{d,r,i,t,o\}$ is a subset of the set $B=\{p,e,d,r,i,t,o\}$ because every element of $A$ is also in $B$. That is, $A\subseteq B$. 

On the other hand, if $C=\{p,e,d,r,o\}$, then neither $A \subseteq C$ (because $t\in A$ but $t\not\in C$) nor $C \subseteq A$ (because $p\in C$ but $p\not\in A$).  The fact that $A$ is not a subset of $C$ is written as $A\not\subseteq C$.  Similarly, we have $C\not\subseteq A$.

If $X\subseteq Y$ and $Y\subseteq X$, it must be the case that $X=Y$.

Every set is a subset of itself, and the empty set is a subset of every other set.  The set $A$ is called a proper subset of $B$, if $A\subset B$ and $A\neq B$.  In this case, we do not use $A\subseteq B$.
\end{document}
