\documentclass{article}
\usepackage{planetmath-specials}
\usepackage{pmath}
\usepackage{amsmath}
\usepackage{amsfonts}
\usepackage{amssymb}

\newcommand{\reals}{\mathbb{R}}
\newcommand{\natnums}{\mathbb{N}}
\newcommand{\cnums}{\mathbb{C}}

\newcommand{\lp}{\left(}
\newcommand{\rp}{\right)}
\newcommand{\lb}{\left[}
\newcommand{\rb}{\right]}

\newtheorem{proposition}{Proposition}
\newcommand{\va}{\mathbf{a}}
\newcommand{\vb}{\mathbf{b}}
\newcommand{\hva}{\mathbf{a}'}
\newcommand{\hvb}{\mathbf{b}'}
\newcommand{\kfield}{\mathbb{K}}
\newcommand{\hI}{I'}
\newcommand{\hJ}{J'}
\newcommand{\hi}{{i'}}
\newcommand{\hj}{{j'}}
\newcommand{\hc}{c'}
\begin{document}
{\bf Definition.} The classical conception of the tensor product
operation involved finite dimensional vector spaces $A$, $B$, say over
a field $\kfield$.  To describe the tensor product $A\otimes B$ one
was obliged to chose bases
$$\va_i\in A,\; i\in I,\quad \vb_j\in B,\; j\in J$$
of $A$ and $B$ indexed
by finite sets $I$ and $J$, respectively, and represent elements of
$\va\in A$ and $\vb\in B$ by their coordinates relative to these bases,
i.e. as mappings $a:I\rightarrow \kfield$ and $b:J\rightarrow \kfield$
such that
$$\va = \sum_{i\in I} a^i \va_i,\qquad \vb=\sum_{j\in J} b^j \vb_j.$$
One then represented $A\otimes B$ relative to this particular choice
of bases as the vector space of mappings $c:I\times J\rightarrow
\kfield$.  These mappings were called ``second-order contravariant
tensors'' and their values were customarily denoted by superscripts,
a.k.a. contravariant indices:
$$c^{ij}\in \kfield,\quad i\in I,\; j\in J.$$
The canonical bilinear
multiplication (also known as outer multiplication)
$$\otimes:A\times B\rightarrow A\otimes B$$
was defined by
representing $\va\otimes \vb$, relative to the chosen bases, as the
tensor
$$c^{ij} = a^ib^j,\quad i\in I,\;j\in J.$$
In this system, the
products
$$\va_i\otimes \vb_j,\quad i\in I,\; j\in J$$
were represented by basic
tensors, specified in terms of the Kronecker deltas as the mappings
$$(i',j')\mapsto \delta^{i'}_i \delta^{j'}_j,\quad i'\in I,\; j'\in J.$$
These gave a basis of $A\otimes B$.


The construction is independent of the choice of bases in the
following sense.  Let
$$\hva_i\in A,\; i\in \hI,\qquad \hvb_j\in B,\; j\in \hJ$$
be different bases of $A$ and $B$ with indexing sets $\hI$
and $\hJ$ respectively.  Let
$$
r:I\times \hI\rightarrow \kfield,\qquad
s:J\times \hJ\rightarrow \kfield,
$$
be the corresponding change of basis matrices determined by
\begin{align*}
\hva_{\hi} &= \sum_{i\in I} \lp r^i_{i'}\rp \va_i,\quad \hi\in I'\\
\hvb_{\hj} &= \sum_{j\in I} \lp s^j_{j'}\rp \vb_j,\quad \hj\in J'.
\end{align*}
One then stipulated that tensors $c:I\times J\rightarrow \kfield$ and 
$\hc:\hI\times \hJ\rightarrow \kfield$ represent the same element of
$A\otimes B$ if 
\begin{equation}
  \label{eq:eq1}
c^{ij} = \sum_{\genfrac{}{}{0pt}{}{i'\in I'}{j'\in J'}}
\lp r^i_{i'}\rp \lp s^j_{j'}\rp (c')^{i'j'}  
\end{equation}
for all $i\in I, j\in J$.  This relation corresponds to the fact that
the products
$$\hva_i\otimes \hvb_j,\quad i\in I',\; j\in J'$$
constitute an alternate basis of $A\otimes B$, and that the change of
basis relations are
\begin{equation}
  \label{eq:eq2}
  \hva_\hi \otimes \hvb_\hj = \sum_{\genfrac{}{}{0pt}{}{i\in I}{j\in J}}
  \lp r^i_{i'}\rp \lp s^j_{j'}\rp \va_i\otimes \vb_j,\quad
  \hi\in I',\; \hj\in J'.  
\end{equation}

{\bf Notes.} Historically, the tensor product was called the {\em
  outer} product, and has its origins in the absolute differential
calculus (the theory of manifolds).  The old-time tensor calculus is
difficult to understand because it is afflicted with a particularly
lethal notation that makes coherent comprehension all but impossible.
Instead of talking about an element $\va$ of a vector space, one was
obliged to contemplate a symbol $\va^i$, which signified a list of
real numbers indexed by $1,2,\ldots,n$, and which was understood to
represent $\va$ relative to some specified, but unnamed basis.  

What makes this notation truly lethal is the fact a symbol $\va^j$ was
taken to signify an alternate list of real numbers, also indexed by
$1,\ldots, n$, and also representing $\va$, albeit relative to a
different, but equally unspecified basis.  Note that the choice of
{\em dummy variables make all the difference.}  Any sane system of
notation would regard the expression
$$\va^i,\quad i=1,\ldots,n$$
as representing a list of $n$ symbols
$$\va^1,\va^2,\ldots,\va^n.$$
However, in the classical system, one was strictly forbidden from
using
$$\va^1,\va^2,\ldots,\va^n$$
because where, after all, is the all important
dummy variable to indicate choice of basis?

Thankfully, it is possible to shed some light onto this confusion (I
have read that this is credited to Roger Penrose) by interpreting the
symbol $\va^i$ as a mapping from some finite index set $I$ to
$\reals$, whereas $\va^j$ is interpreted as a mapping from another
finite index set $J$ (of equal cardinality) to $\reals$.  


My own surmise is that the source of this notational difficulty stems
from the reluctance of the ancients to deal with geometric objects
directly.  The prevalent superstition of the age held that in order to
have meaning, a geometric entity had to be {\em measured} relative to
some basis.  Of course, it was understood that geometrically no one
basis could be preferred to any other, and this leads directly to the
definition of geometric entities as lists of measurements modulo the
equivalence engendered by changing the basis.

It is also worth remarking on the contravariant nature of the
relationship between the actual elements of $A\otimes B$ and the
corresponding representation by tensors relative to a basis --- compare
equations (1) and (2).
This relationship is the source of the
terminology ``contravariant tensor'' and ``contravariant index'', and
I surmise that  it
is this very medieval pit of darkness and confusion that spawned the
present-day notion of ``contravariant functor''.
\bigskip

{\bf References.}
\begin{enumerate}
\item Levi-Civita, ``The Absolute Differential Calculus.''
\end{enumerate}
\end{document}
