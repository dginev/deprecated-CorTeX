\documentclass{article}
\usepackage{ids}
\usepackage{planetmath-specials}
\usepackage{pmath}
\usepackage{amssymb}
\usepackage{amsmath}
\usepackage{amsfonts}

\def\cf#1{\operatorname{cf}(#1)}
\def\cof#1{\operatorname{cof}(#1)}
\begin{document}
\PMlinkescapeword{case}
\PMlinkescapeword{cases}
\PMlinkescapeword{contain}
\PMlinkescapeword{inequality}
\PMlinkescapeword{ordinal}
\PMlinkescapeword{property}
\PMlinkescapeword{satisfies}

\section*{Definitions}

Let $(P,\leq)$ be a poset. A subset $A\subseteq P$ is said to be \emph{cofinal} in $P$ if for every $x\in P$ there is a $y\in A$ such that $x\le y$.
A function $f\colon X\to P$ is said to be \emph{cofinal} if $f(X)$ is cofinal in $P$.
The least cardinality of a cofinal set of $P$ is called the \emph{cofinality} of $P$.
Equivalently, the cofinality of $P$ is the least \PMlinkid{ordinal}{2787} $\alpha$ such that there is a cofinal function $f\colon\alpha\to P$.
The cofinality of $P$ is written $\cf{P}$, or $\cof{P}$.

\section*{Cofinality of totally ordered sets}

If $(T,\leq)$ is a totally ordered set, then it must contain a well-ordered cofinal subset which is order-isomorphic to $\cf{T}$.
Or, put another way, there is a cofinal function $f\colon\cf{T}\to T$ with the property that $f(x)<f(y)$ whenever $x<y$.

For any ordinal $\beta$ we must have $\cf{\beta}\leq\beta$, because the identity map on $\beta$ is cofinal.
In particular, this is true for cardinals, so any cardinal $\kappa$ either satisfies $\cf{\kappa}=\kappa$, in which case it is said to be \emph{regular}, or it satisfies $\cf{\kappa}<\kappa$, in which case it is said to be \emph{singular}.

The cofinality of any totally ordered set is necessarily a regular cardinal.

\section*{Cofinality of cardinals}

$0$ and $1$ are regular cardinals. All other finite cardinals have cofinality $1$ and are therefore singular.

It is easy to see that $\cf{\aleph_0}=\aleph_0$, so $\aleph_0$ is regular.

$\aleph_1$ is regular, because the union of countably many countable sets is countable.
More generally, all infinite successor cardinals are regular.

The smallest infinite singular cardinal is $\aleph_{\omega}$.
In fact, the function $f\colon\omega\to\aleph_{\omega}$ given by $f(n)=\omega_n$ is cofinal, so $\cf{\aleph_\omega}=\aleph_0$.
More generally, for any nonzero limit ordinal $\delta$, the function $f\colon\delta\to\aleph_\delta$ given by $f(\alpha)=\omega_\alpha$ is cofinal, and this can be used to show that $\cf{\aleph_\delta}=\cf{\delta}$.

Let $\kappa$ be an infinite cardinal.
It can be shown that $\cf{\kappa}$ is 
the least cardinal $\mu$ such that $\kappa$ is 
the sum of $\mu$ cardinals each of which is less than $\kappa$.
This fact together with K\"onig's theorem tells us that  
$\kappa<\kappa^{\cf{\kappa}}$.
Replacing $\kappa$ by $2^\kappa$ in this inequality 
we can further deduce that $\kappa<\cf{2^\kappa}$.
In particular, $\cf{2^{\aleph_0}}>\aleph_0$, from which it follows that $2^{\aleph_0}\neq\aleph_\omega$ (this being the smallest uncountable aleph which is provably not the cardinality of the continuum).
\end{document}
