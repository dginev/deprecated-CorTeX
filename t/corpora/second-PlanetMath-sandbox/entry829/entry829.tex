\documentclass{article}
\usepackage{planetmath-specials}
\usepackage{pmath}
\usepackage{amssymb}
\usepackage{amsmath}
\usepackage{amsfonts}

%\usepackage{psfrag}
%\usepackage{graphicx}
%\usepackage{xypic}
\begin{document}
A sequence $(s_n)$ is \emph{monotonically nonincreasing} if 

$$ s_m \le s_n \text{ for all } m > n $$

Similarly, a real function $f(x)$ is monotonically nonincreasing if 

$$ f(x) \le f(y) \text{ for all } x > y $$

Compare this to monotonically decreasing.

\textbf{Conflict note.} In some contexts, such as \cite{NIST}, this is called \emph{monotonically decreasing} (in turn, our ``monotonically decreasing'' is called ``strictly decreasing'').  This is unfortunately counter-intuitive, since a sequence or function that is ``flat'' (such as $f(x) = 1$) is somehow ``decreasing.''  Beware!

\section{Examples}

\begin{itemize}

\item $(s_n) = 1, 0, -1, -2, \ldots$ is monotonically nonincreasing.  It is also monotonically decreasing.

\item $(s_n) = 1, 1, 1, 1, \ldots$ is nonincreasing but {\bf not} monotonically decreasing.

\item $(s_n) = (\frac{1}{n+1})$  is nonincreasing (note that $n$ is nonnegative).

\item $(s_n) = 1, 1, 2, 1, 1, \ldots$ is {\bf not} nonincreasing.  It also happens to fail to be monotonically nondecreasing.

\item $(s_n) = 1, 2, 3, 4, 5, \ldots$ is {\bf not} nonincreasing, rather it is nondecreasing (and monotonically increasing).

\end{itemize}

\begin{thebibliography}{3}
\bibitem{NIST} ``\PMlinkexternal{monotonically decreasing}{http://www.nist.gov/dads/HTML/monotoncdecr.html},'' from the NIST Dictionary of Algorithms and Data Structures, Paul E. Black, ed.
\end{thebibliography}
\end{document}
