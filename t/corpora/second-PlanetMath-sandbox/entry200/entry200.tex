\documentclass{article}
\usepackage{planetmath-specials}
\usepackage{pmath}
\usepackage{amssymb}
\usepackage{amsmath}
\usepackage{amsfonts}
\usepackage{graphicx}
\usepackage{xypic}
\begin{document}
Let $S$ be a set with an equivalence relation $\sim$. An {\em equivalence class} of $S$ under $\sim$ is a subset $T\subset S$ such that
\begin{itemize}
\item If $x \in T$ and $y \in S$, then $x \sim y$ if and only if $y \in T$
\item If $S$ is nonempty, then $T$ is nonempty
\end{itemize}

For $x \in S$, the equivalence class containing $x$ is often denoted by $[x]$, so that
$$
[x] := \{ y \in S \mid x \sim y \}.
$$

The {\em set of all equivalence classes} of $S$ under $\sim$ is defined to be the set of all subsets of $S$ which are equivalence classes of $S$ under $\sim$, and is denoted by $S/\sim$.  The map $x\mapsto [x]$ is sometimes referred to as the \PMlinkescapetext{canonical projection}.

For any equivalence relation $\sim$, the set of all equivalence classes of $S$ under $\sim$ is a partition of $S$, and this correspondence is a bijection between the set of equivalence relations on $S$ and the set of partitions of $S$ (consisting of nonempty sets).
\end{document}
