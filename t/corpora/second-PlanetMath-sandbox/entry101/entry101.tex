\documentclass{article}
\usepackage{ids}
\usepackage{planetmath-specials}
\usepackage{pmath}
\usepackage{amssymb}
\usepackage{amsmath}
\usepackage{amsfonts}
%\usepackage{graphicx}
%\usepackage{xypic}
\begin{document}
For integers $n\ge r \ge 0$ we define 
$$ 
{n\choose r} = \frac{n!}{(n-r)!r!}
$$ 
and call such numbers \emph{binomial coefficients}.

\subsubsection*{Properties.}
\begin{enumerate}
\item \label{isinteger}
   $n\choose r$ is an integer (\PMlinkname{proof.}{NchooseRIsAnInteger}). 
\item \label{reverse}
   ${n\choose r}={n\choose n-r}$.
\item \label{pascal}
   ${n\choose r-1}+{n\choose r}={n+1\choose r}$ (Pascal's rule).
\item \label{boundaryterms}
   ${n\choose 0}=1={n\choose n}$ for all $n$.
\item \label{sum2n}
   ${n\choose 0}+{n\choose 1}+{n\choose 2}+\cdots+{n\choose n}=2^n$.
\item \label{sum0}
   ${n\choose 0}-{n\choose 1}+{n\choose 2}-\cdots+(-1)^n{n\choose n}=0$ for $n>0$.
\item \label{sump1}
   $\sum_{t=k}^n {t\choose k }={n+1\choose k+1}$.
\end{enumerate}

Properties \ref{sum2n} and \ref{sum0} are the binomial theorem
applied to $(1+1)^n$ and $(1-1)^n$, respectively, although they also have purely combinatorial meaning.

\subsubsection*{Motivation}
Suppose $n\ge r$ are integers. 
The below list shows some examples where the binomial coefficients
appear. 
\begin{itemize}
\item $n\choose r$ constitute the coefficients when expanding the
binomial $(x+y)^n$ -- hence the name binomial coefficients.
See Binomial Theorem. 
\item $n\choose r$ is the number of ways to choose $r$ objects from a set with $n$ elements.
\item On the context of computer science, it also helps to see ${n\choose r}$ as the number 
of strings consisting of ones and zeros with $r$ ones and $n-r$ zeros.
This equivalency comes from the fact that
if $S$ be a finite set with $n$ elements, ${n\choose r}$ is the number of distinct 
subsets of $S$ with $r$ elements.  For each subset $T$ of $S$, consider the function
$$
  X_T\colon S \rightarrow \{0,1\}
$$
where $X_T(x) = 1$ whenever $x \in T$ and $0$ otherwise (so $X_T$ is the characteristic function 
for $T$).  For each $T \in \mathcal{P}(S)$, $X_T$ can be used to produce a unique bit 
string of length $n$ with exactly $r$ ones.
\end{itemize}


\subsubsection*{Notes} 
The ${n\choose r}$ notation was first introduced by 
von Ettinghausen \cite{Higham} in 1826, altough
these numbers have been used long before that. See 
\PMlinkname{this page}{PascalsTriangle} for some notes on their history. 
Although the standard mathematical notation for the binomial coefficients is $n\choose r$, there
are also several variants. Especially in high school environments one encounters also 
${C}(n,r)$ or ${C}^n_r$ for ${n\choose r}$.

\textbf{Remark}.  It is sometimes convenient to set ${n \choose r}:=0$ when $r>n$.  For example, property 7 above can be restated: $\sum_{t=1}^n {t\choose k }={n+1\choose k+1}$.  It can be shown that ${n \choose r}$ is elementary recursive.
  
\begin{thebibliography}{9}
\bibitem{Higham} N. Higham, \emph{Handbook of writing for the mathematical sciences},
Society for Industrial and Applied Mathematics, 1998. 
\end{thebibliography}
\end{document}
