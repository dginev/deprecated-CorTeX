\documentclass{article}
\usepackage{planetmath-specials}
\usepackage{pmath}
\usepackage{amssymb}
\usepackage{amsmath}
\usepackage{amsfonts}
\usepackage{graphicx}
\usepackage{xypic}
\begin{document}
\paragraph{Sequences}

Given any set $X$, a \emph{sequence} in $X$ is a function $f\colon \mathbb{N} \to X$ from the set of natural numbers to $X$. Sequences are usually written with subscript notation: $x_0, x_1, x_2 \dots$, instead of $f(0), f(1), f(2) \dots $.

\paragraph{Generalized sequences}

One can generalize the above definition to any arbitrary ordinal. For any set $X$, a \emph{generalized sequence} or \emph{transfinite sequence} in $X$ is a function $f\colon \omega \to X$ where $\omega$ is any ordinal number. If $\omega$ is a finite ordinal, then we say the sequence is a \emph{finite sequence}.
\end{document}
