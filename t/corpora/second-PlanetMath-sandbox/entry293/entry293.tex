\documentclass{article}
\usepackage{planetmath-specials}
\usepackage{pmath}
% this is the default PlanetMath preamble.  as your knowledge
% of TeX increases, you will probably want to edit this, but
% it should be fine as is for beginners.

% almost certainly you want these
\usepackage{amssymb}
\usepackage{amsmath}
\usepackage{amsfonts}

% used for TeXing text within eps files
%\usepackage{psfrag}
% need this for including graphics (\includegraphics)
%\usepackage{graphicx}
% for neatly defining theorems and propositions
%\usepackage{amsthm}
% making logically defined graphics
%\usepackage{xypic}

% there are many more packages, add them here as you need them

% define commands here
\begin{document}
The \emph{Legendre polynomials} are a set of polynomials $\{P_n\}_{n=0}^{\infty}$ each of order $n$ that satisfy Legendre's ODE: $$\frac{d}{dx}[(1-x^2) P_n'(x)] + n(n+1)P_n(x) = 0.$$

Alternatively $P_n$ is an eigenfunction of the self-adjoint differential operator $\frac{d}{dx}(1-x^2) \frac{d}{dx}$ with eigenvalue $-n(n+1)$.

The Legendre polynomials are also known as Legendre functions of the first kind.

By Sturm-Liouville theory, this means they're orthogonal over some interval with
some weight function.  In fact it can be shown that they're orthogonal on $[-1, 1]$ with weight function $W(x) = 1$.  As with any set of orthogonal polynomials, this can be used to generate them (up to normalization) by Gram-Schmidt orthogonalization of the monomials $\{x^i\}$.  The normalization used
is $\langle P_n \| P_n \rangle = 2 / (2n + 1)$, which makes $P_n(\pm 1) = (\pm 1)^n$

Rodrigues's Formula (which can be generalized to some other polynomial sets) is a sometimes convenient form of $P_n$ in terms of derivatives:
$$P_n(x) = \frac{1}{2^n n!} \left( \frac{d}{dx} \right)^n (x^2 - 1)^n$$

The first few explicitly are: 

\begin{eqnarray*}
P_0(x) &=& 1 \\
P_1(x) &=& x \\
P_2(x) &=& \frac{1}{2} (3x^2 - 1) \\
P_3(x) &=& \frac{1}{2} (5x^3 - 3x) \\
P_4(x) &=& \frac{1}{8} (35x^4 - 30x^2 + 3) \\
...
\end{eqnarray*}

As all orthogonal polynomials do, these satisfy a three-term recurrence relation:
$$(n+1)P_{n+1}(x) = (2n+1)xP_{n}(x) - (n)P_{n-1}(x)$$

The Legendre functions of the second kind also satisfy the Legendre ODE but are not regular at the origin.

Related are the associated Legendre functions, and spherical harmonics.
\end{document}
