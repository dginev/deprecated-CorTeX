\documentclass{article}
\usepackage{planetmath-specials}
\usepackage{pmath}
% this is the default PlanetMath preamble.  as your knowledge
% of TeX increases, you will probably want to edit this, but
% it should be fine as is for beginners.

% almost certainly you want these
\usepackage{amssymb}
\usepackage{amsmath}
\usepackage{amsfonts}

% used for TeXing text within eps files
%\usepackage{psfrag}
% need this for including graphics (\includegraphics)
%\usepackage{graphicx}
% for neatly defining theorems and propositions
%\usepackage{amsthm}
% making logically defined graphics
%\usepackage{xypic} 

% there are many more packages, add them here as you need them

% define commands here
\newcommand{\GL}{{\operatorname{GL}}}
\begin{document}
Given a vector space $V$, the {\em general linear group} $\GL(V)$ is defined to be the group of invertible linear transformations from $V$ to $V$. The group operation is defined by composition: given $T: V \longrightarrow V$ and $T': V \longrightarrow V$ in $\GL(V)$, the product $TT'$ is just the composition of the maps $T$ and $T'$.

If $V = \mathbb{F}^n$ for some field $\mathbb{F}$, then the group $\GL(V)$ is often denoted $\GL(n,\mathbb{F})$ or $\GL_n(\mathbb{F})$. In this case, if one identifies each linear transformation $T: V \longrightarrow V$ with its matrix with respect to the standard basis, the group $\GL(n,\mathbb{F})$ becomes the group of invertible $n \times n$ matrices with entries in $\mathbb{F}$, under the group operation of matrix multiplication.

One also discusses the general linear group on a module $M$ over some ring $R$.  There it is the set of automorphisms of $M$ as an $R$-module.  For example, one might take $\GL(\mathbb{Z}\oplus\mathbb{Z})$; this is isomorphic to the group of two-by-two matrices with integer entries having determinant $\pm 1$.  If $M$ is a general $R$-module, there need not be a natural interpretation of $\GL(M)$ as a matrix group.

The general linear group is an example of a group scheme; viewing it in this way ties together the properties of $\GL(V)$ for different vector spaces $V$ and different fields $F$.  The general linear group is an algebraic group, and it is a Lie group if $V$ is a real or complex vector space. 

When $V$ is a finite-dimensional Banach space, $\GL(V)$ has a natural topology coming from the operator norm; this is isomorphic to the topology coming from its embedding into the ring of matrices.  When $V$ is an infinite-dimensional vector space, some elements of $\GL(V)$ may not be continuous and one generally looks instead at the set of bounded operators.
\end{document}
