\documentclass{article}
\usepackage{planetmath-specials}
\usepackage{pmath}
\usepackage{amssymb}
\usepackage{amsmath}
\usepackage{amsfonts}
%\usepackage{graphicx}
%\usepackage{xypic}
\usepackage{pstricks}
\begin{document}
Let $ABC$ be a given triangle. A height of $ABC$ is a line segment drawn from a vertex to the opposite side (or its prolongations) and perpendicular to it. So we have three heights in any triangle. The three heights are always concurrent and the common point is called the orthocenter. In Euclidean geometry the
length of the segment "height" is sometimes referred to as the height.

In the following figure, $AD,BE$ and $CF$ are heights of $ABC$.
\begin{center}
\begin{pspicture}(0,-0.4)(5,4.1)
\pspolygon(0,0)(5,0)(2,4)
\psline(2,0)(2,4)
\psline(0,0)(3.2,2.4)
\psline(1,2)(5,0)
\psline(2,0.2)(2.2,0.2)(2.2,0)
\psline(3.04,2.28)(3.16,2.12)(3.32,2.24)
\psline(0.91,1.82)(1.09,1.73)(1.18,1.91)
\rput[b](2,4.1){$A$}
\rput[b](0,-0.4){$B$}
\rput[b](5,-0.4){$C$}
\rput[b](2,-0.4){$D$}
\rput[l](3.3,2.5){$E$}
\rput[r](1,2.1){$F$}
\end{pspicture}
\end{center}

\end{document}
