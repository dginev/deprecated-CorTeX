\documentclass{article}
\usepackage{ids}
\usepackage{planetmath-specials}
\usepackage{pmath}
\usepackage{amssymb}
\usepackage{amsmath}
\usepackage{amsfonts}
\usepackage{graphicx}
\usepackage{xypic}
\begin{document}
A \emph{ring} is a set $R$ together with two binary operations, denoted $+: R \times R \longrightarrow R$ and $\cdot: R \times R \longrightarrow R$, such that
\begin{enumerate}
\item $(a+b)+c = a+(b+c)$ and $(a \cdot b) \cdot c = a \cdot (b \cdot c)$ for all $a,b,c \in R$ (associative law)
\item $a+b = b+a$ for all $a,b \in R$ (commutative law)
\item There exists an element $0 \in R$ such that $a+0 = a$ for all $a \in R$ (additive identity)
\item For all $a \in R$, there exists $b \in R$ such that $a+b = 0$ (additive inverse)
\item $a\cdot(b+c) = (a \cdot b) + (a \cdot c)$ and $(a+b) \cdot c = (a \cdot c) + (b \cdot c)$ for all $a,b,c \in R$ (distributive law)
\end{enumerate}
Equivalently, a ring is an abelian group $(R,+)$ together with a second binary operation $\cdot$ such that $\cdot$ is associative and distributes over $+$. Additive inverses are unique, and one can define \emph{subtraction} in any ring using the formula $a-b := a + (-b)$ where $-b$ is the additive inverse of $b$.

We say $R$ has a \emph{multiplicative identity} if there exists an element $1 \in R$ such that $a \cdot 1 = 1 \cdot a = a$ for all $a \in R$. Alternatively, one may say that $R$ is a \emph{ring with unity}, a \emph{unital ring}, or a \emph{unitary ring}. Oftentimes an author will adopt the convention that all rings have a multiplicative identity. If $R$ does have a multiplicative identity, then a \emph{multiplicative inverse} of an element $a \in R$ is an element $b \in R$ such that $a \cdot b = b \cdot a = 1$. An element of $R$ that has a multiplicative inverse is called a \emph{unit} of $R$.

A ring $R$ is \emph{commutative} if $a \cdot b = b \cdot a$ for all $a,b \in R$.


\end{document}
