\documentclass{article}
\usepackage{ids}
\usepackage{planetmath-specials}
\usepackage{pmath}
% this is the default PlanetMath preamble.  as your knowledge
% of TeX increases, you will probably want to edit this, but
% it should be fine as is for beginners.

% almost certainly you want these
\usepackage{amssymb}
\usepackage{amsmath}
\usepackage{amsfonts}

% used for TeXing text within eps files
%\usepackage{psfrag}
% need this for including graphics (\includegraphics)
%\usepackage{graphicx}
% for neatly defining theorems and propositions
%\usepackage{amsthm}
% making logically defined graphics
%\usepackage{xypic}

% there are many more packages, add them here as you need them

% define commands here

\begin{document}
Let $x$, $y$, $z$ be positive real numbers.

For each positive integer $r$, let

$a_r = (x^r+y^r)/r!$ and $b_r=z^r/r!$.

For $s$ divisible by 4, let

$A_s=a_2-a_4+a_6- \cdots +a_{s-2}-a_s$,

$B_s=b_2-b_4+b_6- \cdots +b_{s-2}-b_s$.

Then Fermat's last theorem is equivalent (by elementary means) to: 

{\bf Theorem} If $a_n=b_n$ for some odd integer $n>2$, then either

(i) $A_N > 0$ for some $N>x,y$,

\hspace{2mm} or

(ii) $B_M>0$ for some $M>z$.

For a proof that these theorems are equivalent see:

proof of equivalence of Fermat's Last Theorem to its analytic form
\end{document}
