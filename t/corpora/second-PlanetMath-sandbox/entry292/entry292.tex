\documentclass{article}
\usepackage{ids}
\usepackage{planetmath-specials}
\usepackage{pmath}
\usepackage{amssymb}
\usepackage{amsmath}
\usepackage{amsfonts}
%\usepackage{amsthm}
\usepackage{enumerate}

% used for TeXing text within eps files
%\usepackage{psfrag}
% need this for including graphics (\includegraphics)
%\usepackage{graphicx}
% making logically defined graphics
%\usepackage{xypic}

% define commands here
\newcommand{\complex}{\mathbb{C}}
\newcommand{\real}{\mathbb{R}}
\newcommand{\rat}{\mathbb{Q}}
\newcommand{\nat}{\mathbb{N}}

\providecommand{\abs}[1]{\lvert#1\rvert}
\providecommand{\absW}[1]{\left\lvert#1\right\rvert}
\providecommand{\absB}[1]{\Bigl\lvert#1\Bigr\rvert}
\providecommand{\norm}[1]{\lVert#1\rVert}
\providecommand{\normW}[1]{\left\lVert#1\right\rVert}
\providecommand{\normB}[1]{\Bigl\lVert#1\Bigr\rVert}
\providecommand{\defnterm}[1]{\emph{#1}}

\DeclareMathOperator{\D}{D}
\DeclareMathOperator{\linspan}{span}
\begin{document}
\tableofcontents

\section{Real Taylor series}
Let $f\colon I \to \real$  be a function defined on an open interval $I$,
possessing derivatives of all orders at $a \in I$.
Then the power series
\[
T(x) = \sum_{k=0}^{\infty} \frac{f^{(k)}(a)}{k!}(x-a)^k 
\]
is called the \defnterm{Taylor series} for $f$ centered at $a$.

Often the case $a = 0$  is considered, and we have the simpler
\[
T(x) = \sum_{k=0}^{\infty} \frac{f^{(k)}(0)}{k!}x^k\,, 
\]
called the \defnterm{Maclaurin series} for $f$ by some authors.

If we perform formal term-by-term differentiation of $T(x)$,
we find that $T^{(k)}(a) = f^{(k)}(a)$, so it is plausible 
that $T$ is an extrapolation or an approximation to $f$ based 
on the derivatives of $f$ at a single point $a$.

In general, $T$ may not extrapolate or approximate $f$ in the strictest sense:
 a Taylor series does not necessarily converge, and even if it does, 
it \PMlinkname{may not necessarily converge to the original function $f$}{InfinitelyDifferentiableFunctionThatIsNotAnalytic}.  It is also not necessarily true that a Taylor series about $a$ equals the Taylor series of $f$ about some other point $b$,
when considered as functions.

Those functions whose Taylor series do converge
to the function are termed 
\defnterm{\PMlinkname{analytic}{Analytic} functions}.

If we start with a \emph{convergent} power series 
$f(x) = c_0 + c_1 (x-a) + c_2 (x-a)^2 + \dotsb$ to 
define a function $f$, then the Taylor series of $f$
about $a$ will turn out to be the same as our original power series.

\section{Taylor polynomials}
The $n$th degree \defnterm{Taylor polynomial} for $f$ centered at $a$
is the polynomial
\[
P_{n,a}(x) = \sum_{k=0}^{n} \frac{f^{(k)}(a)}{k!}(x-a)^k \,.
\]
In general, $P_{n,a}$ has degree $\leq n$;
it may be $< n$ if some of the terms $f^{(k)}(a)$
vanish.  Nevertheless, $P_{n,a}$
is characterized by the following properties:
it is the unique polynomial of degree $\leq n$
whose derivatives up to the $n$th order at $a$ agree
with those of $f$;
it is also the unique polynomial $p$ of degree $\leq n$
such that
\[
f(x) - p(x)  = o(\abs{x-a}^n)\,,  \quad \textrm{as $x \to a$}.
\]
(Landau notation is being used here.)  
These characterizations are sometimes helpful in
actually computing Taylor polynomials.

The Taylor polynomial $P_{n,a}$ is applicable even if $f$ is only
differentiable $n$ times at $a$, or when its Taylor series does 
not converge to $f$.

The error from the approximation of $f$ by $P_{n,a}$,  
or \defnterm{remainder term}, $R_{n,a}(x) = f(x) - P_{n,a}(x)$, 
can be quantified precisely
using  \PMlinkname{Taylor's theorem}{TaylorsTheorem}.
In particular, Taylor's Theorem is often used to show
\[
 \lim_{n\rightarrow \infty} R_{n,a}(x) = 0 \,,
\]
which is equivalent to $T(x) = f(x)$, i.e. the Taylor series
converges to the original function.

A term that is often heard is that of a ``Taylor expansion'';
depending on the circumstance, this may mean either the Taylor series
or the $n$th degree Taylor polynomial.
Both are useful to linearize or otherwise reduce the analytical complexity of a function.  They are also useful for numerical approximation of functions,
when the magnitude of the later terms fall off rapidly.  

\section{Examples}
Using the above definition of a Taylor series about $0$, we have the following important series representations:
\begin{align*}
 e^x &= 1 + \frac{x}{1!} + \frac{x^2}{2!} + \frac{x^3}{3!} + \cdots \\ 
  \sin x &= \frac{x}{1!} - \frac{x^3}{3!} + \frac{x^5}{5!} - \frac{x^7}{7!} + \cdots \\
  \cos x &= 1-\frac{x^2}{2!} + \frac{x^4}{4!} - \frac{x^6}{6!} + \cdots 
\end{align*}
That the series on the right converge to the functions on the left
can be proven by Taylor's Theorem.

\section{Complex Taylor series}
If $f\colon U \to \complex$ is a holomorphic function from an open subset $U$ of the complex plane, and $a \in U$, we may also consider its Taylor series about $a$ (defined with the same formulae as before, but with complex numbers).

In contrast with the complex case, it turns out that all holomorphic functions
are infinitely differentiable and have Taylor series that converge to them.
(The radius of convergence of the Taylor series at $a$ being the radius of the largest open disk about $a$ on which the domain of $f$ can be extended.)

This of course makes the theory of analytic functions very nice, and
many questions about real power series and real analytic functions are more easily  
answered by looking at the complex case.
For example, we can immediately tell that 
the Taylor series about the origin for the \PMlinkname{tangent function}{DefinitionsInTrigonometry}
\[
\frac{\sin z}{\cos z} = \tan z = z+\frac{1}{3} z^3+\frac{2}{15}z^5+ \frac{17}{315}z^7+ \frac{62}{2835} z^9+ \dotsb
\]
has a radius of convergence of $\pi/2$,
because $\tan$ is holomorphic everywhere except at its poles at
$z = \pi/2 +k\pi, k \in \mathbb{Z}$, 
and $\pi/2$ is the distance that the closest of these poles get to the origin.

\section{Taylor series and polynomials in Banach spaces}
Taylor series and polynomials can be generalized to Banach spaces:
for details, see Taylor's formula in Banach spaces.

\section{Taylor series and polynomials for functions of several variables}
The simplest Banach spaces are the spaces $\real^n$,
and in this case Taylor series and Taylor polynomials for functions
$f \colon \real^n \to \real$ (``functions of $n$ variables'')
look like this:
\begin{align*}
 T(x) &= \sum_{i_1=0}^\infty \cdots \sum_{i_n=0}^\infty 
\frac{f^{(i_1,i_2,\ldots,i_n)}(0)}{i_1!i_2!\cdots i_n!}x_1^{i_1}x_2^{i_2} \cdots x_n^{i_n}
\,,
\quad
f^{(i_1,i_2,\ldots,i_n)}(0) = \left.\frac{\partial^{i_1+\dotsb+i_n} f}{\partial x_1^{i_1} \cdots \partial x_n^{i_n}}\right|_{x=0}\,,
\\
P_{N,0}(x) &= \sum_{i_1 + \dotsb + i_n  \leq N}
\frac{f^{(i_1, i_2, \ldots, i_n)}(0)}{i_1!i_2!\cdots i_n!}
  x_1^{i_1}x_2^{i_2} \cdots x_n^{i_n}  = \sum_{\abs{I} \leq N} \frac{1}{I!} \left.\frac{\partial^{\abs{I}} f}{\partial x^I}\right|_{x=0} \, x^I\,.
\end{align*}
(For simplicity, we have put the centre at $a= 0$. 
The last expression employs a commonly-used multi-index notation.)

For example, the second-degree Taylor polynomial for $f(x,y) = \cos(x+y)$
centered about $(0,0)$
is
\[
P_{2,0}(x,y) = 1 - \frac{1}{2} x^2  - xy -   \frac{1}{2}y^2\,.
\]
Note that $P_{2,0}(x,y)$ can also be obtained by
taking the one-variable Taylor series $\cos t = 1 - t^2 / 2 + \dotsb $ and substituting
$t = x+y$, and keeping only the terms of degree $\leq 2$.
This procedure works because of the uniqueness characterization of Taylor polynomials.

\section{Taylor expansion of formal polynomials}
If $f$ is a polynomial function, of degree $n$,
then its Taylor series and its Taylor polynomial of degree $\geq n$
actually equal $f$.  For this reason, we can consider Taylor series and polynomials
applied to formal polynomials, without any notion of convergence.
(The usual derivative is replaced by formal differentiation.)
In this setting, a ``Taylor expansion'' of a formal polynomial $p(x)$
about $a$ amounts to nothing more than rewriting $p(x)$
in the form $c_0 + c_1 (x-a) + \dotsb + c_n (x-a)^n$.

Similar considerations apply to formal power series,
or to formal polynomials of several variables.



\begin{thebibliography}{3}
\bibitem{Ahlfors}
Lars V. Ahlfors. {\it Complex Analysis}, third edition. McGraw-Hill, 1979.
\bibitem{Spivak}
Michael Spivak. {\it Calculus}, third edition. Publish or Perish, 1994.
\end{thebibliography}
\end{document}
