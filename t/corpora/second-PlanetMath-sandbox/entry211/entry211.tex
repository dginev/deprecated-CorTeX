\documentclass{article}
\usepackage{planetmath-specials}
\usepackage{pmath}
\usepackage{amssymb}
\usepackage{amsmath}
\usepackage{amsfonts}
\usepackage{graphicx}
\usepackage{xypic}
\begin{document}
An implication is a logical construction that essentially tells us if one condition is true, then another condition must be also true.  Formally it is written $$ a \rightarrow b $$ or $$a \Rightarrow b$$ which would be read ``$a$ implies $b$'', or ``$a$ therefore $b$'', or ``if $a$, then $b$'' (to name a few).

Implication is often confused for ``if and only if'', or the biconditional truth function ($\Leftrightarrow$).  They are not, however, the same.  The implication $a \rightarrow b$ is true even if only $b$ is true.  So the statement ``pigs have wings, therefore it is raining today'', is true if it is indeed raining, despite the fact that the first item is false.

In fact, any implication $a \rightarrow b$ is called \emph{vacuously true} when $a$ is false.  By contrast, $a \Leftrightarrow b$ would be false if either $a$ or $b$ was by itself false ($a \Leftrightarrow b \equiv (a \land b) \lor (\lnot a \land \lnot b)$, or in terms of implication as $(a \rightarrow b) \land (b \rightarrow a)$).

It may be useful to remember that $a \rightarrow b$ only tells you that it \emph{cannot} be the case that $b$ is false while $a$ is true; $b$ must ``follow'' from $a$ (and ``false'' does follow from ``false'').  Alternatively, $a \rightarrow b$ is in fact equivalent to 

$$ b \lor \lnot a $$

The truth table for implication is therefore

\begin{center}
\begin{tabular}{ccc}
a & b & $a \rightarrow b$ \\
\hline
F & F & T \\
F & T & T \\
T & F & F \\
T & T & T \\
\end{tabular}
\end{center}
\end{document}
