\documentclass{article}
\usepackage{ids}
\usepackage{planetmath-specials}
\usepackage{pmath}
\usepackage{amssymb}
\usepackage{amsmath}
\usepackage{amsfonts}
\usepackage{graphicx}
\usepackage{xypic}
\begin{document}
A \PMlinkid{connected}{4811} non-empty open set in $\mathbb{C}^n$ is called a \emph{domain}.

The topology considered is the Euclidean one (viewing $\mathbb{C}$ as $\mathbb{R}^2$). So we have that for a domain $D$ being connected is equivalent to being path-connected.

Since we have that every component of a region $D$ will be a domain, we have that every region has at most countably many components.

This definition has no particular relationship to the notion of an \PMlinkname{integral domain}{IntegralDomain}, used in algebra.  In number theory, one sometimes talks about fundamental domains in the upper half-plane, these have a different definition and are not normally open.  In set theory, one often talks about the \PMlinkname{domain}{Function} of a function.  This is a separate concept.  However, when one is interested in complex analysis, it is often reasonable to consider only functions defined on connected open sets in $\mathbb{C}^n$, which we have called domains in this entry.  In this context, the two notions coincide.\\

A \emph{domain} in a metric space (or more generally in a topological space) is a connected open set.

Cf. \PMlinkexternal{Mathworld}{http://mathworld.wolfram.com/Domain.html}, 
\PMlinkexternal{Wikipedia}{http://en.wikipedia.org/wiki/Domain}.
\end{document}
