\documentclass{article}
\usepackage{ids}
\usepackage{planetmath-specials}
\usepackage{pmath}
\usepackage{graphicx}
%\usepackage{xypic} 
\usepackage{bbm}
\newcommand{\Z}{\mathbbmss{Z}}
\newcommand{\C}{\mathbbmss{C}}
\newcommand{\R}{\mathbbmss{R}}
\newcommand{\Q}{\mathbbmss{Q}}
\newcommand{\mathbb}[1]{\mathbbmss{#1}}
\newcommand{\figura}[1]{\begin{center}\includegraphics{#1}\end{center}}
\newcommand{\figuraex}[2]{\begin{center}\includegraphics[#2]{#1}\end{center}}
\begin{document}
Let $ABC$ be a given triangle and $P$ any point of the plane. If $X$ is the intersection point of $AP$ with $BC$, $Y$ the intersection point of $BP$ with $CA$ and $Z$ is the intersection point of $CP$ with $AB$, then
$$\frac{AZ}{ZB}\cdot\frac{BX}{XC}\cdot\frac{CY}{YA}=1.$$

Conversely, if $X,Y,Z$ are points on $BC,CA,AB$ respectively, and if 
$$\frac{AZ}{ZB}\cdot\frac{BX}{XC}\cdot\frac{CY}{YA}=1$$
then $AX,BY,CZ$ are concurrent.

Remarks: All the segments are directed segments (that is $AB=-BA$), and so theorem is valid even if the points $X,Y,Z$ are in the prolongations (even at the infinity) and $P$ is any point on the plane (or at the infinity).

\figuraex{ceva}{scale=0.75}
\end{document}
