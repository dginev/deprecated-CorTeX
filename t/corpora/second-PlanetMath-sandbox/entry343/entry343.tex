\documentclass{article}
\usepackage{ids}
\usepackage{planetmath-specials}
\usepackage{pmath}
\usepackage{amsmath}
\usepackage{amsfonts}
\usepackage{amssymb}

\usepackage{enumerate}

% define commands here
\newcommand{\complex}{\mathbb{C}}
\newcommand{\real}{\mathbb{R}}
\newcommand{\rat}{\mathbb{Q}}
\newcommand{\nat}{\mathbb{N}}

\providecommand{\abs}[1]{\lvert#1\rvert}
\providecommand{\absW}[1]{\left\lvert#1\right\rvert}
\providecommand{\absB}[1]{\Bigl\lvert#1\Bigr\rvert}
\providecommand{\norm}[1]{\lVert#1\rVert}
\providecommand{\normW}[1]{\left\lVert#1\right\rVert}
\providecommand{\normB}[1]{\Bigl\lVert#1\Bigr\rVert}
\providecommand{\defnterm}[1]{\emph{#1}}

\DeclareMathOperator{\D}{D}


\newcommand{\lp}{\left(}
\newcommand{\rp}{\right)}
\newcommand{\lb}{\left[}
\newcommand{\rb}{\right]}
\newcommand{\supth}{^{\text{th}}}

\newcommand{\vF}{\mathbf{F}}
\newcommand{\vs}{\mathbf{s}}
\newcommand{\vv}{\mathbf{v}}

\newcommand{\vi}{\mathbf{i}}
\newcommand{\vj}{\mathbf{j}}
\newcommand{\vk}{\mathbf{k}}
\newcommand{\ve}{\mathbf{e}}

\newcommand{\vx}{\mathbf{x}}
\newcommand{\vX}{\mathbf{X}}
\newcommand{\vA}{\mathbf{A}}

\newcommand{\grad}{\operatorname{grad}}
\newcommand{\curl}{\operatorname{curl}}
\begin{document}
\paragraph{Summary.}

The \defnterm{gradient} is a first-order differential operator that maps
scalar functions to vector fields. It is a generalization of the ordinary
derivative, and as such conveys information about the rate of change
of a function relative to small variations in the independent
variables. The gradient of a function $f$ is customarily denoted by
$\nabla f$ or by $\grad f$.
\tableofcontents

%\paragraph{Definition: Euclidean space.}
\section{Definition: Euclidean space}

Let $f\colon \real^n \to \real$ be continuously differentiable.
The \defnterm{gradient} of $f$, denoted by $\nabla f$, 
is defined 
by the property:
\begin{equation}\label{inner-prod-def}
\D_\vv f = \nabla f \cdot \vv \quad \text{for all vectors $\vv \in \real^n$.}
\end{equation}
The middle dot is the dot product,
and $\D_\vv$ is the directional derivative with respect to $\vv$.

\smallskip

If $x^1, \dotsc, x^n$ are Euclidean coordinates,
corresponding to the orthonormal basis $\ve_1, \dotsc, \ve_n$,
then
\begin{equation}\label{euclidean-def}
\nabla f = \sum_{i=1}^n \frac{\partial f}{\partial x^i}\, \ve_i\,.
\end{equation}
The formula \eqref{euclidean-def} is sometimes given as the definition of $\nabla f$.
We prefer to define $\nabla f$ by the coordinate-free formula \eqref{inner-prod-def} instead,
because then the geometric interpretations (see below) become obvious,
and \eqref{inner-prod-def} also indicates how we would go about
calculating the gradient in other curvilinear coordinate systems.
Formula \eqref{inner-prod-def} also makes it clear that the gradient
is a physical vector, depending only on the inner product structure of $\real^n$,
and not on the specific coordinate system used to calculate it.

There is the issue of whether the $\nabla f$ as defined by \eqref{inner-prod-def}
exists; but this is proved easily enough, by substituting the concrete expression \eqref{euclidean-def}
and seeing that it satisfies \eqref{inner-prod-def}.

\medskip

The gradient can be considered to be a
vector-valued differential operator, written as
\[
\nabla = \sum_{i=1}^n \ve_i \, \frac{\partial}{\partial x^i}\,,
\]
or, in the context of Euclidean 3-space, as
\[
\nabla = \vi \, \frac{\partial}{\partial x} +\vj \,
\frac{\partial}{\partial y} +\vk \, \frac{\partial}{\partial z} \,,
\]
where
$\vi, \vj, \vk$ are the unit vectors lying along the positive
direction of the $x, y, z$ axes, respectively. 

\section{Geometric and physical interpretations}
\begin{enumerate}[(a)]
\item
The direction of the vector $\nabla f$ is the direction of the greatest positive change, or increase, in $f$.
The magnitude of $\nabla f$ is the magnitude of this increase.
This follows immediately from \eqref{inner-prod-def}:
\[
\D_\vv f = \nabla f \cdot \vv = \norm{\nabla f} \, \norm{\vv} \, \cos \theta\,,
\]
where $\theta$ is the angle between $\nabla f$ and $\vv$.
So among all unit directions $\vv$ of change, if $\vv$ is perpendicular to $\nabla f$
then the change $\D_\vv f$ is zero; if $\vv$ is parallel to $\nabla f$ then the change is maximized.

Similarly,
$-\nabla f$ is the direction of the greatest negative change, or decrease, in $f$.

\item
If $M$ is the hypersurface in $\real^n$ defined by
\[
M = \{ p \in \real^n : f(p) = 0 \,, \: \D f(p) \neq 0 \}\,,
\]
then $\nabla f(p)$ is the normal to the hypersurface $M$ at the point $p$.  For $\ker \D f(p)$ is the tangent space $\mathrm{T}_p M$
to $M$ at $p$, that is, $\D_\vv f(p) = 0$ for all $\vv \in \mathrm{T}_p M$,
and by definition \eqref{inner-prod-def}, $\nabla f(p)$ must be perpendicular to all $\vv \in \mathrm{T}_p M$.

Note that $\D f \neq 0$ is equivalent to $\nabla f \neq 0$.  Consequently, $\nabla f$ also gives
an orientation to the hypersurface $M$.

For example, if $f(\vx) = \norm{\vx} - 1$ for $\vx \in \real^n$,
$M$ is the $(n-1)$-dimensional sphere of unit radius, embedded in $\real^n$.
Its normal, $\nabla f(\vx) = \vx/\norm{\vx}$, as one would expect, points outward radially.

\item
As a simple case of (b), consider the 
surface $z = f(x, y)$ in $\real^3$, with Cartesian coordinates $(x, y, z)$.
Think of this surface as describing a hill, with height $z$.
Then the direction of the gradient vector $\nabla f$
is the direction of steepest ascent of the hill, while its
magnitude 
\[
\Vert \nabla f \Vert = \sqrt{ \lp \frac{\partial f}{\partial x}
\rp^2 + \lp \frac{\partial f}{\partial y} \rp^2 }
\]
is the slope or steepness in that direction. 

If a ball is placed on the hill at a point $(x, y, z)$,
theoretically it should roll down the hill in the direction of the gradient
vector $-\nabla f(x, y)$.  This may be easily derived by considering the mechanical forces
on the ball.  The direction of $-\nabla f(x, y)$ is, in fact, the projection
to the xy-plane of an outward normal vector to the hill at $(x, y, z)$;
the normal vector is involved because the movement of the ball
arises from the normal force from the hill.

\item
Suppose the surface $z = f(x, y)$ in (c) describes a bowl instead of a hill,
and we place a marble at any point $(x, y, z)$ on this bowl.
We would expect the marble to roll down to a local minimum point of $f(x, y)$.
Since the marble should roll down in the direction of $-\nabla f$,
we might hope that we can \emph{find} local minima
of a \emph{given function} $f$ by following the path mapped out
by the gradients $-\nabla f$.  Formally, this method of
finding local extrema (with some modifications) is called \defnterm{gradient descent}.


\item
If $U$ is the potential function corresponding to a
conservative physical force, then $\vF = -\nabla U$
is the corresponding force field.

Consequently, the gradient theorem,
\[
\int_\gamma \vF \cdot d\vs = -\int_\gamma \nabla U \cdot d\vs = -U(\gamma(b)) + U(\gamma(a)), \quad \gamma\colon [a, b] \to \real^3
\]
simply gives the formula for the change in the potential energy $U$
when an object ``does work'' along a path $\gamma$ in a conservative force field $\vF$.

\end{enumerate}



%\paragraph{Physical Interpretation.}


\section{Definition: Riemannian geometry}

%\paragraph{Definition: Riemannian geometry.}
It is obvious how \eqref{inner-prod-def}
can be generalized to the setting of Riemannian manifolds:
the dot product of $\real^n$ must be replaced
by the Riemannian metric, and the analogue
of $\D_\vv f$ is the directional derivative $\vv[f]$, for tangent vectors $\vv$ on the Riemannian manifold.
Thus for a smooth scalar-valued function $f$ on a Riemannian manifold,
\begin{equation}\label{eq:riemann-def}
\vX = \grad f  \: \Leftrightarrow \: df_p(\vv) = \vv[f] = \langle \vX, \vv \rangle_p\,.
\end{equation}
We can calculate $\vX$ explicitly as follows.
If $x^i$ are local coordinates on the manifold (not necessarily orthonormal),
set $\vX = X^i \frac{\partial}{\partial x^i}$ (the Einstein summation convention is being used).
Let $ g_{ij}$ and $ g^{ij}$ be the covariant and contravariant metric tensors, respectively.
Then from \eqref{eq:riemann-def},
\[
\frac{\partial f}{\partial x^j} = \left\langle X^i 
\frac{\partial}{\partial x^i}, \frac{\partial}{\partial x^j} \right\rangle
= X^i \, \left\langle \frac{\partial}{\partial x^i}, \frac{\partial}{\partial x^j} \right\rangle 
=  g_{ij} \, X^i\,,
\]
and taking inverses,
\begin{equation}\label{eq:gendef}
X^i = g^{ij} \, \frac{\partial f}{\partial x^j}\,.
\end{equation}


\section{Duality with differential one-forms}
%\paragraph{Duality with differential one-forms.}
Definitions \eqref{inner-prod-def} and \eqref{eq:riemann-def}
exhibit $\nabla f$ as the vector field
dual to the differential form $df$.
The isomorphism is given by applying the inner product or Riemannian metric.
This isomorphism is, of course, linear;
in particular it leads to the identity
\begin{equation}\label{eq:linearity}
\nabla f = \frac{\partial f}{\partial x^i} \, \nabla x^i\,, 
\end{equation}
which is the dual to the standard formula of differential one-forms:
\[
df = \frac{\partial f}{\partial x^i} \, dx^i \,.
\]
Using \eqref{eq:riemann-def} and \eqref{eq:gendef},
we have
\begin{equation}\label{eq:grad-coord}
\nabla x^i = g^{ij} \frac{\partial}{\partial x^j} \,, \quad
\frac{\partial}{\partial x^i} = g_{ij} \nabla x^j\,.
\end{equation}
So the isomorphism between vector fields and one-forms
is expressed by changing
the $\nabla$'s in \eqref{eq:grad-coord} to $d$'s, and vice versa.  That is,
\begin{equation}
d x^i \leftrightarrow g^{ij}  \frac{\partial}{\partial x^j} \,, \quad
\frac{\partial}{\partial x^i} \leftrightarrow g_{ij} \, d x^j\,.
\end{equation}
It is commonly said that this isomorphism is expressed by 
``raising and lowering the indices of a tensor field,
using contractions with $g_{ij}$ and $g^{ij}$''.

\smallskip

Notice that when $x^i$ are orthonormal coordinates
on $\real^n$, equation \eqref{eq:linearity}
reduces to equation \eqref{euclidean-def}, because
$g_{ij} = \ve_i \cdot \ve_j = \delta_{ij}$ (Kronecker delta).

The formulae presented in this section are useful in the Euclidean setting as
well, for deriving the formulae for the \PMlinkname{gradient in various curvilinear coordinate systems}{GradientInCurvilinearCoordinates}.



%\paragraph{Differential identities.}
\section{Differential identities}
Several properties of the one-dimensional derivative generalize to a
multi-dimensional setting
\begin{align*}
\nabla(af+bg) &= a\nabla f + b\nabla g & \text{Linearity}\\
\nabla(fg) &= f\nabla g + g\nabla f & \text{Product rule}\\
\nabla( f \circ \phi ) (p) &= \D \phi(p)^* \nabla f(\phi(p)) & \text{Chain rule} \\
\nabla(h \circ f)(p) &= h'(f(p)) \, \nabla f(p) & \text{Another Chain rule} 
\end{align*}
The function $h$ is $h\colon \real \to \real$.
The notation $(\D \phi)^*$ denotes the transpose of the Jacobian matrix,
in Euclidean coordinates, of $\phi \colon \real^m \to \real^n$.
In the abstract setting, $(\D \phi)^*$ is the adjoint to the tangent map $\D \phi$
between the tangent bundles of two Riemannian manifolds.  


These identities can be proved directly from the definition,
but the first three are really just the duals
of the following well-known identities for differential forms:
\begin{align*}
d(af + bg) &= a df + b dg \\
d(fg) &= f dg + g df \\
d(\phi^* f) &= \phi^* df
\end{align*}
and so may be derived by changing the $d$'s here
to $\nabla$'s!  (Though the third identity may take a bit of thought.)

\smallskip

The following identity
\[
\curl \grad f = \nabla \times \nabla f = 0\,,
\]
is a special case of the differential forms identity $d^2 = 0$.
Conversely, if $\curl g = 0$ on a simply connected domain, then there
is $f$ such that $g = \grad f$.  See laminar field for details.


\section{The $\nabla$ symbolism}
(This discussion does not really belong here, but should be moved
to the nabla entry.)

Using the $\nabla$ formalism,
the divergence operator can be expressed as
$\nabla\cdot$, the curl operator as $\nabla\times$, and the
Laplacian operator as $\nabla^2$. To wit, for a given vector field
$$\vA = A_x\, \vi + A_y\, \vj + A_z\, \vk,$$ and a given function $f$
we have
\begin{align*}
\nabla\cdot \vA &= \frac{\partial A_x}{\partial x} +
\frac{\partial A_y}{\partial y} +\frac{\partial A_z}{\partial z} \\
\nabla\times \vA &=
\lp\frac{\partial A_z}{\partial y} - \frac{\partial A_y}{\partial z}
\rp \vi+
\lp \frac{\partial A_x}{\partial z} -
\frac{\partial A_z}{\partial x}\rp \vj+
\lp \frac{\partial A_y}{\partial x} -
\frac{\partial A_x}{\partial y}\rp \vk\\
\nabla^2 f &= \frac{\partial^2 f}{\partial x^2} +
\frac{\partial^2 f}{\partial y^2} +\frac{\partial^2 f}{\partial z^2}.
\end{align*}

\begin{thebibliography}{3}
\bibitem{Spivak1}
Michael Spivak. \emph{A Comprehensive Introduction to Differential Geometry},
Volume I. Publish or Perish, 1979.
\end{thebibliography}
\end{document}
