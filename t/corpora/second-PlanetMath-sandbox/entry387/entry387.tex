\documentclass{article}
\usepackage{planetmath-specials}
\usepackage{pmath}
\usepackage{amssymb}
\usepackage{amsmath}
\usepackage{amsfonts}
\usepackage{graphicx}
\usepackage{xypic}
\begin{document}
Let $R$ be a ring, and suppose that $M$ is a left $R$-module and $N$ a right $R$-module.

\subsubsection*{Annihilator of a Subset of a Module}

\begin{enumerate}
\item
If $X$ is a subset of $M$, 
then we define the {\it left annihilator} of $X$ in $R$:
$${\rm l.ann}(X) = \{ r \in R \mid rx = 0 \text{ for all } x \in X \}.$$
If $a,b\in {\rm l.ann}(X)$, then so are $a-b$ and $ra$ for all $r\in R$.  Therefore, ${\rm l.ann}(X)$ is a left ideal of $R$.
\item
If $Y$ is a subset of $N$, 
then we define the {\it right annihilator} of $Y$ in $R$:
$${\rm r.ann}(Y) = \{ r \in R \mid yr = 0 \text{ for all } y \in Y \}.$$
Like above, it is easy to see that ${\rm r.ann}(Y)$ is a right ideal of $R$.
\end{enumerate}

\textbf{Remark}.  ${\rm l.ann}(X)$ and ${\rm r.ann}(Y)$ may also be written as ${\rm l.ann}_R(X)$ and ${\rm r.ann}_R(Y)$ respectively, if we want to emphasize $R$.

\subsubsection*{Annihilator of a Subset of a Ring}

\begin{enumerate}
\item
If $Z$ is a subset of $R$, 
then we define the {\it right annihilator} of $Z$ in $M$:
$${\rm r.ann}_M(Z) = \{ m \in M \mid zm = 0 \text{ for all } z \in Z \}.$$
If $m,n\in {\rm r.ann}_M(Z)$, then so are $m-n$ and $rm$ for all $r\in R$.  Therefore, ${\rm r.ann}_M(Z)$ is a left $R$-submodule of $M$.
\item
If $Z$ is a subset of $R$, 
then we define the {\it left annihilator} of $Z$ in $N$:
$${\rm l.ann}_N(Z) = \{ n \in N \mid nz = 0 \text{ for all } z \in Z \}.$$
Similarly, it can be easily seen that ${\rm l.ann}_N(Z)$ is a right $R$-submodule of $N$.
\end{enumerate}
\end{document}
