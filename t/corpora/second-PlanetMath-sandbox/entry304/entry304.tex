\documentclass{article}
\usepackage{planetmath-specials}
\usepackage{pmath}
\usepackage{amssymb}
\usepackage{amsmath}
\usepackage{amsfonts}
\usepackage{graphicx}
\usepackage{xypic}
\newcommand{\union}{\cup}
\begin{document}
Let $(E, \mathcal{B}(E))$ be a measurable space. A \emph{measure} on $(E,\mathcal{B}(E))$ is a function $\mu\colon \mathcal{B}(E) \to \mathbb{R} \union \{\infty\}$ with values in the extended real numbers such that:
\begin{enumerate}
\item $\mu(A) \geq 0$ for $A \in \mathcal{B}(E)$, with equality if $A = \emptyset$
\item $\mu(\bigcup_{i=0}^\infty A_i) = \sum_{i=0}^\infty \mu(A_i)$ for any sequence of pairwise disjoint sets $A_i \in \mathcal{B}(E)$.
\end{enumerate}

Occasionally, the term \emph{positive measure} is used to distinguish measures as defined here from more general notions of measure which are not necessarily restricted to the non-negative extended reals.

The second property above is called countable additivity, or $\sigma$-additivity.  A \emph{finitely additive measure} $\mu$ has the same definition except that $\mathcal{B}(E)$ is only required to be an algebra and the second property above is only required to hold for finite unions.  Note the slight abuse of terminology: a finitely additive measure is not necessarily a measure.

The triple $(E, \mathcal{B}(E), \mu)$ is called a \emph{measure space}. If $\mu(E) = 1$, then it is called a \emph{probability space}, and the measure $\mu$ is called a \emph{probability measure}.

Lebesgue measure on $\mathbb{R}^n$ is one important example of a measure.
\end{document}
