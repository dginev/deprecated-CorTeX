\documentclass{article}
\usepackage{ids}
\usepackage{planetmath-specials}
\usepackage{pmath}
\usepackage{amssymb}
\usepackage{amsmath}
\usepackage{amsfonts}
\usepackage{graphicx}
\usepackage{xypic}
\begin{document}
Given random variables $X_1, X_2, ... , X_n$ and a subset $I \subset \{1,2,...,n\}$, the \textbf{marginal distribution} of the random variables ${X_i : i \in I}$ is the following: \\

\par$f_{\{X_i : i \in I\}}(\mathbf{x}) = \sum_{\{x_i : i \notin I\}}^{}{ f_{X_1,...,X_n}(x_1,...,x_n) }$  or\\
$f_{\{X_i : i \in I\}}(\mathbf{x}) = \int_{\{x_i : i \notin I\}}^{}{ f_{X_1,...,X_n}(u_1,...,u_n) \prod_{ \{u_i : i \notin I\} }^{} du_i}$, \\
\par
summing if the variables are discrete and integrating if the variables are continuous.\\
\par
This is, the marginal distribution of a set of random variables $X_1,...,X_n$ can be obtained by summing (or integrating) the joint distribution over all values of the other variables.\\
\par
The most common marginal distribution is the individual marginal distribution (ie, the marginal distribution of ONE random variable).
\end{document}
