\documentclass{article}
\usepackage{planetmath-specials}
\usepackage{pmath}
\usepackage{graphicx}
%\usepackage{xypic} 
\usepackage{bbm}
\newcommand{\Z}{\mathbbmss{Z}}
\newcommand{\C}{\mathbbmss{C}}
\newcommand{\R}{\mathbbmss{R}}
\newcommand{\Q}{\mathbbmss{Q}}
\newcommand{\mathbb}[1]{\mathbbmss{#1}}
\newcommand{\figura}[1]{\begin{center}\includegraphics{#1}\end{center}}
\newcommand{\figuraex}[2]{\begin{center}\includegraphics[#2]{#1}\end{center}}
\begin{document}
On every triangle there are points where special lines or circles intersect, and those points usually have very interesting geometrical properties. Such points are called \emph{triangle centers.}

Some examples of triangle centers are incenter, orthocenter, centroid, circumcenter, excenters, Feuerbach point, Fermat points, etc.

For an online reference please check the
\PMlinkexternal{Triangle Centers}{http://faculty.evansville.edu/ck6/tcenters/} page.

Here is a drawing showing the most important lines and centers of a triangle
\begin{center}
\figuraex{triangulo-rev}{scale=0.75}
\end{center}
{\footnotesize(XEukleides \PMlinktofile{source code}{triangulo-rev.euk} for the drawing)}
\end{document}
