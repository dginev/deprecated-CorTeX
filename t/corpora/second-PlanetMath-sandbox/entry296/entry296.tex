\documentclass{article}
\usepackage{ids}
\usepackage{planetmath-specials}
\usepackage{pmath}
\usepackage{amsfonts}

\def\Z{\mathbb{Z}}
\begin{document}
\PMlinkescapeword{endomorphism}
\PMlinkescapeword{structure}

Let $(G,\ast)$ and $(K,\star)$ be two groups.  A \emph{group homomorphism} is a
function $\phi\colon G \to K$ such that
$\phi (s \ast t) = \phi(s) \star \phi(t)$ for all $s,t \in G$.

A composition of group homomorphisms is again a homomorphism.

Let $\phi\colon G\to K$ a group homomorphism.
Then the kernel of $\phi$ is a normal subgroup of $G$,
and the image of $\phi$ is a subgroup of $K$.
Also, $\phi(g^n)=\phi(g)^n$ for all $g\in G$ and for all $n \in\Z$.
In particular,
taking $n=-1$ we have $\phi(g^{-1})=\phi(g)^{-1}$ for all $g \in G$,
and taking $n=0$ we have $\phi(1_G)=1_K$,
where $1_G$ and $1_K$ are the identity elements of $G$ and $K$,
respectively.

Some special homomorphisms have special names.
If the homomorphism $\phi\colon G\to K$ is injective,
we say that $\phi$ is a \emph{monomorphism},
and if $\phi$ is surjective we call it an \emph{epimorphism}.
When $\phi$ is both injective and surjective (that is, bijective) we call it an \emph{isomorphism}.
In the latter case we also say that $G$ and $K$ are \emph{isomorphic}, meaning they are basically the same group (have the same structure).
A homomorphism from $G$ on itself is called an \emph{endomorphism},
and if it is bijective then it is called an \emph{automorphism}.
\end{document}
