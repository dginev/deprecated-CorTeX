\documentclass{article}
\usepackage{planetmath-specials}
\usepackage{pmath}
\usepackage{amsthm}

\newtheorem*{thm*}{Theorem}
\begin{document}
\PMlinkescapeword{theorem}

Let $G$ be a group.

A subgroup $H$ of $G$
is said to be a \emph{maximal subgroup} of $G$
if $H\neq G$ and there is no subgroup $K$ of $G$
such that $H<K<G$.
Note that a maximal subgroup of $G$ is not \PMlinkname{maximal}{MaximalElement} among all subgroups of $G$,
but only among all proper subgroups of $G$.
For this reason, maximal subgroups are sometimes called \emph{maximal proper subgroups}.

Similarly, a normal subgroup $N$ of $G$
is said to be a \emph{maximal normal subgroup} 
(or \emph{maximal proper normal subgroup}) of $G$
if $N\neq G$ and there is no normal subgroup $K$ of $G$
such that $N<K<G$.
We have the following theorem:

\begin{thm*}
A normal subgroup $N$ of a group $G$ is a maximal normal subgroup
if and only if the \PMlinkname{quotient}{QuotientGroup} $G/N$
is \PMlinkname{simple}{Simple}.
\end{thm*}
\end{document}
