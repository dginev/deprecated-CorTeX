\documentclass{article}
\usepackage{planetmath-specials}
\usepackage{pmath}
\usepackage{amssymb}
\usepackage{amsmath}
\usepackage{amsfonts}
\usepackage{graphicx}
\usepackage{xypic}
\begin{document}
Let $R$ be a ring, 
and let $M$ be an $R$-module.
A prime ideal $P$ of $R$ 
is an {\PMlinkescapetext {\it annihilator prime}} for $M$
if $P={\rm ann}(X)$, the annihilator of some nonzero submodule $X$ of $M$.

Note that if this is the case, then the module ${\rm ann}_M(P)$ contains $X$, has $P$ as its annihilator,
and is a \PMlinkname{faithful}{FaithfulModule} $(R/P)$-module.

If, in addition, $P$ is equal to the annihilator of a submodule of $M$ that is a \PMlinkname{fully faithful}{FaithfulModule} $(R/P)$-module, then we call $P$ an {\PMlinkescapetext {\it associated prime}} of $M$.
\end{document}
