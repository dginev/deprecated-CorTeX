\documentclass{article}
\usepackage{ids}
\usepackage{planetmath-specials}
\usepackage{pmath}
% this is the default PlanetMath preamble.  as your knowledge
% of TeX increases, you will probably want to edit this, but
% it should be fine as is for beginners.

% almost certainly you want these
\usepackage{amssymb}
\usepackage{amsmath}
\usepackage{amsfonts}

% used for TeXing text within eps files
%\usepackage{psfrag}
% need this for including graphics (\includegraphics)
%\usepackage{graphicx}
% for neatly defining theorems and propositions
%\usepackage{amsthm}
% making logically defined graphics
%\usepackage{xypic}

% there are many more packages, add them here as you need them

% define commands here
\begin{document}
The \emph{double factorial} of a positive integer $n$ is the product $n!!$ of the positive integers less than or equal to $n$ that have the same parity as $n$, that is,
\[n!! = n (n-2) (n-4)\cdots k_n\]
where $k_n$ denotes $1$ if $n$ is an odd number and $2$ if $n$ is an even number.

For example,
\[ 7!! = 7 \cdot 5 \cdot 3 \cdot 1 = 105 \]
\[ 10!! = 10\cdot 8\cdot 6\cdot 4\cdot 2 = 3840 \]

Note that $n!!$ is not the same as $(n!)!$.

Observe that $(2n)!! = 2^n n!$ and $(2n+1)!! = \frac{(2n+1)!}{2^n n!}$.
\end{document}
