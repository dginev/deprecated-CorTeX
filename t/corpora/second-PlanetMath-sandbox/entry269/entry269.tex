\documentclass{article}
\usepackage{ids}
\usepackage{planetmath-specials}
\usepackage{pmath}
\usepackage{amssymb}
\usepackage{amsmath}
\usepackage{amsfonts}

\def\Q{\mathbb{Q}}
\def\Z{\mathbb{Z}}
\begin{document}
A \emph{transcendental number} is a complex number
that is not an algebraic number.
That is, it is a complex number that is transcendental over $\Q$
(or, equivalently, over $\Z$).

Well known transcendental numbers include $\pi$ and $e$
(the base of natural logarithms).

Cantor showed that, in a sense,
``almost all'' complex numbers are transcendental:
there are uncountably many complex numbers, but
\PMlinkname{only countably many algebraic numbers}{AlgebraicNumbersAreCountable}.

\end{document}
