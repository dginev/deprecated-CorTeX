\documentclass{article}
\usepackage{planetmath-specials}
\usepackage{pmath}
\usepackage{graphicx}
%\usepackage{xypic} 
\usepackage{bbm}
\newcommand{\Z}{\mathbbmss{Z}}
\newcommand{\C}{\mathbbmss{C}}
\newcommand{\R}{\mathbbmss{R}}
\newcommand{\Q}{\mathbbmss{Q}}
\newcommand{\mathbb}[1]{\mathbbmss{#1}}
\begin{document}
\PMlinkescapeword{order}
Let $A,B,C$ be points on a line (not necessarily in that order) and let $D,E,F$ points on another line (not necessarily in that order). Then the intersection points of $AD$ with $FC$, $DB$ with $CE$, and $BF$ with $EA$, are collinear.

This is a special case of Pascal's mystic hexagram.
\begin{center}
\includegraphics[scale=2.0]{pappus}
\end{center}

\textbf{Remark}.  Pappus's theorem is a statement about the incidence relation between points and lines in any geometric structure with points, lines, and an incidence relation between the points and the lines.  Generally speaking, an incidence geometry is \emph{Pappian} or satisfies the \emph{Pappian property} if the statement of Pappus's theorem is true.  In both Euclidean and affine geometry, Pappus theorem is true.  In plane projective geometry, both Pappian and non-Pappian planes exist.  Furthermore, it can be shown that every Pappian plane is Desarguesian, and the converse is true if the plane is finite (the result of Wedderburn's theorem).
\end{document}
