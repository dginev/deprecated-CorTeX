\documentclass{article}
\usepackage{planetmath-specials}
\usepackage{pmath}
\usepackage{amssymb}
\usepackage{amsmath}
\usepackage{amsfonts}
\usepackage{graphicx}
\usepackage{xypic}

\begin{document}
$X$ is a \textbf{Bernoulli random variable} with parameter \textbf{$p$} if\\
\par
$f_X(x) = p^x (1-p)^{1-x}$,     $x=\{0,1\}$	\\
\par
Parameters:\\
\par
\begin{list}{$\star$ }{}
\item $p \in [0,1]$
\end{list}
\par
Syntax:\\
\par
$X\sim Bernoulli(p)$\\
\par
Notes:\\
\par
\begin{enumerate}

\item $X$ represents the number of successful results in a Bernoulli trial. A Bernoulli trial is an experiment in which only two outcomes are possible: success, with probability $p$, and failure, with probability $1-p$.
\item $E[X] = p$
\item $Var[X] = p(1-p)$
\item $M_X(t) = p e^t + (1-p)$

\end{enumerate}
\end{document}
