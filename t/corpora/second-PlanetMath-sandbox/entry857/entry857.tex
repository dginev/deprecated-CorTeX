\documentclass{article}
\usepackage{ids}
\usepackage{planetmath-specials}
\usepackage{pmath}
\usepackage{amsmath}
\usepackage{amsfonts}
\usepackage{amssymb}

\newcommand{\reals}{\mathbb{R}}
\newcommand{\natnums}{\mathbb{N}}
\newcommand{\cnums}{\mathbb{C}}

\newcommand{\lp}{\left(}
\newcommand{\rp}{\right)}
\newcommand{\lb}{\left[}
\newcommand{\rb}{\right]}

\newcommand{\supth}{^{\text{th}}}


\newtheorem{proposition}{Proposition}
\begin{document}
The binomial formula gives the power series expansion of the
$p\supth$ power function. The power $p$ can be an integer,
rational, real, or even a complex number.  The formula is
\begin{align*}
  (1+x)^p &= \sum_{n=0}^\infty  \frac{p^{\underline{n}}}{n!} \, x^n\\
          &=  \sum_{n=0}^\infty  \binom{p}{n} x^n
\end{align*}
where $p^{\underline{n}}= p(p-1)\ldots (p-n+1)$ denotes the falling
factorial, and where $\binom{p}{n}$ denotes the generalized binomial
coefficient.  

For $p=0,1,2,\ldots$ the power series reduces to a polynomial, and we
obtain the usual binomial theorem.  For other values of $p$, the
radius of convergence of the series is $1$; the right-hand series
converges pointwise for all complex $|x|<1$ to the value on the left
side.  Also note that the binomial formula is valid at $x=\pm 1$, but
for certain values of $p$ only.  Of course, we have convergence if $p$
is a natural number.  Furthermore, for $x=1$ and real $p$, we have
absolute convergence if $p>0$, and conditional convergence if
$-1<p<0$.  For $x=-1$ we have absolute convergence for $p>0$.
\end{document}
