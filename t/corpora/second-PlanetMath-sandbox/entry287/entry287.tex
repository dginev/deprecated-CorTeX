\documentclass{article}
\usepackage{planetmath-specials}
\usepackage{pmath}
\usepackage{amssymb}
\usepackage{amsmath}
\usepackage{amsfonts}
\usepackage{graphicx}
\usepackage{xypic}
\begin{document}
Suppose $A,B,C$ are real numbers, with $A\neq 0$, and suppose
 \[Ax^2+Bx+C=0.\]

Since $A$ is nonzero, we can divide by $A$ and obtain the equation
\[
x^2 + bx+c = 0,
\]
where $b=\frac{B}{A}$ and $c=\frac{C}{A}$.  
This equation can be written as
$$x^2 + bx + \frac{b^2}{4} -\frac{b^2}{4} +c =0,$$
so completing the square, i.e., applying the identity $(p+q)^2=p^2+2pq + q^2$, yields
$$\left(x+\frac{b}{2}\right)^2 = \frac{b^2}{4} - c.$$
Then, taking the square root of both sides, and solving for $x$, we obtain 
the solution formula
\begin{eqnarray*}
x &=& -\frac{b}{2} \pm \sqrt{\frac{b^2}{4}-c}\nonumber\\
 &=& \frac{B}{2A} \pm \sqrt{\frac{B^2}{4A^2}-\frac{C}{A}}\nonumber\\
 &=& \frac{-B\pm\sqrt{B^2-4AC}}{2A},
\end{eqnarray*}
and the derivation is completed.

A slightly less intuitive but more aesthetically pleasing approach to this derivation can be achieved by multiplying both sides of the equation
\begin{align*}
ax^2+bx+c=0
\end{align*}
by $4a$, resulting in the equation
\begin{align*}
4a^2x^2+4abx+b^2=b^2-4ac,
\end{align*}
in which the left-hand side can be expressed as $(2ax+b)^2$.  From here, the proof is identical.
\end{document}
