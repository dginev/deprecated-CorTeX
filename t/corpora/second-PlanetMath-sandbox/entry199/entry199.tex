\documentclass{article}
\usepackage{planetmath-specials}
\usepackage{pmath}
\usepackage{amssymb}
\usepackage{amsmath}
\usepackage{amsfonts}
\usepackage{graphicx}
\usepackage{xypic}
\begin{document}
\section{Polynomial rings in one variable}
Let $R$ be a ring. The \emph{polynomial ring} over $R$ in one variable $X$ is the set $R[X]$ of all sequences in $R$ with only finitely many nonzero terms. If $(a_0, a_1, a_2, a_3, \dots)$ is an element in $R[X]$, with $a_n = 0$ for all $n > N$, then we usually write this element as
$$
\sum_{n=0}^N a_n X^n = a_0 + a_1 X + a_2 X^2 + a_3 X^3 + \cdots + a_N X^N.
$$
Elements of $R[X]$ are called \emph{polynomials} in the indeterminate $X$ with coefficients in $R$. The ring elements $a_0, \ldots, a_N$ are called \emph{coefficients} of the polynomial, and the \emph{degree} of a polynomial is the largest natural number $N$ for which $a_N \neq 0$, if such an $N$ exists. When a polynomial has all of its coefficients equal to $0$, its degree is usually considered to be undefined, although some people adopt the convention that its degree is $-\infty$.

A \emph{monomial} is a polynomial with exactly one nonzero coefficient. Similarly, a \emph{binomial} is a polynomial with exactly two nonzero coefficients, and a \emph{trinomial} is a polynomial with exactly three nonzero coefficients.

Addition and multiplication of polynomials is defined by
\begin{eqnarray}
\sum_{n=0}^N a_n X^n + \sum_{n=0}^N b_n X^n & = & \sum_{n=0}^N (a_n+b_n) X^n \\
\sum_{n=0}^N a_n X^n \cdot \sum_{n=0}^N b_n X^n & = & \sum_{n=0}^{2N} \left(\sum_{k=0}^n a_k b_{n-k}\right) X^n
\end{eqnarray}
$R[X]$ is a $\mathbb{Z}$--graded ring under these operations, with the monomials of degree exactly $n$ comprising the $n^\mathrm{th}$ graded component of $R[X]$. The zero element of $R[X]$ is the polynomial whose coefficients are all $0$, and when $R$ has a multiplicative identity $1$, the polynomial whose coefficients are all $0$ except for $a_0 = 1$ is a multiplicative identity for the polynomial ring $R[X]$.

\section{Polynomial rings in finitely many variables}

The \emph{polynomial ring} over $R$ in two variables $X,Y$ is defined to be $R[X,Y] := R[X][Y] \cong R[Y][X]$. Elements of $R[X,Y]$ are called \emph{polynomials} in the indeterminates $X$ and $Y$ with coefficients in $R$. A \emph{monomial} in $R[X,Y]$ is a polynomial which is simultaneously a monomial in both $X$ and $Y$, when considered as a polynomial in $X$ with coefficients in $R[Y]$ (or as a polynomial in $Y$ with coefficients in $R[X]$). The \emph{degree} of a monomial in $R[X,Y]$ is the sum of its individual degrees in the respective indeterminates $X$ and $Y$ (in $R[Y][X]$ and $R[X][Y]$), and the degree of a polynomial in $R[X,Y]$ is the supremum of the degrees of its monomial summands, if it has any.

In three variables, we have $R[X,Y,Z] := R[X,Y][Z] = R[X][Y][Z] \cong R[X][Z][Y] \cong \cdots$, and in any finite number of variables, we have inductively $R[X_1,X_2,\dots,X_n] := R[X_1,\dots,X_{n-1}][X_n] = R[X_1][X_2]\cdots[X_n]$, with monomials and degrees defined in analogy to the two variable case. In any number of variables, a polynomial ring is a graded ring with $n^\mathrm{th}$ graded component equal to the $R$-module generated by the monomials of degree $n$.

\section{Polynomial rings in arbitrarily many variables}

For any nonempty set $M$, let $E(M)$ denote the set of all finite subsets of $M$. For each element $A = \{a_1, \ldots ,a_n\}$ of $E(M)$, set $R[A]:=R[a_1,\ldots,a_n]$.  Any two elements $A,B \in E(M)$ satisfying $A \subset B$ give rise to the relationship $R[A] \subset R[B]$ if we consider $R[A]$ to be embedded in $R[B]$ in the obvious way. The union of the rings $\{R[A] : A \in E(M)\}$ (or, more formally, the categorical direct limit of the direct system of rings $\{R[A] : A \in E(M)\}$) is defined to be the ring $R[M]$.
\end{document}
