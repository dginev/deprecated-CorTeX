\documentclass{article}
\usepackage{planetmath-specials}
\usepackage{pmath}
\usepackage{graphicx}
%\usepackage{xypic} 
\usepackage{bbm}
\newcommand{\Z}{\mathbbmss{Z}}
\newcommand{\C}{\mathbbmss{C}}
\newcommand{\R}{\mathbbmss{R}}
\newcommand{\Q}{\mathbbmss{Q}}
\newcommand{\mathbb}[1]{\mathbbmss{#1}}
\newcommand{\figura}[1]{\begin{center}\includegraphics{#1}\end{center}}
\newcommand{\figuraex}[2]{\begin{center}\includegraphics[#2]{#1}\end{center}}
\begin{document}
A \emph{well-ordered} set is a totally ordered set in which every nonempty subset has a least member.

An example of well-ordered set is the set of positive integers with the standard order relation $(\mathbbmss{Z}^+,<)$, because any nonempty subset of it has least member. However,  $\mathbbmss{R}^+$ (the positive reals) is not a well-ordered set with the usual order, because $(0,1)=\{x:0<x<1\}$ is a nonempty subset but it doesn't contain a least number.

A \textbf{well-ordering} of a set $X$ is the result of defining a binary relation $\leq$ on $X$ to itself in such a way that $X$ becomes well-ordered with respect to $\leq$.
\end{document}
