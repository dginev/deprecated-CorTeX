\documentclass{article}
\usepackage{ids}
\usepackage{planetmath-specials}
\usepackage{pmath}
\usepackage{amssymb}
\usepackage{amsmath}
\usepackage{amsfonts}
\usepackage{graphicx}
\usepackage{xypic}
\begin{document}
Two elements in a ring with unity are \emph{associates} or {\em associated elements} of each other
if one can be obtained from the other by multiplying by some unit,
that is, $a$ and $b$ are associates if there is a unit $u$ such that\,
$a = bu$.\, Equivalently, one can say that two associates are divisible by each other.

The binary relation ``is an associate of'' is an equivalence relation
on any ring with unity.  For example, the equivalence class of the
unity of the ring consists of all units of the ring.

\textbf{Examples.} In the ring $\mathbb{Z}$ of the rational integers, only opposite numbers $\pm n$ are associates.\, Among the polynomials, the associates of a polynomial are gotten by multiplying the polynomial by an element belonging to the coefficient ring in question (and being no zero divisor).

In an integral domain, two elements are associates if and only if they
generate the same principal ideal.

\end{document}
