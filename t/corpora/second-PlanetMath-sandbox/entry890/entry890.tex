\documentclass{article}
\usepackage{planetmath-specials}
\usepackage{pmath}
%\usepackage{graphicx}
%\usepackage{xypic} 
\usepackage{bbm}
\newcommand{\Z}{\mathbbmss{Z}}
\newcommand{\C}{\mathbbmss{C}}
\newcommand{\R}{\mathbbmss{R}}
\newcommand{\Q}{\mathbbmss{Q}}
\newcommand{\mathbb}[1]{\mathbbmss{#1}}
\begin{document}
\PMlinkescapeword{character}
\PMlinkescapeword{mean}
\PMlinkescapeword{name}
\PMlinkescapeword{universe}
\PMlinkescapeword{difference}
{\bf Definition}
A \emph{googol} equals to the number $10^{100}$, that is, a one followed by
one hundred zeros.
A \emph{googolplex} is
is  ten raised to the power of a googol, i.e., $10^{(10^{100})}$.

This is already a huge number. For instance,
it is more than the number of atoms in the known universe.
A googoplex is even larger. In fact, since a googolplex has 
a googol number of zeros in its decimal
representation, 
a googoplex has more digits than there are atoms in our universe. 
Thus, even if all matter in the universe were at disposal,
it would not be possible to write down the decimal representatio{n} of
a googolplex \cite{wikigoogol}.

\subsubsection*{Properties}
\begin{enumerate}
\item
A googol is approximately the factorial of $70$.
The only prime factors in a googol and a googolplex are $2$ and $5$ \cite{wikigoogol}.
\item Using Stirling's formula  we can approximate the factorial of  a googol to obtain
$$ 
  10^{100}! \approx 10^{9.95\, \cdot 10^{101}}.$$
\item A googoplex plus one is not a prime. One factor is 316912650057057350374175801344000001 \cite{mrob}.
\end{enumerate}

\subsubsection*{History and etymology}
The googol was created by the American mathematician Edward Kasner (1878-1955) \cite{wikikasner}
in \cite{kasner} to  illustrate the difference between an
 unimaginably large number and infinity.
The name 'googol' was supposedly coined by Kasner's
nine-year-old nephew Milton Sirotta in 1938 when asked to give a name for a huge number.
The name googol was perhaps influenced by
the comic strip character Barney Google \cite{wikigoogol, harper}.
The name of the search engine google was inspired by the number 
googol \cite{google}.

\begin{thebibliography}{9}
 \bibitem {wikigoogol} Wikipedia's \PMlinkexternal{entry on googol}{http://en2.wikipedia.org/wiki/Googol}.
\bibitem{kasner} Kasner, Edward\&Newman, James Roy,\emph{Mathematics and the Imagination}, (New York, NY, U
SA:
Simon and Schuster, 1967; Dover Pubns, April 2001; London: Penguin, 1940, ISBN 0486417034).
 \bibitem {harper} Douglas Harper's 
\PMlinkexternal{Etymology online dictionary, Googol}{http://www.etymonline.com/g3etym.htm} 12/2003.
 \bibitem {wikikasner} Wikipedia's \PMlinkexternal{entry on Edward Kasner}{http://en2.wikipedia.org/wiki/Edward_Kasner}.
 \bibitem {mrob} Robert Munafo, \PMlinkexternal{web page on large numbers}{http://www.mrob.com/pub/math/numbers-21.html}.
\bibitem{google} Google corporate information: 
\PMlinkexternal{Google Milestones}{http://www.google.com/corporate/history.html}.
\end{thebibliography}
\end{document}
