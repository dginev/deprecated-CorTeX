\documentclass{article}
\usepackage{planetmath-specials}
\usepackage{pmath}
\usepackage{amssymb}
\usepackage{amsmath}
\usepackage{amsfonts}

\begin{document}
The \emph{ceiling} of a real number is the smallest integer greater than or equal to the number. The ceiling of $x$ is usually denoted by $\lceil x\rceil$.

Some examples:
$\lceil 6.2\rceil=7$, $\lceil 0.4\rceil=1$, $\lceil 7\rceil=7$, $\lceil -5.1\rceil=-5$, $\lceil \pi\rceil=4$, $\lceil -4\rceil=-4$.

Note that this function is not the integer part ($[x]$), since 
$\lceil 3.5\rceil = 4$ and $[3.5]=3$.

The notation for floor and ceiling was introduced by Iverson in 1962\cite{Higham}.
 
\begin{thebibliography}{9}
\bibitem{Higham} N. Higham, Handbook of writing for the mathematical sciences, Society for Industrial and Applied Mathematics, 1998.
\end{thebibliography}
\end{document}
