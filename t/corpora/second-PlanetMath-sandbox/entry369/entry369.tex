\documentclass{article}
\usepackage{ids}
\usepackage{planetmath-specials}
\usepackage{pmath}
\usepackage{amssymb}
\usepackage{amsmath}
\usepackage{amsfonts}
\usepackage{graphicx}
\usepackage{xypic}

\def\C{\mathbb{C}}
\def\Re{\operatorname{Re}}
\begin{document}
\PMlinkescapeword{arguments}
\PMlinkescapeword{calculate}
\PMlinkescapeword{equivalent}
\PMlinkescapeword{euler}
\PMlinkescapeword{formula}
\PMlinkescapephrase{generated by}
\PMlinkescapeword{natural}
\PMlinkescapeword{satisfies}

\section*{Introduction}

The \emph{gamma function} can be thought of as
the natural way to generalize the concept of the factorial
to non-integer arguments.

\PMlinkname{Leonhard Euler}{EulerLeonhard}
came up with a formula for such a generalization in 1729.
At around the same time,
James Stirling independently arrived at a different formula,
but was unable to show that it always converged.
In 1900, Charles Hermite showed that the formula given by Stirling does work,
and that it defines the same function as \PMlinkescapetext{Euler's}.

\section*{Definitions}

\PMlinkescapetext{Euler's original formula for the gamma function was}
\[
  \Gamma(z) = \lim_{n\to\infty}\frac{n^z n!}{\prod_{k=0}^n(z+k)}.
\]
However, it is now more commonly defined by
\[
  \Gamma(z) = \int_0^\infty \! e^{-t} t^{z-1} \, dt
\]
for $z\in\C$ with $\Re(z)>0$,
and by analytic continuation for the rest of the complex plane,
except for the non-positive integers (where it has simple poles).

Another equivalent definition is
\[
  \Gamma(z) = \frac{e^{-\gamma z}}{z}
  \prod_{n=1}^\infty \left(1 + \frac{z}{n}\right)^{-1} e^{z/n},
\]
where $\gamma$ is Euler's constant.

\section*{Functional equations}

The gamma function satisfies the functional equation
\[
  \Gamma(z+1) = z \Gamma(z)
\]
except when $z$ is a non-positive integer.
As $\Gamma(1)=1$, it follows by induction that
\[
  \Gamma(n) = (n-1)!
\]
for positive integer values of $n$.

Another functional equation satisfied by the gamma function is
\[
  \Gamma(z) \Gamma(1-z) = \frac{\pi}{\sin \pi z}
\]
for non-integer values of $z$.

\section*{Approximate values}

The gamma function for real $z$ looks like this:
\begin{center}
\begin{tabular}{c}
\includegraphics[scale=1]{gammafunc.eps} \\
{\tiny (generated by GNU Octave and gnuplot) }
\end{tabular}
\end{center}

It can be shown that $\Gamma(1/2)=\sqrt{\pi}$.
Approximate values of $\Gamma(x)$ for some other $x\in(0,1)$ are:
\[
  \begin{array}{cc}
  \Gamma(1/5) \approx 4.5908 & \Gamma(1/4) \approx 3.6256 \\
  \Gamma(1/3) \approx 2.6789 & \Gamma(2/5) \approx 2.2182 \\
  \Gamma(3/5) \approx 1.4892 & \Gamma(2/3) \approx 1.3541 \\
  \Gamma(3/4) \approx 1.2254 & \Gamma(4/5) \approx 1.1642 
  \end{array}
\]

If the value of $\Gamma(x)$ is known for some $x\in(0,1)$,
then one may calculate the value of $\Gamma(n+x)$ for any integer $n$
by making use of the formula $\Gamma(z+1)=z\Gamma(z)$.
We have
\begin{eqnarray*}
  \Gamma(n+x) & = & (n+x-1)\Gamma(n+x-1) \\
  & = & (n+x-1)(n+x-2)\Gamma(n+x-2) \\
  & \vdots & \\
  & = & (n+x-1)(n+x-2)\cdots(x)\Gamma(x) 
\end{eqnarray*}
which is easy to calculate if we know $\Gamma(x)$.

\begin{thebibliography}{9}
\bibitem{havil}
 Julian Havil,
 {\it Gamma: Exploring Euler's Constant},
 Princeton University Press, 2003.
 (Chapter 6 is about the gamma function.)
\end{thebibliography}

\end{document}
