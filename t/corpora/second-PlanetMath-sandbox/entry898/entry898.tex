\documentclass{article}
\usepackage{planetmath-specials}
\usepackage{pmath}
% this is the default PlanetMath preamble.  as your knowledge
% of TeX increases, you will probably want to edit this, but
% it should be fine as is for beginners.

% almost certainly you want these
\usepackage{amssymb}
\usepackage{amsmath}
\usepackage{amsfonts}

% used for TeXing text within eps files
%\usepackage{psfrag}
% need this for including graphics (\includegraphics)
%\usepackage{graphicx}
% for neatly defining theorems and propositions
%\usepackage{amsthm}
% making logically defined graphics
%\usepackage{xypic} 

% there are many more packages, add them here as you need them

% define commands here
\begin{document}
Let $Q$ be a non-degenerate symmetric bilinear form over the real vector space $\mathbb{R}^n$. A linear transformation $T\colon V \to V$ is said to \emph{preserve} $Q$ if $Q(Tx,Ty) = Q(x,y)$ for all vectors $x,y \in V$. The subgroup of the general linear group $\operatorname{GL}(V)$ consisting of all linear transformations that preserve $Q$ is called the \emph{orthogonal group} with respect to $Q$, and denoted  $\operatorname{O}(n,Q)$.

If $Q$ is also positive definite (i.e., $Q$ is an inner product), then $\operatorname{O}(n,Q)$ is equivalent to the group of invertible linear transformations that preserve the standard inner product on $\mathbb{R}^n$, and in this case the group $\operatorname{O}(n,Q)$ is usually denoted $\operatorname{O}(n)$.

Elements of $\operatorname{O}(n)$ are called \emph{orthogonal transformations}. 
One can show that a linear transformation $T$ is an orthogonal transformation if and only if $T^{-1} = T^{\operatorname{T}}$ (i.e., the inverse of $T$ equals the transpose of $T$).
\end{document}
