\documentclass{article}
\usepackage{planetmath-specials}
\usepackage{pmath}
\usepackage{amssymb}
\usepackage{amsmath}
\usepackage{amsfonts}
\usepackage{graphicx}
\usepackage{xypic}
\begin{document}
Let $S$ be a set. An {\em ordering relation} is a relation $\leq$ on $S$ such that, for every $a,b,c \in S$:
\begin{itemize}
\item Either $a \leq b$, or $b \leq a$,
\item If $a \leq b$ and $b \leq c$, then $a \leq c$,
\item If $a \leq b$ and $b \leq a$, then $a = b$.
\end{itemize}

Equivalently, an ordering relation is a relation $\leq$ on $S$ which makes the pair $(S,\leq)$ into a totally ordered set. {\bf Warning:} In some cases, an author may use the term ``ordering relation'' to mean a partial order instead of a total order.

Given an ordering relation $\leq$, one can define a relation $<$ by: $a < b$ if $a \leq b$ and $a \neq b$. The {\em opposite ordering} is the relation $\geq$ given by: $a \geq b$ if $b \leq a$, and the relation $>$ is defined analogously.
\end{document}
