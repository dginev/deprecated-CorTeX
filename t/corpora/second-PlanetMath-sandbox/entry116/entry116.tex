\documentclass{article}
\usepackage{ids}
\usepackage{planetmath-specials}
\usepackage{pmath}
\usepackage{amssymb}
\usepackage{amsmath}
\usepackage{amsfonts}
\usepackage{graphicx}
\usepackage{xypic}
\begin{document}
The \emph{infimum} of a set $S$ is the greatest lower bound of $S$ and is denoted $\inf(S)$.

Let $A$ be a set with a partial order $\leq$, and let $S \subseteq A$.  For any $x \in A$, $x$ is a lower bound of $S$ if $x \leq y$ for any $y \in S$.  The  infimum of $S$, denoted $\inf(S)$, is the greatest such lower bound; that is, if $b$ is a lower bound of $S$, then $b \leq \inf(S)$.

Note that it is not necessarily the case that $\inf(S) \in S$.  Suppose $S = (0, 1)$; then $\inf(S) = 0$, but $0 \not\in S$.

Also note that a set does not necessarily have an infimum. See the attachments to this entry for examples.
\end{document}
