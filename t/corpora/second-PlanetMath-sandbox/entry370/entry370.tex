\documentclass{article}
\usepackage{planetmath-specials}
\usepackage{pmath}
\usepackage{amssymb}
\usepackage{amsmath}
\usepackage{amsfonts}
\usepackage{amsthm}

\theoremstyle{definition}
\newtheorem*{defn}{Cardinality}

\newtheorem*{altdefn}{Cardinality (alt. def.)}

\def\Z{\mathbb{Z}}
\begin{document}
\section*{Cardinality}

\emph{Cardinality} is a notion of the size of a set which does not rely on numbers.  It is a relative notion.  For instance, two sets may each have an infinite number of elements, but one may have a greater cardinality.  That is, in a sense, one may have a ``more infinite'' number of elements.  See Cantor diagonalization for an example of how the reals have a greater cardinality than the natural numbers.

The formal definition of cardinality rests upon the notion mappings between sets:

\begin{defn}
The cardinality of a set $A$ is greater than or equal to 
the cardinality of a set $B$
if there is a one-to-one function (an injection) from $B$ to $A$.
Symbolically, we write $|A| \geq |B|$.
\end{defn}

and

\begin{defn}
Sets $A$ and $B$ have the same cardinality
if there is a one-to-one and onto function (a bijection) from $A$ to $B$.
Symbolically, we write $|A| = |B|$.
\end{defn}

It can be shown that if $|A| \geq |B|$ and $|B| \geq |A|$ then $|A| = |B|$.
This is the Schröder-Bernstein Theorem.

Equality of cardinality is variously called \emph{equipotence}, \emph{equipollence}, \emph{equinumerosity}, or \emph{equicardinality}.
For $|A| = |B|$, we would say that ``$A$ is \emph{equipotent} to $B$'',
``$A$ is \emph{equipollent} to $B$'', or ``$A$ is \emph{equinumerous} to $B$''.

An equivalent definition of cardinality is

\begin{altdefn}
The cardinality of a set $A$ is the unique cardinal number $\kappa$ such that $A$ is equinumerous with $\kappa$. The cardinality of $A$ is written $|A|$.
\end{altdefn}

This definition of cardinality makes use of a special class of numbers, called the cardinal numbers. This highlights the fact that, while cardinality can be understood and defined without appealing to numbers, it is often convenient and useful to treat cardinality in a ``numeric'' manner.

\section*{Results}

Some results on cardinality:

\begin{enumerate}

\item $A$ is equipotent to $A$.

\item If $A$ is equipotent to $B$, then $B$ is equipotent to $A$.

\item If $A$ is equipotent to $B$ and $B$ is equipotent to $C$, then $A$ is equipotent to $C$.

\end{enumerate}

\begin{proof} Respectively:

\begin{enumerate} 

\item The identity function on $A$ is a bijection from $A$ to $A$.

\item If $f$ is a bijection from $A$ to $B$, then $f^{-1}$ exists and is a bijection from $B$ to $A$.

\item If $f$ is a bijection from $A$ to $B$ and $g$ is a bijection from $B$ to $C$, then $f \circ g$ is a bijection from $A$ to $C$.

\end{enumerate}

\end{proof}

\section*{Example}

The set of even integers $2\Z$ has the same cardinality as the set of integers $\Z$: if we define $f\colon 2\Z \to \Z$ such that $f(x) = \frac{x}{2}$,
then $f$ is a bijection, and therefore $|2\Z| = |\Z|$.
\end{document}
