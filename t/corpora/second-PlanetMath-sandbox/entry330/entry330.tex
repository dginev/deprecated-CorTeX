\documentclass{article}
\usepackage{ids}
\usepackage{planetmath-specials}
\usepackage{pmath}
\usepackage{amssymb}
\usepackage{amsmath}
\usepackage{amsfonts}

\newcommand{\bu}{\mathbf{u}}
\newcommand{\bv}{\mathbf{v}}
\begin{document}
A \emph{unit vector} is a unit-length element of Euclidean space.
Equivalently, one may say that the norm of a unit vector is equal
to $1$, and write $\Vert \bu\Vert=1$, where $\bu$ is the vector in
question.

Let $\bv$ be a non-zero vector.  To \emph{normalize} $\bv$ is to find
the unique unit vector with the same direction as $\bv$. This is done
by multiplying $\bv$ by the reciprocal of its length; the
corresponding unit vector is given by $\bu=\frac{\bv}{\Vert
  \bv\Vert}$.

\paragraph{Note:} The concept of a unit vector and normalization makes
sense in any vector space equipped with a real or complex norm.
Thus, in quantum mechanics one  represents states as unit vectors
belonging to a (possibly) infinite-dimensional Hilbert space.  To
obtain an expression for such states one  normalizes
the results of a calculation. 

\paragraph{Example:} Consider $\mathbb{R}^3$ and the vector
$\bv=(1,2,3)$. The norm (length) is $\sqrt{14}$.  Normalizing, we obtain
the unit vector $\bu$ pointing in the same direction, namely
$\bu=\left(\frac{1}{\sqrt{14}},\frac{2}{\sqrt{14}},\frac{3}{\sqrt{14}}\right)$.
\end{document}
