\documentclass{article}
\usepackage{planetmath-specials}
\usepackage{pmath}
\usepackage{amssymb}
\usepackage{amsmath}
\usepackage{amsfonts}
\usepackage{graphicx}
\usepackage{xypic}
\begin{document}
\PMlinkescapeword{range}
We give a proof for the ``strong" formulation.

Let $S$ be a set of natural numbers such that $n$ belongs to $S$ whenever all numbers less than $n$ belong to $S$ (i.e., assume $\forall n(\forall m<n\ m\in S)\Rightarrow n\in S$, where the quantifiers range over all natural numbers). For indirect proof, suppose that $S$ is not the set of natural numbers $\mathbb{N}$. That is, the complement $\mathbb{N}\setminus S$ is nonempty.  The well-ordering principle for natural numbers says that $\mathbb{N}\setminus S$ has a smallest element; call it $a$. By assumption, the statement $(\forall m<a\ m\in S)\Rightarrow a\in S$ holds. Equivalently, the contrapositive statement $a\in \mathbb{N}\setminus S \Rightarrow \exists m<a\ m\in \mathbb{N}\setminus S$ holds. This gives a contradition since the element $a$ is an element of $\mathbb{N}\setminus S$ and is, moreover, the \emph{smallest} element of $\mathbb{N}\setminus S$.
\end{document}
