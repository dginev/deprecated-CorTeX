\documentclass{article}
\usepackage{ids}
\usepackage{planetmath-specials}
\usepackage{pmath}
\usepackage{amssymb}
\usepackage{amsmath}
\usepackage{amsfonts}
\usepackage{graphicx}
\usepackage{xypic}
\begin{document}
A \textbf{gamma random variable} with parameters $\alpha>0$ and $\lambda>0$ is one whose probability density function is given by
\begin{align*}
f_X(x) = \frac{ \lambda^\alpha}{\Gamma(\alpha)} x^{\alpha - 1} e^{-\lambda x}  
\end{align*}
for $x>0$, and is denoted by $X\sim Gamma(\alpha, \lambda)$.

Notes:\\
\begin{enumerate}
\item Gamma random variables are widely used in many applications. Taking $\alpha = 1$ reduces the form to that of an exponential random variable. If $\alpha = \frac{n}{2}$ and $\lambda = \frac{1}{2}$, this is a chi-squared random variable.
\item The function $\Gamma: [0,\infty] \to R$ is the gamma function, defined as $\Gamma(t) = \int_{0}^{\infty}{x^{t-1} e^{-x} dx}$. 
\item The expected value of a gamma random variable is given by $E[X]=\frac{\alpha}{\lambda}$, and the variance by $Var[X] = \frac{\alpha}{\lambda^2}$
\item The moment generating function of a gamma random variable is given by $M_X(t) = (\frac{\lambda}{\lambda - t})^\alpha$.
\end{enumerate}

If the first parameter is a positive integer, the variate is usually called Erlang random variate. The sum of $n$ exponentially distributed variables with parameter $\lambda$ is a gamma (Erlang) variate with parameters $n, \lambda$.
\end{document}
