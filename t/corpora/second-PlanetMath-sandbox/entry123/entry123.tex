\documentclass{article}
\usepackage{ids}
\usepackage{planetmath-specials}
\usepackage{pmath}
\usepackage{amssymb}
\usepackage{amsmath}
\usepackage{amsfonts}
%\usepackage{graphicx}
%\usepackage{xypic}

%\makeatletter
%\@ifundefined{bibname}{}{\renewcommand{\bibname}{References}}
%\makeatother
\begin{document}
{\bf Definition} Suppose $A$ is a subset of a set $X$. Then the
function
\begin{equation*}
\chi_A(x) =
\begin{cases}
1,&\text{when }x\in A,\\
0,&\text{when }x\in X\setminus A
\end{cases}
\end{equation*}
is the \emph{characteristic function} for $A$.

%From the definition, it follows that there is  a natural correspondence
%between charactersitic functions in a set $X$ and the power set of $X$.

\subsubsection{Properties}
Suppose $A,B$ are subsets of a set $X$.
\begin{enumerate}
\item For set intersections and set unions, we have
   \begin{eqnarray*}
   \chi_{A\cap B} &=& \chi_A \chi_B, \\
   \chi_{A\cup B} &=& \chi_A + \chi_B - \chi_{A\cap B},\\
   \chi_{A\cap B} &=& \min(\chi_A,\chi_B),\\
   \chi_{A\cup B} &=& \max(\chi_A,\chi_B).
   \end{eqnarray*}
\item For the symmetric difference,
$$\chi_{A\bigtriangleup B} = \chi_A + \chi_B - 2\chi_{A\cap B}.$$
\item For the set complement,
$$\chi_{A^\complement} = 1-\chi_A. $$
\end{enumerate}


\subsubsection{Remarks}
A synonym for characteristic function is \emph{indicator function}
\cite{folland}.


\begin{thebibliography}{9}
 \bibitem{folland}
 G.B. Folland, \emph{Real Analysis: Modern Techniques and Their Applications}, 2nd ed, John Wiley \& Sons, Inc., 1999.
 \end{thebibliography}
\end{document}
