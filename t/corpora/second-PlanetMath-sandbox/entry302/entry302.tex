\documentclass{article}
\usepackage{planetmath-specials}
\usepackage{pmath}
\usepackage{amssymb}
\usepackage{amsmath}
\usepackage{amsfonts}
\usepackage{graphicx}
\usepackage{xypic}
\begin{document}
\PMlinkescapeword{index}

The phrase ``\emph{the $n$-th root of a number}'' is a somewhat misleading concept that requires a fair amount of thought to make rigorous.

For $n$ a positive integer, we define \emph{an} $n$-th root of a number $x$ to be a number $y$ such that $y^n=x$.  The number $n$ is said to be the \emph{index} of the root.  Note that the term ``number'' here is ambiguous, as the discussion can apply in a variety of contexts (groups, rings, monoids, etc.)  The purpose of this entry is specifically to deal with $n$-th roots of real and complex numbers.

In an effort to give meaning to the term \emph{the} $n$-th root of a real number $x$, we define it to be the unique real number that $y$ is \emph{an} $n$th root of $x$ and such that $\operatorname{sign}(x)=\operatorname{sign}(y)$, if such a number exists.  We denote this number by $\sqrt[n]{x}$, or by $x^{\frac{1}{n}}$ if $x$ is positive.  This specific $n$th root is also called the \emph{principal $n$th root}.

Example:  $\sqrt[4]{81} = 3$ because $3^4 = 3 \times 3 \times 3 \times 3= 81$, and $3$ is the unique positive real number with this property.

Example:  If $x+1$ is a positive real number, then we can write $\sqrt[5]{x^5 + 5x^4 + 10x^3 + 10x^2 + 5x + 1} = x + 1$ because
$(x + 1)^5 = (x^2 + 2x + 1)^2(x + 1) = x^5 + 5x^4 + 10x^3 + 10x^2 + 5x + 1$.  (See the Binomial Theorem
and \PMlinkescapetext{Pascal's Triangle}.)

The nth root operation is distributive for multiplication and division, but not for addition and
subtraction.  That is, $\sqrt[n]{x \times y} = \sqrt[n]{x} \times \sqrt[n]{y}$, and
$\sqrt[n]{\frac{x}{y}} = \frac{\sqrt[n]{x}}{\sqrt[n]{y}}$.  However, except in special cases,
$\sqrt[n]{x + y} \not= \sqrt[n]{x} + \sqrt[n]{y}$ and $\sqrt[n]{x - y} \not= \sqrt[n]{x} - \sqrt[n]{y}$.

Example:  $\sqrt[4]{\frac{81}{625}} = \frac{3}{5}$ because
$\left(\frac{3}{5}\right)^4 = \frac{3^4}{5^4} = \frac{81}{625}$.

Note that when we restrict our attention to real numbers, expressions like $\sqrt{-3}$ are undefined.  Thus, for a more full definition of $n$th roots, we will have to incorporate the notion of complex numbers:  \emph{The nth roots of a complex number} $t = x + yi$ are all the complex numbers $z_1, z_2, \ldots, z_n \in \mathbb{C}$ that satisfy the condition $z_k^n = t$.  Applying the fundamental theorem of algebra (complex version) to the function $x^n-t$ tells us that $n$ such complex numbers always exist (counting multiplicity).

One of the more popular methods of finding these roots is through trigonometry and the geometry of complex numbers.  For a complex number $z=x+iy$, recall that we can put $z$ in polar form:  $z=(r, \theta)$, where $r = \sqrt[2]{x^2 + y^2}$, and $\theta = \frac{\pi}{2}$ if $x = 0$, and $\theta = \arctan{\frac{y}{x}}$ if $x \not= 0$.  (See the Pythagorean Theorem.)  For the specific procedures involved, see calculating the nth roots of a complex number.
\end{document}
