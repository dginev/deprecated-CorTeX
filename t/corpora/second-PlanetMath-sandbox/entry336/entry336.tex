\documentclass{article}
\usepackage{planetmath-specials}
\usepackage{pmath}
% this is the default PlanetMath preamble.  as your knowledge
% of TeX increases, you will probably want to edit this, but
% it should be fine as is for beginners.

% almost certainly you want these
\usepackage{amssymb}
\usepackage{amsmath}
\usepackage{amsfonts}

% used for TeXing text within eps files
%\usepackage{psfrag}
% need this for including graphics (\includegraphics)
\usepackage{graphicx}
% for neatly defining theorems and propositions
 \usepackage{amsthm}
% making logically defined graphics
%\usepackage{xypic}

% there are many more packages, add them here as you need them

% define commands here

\theoremstyle{definition}
\newtheorem*{thmplain}{Theorem}
\begin{document}
\PMlinkescapeword{words}

When a straight line moves in the space without changing its direction, the ruled surface it sweeps is called a {\em cylindrical surface} (or, in some special cases, simply a cylinder).\, Formally, a cylindrical surface $S$ is a ruled surface with the given condition: 
\begin{quote}
If $p,\,q$ are two distinct points in $S$, and $l$ and $m$ are the rulings passing through $p$ and $q$ respectively, then\, $l \parallel m$ (this includes the case when\, $l = m$).
\end{quote}
If the moving line returns to its starting point, the cylindrical surface $S$ is said to be \PMlinkescapetext{{\em closed}}.\, In other words, if we take any plane $\pi$ perpendicular to any of its rulings, and observe the curve $c$ of intersection of $\pi$ and $S$, then $S$ is \PMlinkescapetext{closed} if $c$ is a closed curve.

% This is a picture of a circular cylinder.
% (I know, not very creative, but if someone can come up with a nice 
% parameterized curve to use, I can change that. Look at the Python 
% source file cylinder.py if you are curious.) // SteveCheng
\begin{figure}
\begin{center}
\includegraphics{cylinder.eps}
\end{center}
\caption{A closed cylindrical surface}
\end{figure}

The solid \PMlinkescapetext{bounded by a closed} cylindrical surface and two parallel planes is a {\em cylinder}.\, The portion of the surface of the cylinder belonging to the cylindrical surface is called the {\em lateral surface} or the {\em mantle} of the cylinder and the portions belonging to the planes are the {\em bases} of the cylinder.

The bases of any cylinder are congruent.\, The line segment of a generatrix between the planes is a \PMlinkescapetext{{\em side line}} of the cylinder.\, All side lines are equally long.\, If the side lines are perpendicular to the planes of the bases, one speaks of a {\em right cylinder}, otherwise of a {\em skew cylinder}.

The perpendicular distance of the planes of the bases is the \PMlinkescapetext{{\em height}} of the cylinder.\, The volume ($V$) of the cylinder equals the product of the base area ($A$) and the height ($h$):
                              $$V = Ah$$
If the base is a polygon, the cylinder is called a {\em prism} (which is a polyhedron).\, The faces of the mantle of a prism are parallelograms.\, If also the bases of a prism are parallelograms, the prism is a {\em parallelepiped}.\, If the faces of the mantle of a prism are rectangles, one speaks of a {\em right prism}, otherwise of a {\em skew prism}.\, 

For any integer $n \ge 3$, the following are equivalent statements about a prism $P$:

\begin{enumerate}
\item $P$ has a base that is an $n$-gon;
\item $P$ has $n+2$ faces;
\item $P$ has $2n$ vertices;
\item $P$ has $3n$ edges.
\end{enumerate}

\textbf{Note.}\, The notion of the prism (or cylinder) of a polygon in $\mathbb{R}^3$ has a higher-dimensional analogue.\, Given any polytope $P$, the
prism of P is the polytope\, $\mbox{Prism}(P) := P\!\times\![0,\,1]$.\, The
vertices of $\mbox{Prism}(P)$ are the points\, $(x,\,0)$ and\, $(x,\,1)$, where
$x$ \PMlinkescapetext{ranges} over the vertices of $P$.\, In other words, we drag
$P$ a short distance through a vector orthogonal to everything
in $P$, just as we would to obtain the prism of a polygon.
\end{document}
