\documentclass{article}
\usepackage{ids}
\usepackage{planetmath-specials}
\usepackage{pmath}
\usepackage{amssymb}
\usepackage{amsmath}
\usepackage{amsfonts}
\usepackage{graphicx}
\usepackage{xypic}
\begin{document}
Let $R$ be a \PMlinkname{domain}{IntegralDomain}.  
We say that $R$ is a {\it right Ore domain} 
if any two nonzero elements of $R$ 
have a nonzero common right multiple, 
i.e. for every pair of nonzero $x$ and $y$, 
there exists a pair of elements $r$ and $s$ of $R$ 
such that $xr = ys \neq 0$.\par
This condition turns out to be equivalent 
to the following conditions on $R$ when viewed as a right $R$-module:\\
(a) $R_R$ is a uniform module.\\
(b) $R_R$ is a module of finite rank.\par
The definition of a {\it left Ore domain} is similar.\par
If $R$ is a commutative \PMlinkname{domain}{IntegralDomain}, 
then it is a right (and left) Ore domain.
\end{document}
