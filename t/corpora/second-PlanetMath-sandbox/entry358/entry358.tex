\documentclass{article}
\usepackage{planetmath-specials}
\usepackage{pmath}
\usepackage{amssymb}
\usepackage{amsmath}
\usepackage{amsfonts}
%\usepackage{graphicx}
%\usepackage{xypic}
\begin{document}
\PMlinkescapeword{contains}
\PMlinkescapeword{primes}

Let $C$ be a countable set of countable sets.  We will show that $\cup C$ is countable.

Let $P$ be the set of positive \PMlinkname{primes}{Prime}.  $P$ is countably infinite, so there is a bijection between $P$ and $\mathbb{N}$.  Since there is a bijection between $C$ and a subset of $\mathbb{N}$, there must in turn be a one-to-one function $f: C \rightarrow P$.

Each $S \in C$ is countable, so there exists a bijection between $S$ and some subset of $\mathbb{N}$.  Call this function $g$, and define a new function $h_S: S \rightarrow \mathbb{N}$ such that for all $x \in S$,
$$h_S(x) = f(S)^{g(x)}$$
Note that $h_S$ is one-to-one.  Also note that for any distinct pair $S, T \in C$, the range of $h_S$ and the range of $h_T$ are disjoint due to the fundamental theorem of arithmetic.

We may now define a one-to-one function $h: \cup C \rightarrow \mathbb{N}$, where, for each $x \in \cup C$, $h(x) = h_S(x)$ for some $S \in C$ where $x \in S$ (the choice of $S$ is irrelevant, so long as it contains $x$).  Since the range of $h$ is a subset of $\mathbb{N}$, $h$ is a bijection into that set and hence $\cup C$ is countable.
\end{document}
