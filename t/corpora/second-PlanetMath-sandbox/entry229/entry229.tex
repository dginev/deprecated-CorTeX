\documentclass{article}
\usepackage{ids}
\usepackage{planetmath-specials}
\usepackage{pmath}
\usepackage{amssymb}
\usepackage{amsmath}
\usepackage{amsfonts}
\begin{document}
\PMlinkescapeword{word}

For any non-negative integer $n$, the {\em factorial} of $n$, denoted $n!$, can be defined by
$$n!=\prod_{r=1}^n r$$
where for $n=0$ the empty product is taken to be $1$.

Alternatively, the factorial can be defined recursively by $0!=1$ and $n!=n(n-1)!$ for $n>0$.

$n!$ is equal to the number of permutations of $n$ distinct objects.
For example, there are $5!$ ways to arrange the five letters A, B, C, D and E into a word.

For every non-negative integer $n$ we have
$$\Gamma(n+1) = n!$$
where $\Gamma$ is Euler's gamma function.
In this way the notion of factorial can be generalized to all \PMlinkname{complex}{Complex} values except the negative integers.
\end{document}
