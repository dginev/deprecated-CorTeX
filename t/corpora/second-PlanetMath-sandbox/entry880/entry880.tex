\documentclass{article}
\usepackage{ids}
\usepackage{planetmath-specials}
\usepackage{pmath}
\usepackage{amsmath}
\usepackage{amsfonts}
\usepackage{amssymb}
\newcommand{\reals}{\mathbb{R}}
\newcommand{\natnums}{\mathbb{N}}
\newcommand{\cnums}{\mathbb{C}}
\newcommand{\znums}{\mathbb{Z}}
\newcommand{\lp}{\left(}
\newcommand{\rp}{\right)}
\newcommand{\lb}{\left[}
\newcommand{\rb}{\right]}
\newcommand{\supth}{^{\text{th}}}
\newtheorem{proposition}{Proposition}
\newtheorem{definition}[proposition]{Definition}
\newcommand{\nl}[1]{\PMlinkescapetext{{#1}}}
\newtheorem{theorem}[proposition]{Theorem}
\newcommand{\bu}{\mathbf{u}}
\newcommand{\bv}{\mathbf{v}}
\newcommand{\pnorm}{\operatorname{p}}
\newcommand{\snorm}[1]{\pnorm(#1)}
\begin{document}
Let $V$ be a real, or a complex vector space, with $K$ denoting the
corresponding field of scalars. A \emph{seminorm}  is a function
$$\operatorname{p}:V\to\reals^+,$$
from $V$ to the set of non-negative real numbers, that satisfies the
following two properties.
\begin{align*}
\snorm{k\, \bu} &= \vert k\vert \snorm{\bu},\quad k\in K,\; \bu\in V
& \text{\bf 
Homogeneity}\\
\snorm{\bu+\bv} &\leq \snorm{\bu}+\snorm{\bv},\quad \bu,\bv\in V,
&\text{\bf Sublinearity} 
\end{align*}
A seminorm differs from a norm in that it is permitted that $\snorm{\bu}=0$
for some non-zero $\bu\in V.$

It is possible to characterize the seminorms properties
geometrically. For $k>0$, let
$$B_k = \{ \bu \in V: \snorm{\bu}\leq k\}$$ denote
the ball of radius $k$. The homogeneity property is equivalent to the
assertion that
$$B_k = k B_1,$$
in the sense that
$\bu \in B_1$ if and only if
$k\bu\in B_k.$
Thus, we see that a seminorm is fully determined by its unit ball.
Indeed, given $B\subset V$ we may define a function
$\pnorm_B:V\to \reals^+$
by
$$\pnorm_B(\bu) = \inf\{ \lambda\in\reals^+ : \lambda^{-1}\bu\in B\}.$$
The geometric nature of the unit ball is described by the following.
\begin{proposition}
The function $\pnorm_B$ satisfies the homegeneity property if and only if
for every $\bu\in
V$, there exists a $k\in\reals^+\cup \{\infty\}$ such that
$$\lambda\,\bu \in B\quad\text{if and only if}\quad \Vert \lambda\Vert
\leq k.$$

\end{proposition}
\begin{proposition}
  Suppose that $\pnorm$ is homogeneous. Then, it is sublinear if and
  only if its unit ball, $B_1$, is a convex subset of $V$.
\end{proposition}
\emph{Proof.} First, let us suppose that the seminorm is both sublinear
and homogeneous, and prove that $B_1$ is necessarily convex. Let
$\bu,\bv\in B_1$, and let $k$ be a real number between $0$ and $1$.
We must show that the weighted average $k\bu+(1-k)\bv$ is in $B_1$ as
well. By assumption,
$$\snorm{k\,\bu +(1-k)\bv} \leq k\snorm{\bu} +(1-k)\snorm{\bv}.$$
The right side is a
weighted average of two numbers between $0$ and $1$, and is therefore
between $0$ and $1$ itself. Therefore
$$k\,\bu+(1-k)\bv \in B_1,$$
as desired.

Conversely, suppose that the seminorm function is homogeneous, and
that the unit ball is convex. Let $\bu,\bv\in
V$ be given, and let us show that
$$\snorm{\bu+\bv}\leq \snorm{\bu}+\snorm{\bv}.$$
The essential complication here is that we do not exclude the
possibility that $\snorm{\bu}=0$, but that $\bu\neq 0$.
First, let us consider the case where
$$\snorm{\bu}=\snorm{\bv}=0.$$
By homogeneity, for every $k>0$ we have
$$k\bu,\,k\bv\in B_1,$$
and hence
$$\frac{k}{2}\, \bu + \frac{k}{2}\,\bv\in B_1, $$
as well. By homogeneity, again,
$$\snorm{\bu+\bv}\leq \frac{2}{k}.$$
Since the above is true for all positive $k$, we infer that
$$\snorm{\bu+\bv} = 0,$$
as desired.

Next suppose that $\snorm{\bu}=0$, but that $\snorm{\bv}\neq 0$. We will show
that in this case, necessarily,
$$\snorm{\bu+\bv} = \snorm{\bv}.$$
Owing to the homogeneity assumption, we may without loss of generality
assume that
$$\snorm{\bv}=1.$$
For every $k$ such that $0\leq k<1$ we have
$$ k\,\bu+k\,\bv = (1-k) \frac{k\,\bu}{1-k} + k\,\bv.$$
The right-side expression is an element of $B_1$ because
$$\frac{k\,\bu}{1-k},\, \bv \in B_1.$$
Hence
$$k\snorm{\bu+\bv} \leq 1,$$
and since this holds for $k$ arbitrarily close to $1$ we conclude that
$$\snorm{\bu+\bv}\leq \snorm{\bv}.$$
The same argument also shows that
$$\snorm{\bv} = \snorm{-\bu+(\bu+\bv)} \leq \snorm{\bu+\bv},$$
and hence
$$\snorm{\bu+\bv}=\snorm{\bv},$$
as desired.

Finally, suppose that neither $\snorm{\bu}$ nor $\snorm{\bv}$ is zero. Hence,
$$\frac{\bu}{\snorm{u}},\, \frac{\bv}{\snorm{v}}$$
are both in $B_1$, and hence
$$\frac{\snorm{u}}{\snorm{u}+\snorm{v}} \frac{\bu}{\snorm{u}}+
\frac{\snorm{v}}{\snorm{u}+\snorm{v}} \frac{\bv}{\snorm{v}} =
\frac{\bu+\bv}{\snorm{u}+\snorm{v}} 
$$
is in $B_1$ also. Using homogeneity, we conclude that
$$\snorm{u+v}\leq \snorm{u}+\snorm{v},$$
as desired.
\end{document}
