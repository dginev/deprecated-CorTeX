\documentclass{article}
\usepackage{planetmath-specials}
\usepackage{pmath}
\usepackage{amssymb}
\usepackage{amsmath}
\usepackage{amsfonts}
\usepackage{graphicx}
\usepackage{xypic}
\usepackage{amsthm}

\newtheorem*{defn}{Definition}
\theoremstyle{remark}
\newtheorem*{example}{Example}
\begin{document}
\PMlinkescapeword{calculate}
\PMlinkescapeword{one way}
\PMlinkescapeword{support}
\PMlinkescapeword{link}
\PMlinkescapeword{fixed}
\PMlinkescapeword{represents}
\PMlinkescapeword{satisfy}
\PMlinkescapeword{mean}
\PMlinkescapeword{bases}
\PMlinkescapeword{associates}
\PMlinkescapeword{interpretation}
\PMlinkescapeword{information}
\PMlinkescapeword{pole}
\section*{Vector fields in $\mathbb{R}^n$}

\begin{defn}
A \emph{vector field} $v$ on some open set $U \subset \mathbb{R}^n$ is a function which associates a vector $v(x)$ to each point $x$ of $U$. That is, $v$ is a function from $\mathbb{R}^n$ to $\mathbb{R}^n$.  However, we give it a rather special interpretation.  This distinction will become clear when we generalize.  If $v$ is differentiable, then we say the vector field is \emph{differentiable}.
\end{defn}

Here and throughout this entry we use ``differentiable'' to mean ``infinitely differentiable''.  Other smoothness criteria (continuity, $C^r$, real analytic) can of course be substituted in the obvious way.

\begin{example}
Let $v(x,y)=(-y,x)$.  At the point $(x,y)$, we view this as an arrow pointing counterclockwise around the origin, with a length proportional to (in fact equal to) the distance from the origin.  So this might be the field of velocity vectors of a fluid rotating en masse.
\end{example}

\section*{Vector fields on manifolds}

Suppose first that $M$ is a manifold of dimension $n$ embedded in $\mathbb{R}^m$.  Then we would like to define a vector field on $M$ to be a function as before.  But $M$ has a natural notion of tangent space at each point, so now we would like all the vectors to be tangent to $M$.  If we are to define a vector field as before, as simply a function from some open set $U$ of $M$ to $\mathbb{R}^n$, we must pick a basis for the tangent space at each point; the basis elements must, however, be differentiable.  It is not obvious that we can always pick such a differentiable basis (think of the tangent spaces to the M\"obius strip).  The problem is that the tangent spaces form a fiber bundle, and this may not be trivial.  We could get around this by shrinking $U$ until we could always construct such a basis, but then we would have to describe how to convert bases on the overlap.  This can be done, and it is one way to approach the theory of differential manifolds; at this point one might as well do away with the ambient space $\mathbb{R}^n$.

Instead of doing this, we will take a coordinate-free approach.  The first thing to notice is that given a tangent vector to a manifold, it makes sense to take a differentiable function on the manifold and ask what the directional derivative along the vector is.  If we have two different tangent vectors, then we can find a function whose directional derivative along each vector is different.  So we could identify tangent vectors with directional derivative operators.  This is how we will define them in a general setting.

\begin{defn}
Let $M$ be a differential manifold of dimension $n$, and let $x$ be a point on $M$.  Let $V$ be the space of differentiable functions defined in some neighborhood of $x$.  Then a \emph{tangent vector} $X$ at $x$ is a linear operator on $V$ such that
\begin{enumerate}
\item $X(fg) = fX(g)+gX(f)$, and
\item if $f$ and $g$ agree on some neighborhood of $x$, then $X(f)=X(g)$.
\end{enumerate}
We write the set of tangent vectors at $x$ as $T_xM$; it is a vector space of dimension $n$.

A \emph{vector field} on an open set $U\subset M$ is a family of tangent vectors $X_x$ at each $x$ such that for every differentiable function $f$ on an open set $V\subset U$, the function $x\mapsto X_x(f)$ is differentiable.
\end{defn}

What does this definition actually mean?  Suppose we have a coordinate chart $\phi\colon V\to \mathbb{R}^n$ on some open set $V\subset M$.  Then we can define a differentiable function $x_i$ on $V$ by applying $\phi$ followed by extracting the $i$th coordinate.  So the function $x_i$ really extracts the $x_i$ coordinate in this coordinate system.  Now, let $X$ be a vector field on $V$.  With some thought, we see that
\[
X(f) = \sum_{i=1}^n X(x_i)\frac{\partial}{\partial x_i} f,
\]
or, in some sense
\[
X = \sum_{i=1}^n X(x_i)\frac{\partial}{\partial x_i}.
\]
Each $X(x_i)$ is by definition just a differentiable function on $V$. 

If we had chosen a different coordinate chart $\psi$ on an open set $W\subset M$, we would obtain coordinate functions $y_i$ by analogy with the $x_i$.  Then in this coordinate system we would have
\[
X = \sum_{i=1}^n X(y_i)\frac{\partial}{\partial y_i}.
\]

If $V$ and $W$ overlap, then on their overlap we can compare the components of $X$ in these two coordinate systems.  A calculation will reveal
\[
X = \sum_{i=1}^n X(y_i)\frac{\partial}{\partial y_i}
  = \sum_{i=1}^n \sum_{j=1}^n X(x_j)\frac{\partial y_i}{\partial x_j}\frac{\partial}{\partial y_i}.
\]

An alternative definition of vector fields defines tangent vectors locally and then requires that they satisfy this transformation law.  

Observe that this transformation law means that we can't compare vectors at different points in a coordinate-independent way, or at least, it will require some significant cleverness to transport a vector from one point to another.  This is in fact possible with some extra information in the form of a connection on $M$, allowing parallel transport of vectors along curves.  The result will continue to depend on the curve, as you can imagine if you imagine trying to parallel-transport a vector from the North Pole to Baghdad to Mexico City and then back to the North Pole: it will have rotated.  This is in fact a direct result of the curvature of the Earth. 

\begin{example}
Let $M$ be $\mathbb{R}^2$, and let 
\[
\nu_t(x,y):=\begin{pmatrix}
\cos t & -\sin t \\
\sin t & \cos t
\end{pmatrix}\begin{pmatrix}
x \\
y
\end{pmatrix}.
\]
Then define a vector field $X$ by
\[
f \mapsto \frac{\partial}{\partial t}(f\circ\nu_t).
\]
We have a natural coordinate patch defined by the identity function; in this coordinate system, we can easily calculate that
\[
X = -y\frac{\partial}{\partial x} + x\frac{\partial}{\partial y};
\]
this $X$ is precisely the vector field we had before, viewed in a new and more confusing light.  Now, with a little imagination, we can see that even for fixed $t$, the function $\nu_t$ is a different kind of object than $X$; while $\nu_t$ represents a rotation, a smooth map from $M$ to itself, while $X$ is a piece of information attached in an essential way to each point, perhaps representing a velocity vector field. 
\end{example}

\section*{The Tangent Bundle}

\begin{defn}
Let $TM$ denote the vector bundle with base space $M$ having fiber over $x$ given by $T_xM$, and having local trivializations over coordinate charts with change of coordinates given by the formulas in the previous section.  Then the bunlde $TM$ is called the \emph{tangent bundle} of $M$.
\end{defn}

The tangent bundle is extremely useful in its own right, and other bundles of interest are also constructed from it.  For example, the cotangent bundle is obtained by taking the dual vector space at each point.  Sections of the cotangent bundle are one-forms, and since they are obtained by taking the dual, they transform according to the inverse matrix at each point.  Higher wedge and tensor products of these bundles are used to construct tensors and differential forms.

Vector fields on a smooth manifold support an operation called the Lie bracket, making them into a Lie algebra; this construction produces an intimate link between Lie algebras and Lie groups, which are of great interest to physicists and mathematicians alike.

This viewpoint on vector fields emphasizes the machinery of modern geometry, namely sheaves, local rings, and bundles; this machinery is useful in differential geometry, important in complex analtyic geometry, and foundational in algebraic geometry --- schemes cannot be described without it.

\section*{References}

See the bibliography for differential geometry.
\end{document}
