\documentclass{article}
\usepackage{ids}
\usepackage{planetmath-specials}
\usepackage{pmath}
\usepackage{amssymb}
\usepackage{amsmath}
\usepackage{amsfonts}
\usepackage{graphicx}
\usepackage{xypic}
\begin{document}
Let $R$ be a ring.

The abelian group which contains only an identity element (zero) 
gains a trivial $R$-module structure, 
which we call the {\PMlinkescapetext {\it zero module}}.

Every $R$-module $M$ has an zero element 
and thus a submodule consisting of that element.
This is called the {\it zero submodule} of $M$.
\end{document}
