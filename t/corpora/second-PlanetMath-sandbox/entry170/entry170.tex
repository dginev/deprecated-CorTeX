\documentclass{article}
\usepackage{planetmath-specials}
\usepackage{pmath}
\usepackage{amssymb}
\usepackage{amsmath}
\usepackage{amsfonts}
\usepackage{graphicx}
\usepackage{xypic}
\begin{document}
Let $R$ be a ring with identity. A proper left (right, two-sided) ideal $\mathfrak{m} \subsetneq R$ is said to be {\em maximal} if $\mathfrak{m}$ is not a proper subset of any other proper left (right, two-sided) ideal of $R$.

One can prove:
\begin{itemize}
\item A left ideal $\mathfrak{m}$ is maximal if and only if $R/\mathfrak{m}$ is a simple left $R$-module.
\item A right ideal $\mathfrak{m}$ is maximal if and only if $R/\mathfrak{m}$ is a simple right $R$-module.

\item A two-sided ideal $\mathfrak{m}$ is maximal if and only if $R/\mathfrak{m}$ is a simple ring.
\end{itemize}

All maximal ideals are prime ideals. If $R$ is commutative, an ideal $\mathfrak{m} \subset R$ is maximal if and only if the quotient ring $R/\mathfrak{m}$ is a field.
\end{document}
