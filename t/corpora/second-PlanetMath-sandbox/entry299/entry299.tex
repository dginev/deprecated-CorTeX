\documentclass{article}
\usepackage{planetmath-specials}
\usepackage{pmath}
\usepackage{amssymb}
\usepackage{amsmath}
\usepackage{amsfonts}
\usepackage{amsthm}

\theoremstyle{definition}
\newtheorem*{ex*}{Example}
\begin{document}
The square root of a nonnegative real number $x$, written as $\sqrt{x}$, is the unique nonnegative real number $y$ such that $y^2=x$.  Thus, $(\sqrt{x})^2 \equiv x$.  Or, $\sqrt{x}\times\sqrt{x}\equiv x$.

\begin{ex*}
$\sqrt{9}=3$ because $3\geq 0$ and $3^2=3\times 3=9$.
\end{ex*}

\begin{ex*}
$\sqrt{x^2+2x+1}=\vert x+1\vert$ (see absolute value and even-even-odd rule) because $(x+1)^2=(x+1)(x+1)=x^2+x+x+1=x^2+2x+1$.
\end{ex*}

In some situations it is better to allow two values for $\sqrt{x}$.  For example, $\sqrt{4}=\pm 2$ because $2^2=4$ and $(-2)^2=4$.

Over nonnegative real numbers, the square root operation is left distributive over multiplication and division, but not over addition or subtraction.  That is, if $x$ and $y$ are nonnegative real numbers, then $\sqrt{x\times y}=\sqrt{x}\times\sqrt{y}$ and $\displaystyle \sqrt{\frac{x}{y}}=\frac{\sqrt{x}}{\sqrt{y}}$.

\begin{ex*}
$\sqrt{x^{2}y^{2}}=xy$ because $(xy)^2=xy\times xy=x\times x\times y\times y=x^2\times y^2=x^{2}y^{2}$.
\end{ex*}

\begin{ex*}
$\displaystyle \sqrt{\frac{9}{25}}=\frac{3}{5}$ because $\displaystyle\left(\frac{3}{5}\right)^2=\frac{3^2}{5^2}=\frac{9}{25}$.
\end{ex*}

On the other hand, in general, $\sqrt{x+y}\neq\sqrt{x}+\sqrt{y}$ and $\sqrt{x-y}\neq\sqrt{x}-\sqrt{y}$.  This error is an instance of the freshman's dream error.

The square root notation is actually an alternative to exponentiation.  That is, $\sqrt{x} \equiv x^\frac{1}{2}$.  When it is defined, the square root operation is commutative with exponentiation.  That is, $\sqrt{x^a}=x^\frac{a}{2}=(\sqrt{x})^a$ whenever both $x^a>0$ and $x>0$.  The restrictions can be lifted if we extend the domain and codomain of the square root function to the complex numbers.

Negative real numbers do not have real square roots.  For example, $\sqrt{-4}$ is not a real number.  This fact can be \PMlinkname{proven by contradiction}{ProofByContradiction} as follows:  Suppose $\sqrt{-4}=x\in\mathbb{R}$.  If $x$ is negative, then $x^2$ is positive, and if $x$ is positive, then $x^2$ is also positive.  Therefore, $x$ cannot be positive or negative.  Moreover, $x$ cannot be zero either, because $0^2=0$.  Hence, $\sqrt{-4}\notin\mathbb{R}$.

For additional discussion of the square root and negative numbers, see the discussion of complex numbers.

The square root function generally maps rational numbers to algebraic numbers; $\sqrt{x}$ is rational if and only if $x$ is a rational number which, after cancelling, is a fraction of two \PMlinkescapetext{perfect} squares.  In particular, $\sqrt{2}$ is irrational. 

The function is continuous for all nonnegative $x$, and differentiable for all positive $x$ (it is not differentiable for $x=0$). Its derivative is given by:
\[
\frac{d}{dx}\big(\sqrt{x}\big)=\frac{1}{2\sqrt{x}}
\]

It is possible to consider square roots in rings other than the integers or the rationals. For any ring $R$, with $x,y\in R$, we say that $y$ is a square root of $x$ if $y^2=x$. 

When working in the ring of \PMlinkname{integers modulo $n$}{MathbbZ_n}, we give a special name to members of the ring that have a square root. We say $x$ is a quadratic residue modulo $n$ if there exists $y$ coprime to $x$ such that $y^2\equiv x\pmod{n}$. Rabin's cryptosystem is based on the difficulty of finding square roots modulo an integer $n$.
\end{document}
