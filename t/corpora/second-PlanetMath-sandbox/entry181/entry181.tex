\documentclass{article}
\usepackage{planetmath-specials}
\usepackage{pmath}

\begin{document}
\PMlinkescapeword{order}

If $f$ is a real-valued continuous function on the interval $[a,b]$,
and $x_1$ and $x_2$ are points with $a\le x_1<x_2\le b$
such that $f(x_1)\ne f(x_2)$,
then for every $y$ strictly between $f(x_1)$ and $f(x_2)$
there is a $c\in(x_1,x_2)$ such that $f(c)=y$.

Bolzano's theorem is a special case of this.

The theorem can be generalized as follows:
If $f$ is a real-valued continuous function
on a connected topological space $X$,
and $x_1, x_2 \in X$ with $f(x_1) \ne f(x_2)$,
then for every $y$ between $f(x_1)$ and $f(x_2)$
there is a $\xi \in X$ such that $f(\xi) = y$.
(However, this ``generalization'' is essentially trivial,
and in order to derive the intermediate value theorem from it
one must first establish the less trivial fact that $[a,b]$ is connnected.)
This result remains true
if the codomain is an arbitrary ordered set with its order topology;
see the entry
\PMlinkname{proof of generalized intermediate value theorem}{ProofOfGeneralizedIntermediateValueTheorem}
for a proof.





\end{document}
