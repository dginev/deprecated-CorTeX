\documentclass{article}
\usepackage{ids}
\usepackage{planetmath-specials}
\usepackage{pmath}
\usepackage{amssymb}
\usepackage{amsmath}
\usepackage{amsfonts}
\usepackage{graphicx}
\usepackage{xypic}
\begin{document}
A \emph{homeomorphism} $f$ of topological spaces is a continuous, bijective map such that $f^{-1}$ is also continuous. We also say that two spaces are \emph{homeomorphic} if such a map exists.

If two topological spaces are homeomorphic, they are topologically equivalent --- using the techniques of topology, there is no way of distinguishing one space from the other.

An \emph{autohomeomorphism} (also known as a \emph{self-homeomorphism}) is a 
homeomorphism from a topological space to itself.
\end{document}
