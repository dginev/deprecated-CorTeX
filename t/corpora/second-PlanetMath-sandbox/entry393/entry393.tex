\documentclass{article}
\usepackage{ids}
\usepackage{planetmath-specials}
\usepackage{pmath}
\usepackage{amssymb}
\usepackage{amsmath}
\usepackage{amsfonts}
\usepackage{graphicx}
\usepackage{xypic}
\begin{document}
Often in a family of algebraic structures, 
those structures are generated by certain elements, 
and subject to certain relations.  
One often refers to functions between structures 
which are said to preserve those relations. 
These functions are typically called {\it homomorphisms}.

This situation is generalized by category theory, where
homomorphisms are regarded abstractly, not necessarily being
maps between sets.  For more information on category theory,
please see its entry.  In this entry, we only mention
category theory in passing and references to category
theory safely may be ignored by the reader not familiar with 
the subject.

An example is the category of groups.  
Suppose that $f:A \to B$ 
is a function between two groups.  
We say that $f$ is a group homomorphism if:

(a) the binary operator is preserved:
$f(a_1 \cdot a_2) = f(a_1) \cdot f(a_2)$ for all $a_1, a_2 \in A$;

(b) the identity element is preserved: 
$f(e_A) = e_B$;

(c) inverses of elements are preserved: 
$f(a^{-1}) = [f(a)]^{-1}$ for all $a \in A$.

One can define similar 
natural concepts of homomorphisms
for other algebraic structures,
giving us ring homomorphisms, 
module homomorphisms,
and a host of others.

We give special names to homomorphisms 
when their functions have interesting properties.

If a homomorphism is an injective function 
(i.e. one-to-one), 
then we say that it is a {\it monomorphism}.
These are typically monic in their category.

If a homomorphism is an surjective function 
(i.e. onto), 
then we say that it is an {\it epimorphism}.
These are typically epic in their category.
Sometimes the term "epimorphism" is used as
a synonym for "epic" --- in cases such as
the term "ring epimorphism", this usage can
lead to ambiguity, so one needs to check the
author's usage.

If a homomorphism has an inverse, 
then we say that it is an {\it isomorphism}.
Often, but not always, this is the same as
requiring that the homomorphism be a bijective function 
(i.e. both one-to-one and onto), 

If the domain of a homomorphism is the same as its codomain
(e.g. a homomorphism $f:A \to A$),
then we say that it is an {\it endomorphism}.
We often denote the collection of endomorphisms on $A$ as ${\rm End}(A)$.

If a homomorphism is both an endomorphism and an isomorphism,
then we say that it is an {\it automorphism}.
We often denote the collection of automorphisms on $A$ as ${\rm Aut}(A)$.
\end{document}
