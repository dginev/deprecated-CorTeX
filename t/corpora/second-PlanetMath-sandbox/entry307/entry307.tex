\documentclass{article}
\usepackage{planetmath-specials}
\usepackage{pmath}
\usepackage{amssymb}
\usepackage{amsmath}
\usepackage{amsfonts}
\usepackage{graphicx}
\usepackage{xypic}
\begin{document}
The goal is to prove the sine law:
\[
\frac{\sin A}{a} = \frac{\sin B}{b}=\frac{\sin C}{c}=\frac{1}{2R}
\]
where the variables are defined by the triangle
\[
\begin{xy}
,(0,0)
;(40,0)**@{-}
;(60,30)**@{-}
;(0,0)**@{-}
,(50,12)*{a}
,(30,18)*{b}
,(20,-3)*{c}
,(9,2)*{A}
,(39,3)*{B}
,(54,25)*{C}
\end{xy}
\]
and where $R$ is the radius of the circumcircle that encloses our triangle.

Let's add a couple of lines and define more variables.  
\[
\begin{xy}
,(0,0)
;(40,0)**@{-}
;(60,30)**@{-}
;(0,0)**@{-}
,(50,12)*{a}
,(30,18)*{b}
,(20,-3)*{c}
,(9,2)*{A}
,(39,3)*{B}
,(54,25)*{C}
,(40,0)
;(60,0)**@{--}
;(60,30)**@{--}
,(50,-3)*{x}
,(63,15)*{y}
\end{xy}
\]

So, we now know that
\[
\sin A = \frac{y}{b}
\]
and, therefore, we need to prove
\[
\frac{\sin B}{b} = \frac{y}{ba}
\]
or
\[
\sin B = \frac{y}{a}
\]

From geometry, we can see that
\[
\sin\left(\pi-B\right) = \frac{y}{a}
\]

So the proof is reduced to proving that
\[
\sin\left(\pi-B\right) = \sin B
\]

This is easily seen as
true after examining the top half of the unit circle.  So, putting all
of our results together, we get
\begin{eqnarray}
\frac{\sin A}{a} & = & \frac{y}{ba}\nonumber\\
\frac{\sin A}{a} & = &
\frac{\sin\left(\pi-B\right)}{b}\nonumber\\
\frac{\sin A}{a} & = & \frac{\sin B}{b}
\end{eqnarray}

The same logic may be followed to show that each of these fractions is
also equal to $\frac{\sin C}{c}$.

For the final step of the proof, we must show that 
\[
2R = \frac{a}{\sin A}
\]

We begin by defining our coordinate system.  For this, it is
convenient to find one side that is not shorter than the others and
label it with length $b$.  (The concept of a ``longest'' side is not
well defined in equilateral and some isoceles triangles, but there is
always at least one side that is not shorter than the others.)  We
then define our coordinate system such that the corners of the
triangle that mark the ends of side $b$ are at the coordinates
$\left(0,0\right)$ and $\left(b,0\right)$.  Our third corner (with
sides labelled alphbetically clockwise) is at the point $\left(c\cos
  A,c\sin A\right)$.  Let the center of our circumcircle be at
$\left(x_0,y_0\right)$.  We now have 
\begin{eqnarray}
x_0^2 + y_0^2 &=& R^2 \label{pointA}\\
\left(b-x_0\right)^2 + y_0^2 &=& R^2 \label{pointC}\\
\left(c\cos A - x_0\right)^2 + \left(c\sin A-y_0\right)^2 &=& R^2
\label{pointB} 
\end{eqnarray}
as each corner of our triangle is, by definition of the circumcircle,
a distance $R$ from the circle's center.

Combining equations (3) and (2), we find
\begin{eqnarray*}
\left(b-x_0\right)^2 + y_0^2 &=& x_0^2 + y_0^2\\
b^2 - 2bx_0 &=& 0\\
\frac{b}{2} &=& x_0
\end{eqnarray*}

Substituting this into equation (2) we find that 
\begin{equation}
y_0^2 = R^2 - \frac{b^2}{4}\label{y_0}
\end{equation}

Combining equations (4) and (5) leaves us with
\begin{eqnarray*}
\left(c\cos A-x_0\right)^2 + \left(c \sin A - y_0\right)^2 &=& x_0^2 +
y_0^2\\
c^2\cos^2A-2x_0c\cos A + c^2\sin^2A-2y_0c\sin A &=& 0\\
c-2x_0\cos A - 2y_0 \sin A &=&0\\
\frac{c-b\cos A}{2\sin A} &=& y_0\\
\frac{\left(c-b\cos A\right)^2}{4\sin^2A} &=& R^2 - \frac{b^2}{4}\\
\left(c-b\cos A\right)^2 + b^2 \sin^2 A &=& 4R^2\sin^2 A\\
c^2 - 2bc\cos A +b^2 &=& 4R^2\sin^2A\\
a^2 &=& 4R^2\sin^2A\\
\frac{a}{\sin A} &=& 2R
\end{eqnarray*}
where we have applied the cosines law in the second to last step.
\end{document}
