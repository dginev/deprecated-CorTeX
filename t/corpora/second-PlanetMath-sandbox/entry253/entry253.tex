\documentclass{article}
\usepackage{planetmath-specials}
\usepackage{pmath}
\usepackage{amssymb}
\usepackage{amsmath}
\usepackage{amsfonts}
\usepackage{graphicx}
\usepackage{xypic}
\begin{document}
A sequence $x_0, x_1, x_2, \dots$ in a metric space $(X,d)$ is a \emph{convergent sequence} if there exists a point $x \in X$ such that, for every real number $\epsilon > 0$, there exists a natural number $N$ such that $d(x,x_n) < \epsilon$ for all $n > N$.

The point $x$, if it exists, is unique, and is called the \emph{limit point} or \emph{limit} of the sequence. One can also say that the sequence $x_0, x_1, x_2, \dots$ \emph{converges} to $x$.

A sequence is said to be \emph{divergent} if it does not converge.
\end{document}
