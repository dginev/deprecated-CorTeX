\documentclass{article}
\usepackage{planetmath-specials}
\usepackage{pmath}
\usepackage{amssymb}
\usepackage{amsmath}
\usepackage{amsfonts}

\begin{document}
An \emph{irrational number} is a real number which cannot be represented as a ratio of two integers.  That is, if $x$ is irrational, then 

$$ x \ne \frac{a}{b} $$

with $a,b \in \mathbb{Z}$ and $b \ne 0$.

\subsubsection*{Examples}
\begin{enumerate}
\item $\sqrt[p]{2}$ is irrational for $p=2,3,\ldots$,
\item $\pi, e$, and $\sqrt[p]{2}$ for $p=2,3,\ldots$,
       are irrational,
\item It is not known whether Euler's constant is rational or irrational.
\end{enumerate}

\subsubsection*{Properties}
\begin{enumerate}
\item It $a$ is a real number and $a^n$ is irrational for some $n=2,3,\ldots$, 
then $a$ is irrational (\PMlinkname{proof}{IfAnIsIrrationalThenAIsIrrational}). 
\item The sum, difference, product, and quotient (when defined) of two numbers,
one rational and another irrational, is irrational. 
(\PMlinkname{proof}{RationalAndIrrational}). 
\end{enumerate}
\end{document}
