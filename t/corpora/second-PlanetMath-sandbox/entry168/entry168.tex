\documentclass{article}
\usepackage{ids}
\usepackage{planetmath-specials}
\usepackage{pmath}
\usepackage{amssymb}
\usepackage{amsmath}
\usepackage{amsfonts}
\usepackage{graphicx}
\usepackage{xypic}
\begin{document}
If\, $a_1,\,a_2,\,\ldots,\,a_n$\, are positive numbers, we define their \emph{harmonic mean} as the inverse number of the arithmetic mean of their inverse numbers:
$$H.M. \;=\; \frac{n}{\frac{1}{a_1}+\frac{1}{a_2}+\cdots+\frac{1}{a_n}}$$
\bigskip

\begin{itemize}

\item It follows easily the estimation
$$H.M. \;<\; na_i \quad (i \;=\; 1,\,2,\,\ldots,\,n).$$

\item If you travel from city $A$ to city $B$ at $x$ miles per hour, and then you travel back at $y$ miles per hour.\, What was the average velocity for the whole trip?\\
The harmonic mean of $x$ and $y$. That is, the average velocity is
$$\frac{2}{\frac{1}{x}+\frac{1}{y}} \;=\; \frac{2xy}{x\!+\!y}.$$

\item If one draws through the intersecting point of the diagonals of a trapezoid a line parallel to the parallel sides of the trapezoid, then the segment of the line inside the trapezoid is equal to the harmonic mean of the parallel sides.

\item In the harmonic series
$$1+\frac{1}{2}+\frac{1}{3}+\frac{1}{4}+...$$
every \PMlinkescapetext{term equals to the harmonic mean of the term preceding it and the term} following it.

\end{itemize}
\end{document}
