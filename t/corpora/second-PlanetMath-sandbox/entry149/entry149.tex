\documentclass{article}
\usepackage{ids}
\usepackage{planetmath-specials}
\usepackage{pmath}
\usepackage{amssymb}
\usepackage{amsmath}
\usepackage{amsfonts}
\usepackage{graphicx}
\usepackage{xypic}
\begin{document}
For an indeterminate $q$ and integers $n \ge m \ge 0$
we define the following: \par
(a) $(m)_q = q^{m-1} + q^{m-2} + \cdots + 1$ for $m>0$, \par
(b) $(m!)_q = (m)_q (m-1)_q \cdots (1)_q$ for $m>0$, and $(0!)_q = 1$, \par
(c) ${n \choose m}_q = \frac{(n!)_q}{(m!)_q ((n-m)!)_q}$.
If $m>n$ then we define ${n \choose m}_q=0$. \par
The expressions ${n \choose m}_q$ are called
{\it $q$-binomial coefficients} or {\it Gaussian polynomials}.\par
Note: if we replace $q$ with 1, then we obtain the familiar integers, factorials, and binomial coefficients.  Specifically,\par
(a) $(m)_1 = m$, \par
(b) $(m!)_1 = m!$, \par
(c) ${n \choose m}_1 = {n \choose m}$.\par
(d) ${m \choose m}_q=1$.
\end{document}
