\documentclass{article}
\usepackage{planetmath-specials}
\usepackage{pmath}
\usepackage{amssymb}
\usepackage{amsmath}
\usepackage{amsfonts}
\usepackage{graphicx}
\usepackage{xypic}
\begin{document}
If\, $a_1,\,a_2,\,\ldots,\,a_n$\, are real numbers, their \emph{arithmetic mean} is defined as
$$A.M. \;=\; \frac{a_1+a_2+\ldots+a_n}{n}.$$

The arithmetic mean is what is commonly called the \emph{average} of the numbers.\, The value of $A.M.$ is always between the \PMlinkname{least and the greatest of the numbers}{MinimalAndMaximalNumber} $a_j$.\, If the numbers $a_j$ are all positive, then\, $A.M. \,>\, \frac{a_j}{n}$\, for all $j$.

A generalization of this  concept is that of \emph{weighted mean}, also known as
\emph {weighted average}.\, Let $w_1, \ldots, w_n$ be  numbers whose sum is not zero,
which will be known as \emph{weights}.  (Typically, these will be strictly 
positive numbers, so their sum will automatically differ from zero.)  Then the
weighted mean of $a_1,a_2,\ldots,a_n$ is defined to be
$$W.M. \;=\; 
\frac{w_1 a_1 + w_2 a_2 + \ldots + w_n a_n}{w_1\!+\!w_2\!+\!\ldots+\!w_n}.$$
In the special case where all the weights are equal to each other, the weighted mean equals the arithmetic mean.
\end{document}
