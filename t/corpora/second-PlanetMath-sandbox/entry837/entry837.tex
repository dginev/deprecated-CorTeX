\documentclass{article}
\usepackage{planetmath-specials}
\usepackage{pmath}
%\usepackage{graphicx}
%\usepackage{xypic} 
\usepackage{bbm}
\newcommand{\Z}{\mathbbmss{Z}}
\newcommand{\C}{\mathbbmss{C}}
\newcommand{\R}{\mathbbmss{R}}
\newcommand{\Q}{\mathbbmss{Q}}
\newcommand{\mathbb}[1]{\mathbbmss{#1}}
\begin{document}
The \emph{Chebyshev polynomials of first kind} are defined by the simple 
formula $$T_n(x)=\cos(nt),$$ where $x=\cos t$.

It is an example of a \emph{trigonometric polynomial}.

This can be seen to be a polynomial by expressing $\cos(kt)$ as a polynomial of $\cos(t)$, by using the formula for cosine of angle-sum:

\begin{eqnarray*}
\cos(1t)&=&\cos(t)\\
\cos(2t)&=&\cos(t)\cos(t) - \sin(t)\sin(t) = 2(\cos(t))^2-1\\
\cos(3t)&=&4(\cos(t))^3-3\cos(t)\\
&\vdots&
\end{eqnarray*}

So we have
\begin{eqnarray*}
T_0(x)&=&1\\
T_1(x)&=&x\\
T_2(x)&=&2x^2-1\\
T_3(x)&=&4x^3-3x\\
&\vdots&
\end{eqnarray*}

These polynomials obey the recurrence relation:
$$T_{n+1}(x) \;=\; 2xT_n(x)-T_{n-1}(x)$$
for $n = 1,\,2,\,\ldots$

Related are the \emph{Chebyshev polynomials of the second kind} that are
defined as $$U_{n-1}(\cos t) = \frac{\sin(n t)}{\sin (t)},$$ which
can similarly be seen to be polynomials through either a similar process as the
above or by the relation $U_{n-1}(t) = n T_n'(t)$.

The first few are:
\begin{eqnarray*}
U_0(x)&=&1\\
U_1(x)&=&2x\\
U_2(x)&=&4x^2-1\\
U_3(x)&=&8x^3-4x\\
&\vdots&
\end{eqnarray*}

The same recurrence relation also holds for $U$:
$$U_{n+1}(x) \;=\; 2xU_n(x)-U_{n-1}(x)$$
for $n = 1,\,2,\,\ldots$.
\end{document}
