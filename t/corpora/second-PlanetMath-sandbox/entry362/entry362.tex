\documentclass{article}
\usepackage{planetmath-specials}
\usepackage{pmath}
\usepackage{amssymb}
\usepackage{amsmath}
\usepackage{amsfonts}
\usepackage{graphicx}
\usepackage{xypic}
\begin{document}
Call a continuous map $f:S^m \to S^n$ antipode preserving if $f(-x)=-f(x)$ for all $x \in S^{m}$. 

\textbf{Theorem}: There exists no continuous map $f:S^{n} \to S^{n-1}$ which is antipode preserving for $n>0$.

Some interesting consequences of this theorem have real-world applications. For example, this theorem implies that at any time there exists antipodal points on the surface of the earth which have exactly the same barometric pressure and temperature.

It is also interesting to note a corollary to this theorem which states that no subset of $\mathbb{R}^{n}$ is homeomorphic to $S^{n}$.
\end{document}
