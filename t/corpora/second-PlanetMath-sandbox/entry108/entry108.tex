\documentclass{article}
\usepackage{planetmath-specials}
\usepackage{pmath}
\usepackage{amssymb}
\usepackage{amsmath}
\usepackage{amsfonts}
\usepackage{graphicx}
\usepackage{xypic}
\begin{document}
Suppose that for any coherent proposition $P(x)$, we can construct a set $\{x: P(x)\}$.
\medskip
Let $S = \{x: x \not\in x\}$.  Suppose $S \in S$; then, by definition, $S \not\in S$.  Likewise, if $S \not\in S$, then by definition $S \in S$.  Therefore, we have a contradiction.
\medskip
Bertrand Russell gave this paradox as an example of how a purely intuitive set theory can be inconsistent.  The regularity axiom, one of the Zermelo-Fraenkel axioms, was devised to avoid this paradox by prohibiting self-swallowing sets.\\\\
An interpretation of Russell paradox without any formal language of set theory could be stated like ``If the barber shaves all those who do not themselves shave, does he shave himself?''.  If you answer himself that is false since he only shaves all those who do not themselves shave.  If you answer someone else that is also false because he shaves all those who do not themselves shave and in this case he is part of that set since he does not shave himself.  Therefore we have a contradiction.

\textbf{Remark}.  Russell's paradox is the result of an axiom (due to Frege) in set theory, now obsolete, known as the \emph{axiom of (unrestricted) comprehension}, which states: if $\phi$ is a predicate in the language of set theory, then there is a set that contains exactly those elements $x$ such that $\phi(x)$.  In other words, $\lbrace x\mid \phi(x)\rbrace$ is a set.  So if we take $\phi(x)$ to be $x\notin x$, we arrive at Russell's paradox.
\end{document}
