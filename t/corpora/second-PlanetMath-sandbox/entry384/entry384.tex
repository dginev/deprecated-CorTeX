\documentclass{article}
\usepackage{planetmath-specials}
\usepackage{pmath}
\usepackage{amssymb}
\usepackage{amsmath}
\usepackage{amsfonts}
\usepackage{graphicx}
\usepackage{xypic}
\begin{document}
Let $R$ be a ring.  
An ideal $I$ of $R$ is a {\it semiprime ideal} 
if it satisfies the following equivalent conditions:

(a) $I$ can be expressed as an intersection of prime ideals of $R$;

(b) if $x \in R$, and $xRx \subset I$, then $x \in I$;

(c) if $J$ is a two-sided ideal of $R$ and $J^2 \subset I$, then $J \subset I$ as well;

(d) if $J$ is a left ideal of $R$ and $J^2 \subset I$, then $J \subset I$ as well;

(e) if $J$ is a right ideal of $R$ and $J^2 \subset I$, then $J \subset I$ as well.

Here $J^2$ is the product of ideals $J \cdot J$.

The ring $R$ itself satisfies all of these conditions (including being expressed as an intersection of an empty family of prime ideals) and is thus semiprime.

A ring $R$ is said to be a {\it semiprime ring} if its zero ideal is a semiprime ideal.

Note that an ideal $I$ of $R$ is semiprime if and only if the quotient ring $R/I$ is a semiprime ring.
\end{document}
