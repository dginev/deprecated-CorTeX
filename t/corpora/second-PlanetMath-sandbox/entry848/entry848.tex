\documentclass{article}
\usepackage{ids}
\usepackage{planetmath-specials}
\usepackage{pmath}
% this is the default PlanetMath preamble.  as your knowledge
% of TeX increases, you will probably want to edit this, but
% it should be fine as is for beginners.

% almost certainly you want these
\usepackage{amssymb}
\usepackage{amsmath}
\usepackage{amsfonts}

% used for TeXing text within eps files
%\usepackage{psfrag}
% need this for including graphics (\includegraphics)
%\usepackage{graphicx}
% for neatly defining theorems and propositions
 \usepackage{amsthm}
% making logically defined graphics
%\usepackage{xypic}

% there are many more packages, add them here as you need them

% define commands here

\theoremstyle{definition}
\newtheorem*{thmplain}{Theorem}
\begin{document}
\begin{thmplain}
The equation
\begin{align}
                a_0x^n+a_1x^{n-1}+\cdots+a_{n-1}x+a_n = 0
\end{align}
with odd degree $n$ and real coefficients $a_i$ ($a_0 \ne 0$) has at least one real root $x$.
\end{thmplain}

{\em Proof.}\, Denote by $f(x)$ the left hand side of (1).\, We can write
                      $$f(x) = a_0x^n[1+g(x)]$$
where\, $\displaystyle g(x) := \frac{a_1}{x}\!+\cdots\!+\!\frac{a_{n-1}}{x^{n-1}}\!+\!\frac{a_n}{x^n}$.\, But we have\, 
$\displaystyle\lim_{|x|\to\infty}g(x) = 0$\, because
$$\lim_{|x|\to\infty}\frac{a_i}{x^i} = 0$$
for all\, $i = 1,\,...,\,n$.\, Thus there exists an\, $M > 0$\, such that 
$$|g(x)| < 1\,\, \mbox{for}\,\, |x| \geqq M.$$
Accordingly\, $1+g(\pm M) > 0$\, and
$$\mbox{sign} f(\pm M) = (\mbox{sign} a_0)(\mbox{sign}(\pm M))^n\cdot 1 = 
  (\mbox{sign} a_0)(\pm 1)$$
since $n$ is odd.\, Therefore the real polynomial function $f$ has opposite signs in the end points of the interval\, $[-M,\,M]$.\, Thus the continuity of $f$ guarantees, according to Bolzano's theorem, at least one zero $x$ of $f$ in that interval.\, So (1) has at least one real root $x$.
\end{document}
