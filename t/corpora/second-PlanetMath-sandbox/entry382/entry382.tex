\documentclass{article}
\usepackage{ids}
\usepackage{planetmath-specials}
\usepackage{pmath}
% this is the default PlanetMath preamble.  as your knowledge
% of TeX increases, you will probably want to edit this, but
% it should be fine as is for beginners.

% almost certainly you want these
\usepackage{amssymb}
\usepackage{amsmath}
\usepackage{amsfonts}

% used for TeXing text within eps files
%\usepackage{psfrag}
% need this for including graphics (\includegraphics)
%\usepackage{graphicx}
% for neatly defining theorems and propositions
%\usepackage{amsthm}
% making logically defined graphics
%\usepackage{xypic}

% there are many more packages, add them here as you need them

% define commands here
\newcommand{\opposite}[1]{{#1}^{\,\mathrm{op}}}
\newcommand{\fm}[1]{{\it #1}}
\begin{document}
Let \fm{R} and \fm{S} be rings.  An \emph{(\fm{R},\fm{S})-bimodule} is
an abelian group \fm{M} which is a left module over \fm{R} and a right
module over \fm{S} such that the \fm{r}(\fm{ms})=(\fm{rm})\fm{s} holds
for each \fm{r} in \fm{R}, \fm{m} in \fm{M}, and \fm{s} in \fm{S}.
Equivalently, \fm{M} is an (\fm{R},\fm{S})-bimodule if it is a left
module over $R\otimes\opposite{S}$ or a right module over
$\opposite{R}\otimes S$.

When \fm{M} is an (\fm{R},\fm{S})-bimodule, we sometimes indicate this
by writing the module as ${}_RM_S$.

If \fm{P} is a subgroup of \fm{M} which is also an
(\fm{R},\fm{S})-bimodule, then \fm{P} is an
\emph{(\fm{R},\fm{S})-subbimodule} of \fm{M}.

\end{document}
