\documentclass{article}
\usepackage{ids}
\usepackage{planetmath-specials}
\usepackage{pmath}
% this is the default PlanetMath preamble.  as your knowledge
% of TeX increases, you will probably want to edit this, but
% it should be fine as is for beginners.

% almost certainly you want these
\usepackage{amssymb}
\usepackage{amsmath}
\usepackage{amsfonts}

% used for TeXing text within eps files
%\usepackage{psfrag}
% need this for including graphics (\includegraphics)
%\usepackage{graphicx}
% for neatly defining theorems and propositions
%\usepackage{amsthm}
% making logically defined graphics
\usepackage{xypic} 
\xyoption{all}

% there are many more packages, add them here as you need them

% define commands here
\begin{document}
A \emph{path} in a graph is a finite sequence of alternating vertices and edges, beginning and ending with a vertex, $v_1e_1v_2e_2v_3\dots e_{n-1}v_n$ such that every consecutive pair of vertices $v_x$ and $v_{x+1}$ are adjacent and $e_x$ is incident with $v_x$ and with $v_{x+1}$.  Typically, the edges may be omitted when writing a path (e.g., $v_1v_2v_3\dots v_n$) since only one edge of a graph may connect two adjacent vertices.  In a multigraph, however, the choice of edge may be significant.  

The length of a path is the number of edges in it.

Consider the following graph:

$$\xymatrix{
A \ar@{-}[r] & B \ar@{-}[d] \\
D \ar@{-}[u] & C \ar@{-}[l] }$$

Paths include (but are certainly not limited to) $ABCD$ (length 3), $ABCDA$ (length 4), and $ABABABABADCBA$ (length 12).  $ABD$ is not a path since $B$ is not adjacent to $D$.

In a digraph, each consecutive pair of vertices must be connected by an edge with the proper orientation; if $e=(u,v)$ is an edge, but $(v,u)$ is not, then $uev$ is a valid path but $veu$ is not.

Consider this digraph:

$$\xymatrix{
G \ar[r] \ar[d] & H \ar[d] \ar[l] \\
J & I \ar[l] }$$

$GHIJ$, $GJ$, and $GHGHGH$ are all valid paths.  $GHJ$ is not a valid path because $H$ and $J$ are not connected.  $GJI$ is not a valid path because the edge connecting $I$ to $J$ has the opposite orientation.
\end{document}
