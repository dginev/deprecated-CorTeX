\documentclass{article}
\usepackage{planetmath-specials}
\usepackage{pmath}
\usepackage{amsmath}
\usepackage{amsfonts}
\usepackage{amssymb}
\newcommand{\reals}{\mathbb{R}}
\begin{document}
Let $I\subset\reals$ be an interval, and let $\gamma:I\to\reals^3$ be
an arclength parameterization of an oriented space curve, assumed to
be regular, and free of points of inflection.  We interpret $\gamma(t)$ as
the trajectory of a particle moving through 3-dimensional space.  Let
$T(t), N(t), B(t)$ denote the corresponding moving trihedron.  The
speed of this particle is given by 
\[
   v(t) = \Vert \gamma'(t) \Vert.
\]

The quantity
\[ \kappa(t) = \frac{\Vert T'(t)\Vert}{v(t)} = 
               \frac{\Vert \gamma'(t)\times \gamma''(t)\Vert}
                     {\Vert \gamma'(t)\Vert^3}
\]
is called the
\emph{curvature} of the space curve.  It is invariant with respect to
reparameterization, and is therefore a measure of an intrinsic property
of the curve, a real number geometrically assigned to the point
$\gamma(t)$. If one parameterizes the curve with respect to the arclength $s$, one gets the more concise relation that
\[
   \kappa(s) = \frac{1\cdot\Vert\gamma''(s)\Vert\cdot\sin\frac{\pi}{2}}{1^3} 
             = \Vert\gamma''(s)\Vert.
\]

Physically, curvature may be conceived as the ratio of the normal
acceleration of a particle to the particle's speed.  This ratio
measures the degree to which the curve deviates from the straight line
at a particular point. Indeed, one can show
that of all the circles passing through $\gamma(t)$ and lying on the
osculating plane, the one of radius $1/\kappa(t)$ serves as the best
approximation to the space curve at the point $\gamma(t)$.

To treat curvature analytically, we take the derivative of the relation
\[ \gamma'(t) = v(t) T(t).\]
This yields the following
decomposition of the acceleration vector:
\[
  \gamma''(t) = v'(t) T(t) + v(t) T'(t)
              = v(t) \left\{ (\log v)'(t)\, T(t) + \kappa(t)\, N(t)\right\}.
\]
Thus, to change speed,
one needs to apply acceleration along the tangent vector; to change
heading the acceleration must be applied along the normal.
\end{document}
