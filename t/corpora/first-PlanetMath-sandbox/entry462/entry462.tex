\documentclass{article}
\usepackage{ids}
\usepackage{planetmath-specials}
\usepackage{pmath}
\usepackage{amssymb}
\usepackage{amsmath}
\usepackage{amsfonts}
\usepackage{graphicx}
\usepackage{xypic}
\usepackage{amsthm}

\newtheorem*{proposition}{Proposition}
\begin{document}
Let $(X,\mathcal{T})$ be a topological space.  A subset $\mathcal{B}$ of 
$\mathcal{T}$ is a \emph{basis} for $\mathcal{T}$ if every member of $\mathcal{T}$ is a union of members of $\mathcal{B}$.

Equivalently, $\mathcal{B}$ is a basis if and only if whenever $U$ is open and $x\in U$ then there is an open set $V \in \mathcal{B}$ such that $x\in V\subseteq U$.

The topology generated by a basis $\mathcal{B}$ consists of exactly the unions of the elements of $\mathcal{B}$.

We also have the following easy characterization: (for a proof, see the attachment)

\begin{proposition}
A collection of subsets $\mathcal{B}$ of $X$ is a basis for some topology on $X$ if and only if each $x \in X$ is in some element $B \in \mathcal{B}$ and whenever $B_1, B_2\in\mathcal{B}$ and $x\in B_1\cap B_2$ then there is $B_3\in\mathcal{B}$ such that $x\in B_3\subseteq B_1\cap B_2$.
\end{proposition}

\subsubsection{Examples}

1. A basis for the usual topology of the real line is given by the set of open intervals since every open set can be expressed as a union of open intervals.  One may choose a smaller set as a basis.  For instance, the set of all open intervals with rational endpoints and the set of all intervals whose length is a power of $1/2$ are also bases.  However, the set of all open intervals of length $1$ is not a basis although it is a subbasis (since any interval of length less than $1$ can be expressed as an intersection of two intervals of length $1$).

2. More generally, the set of open balls forms a basis for the topology on a metric space.

3.  The set of all subsets with one element forms a basis for the discrete topology on any set.
\end{document}
