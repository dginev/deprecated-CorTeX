\documentclass{article}
\usepackage{planetmath-specials}
\usepackage{pmath}
\usepackage{amssymb}
\usepackage{amsmath}
\usepackage{amsfonts}
\usepackage{graphicx}
\usepackage{xypic}
\begin{document}
Given a homogeneous polynomial $F$ of degree $d$ in $n+1$ variables $X_0,\ldots,X_n$ and a point $[x_0:\cdots:x_n]$, we cannot evaluate $F$ at that point, because it has multiple such representations, but since $F(\lambda x_0,\ldots,\lambda x_n) = \lambda^d F(x_0,\ldots,x_n)$ we can say whether any such representation (and hence all) vanish at that point.

A \emph{projective variety} over an algebraically closed field $k$ is a subset of some projective space $\mathbb{P}^n_k$ over $k$ which can be described as the common vanishing locus of finitely many homogeneous polynomials with coefficients in $k$, and which is not the union of two such smaller loci.  Also, a \emph{quasi-projective variety} is an open subset of a projective variety.
\end{document}
