\documentclass{article}
\usepackage{ids}
\usepackage{planetmath-specials}
\usepackage{pmath}
% this is the default PlanetMath preamble.  as your knowledge
% of TeX increases, you will probably want to edit this, but
% it should be fine as is for beginners.

% almost certainly you want these
\usepackage{amssymb}
\usepackage{amsmath}
\usepackage{amsfonts}

% used for TeXing text within eps files
%\usepackage{psfrag}
% need this for including graphics (\includegraphics)
%\usepackage{graphicx}
% for neatly defining theorems and propositions
%\usepackage{amsthm}
% making logically defined graphics
%\usepackage{xypic} 

% there are many more packages, add them here as you need them

% define commands here
\begin{document}
A subset $F$ of a topological space $X$ is {\em reducible} if it can be written as a union $F = F_1 \cup F_2$ of two closed proper subsets $F_1$, $F_2$ of $F$ (closed in the subspace topology).  That is, $F$ is reducible if it can be written as a union $F = (G_1\cap F)\cup(G_2\cap F)$ where $G_1$,$G_2$ are closed subsets of $X$, neither of which contains $F$.

A subset of a topological space is {\em irreducible} (or \emph{hyperconnected}) if it is not reducible.

As an example, consider $\{ (x,y)\in\mathbb{R}^2 : xy = 0 \}$ with the subspace topology from $\mathbb{R}^2$.  This space is a union of two lines $\{ (x,y)\in\mathbb{R}^2 : x = 0 \}$ and $\{ (x,y)\in\mathbb{R}^2 : y = 0 \}$, which are proper closed subsets.  So this space is reducible, and thus not irreducible.
\end{document}
