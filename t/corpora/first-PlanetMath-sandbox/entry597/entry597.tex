\documentclass{article}
\usepackage{planetmath-specials}
\usepackage{pmath}
\usepackage{amssymb}
\usepackage{amsmath}
\usepackage{amsfonts}

%\usepackage{psfrag}
%\usepackage{graphicx}
%\usepackage{xypic}
\begin{document}
The \emph{discrete cosine transforms (DCT)} are a family of \PMlinkescapetext{similar} transforms closely related to the discrete sine transform and the discrete Fourier transform. The DCT-II is the most commonly used form and plays an important role in coding signals and images \cite{Jain89}, e.g. in the widely used standard JPEG compression. The discrete cosine transform was first introduced by Ahmed, Natarajan, and Rao \cite{DCT}. Later Wang and Hunt \cite{DWT} introduced the \PMlinkescapetext{complete} set of variants.

The DCT is included in many mathematical packages, such as Matlab, Mathematica and GNU Octave.

\section{Definition}

The orthonormal variants of the DCT, where $x_n$ is the original vector of $N$ real numbers, $C_k$ is the transformed vector of $N$ real numbers and $\delta$ is the Kronecker delta, are defined by the following equations:

\subsection{DCT-I}
\begin{eqnarray*}
C^I_k&=&p_k \sum _{n=0}^{N-1} x_n q_n \cos \frac{\pi n k}{N-1} \quad \quad k=0, 1, 2, \dots, N-1\\
p_k&=&\sqrt{\frac{2-\delta _{k,0}-\delta _{k,N-1}}{N-1}}\\
q_n&=&\sqrt{\frac{1}{1+\delta _{n,0}+\delta _{n,N-1}}}
\end{eqnarray*}
The DCT-I is its own inverse.

\subsection{DCT-II}
\begin{eqnarray*}
C^{II}_k&=&p_k \sum _{n=0}^{N-1} x_n \cos \frac{\pi \left(n+\frac{1}{2}\right) k}{N} \quad \quad k=0, 1, 2, \dots, N-1\\
p_k&=&\sqrt{\frac{2-\delta _{k,0}}{N}}
\end{eqnarray*}
The inverse of DCT-II is DCT-III.

\subsection{DCT-III}
\begin{eqnarray*}
C^{III}_k&=&p \sum _{n=0}^{N-1} x_n q_n \cos \frac{\pi n \left(k+\frac{1}{2}\right)}{N} \quad \quad k=0, 1, 2, \dots, N-1\\
p&=&\sqrt{\frac{2}{N}}\\
q_n&=&\sqrt{\frac{1}{1+\delta _{n,0}}}
\end{eqnarray*}
The inverse of DCT-III is DCT-II.

\subsection{DCT-IV}
\begin{eqnarray*}
C^{IV}_k&=&p \sum _{n=0}^{N-1} x_n \cos \frac{\pi \left(n+\frac{1}{2}\right) \left(k+\frac{1}{2}\right)}{N} \quad \quad k=0, 1, 2, \dots, N-1\\
p&=&\sqrt{\frac{2}{N}}
\end{eqnarray*}
The DCT-IV is its own inverse.

\subsection{DCT-V}
\begin{eqnarray*}
C^V_k&=&p_k \sum _{n=0}^{N-1} x_n q_n \cos \frac{\pi n k}{N-\frac{1}{2}} \quad \quad k=0, 1, 2, \dots, N-1\\
p_k&=&\sqrt{\frac{2-\delta _{k,0}}{N-\frac{1}{2}}}\\
q_n&=&\sqrt{\frac{1}{1+\delta _{n,0}}}
\end{eqnarray*}
The DCT-V is its own inverse.

\subsection{DCT-VI}
\begin{eqnarray*}
C^{VI}_k&=&p_k \sum _{n=0}^{N-1} x_n q_n \cos \frac{\pi \left(n+\frac{1}{2}\right) k}{N-\frac{1}{2}} \quad \quad k=0, 1, 2, \dots, N-1\\
p_k&=&\sqrt{\frac{2-\delta _{k,0}}{N-\frac{1}{2}}}\\
q_n&=&\sqrt{\frac{1}{1+\delta _{n,N-1}}}
\end{eqnarray*}
The inverse of DCT-VI is DCT-VII.

\subsection{DCT-VII}
\begin{eqnarray*}
C^{VII}_k&=&p_k \sum _{n=0}^{N-1} x_n q_n \cos \frac{\pi n \left(k+\frac{1}{2}\right)}{N-\frac{1}{2}} \quad \quad k=0, 1, 2, \dots, N-1\\
p_k&=&\sqrt{\frac{2-\delta _{k,N-1}}{N-\frac{1}{2}}}\\
q_n&=&\sqrt{\frac{1}{1+\delta _{n,0}}}
\end{eqnarray*}
The inverse of DCT-VII is DCT-VI.

\subsection{DCT-VIII}
\begin{eqnarray*}
C^{VII}_k&=&p \sum _{n=0}^{N-1} x_n \cos \frac{\pi \left(n+\frac{1}{2}\right) \left(k+\frac{1}{2}\right)}{N+\frac{1}{2}} \quad \quad k=0, 1, 2, \dots, N-1\\
p&=&\sqrt{\frac{2}{N+\frac{1}{2}}}
\end{eqnarray*}
The DCT-VIII is its own inverse.

\section{Two-dimensional DCT}

The DCT in two dimensions is simply the one-dimensional transform computed in each row and each column. For example, the DCT-II of a $N_1\times N_2$ matrix is given by

\begin{eqnarray*}
C^{II}_{k_1,k_2}&=&p_{k_1}p_{k_2}\sum _{n_1=0}^{N_1-1}\sum _{n_2=0}^{N_2-1} x_{n_1,n_2} \cos \frac{\pi\left( n_1+\frac{1}{2}\right) k_1}{N_1}  \cos \frac{\pi\left( n_2+\frac{1}{2}\right) k_2}{N_2}
\end{eqnarray*}

\begin{thebibliography}{3}

\bibitem{DAB} This entry is based on content from The Data Analysis Briefbook
(\PMlinkexternal{http://rkb.home.cern.ch/rkb/titleA.html}{http://rkb.home.cern.ch/rkb/titleA.html})

\bibitem{Jain89} A.K. Jain, Fundamentals of Digital Image Processing, Prentice Hall, 1989.

\bibitem{Shao07} Xuancheng Shao, Steven G. Johnson. Type-II/III DCT/DST algorithms with reduced number of arithmetic operations. 2007.

\bibitem{Pau06} Markus P\"auschel, Jos\'e M. F. Mouray. The algebraic approach to the discrete cosine and
sine transforms and their fast algorithms. 2006.

\bibitem{DCT} N. Ahmed, T. Natarajan, and K. R. Rao. Discrete Cosine Transform, IEEE Trans. on
Computers, C-23. 1974.

\bibitem{DWT} Z. Wang and B. Hunt, The Discrete W Transform, Applied Mathematics and Computation,
16. 1985.

\end{thebibliography}
\end{document}
