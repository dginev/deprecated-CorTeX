\documentclass{article}
\usepackage{planetmath-specials}
\usepackage{pmath}
% this is the default PlanetMath preamble.  as your knowledge
% of TeX increases, you will probably want to edit this, but
% it should be fine as is for beginners.

% almost certainly you want these
\usepackage{amssymb}
\usepackage{amsmath}
\usepackage{amsfonts}

% used for TeXing text within eps files
%\usepackage{psfrag}
% need this for including graphics (\includegraphics)
%\usepackage{graphicx}
% for neatly defining theorems and propositions
%\usepackage{amsthm}
% making logically defined graphics
%\usepackage{xypic} 

% there are many more packages, add them here as you need them

% define commands here
\begin{document}
We give here an algorithm for reducing a symmetric polynomial into a polynomial in the elementary symmetric polynomials.

We define the {\em height} of a monomial $x_1^{e_1} \cdots x_n^{e_n}$ in $R[x_1,\dots,x_n]$ to be $e_1 + 2 e_2 + \cdots + n e_n$. The height of a polynomial is defined to be the maximum height of any of its monomial terms, or 0 if it is the zero polynomial.

Let $f$ be a symmetric polynomial. We reduce $f$ into elementary symmetric polynomials by induction on the height of $f$. Let $c x_1^{e_1} \cdots x_n^{e_n}$ be the monomial term of maximal height in $f$. Consider the polynomial
$$
g := f - c s_1^{e_n - e_{n-1}} s_2^{e_{n-1} - e_{n-2}} \cdots s_{n-1}^{e_2 - e_1} s_n^{e_1}
$$
where $s_k$ is the $k$--th elementary symmetric polynomial in the $n$ variables $x_1, \dots, x_n$. Then $g$ is a symmetric polynomial of lower height than $f$, so by the induction hypothesis, $g$ is a polynomial in $s_1, \dots, s_n$, and it follows immediately that $f$ is also a polynomial in $s_1, \dots, s_n$.
\end{document}
