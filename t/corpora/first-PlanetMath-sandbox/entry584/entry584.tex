\documentclass{article}
\usepackage{planetmath-specials}
\usepackage{pmath}
\usepackage{pstricks}
\begin{document}
\PMlinkescapeword{regular}

An \emph{hexagon} is a $6$-sided polygon.  The most commonly quoted hexagon is a \PMlinkname{regular}{RegularPolygon} hexagon, having congruent sides and congruent interior angles.  Below is an example of a regular hexagon:

\begin{center}
\begin{pspicture}(0,0)(6,5.2)
\pspolygon(0,2.598)(1.5,0)(4.5,0)(6,2.598)(4.5,5.196)(1.5,5.196)
\psdots(0,2.598)(1.5,0)(4.5,0)(6,2.598)(4.5,5.196)(1.5,5.196)
\end{pspicture}
\end{center}

Below are some properties of regular hexagons in Euclidean geometry:
\begin{itemize}
\item The \PMlinkname{measure}{AngleMeasure} of any interior angle of a regular hexagon is $120^{\circ}$.
\item The smallest $n$ for which a regular $n$-gon has diagonals which are not congruent is $n=6$.  For example, in the regular hexagon below, the diagonal drawn in blue and the one drawn in red are not congruent.
\begin{center}
\begin{pspicture}(0,0)(6,5.2)
\psline[linecolor=blue](1.5,0)(1.5,5.196)
\psline[linecolor=red](4.5,0)(1.5,5.196)
\pspolygon(0,2.598)(1.5,0)(4.5,0)(6,2.598)(4.5,5.196)(1.5,5.196)
\psdots(0,2.598)(1.5,0)(4.5,0)(6,2.598)(4.5,5.196)(1.5,5.196)
\end{pspicture}
\end{center}
\item The side of a regular hexagon has the same length as the radius of the circle circumscribing it.  This fact is illustrated below.
\begin{center}
\begin{pspicture}(0,-0.2)(6,5.2)
\pscircle[linecolor=cyan](3,2.598){3}
\psline[linecolor=cyan](3,2.598)(1.5,5.196)
\pspolygon(0,2.598)(1.5,0)(4.5,0)(6,2.598)(4.5,5.196)(1.5,5.196)
\psdots(3,2.598)(0,2.598)(1.5,0)(4.5,0)(6,2.598)(4.5,5.196)(1.5,5.196)
\end{pspicture}
\end{center}
\end{itemize}

From the last remark, it is easy to see that a regular hexagon is constructible using compass and straightedge.
\end{document}
