\documentclass{article}
\usepackage{planetmath-specials}
\usepackage{pmath}
% almost certainly you want these
\usepackage{amssymb}
\usepackage{amsmath}
\usepackage{amsfonts}

% used for TeXing text within eps files
%\usepackage{psfrag}
% need this for including graphics (\includegraphics)
%\usepackage{graphicx}
% for neatly defining theorems and propositions
%\usepackage{amsthm}
% making logically defined graphics
%\usepackage{xypic} 

% there are many more packages, add them here as you need them

% define commands here
\newcommand{\mv}[1]{\mathbf{#1}}	% matrix or vector
\newcommand{\var}{\mathrm{var}}
\newcommand{\cov}{\mathrm{cov}}
\newcommand{\corr}{\mathrm{corr}}
\newcommand{\mvt}[1]{\mv{#1}^{\mathrm{T}}}
\newcommand{\mvi}[1]{\mv{#1}^{-1}}
\newcommand{\mpderiv}[1]{\frac{\partial}{\partial {#1}}} 
\newcommand{\defined}{:=}
\begin{document}
\PMlinkescapeword{mean}
\PMlinkescapeword{measure}
\PMlinkescapeword{ranges}

The \emph{covariance} of two random variables $X_1$ and $X_2$ with \PMlinkname{mean}{ExpectedValue} $\mu_1$ and $\mu_2$ respectively is defined as

\begin{equation}
\cov(X_1,X_2) \defined E[(X_1 - \mu_1)(X_2 - \mu_2)].
\end{equation}

The covariance of a random variable $X$ with itself is simply the variance, $E[(X - \mu)^2]$.

Covariance captures a measure of the correlation of two variables.  Positive covariance indicates that as $X_1$ increases, so does $X_2$.  Negative covariance indicates $X_1$ decreases as $X_2$ increases and vice versa.  Zero covariance can indicate that $X_1$ and $X_2$ are uncorrelated.

The \emph{correlation coefficient} provides a normalized view of correlation based on covariance:

\begin{equation}
\corr(X,Y) \defined \frac{\cov(X,Y)}{\sqrt{\var(X)\var(Y)}}.
\end{equation}

$\corr(X,Y)$ ranges from -1 (for negatively correlated variables) through zero (for uncorrelated variables) to +1 (for positively correlated variables).

While if $X$ and $Y$ are independent we have $\corr(X,Y)=0$, the latter does not imply the former.
\end{document}
