\documentclass{article}
\usepackage{planetmath-specials}
\usepackage{pmath}
%\usepackage{graphicx}
%\usepackage{xypic} 
\usepackage{bbm}
\newcommand{\Z}{\mathbbmss{Z}}
\newcommand{\C}{\mathbbmss{C}}
\newcommand{\R}{\mathbbmss{R}}
\newcommand{\Q}{\mathbbmss{Q}}
\newcommand{\mathbb}[1]{\mathbbmss{#1}}
\begin{document}
A topological space $(X,\tau)$ is said to be $T_0$
(or to satisfy the $T_0$ axiom )
if for all distinct $x,y\in X$
there exists an open set $U\in\tau$ such that
either $x\in U$ and $y\notin U$ or $x\notin U$ and $y\in U$.

All \PMlinkname{$T_1$ spaces}{T1Space} are $T_0$.
An example of $T_0$ space that is not $T_1$
is the $2$-point Sierpinski space.
\end{document}
