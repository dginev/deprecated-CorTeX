\documentclass{article}
\usepackage{planetmath-specials}
\usepackage{pmath}
\usepackage{amsmath}
\begin{document}
Let $A=(a_{ij})$ and $B=(b_{ij})$ be two $n\times n$ matrices. We define their \emph{\PMlinkescapetext{join}} as the matrix whose $(i,j)$th entry is the pair $(a_{ij},b_{ij})$.

A \emph{Graeco-Latin square} is then the join of two Latin squares.

The name comes from Euler's use of Greek and Latin letters to differentiate the entries on each array.

An example of Graeco-Latin square:
\begin{equation*}
\begin{pmatrix}
a\alpha & b\beta & c\gamma & d\delta\\
d\gamma & c\delta & b\alpha & a\beta\\
b\delta & a\gamma & d\beta &  c\alpha\\
c\beta & d\alpha & a\delta & b\gamma
\end{pmatrix}
\end{equation*}
\end{document}
