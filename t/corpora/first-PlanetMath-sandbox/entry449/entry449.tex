\documentclass{article}
\usepackage{ids}
\usepackage{planetmath-specials}
\usepackage{pmath}
\usepackage{amssymb}
\usepackage{amsmath}
\usepackage{amsfonts}
\usepackage{graphicx}
\usepackage{xypic}
\begin{document}
{\bf Sampling Theorem}

The greyvalues of digitized one- or two-dimensional signals are typically generated by an analogue-to-digital converter (ADC), by sampling a continuous signal at fixed intervals (e.g. in time), and quantizing (digitizing) the samples. The sampling (or point sampling) theorem states that a band-limited analogue signal $x_a(t)$, i.e. a signal in a finite frequency band (e.g. between 0 and BHz), can be completely reconstructed from its samples $x(n) = x(nT)$, if the sampling frequency is greater than $2B$ (the Nyquist rate); expressed in the \PMlinkescapetext{time domain}, this \PMlinkescapetext{means} that the sampling interval $T$ is at most $\frac{1}{2B}$ seconds. Undersampling can produce serious errors (aliasing) by introducing artifacts of low frequencies, both in one-dimensional signals and in digital \PMlinkescapetext{images}.

\begin{center}
\includegraphics[scale=.5]{img988.eps}
\end{center}

{\bf References}
\begin{itemize}
\item Originally from the Data Analysis Briefbook (\PMlinkexternal{http://rkb.home.cern.ch/rkb/titleA.html}{http://rkb.home.cern.ch/rkb/titleA.html})
\end{itemize}
\end{document}
