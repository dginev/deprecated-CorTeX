\documentclass{article}
\usepackage{planetmath-specials}
\usepackage{pmath}
\usepackage{amssymb}
\usepackage{amsmath}
\usepackage{amsfonts}
\usepackage{graphicx}
\usepackage{xypic}
\begin{document}
Let $|G| = N$. We first prove existence, using induction on $N$. If $N=1$ (or, more generally, if $G$ is simple) the result is clear. Now suppose $G$ is not simple. Choose a maximal proper normal subgroup $G_1$ of $G$. Then $G_1$ has a Jordan--H\"older decomposition by induction, which produces a Jordan--H\"older decomposition for $G$.

To prove uniqueness, we use induction on the length $n$ of the decomposition series. If $n=1$ then $G$ is simple and we are done. For $n>1$, suppose that
$$
G \supset G_1 \supset G_2 \supset \cdots \supset G_n = \{1\}
$$
and
$$
G \supset G'_1 \supset G'_2 \supset \cdots \supset G'_m = \{1\}
$$
are two decompositions of $G$. If $G_1 = G'_1$ then we're done (apply the induction hypothesis to $G_1$), so assume $G_1 \neq G'_1$. Set $H := G_1 \cap G'_1$ and choose a decomposition series
$$
H \supset H_1 \supset \cdots \supset H_k = \{1\}
$$
for $H$. By the second isomorphism theorem, $G_1/H = G_1 G'_1/G'_1 = G/G'_1$ (the last equality is because $G_1 G'_1$ is a normal subgroup of $G$ properly containing $G_1$). In particular, $H$ is a normal subgroup of $G_1$ with simple quotient. But then
$$
G_1 \supset G_2 \supset \cdots \supset G_n
$$
and
$$
G_1 \supset H \supset \cdots \supset H_k
$$
are two decomposition series for $G_1$, and hence have the same simple quotients by the induction hypothesis; likewise for the $G'_1$ series. Therefore $n=m$. Moreover, since $G/G_1 = G'_1/H$ and $G/G'_1 = G_1/H$ (by the second isomorphism theorem), we have now accounted for all of the simple quotients, and shown that they are the same.
\end{document}
