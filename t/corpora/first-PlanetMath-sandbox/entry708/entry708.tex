\documentclass{article}
\usepackage{ids}
\usepackage{planetmath-specials}
\usepackage{pmath}
\usepackage{amssymb}
\usepackage{amsmath}
\usepackage{amsfonts}
\begin{document}
\PMlinkescapeword{bound}
\PMlinkescapeword{finite}
\PMlinkescapeword{inverses}
\PMlinkescapeword{closed}
\PMlinkescapeword{contains}
\PMlinkescapeword{generators}

A \emph{finitely generated group} is a group that has a finite generating set.

Every finite group is obviously finitely generated.
Every finitely generated group is countable.

Any \PMlinkname{quotient}{QuotientGroup}
of a finitely generated group is finitely generated.
However, a finitely generated group may have subgroups
that are not finitely generated.
(For example, the free group of rank $2$ is generated by just two elements,
but its commutator subgroup is not finitely generated.)
Nonetheless, a subgroup of finite index in a finitely generated group
is necessarily finitely generated;
a bound on the number of generators required for the subgroup is given by
the \PMlinkname{Schreier index formula}{ScheierIndexFormula}.

The finitely generated groups
all of whose subgroups are also finitely generated
are precisely the groups satisfying the maximal condition.
This includes all finitely generated nilpotent groups and,
more generally, all polycyclic groups.

A group that is not finitely generated
is sometimes said to be \emph{infinitely generated}.

\end{document}
