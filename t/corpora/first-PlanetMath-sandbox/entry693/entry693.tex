\documentclass{article}
\usepackage{planetmath-specials}
\usepackage{pmath}
% this is the default PlanetMath preamble.  as your knowledge
% of TeX increases, you will probably want to edit this, but
% it should be fine as is for beginners.

% almost certainly you want these
\usepackage{amssymb}
\usepackage{amsmath}
\usepackage{amsfonts}

% used for TeXing text within eps files
%\usepackage{psfrag}
% need this for including graphics (\includegraphics)
\usepackage{graphicx}
% for neatly defining theorems and propositions
%\usepackage{amsthm}
% making logically defined graphics
%\usepackage{xypic} 

% there are many more packages, add them here as you need them

% define commands here
\begin{document}
Green's theorem provides a connection between path integrals over a well-connected region in the plane and the area of the region bounded in the plane. Given a closed path $P$ bounding a region $R$ with area $A$, and a vector-valued function $\vec{F}=(f(x,y),g(x,y))$ over the plane,
$$\oint_P\vec{F}\cdot d\vec{x} = \int\!\!\!\int_{\!\!R} [g_1(x,y) - f_2(x,y)] dA$$
where $a_n$ is the derivative of $a$ with respect to the $n$th variable.
\begin{center}
\includegraphics[width=2.694444in]{greensthm}
\end{center}

\paragraph{Corollary:}

The closed path integral over a gradient of a function with continuous partial derivatives is always zero. Thus, gradients are conservative vector fields. The smooth function is called the potential of the vector field.

\paragraph{Proof:}

The corollary states that

$$\oint_P\vec{\nabla}_h\cdot d\vec{x} = 0$$

We can easily prove this using Green's theorem.

$$\oint_P\vec{\nabla}_h\cdot d\vec{x} = \int\!\!\!\int_{\!\!R} [g_1(x,y) - f_2(x,y)] dA$$

But since this is a gradient...

$$\int\!\!\!\int_{\!\!R} [g_1(x,y) - f_2(x,y)] dA = \int\!\!\!\int_{\!\!R} [h_{21}(x,y) - h_{12}(x,y)] dA$$

Since $h_{12}=h_{21}$ for any function with continuous partials, the corollary is proven.
\end{document}
