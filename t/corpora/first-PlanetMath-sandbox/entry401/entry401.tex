\documentclass{article}
\usepackage{planetmath-specials}
\usepackage{pmath}
\usepackage{amssymb}
\usepackage{amsmath}
\usepackage{amsfonts}
\usepackage{graphicx}
\usepackage{xypic}
\begin{document}
\PMlinkescapephrase{states}

Let $(A_n)$ be a sequence of mutually exclusive events whose \PMlinkname{union}{Union} is the sample space and let $E$ be any event.  All of the events have nonzero probability ($P(E) > 0$ and $P(A_n) > 0$ for all $n$).  Bayes' Theorem states

$$ P(A_j|E) = \frac{P(A_j)P(E|A_j)}{\sum_i P(A_i)P(E|A_i)} $$

for any $A_j \in (A_n)$.

A simpler formulation is:

$$ P(A|B) = \frac{P(B|A)P(A)}{P(B)} $$

For two events, $A$ and $B$ (also with nonzero probability).  

\begin{thebibliography}{3}
\bibitem{Milton} Milton, J.S., Arnold, Jesse C., \textsl{Introduction to Probability and Statistics: Principles and Applications for Engineering and the Computing Sciences}, McGraw Hill, 1995.
\end{thebibliography}
\end{document}
