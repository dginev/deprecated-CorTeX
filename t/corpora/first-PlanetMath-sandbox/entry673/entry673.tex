\documentclass{article}
\usepackage{planetmath-specials}
\usepackage{pmath}
\usepackage{amsmath}
\usepackage{amsfonts}
\usepackage{amssymb}
\newcommand{\reals}{\mathbb{R}}
\begin{document}
\paragraph{Kinematic definition.}
A \emph{parameterized space curve} is a parameterized curve taking
values in 3-dimensional Euclidean space.  It may be interpreted as the
trajectory of a particle moving through space.  Analytically, a smooth
space curve is represented by a sufficiently differentiable mapping
$\gamma:I \to \reals^3, $ of an interval $I\subset \reals$ into
3-dimensional Euclidean space $\reals^3$.  Equivalently, a
parameterized space curve can be considered a 3-vector of functions:
$$\gamma(t) =
\begin{pmatrix}
  x(t)\\ y(t) \\ z(t)
\end{pmatrix},\quad t\in I.$$

\paragraph{Regularity hypotheses.}
To preclude the possibility of kinks and corners, it is
necessary to add the hypothesis that the mapping be \PMlinkname{regular}{Curve}, that is
to say that the derivative $\gamma'(t)$ never vanishes.  Also, we say
that $\gamma(t)$ is a point of inflection if the first and second
derivatives $\gamma'(t), \gamma''(t)$ are linearly dependent.  Space curves
with points of inflection are beyond the scope of this entry.
Henceforth we make the assumption that $\gamma(t)$ is both \PMlinkescapetext{regular} and
lacks points of inflection.

\paragraph{Geometric definition.}
A \emph{space curve}, per se, needs to be conceived of as a subset of
$\reals^3$ rather than a mapping.  Formally, we could define a space
curve to be the image of some parameterization $\gamma:I\to\reals^3$.  A
more useful concept, however, is the notion of an \emph{oriented space
  curve}, a space curve with a specified direction of motion.
Formally, an oriented space curve is an equivalence class of
parameterized space curves; with $\gamma_1:I_1\to\reals^3$ and
$\gamma_2:I_2\to\reals^3$ being judged equivalent if there exists a
smooth, monotonically increasing reparameterization function $\sigma:
I_1\to I_2$ such that
$$\gamma_1(t) = \gamma_2(\sigma(t)),\quad t\in I_1.$$

\paragraph{Arclength parameterization.}
We say that $\gamma:I\to\reals^3$ is an arclength parameterization of an
oriented space curve if 
$$\Vert \gamma'(t) \Vert = 1,\quad t\in I.$$
With this hypothesis the
length of the space curve between points $\gamma(t_2)$ and $\gamma(t_1)$ is
just $\vert t_2-t_1 \vert$.  In other words, the parameter in such a
parameterization measures the relative distance along the
curve. 

Starting with an arbitrary parameterization $\gamma: I\to \reals^3$,
one can obtain an arclength parameterization by fixing a $t_0\in I$,
setting
$$\sigma(t) = \int^t_{t_0} \Vert \gamma'(x)\Vert \, dx,$$
and using the
inverse function $\sigma^{-1}$ to reparameterize the curve.  In other
words,
$$\hat{\gamma}(t) = \gamma(\sigma^{-1}(t))$$
is an arclength
parameterization.  Thus, every space curve possesses an arclength
parameterization, unique up to a choice of additive constant in the
arclength parameter.
\end{document}
