\documentclass{article}
\usepackage{planetmath-specials}
\usepackage{pmath}
\usepackage{amssymb}
\usepackage{amsmath}
\usepackage{amsfonts}
\usepackage{graphicx}
\usepackage{xypic}
\begin{document}
Let $U \subset \mathbb{C}$ be a simply connected domain, and suppose $f$ is a complex valued function which is defined and analytic on all but finitely many points $a_1, \dots, a_m$ of $U$. Let $C$ be a closed curve in $U$ which does not intersect any of the $a_i$. Then
$$
\int_C f(z)\ dz = 2 \pi i \sum_{i=1}^m \eta(C,a_i) \operatorname{Res}(f;a_i),
$$
where
$$
\eta(C,a_i) := \frac{1}{2 \pi i} \int_C \frac{dz}{z-a_i}
$$
is the winding number of $C$ about $a_i$, and $\operatorname{Res}(f;a_i)$ denotes the residue of $f$ at $a_i$.

The Cauchy residue theorem generalizes both the Cauchy integral theorem (because analytic functions have no poles) and the Cauchy integral formula (because $f(x)/(x-a)^n$ for analytic $f$ has exactly one pole at $x=a$ with residue $\operatorname{Res}(f(x)/(x-a)^n,a) = f^{(n)}(a)/n!)$.
\end{document}
