\documentclass{article}
\usepackage{planetmath-specials}
\usepackage{pmath}
\usepackage{amssymb}
\usepackage{amsmath}
\usepackage{amsfonts}

\begin{document}
A subset $D$ of a topological space $X$
is said to be \emph{dense} (or \emph{everywhere dense}) in $X$
if the closure of $D$ is equal to $X$.
Equivalently, $D$ is dense if and only if
$D$ intersects every nonempty open set.

In the special case that $X$ is a metric space with metric $d$,
then this can be rephrased as:
for all $\varepsilon > 0$ and all $x\in X$
there is $y\in D$ such that $d(x,y)<\varepsilon$.

For example, both the rationals $\mathbb{Q}$
and the irrationals $\mathbb{R} \setminus \mathbb{Q}$
are dense in the reals $\mathbb{R}$.

The least cardinality of a dense set of a topological space
is called the \emph{density} of the space.
It is conventional to take the density to be $\aleph_0$
if it would otherwise be finite;
with this convention,
the spaces of density $\aleph_0$ are precisely the separable spaces.
The density of a topological space $X$ is denoted $d(X)$.
If $X$ is a Hausdorff space,
it can be shown that $|X| \le 2^{2^{d(X)}}$.
\end{document}
