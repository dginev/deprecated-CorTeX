\documentclass{article}
\usepackage{ids}
\usepackage{planetmath-specials}
\usepackage{pmath}
\usepackage{graphicx}
\begin{document}
The diameter of a circle or a sphere is the length of the segment joining a point with the one symmetrical respect to the center. That is, the length of the longest segment joining a pair of points.

Also, we call any of these segments themselves a diameter. So, in the next picture, $AB$ is a diameter.

%\begin{figure}
\begin{center}
%\fbox{\input{circ.eepic}}
\includegraphics[scale=0.625]{diameter.eps}
\end{center}
%\end{figure}

The diameter is equal to twice the radius.
\end{document}
