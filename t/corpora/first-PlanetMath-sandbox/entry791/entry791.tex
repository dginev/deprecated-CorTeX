\documentclass{article}
\usepackage{planetmath-specials}
\usepackage{pmath}
\usepackage{amssymb}
\usepackage{amsmath}
\usepackage{amsfonts}
\begin{document}
From the Euler relation

\begin{equation*}
e^{i\theta} = \cos \theta + i \sin \theta
\end{equation*}

it follows that

\begin{align*}
e^{i\theta \cdot n} &= (e^{i\theta})^n\\
\cos n \theta + i \sin n \theta &= (\cos \theta + i \sin \theta)^n
\end{align*}
where $n\in\mathbb{Z}$.
This is called \emph{de Moivre's formula}, and besides being generally useful, it's a convenient way to remember double- (and higher-multiple-) angle formulas.  For example,

\begin{equation*}
\cos 2 \theta + i \sin 2 \theta = (\cos \theta + i \sin \theta)^2 = \cos^2 \theta + 2 i \sin \theta \cos \theta - \sin^2 \theta.
\end{equation*}

Since the imaginary parts and real parts on each side must be equal, we must have
\begin{equation*}
\cos 2 \theta = \cos^2 \theta - \sin^2 \theta
\end{equation*}
and
\begin{equation*}
\sin 2 \theta = 2 \sin \theta \cos \theta.
\end{equation*}
\end{document}
