\documentclass{article}
\usepackage{ids}
\usepackage{planetmath-specials}
\usepackage{pmath}
\usepackage{graphicx}
\usepackage{amsmath}
\usepackage{bbm}
\newcommand{\Z}{\mathbbmss{Z}}
\newcommand{\C}{\mathbbmss{C}}
\newcommand{\R}{\mathbbmss{R}}
\newcommand{\F}{\mathbbmss{F}}
\newcommand{\mathbb}[1]{\mathbbmss{#1}}
\newcommand{\figura}[1]{\begin{center}\includegraphics{#1}\end{center}}
\newcommand{\figuraex}[2]{\begin{center}\includegraphics[#2]{#1}\end{center}}
\newcommand{\legsymp}[1]{\left(\frac{#1}{p}\right)}
\begin{document}
\PMlinkescapeword{root}
\PMlinkescapeword{proposition}
\PMlinkescapeword{identity}

Gauss' lemma on quadratic residues is:

% \noindent
\textbf{Proposition 1:} Let $p$ be an odd prime and let $n$
be an integer which is not a multiple of $p$. Let $u$ be the
number of elements of the set
$$\left\{n,2n,3n,\ldots,\frac{p-1}{2}n\right\}$$
whose least positive residues, modulo $p$, are greater than $p/2$.
Then $$\legsymp{n}=(-1)^u$$
where $\legsymp{n}$ is the Legendre symbol.

That is, $n$ is a quadratic residue modulo
$p$ when $u$ is even and it is a quadratic nonresidue when $u$ is odd.

Gauss' Lemma is the special case
\[S=\left\{1,2,\ldots,\frac{p-1}{2}\right\}\]
of the slightly more general statement below.
Write $\F_p$ for the field of $p$ elements, and
identify $\F_p$ with the set $\{0,1,\ldots,p-1\}$,
with its addition and multiplication mod $p$.

% \noindent
\textbf{Proposition 2:} Let $S$ be a subset of $\F_p^{*}$ such that
$x\in S$ or $-x\in S$, but not both, for any $x\in \F_p^{*}$.
For $n\in \F_p^{*}$ let $u(n)$
be the number of elements $k\in S$ such that $kn\notin S$. Then
$$\legsymp{n} = (-1)^{u(n)}.$$
\textbf{Proof:} If $a$ and $b$ are distinct elements of $S$, we cannot have
$an=\pm bn$, in view of the hypothesis on $S$. Therefore
$$\prod_{a\in S}an=(-1)^{u(n)}\prod_{a\in S}a.$$
On the left we have
$$n^{(p-1)/2}\prod_{a\in S}a=\legsymp{n}\prod_{a\in S}a$$
by Euler's criterion. So
$$\legsymp{n}\prod_{a\in S}a=(-1)^{u(n)}\prod_{a\in S}a$$
The product is nonzero, hence can be cancelled, yielding the proposition.

% \noindent
\textbf{Remarks:} Using Gauss' Lemma, it is straightforward to prove that
for any odd prime $p$:
\[
\legsymp{-1}=
\begin{cases}
1 & \textrm{ if $p\equiv 1\pmod 4$} \\
-1 & \textrm{ if $p\equiv -1\pmod 4$}
\end{cases}
\]
\[
\legsymp{2}=
\begin{cases}
1 & \textrm{ if $p\equiv \pm 1\pmod 8$} \\
-1 & \textrm{ if $p\equiv \pm 3\pmod 8$}
\end{cases}
\]

The condition on $S$ can also be stated like this:
for any square $x^2\in \F_p^{*}$, there is a unique $y\in S$ such
that $x^2=y^2$. Apart from the usual choice
$$S=\{1,2,\ldots,(p-1)/2\}\;,$$
the set
$$\{2,4,\ldots,p-1\}$$ has also been used, notably by Eisenstein.
I think it was also Eisenstein who gave us this trigonometric identity,
which is closely related to Gauss' Lemma:
$$\legsymp{n}=\prod_{a\in S}\frac{\sin{(2\pi an/p)}}{\sin{(2\pi a/p)}}$$

It is possible to prove Gauss' Lemma or Proposition 2 ``from scratch'', without
leaning on Euler's criterion, the existence of a primitive
root, or the fact that a polynomial over $\F_p$ has no more zeros
than its degree.
\end{document}
