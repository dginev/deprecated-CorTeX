\documentclass{article}
\usepackage{planetmath-specials}
\usepackage{pmath}
% this is the default PlanetMath preamble.  as your knowledge
% of TeX increases, you will probably want to edit this, but
% it should be fine as is for beginners.

% almost certainly you want these
\usepackage{amssymb}
\usepackage{amsmath}
\usepackage{amsfonts}

% used for TeXing text within eps files
%\usepackage{psfrag}
% need this for including graphics (\includegraphics)
%\usepackage{graphicx}
% for neatly defining theorems and propositions
%\usepackage{amsthm}
% making logically defined graphics
%\usepackage{xypic}

% there are many more packages, add them here as you need them

% define commands here

\begin{document}
An integer $n$ that in a given base $b$ is a palindrome. Given $n$ being $k$ digits $d_x$ long in base $b$, its value being $$n = \sum_{i = 0}^{k - 1} d_kb^i$$ then if the equalities $d_k = d_1$, $d_{k - 1} = d_2$, etc., hold, then $n$ is a {\em palindromic number}. There are infinitely many palindromic numbers in any given base.
\end{document}
