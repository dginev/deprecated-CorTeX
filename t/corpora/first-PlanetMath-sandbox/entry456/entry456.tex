\documentclass{article}
\usepackage{planetmath-specials}
\usepackage{pmath}
\usepackage{amssymb}
\usepackage{amsmath}
\usepackage{amsfonts}
\begin{document}
\paragraph{The formulas.}
Let $D=\{z\in\mathbb{C} : |z-z_0| <R\}$ be an open disk in the
complex plane, and let $f(z)$ be a holomorphic\footnote{It is
  necessary to draw a distinction between holomorphic functions (those having
  a complex derivative) and analytic functions (those representable by
  power series).  The two concepts are, in fact, equivalent, but
  the standard proof of this fact uses the Cauchy Integral Formula
  with the (apparently) weaker holomorphicity hypothesis.} function
defined on some open domain that contains $D$ and its boundary.  Then,
for every $z\in D$ we have
\begin{eqnarray*}
f(z) &=& \frac{1}{2 \pi i} \oint_C \frac{f(\zeta)}{\zeta-z}\ d\zeta \\
f'(z) &=& \frac{1}{2 \pi i} \oint_C \frac{f(\zeta)}{(\zeta-z)^2}\ d\zeta \\
& \vdots & \\
f^{(n)}(z) &=& \frac{n!}{2 \pi i} \oint_C \frac{f(\zeta)}{(\zeta-z)^{n+1}}\ d\zeta  
\end{eqnarray*}
Here $C=\partial D$ is the corresponding circular boundary contour,
oriented counterclockwise, with the most obvious parameterization
given by
$$\zeta=z_0+R e^{it},\quad 0\leq t\leq 2\pi.$$

\paragraph{Discussion.}
The first of the above formulas underscores the ``rigidity'' of
holomorphic functions.  Indeed, the values of the holomorphic function
inside a disk $D$ are completely specified by its values on the
boundary of the disk.  The second formula is useful, because it gives
the derivative in terms of an integral, rather than as the outcome of
a limit process.

\paragraph{Generalization.}
The following technical generalization of the formula is needed for
the treatment of removable singularities.  Let $S$ be a finite subset
of $D$, and suppose that  $f(z)$ is holomorphic for all $z\notin
 S$, but also that
$f(z)$ is bounded near all $z\in S$.  Then, the
above formulas are valid for all $z\in D\setminus S$.

Using the Cauchy residue theorem, one can further generalize the integral formula to the situation where $D$ is any domain and $C$ is any closed rectifiable curve in $D$; in this case, the formula becomes
$$
\eta(C,z) f(z) = \frac{1}{2 \pi i} \oint_C \frac{f(\zeta)}{\zeta-z}\ d\zeta
$$
where $\eta(C,z)$ denotes the winding number of $C$. It is valid for all points $z \in D \setminus S$ which are not on the curve $C$.
\end{document}
