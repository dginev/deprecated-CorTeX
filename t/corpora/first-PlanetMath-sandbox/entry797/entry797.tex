\documentclass{article}
\usepackage{planetmath-specials}
\usepackage{pmath}
\usepackage{amsmath}
\usepackage{amsfonts}
\usepackage{amssymb}

\newcommand{\reals}{\mathbb{R}}
\newcommand{\natnums}{\mathbb{N}}
\newcommand{\cnums}{\mathbb{C}}

\newcommand{\lp}{\left(}
\newcommand{\rp}{\right)}
\newcommand{\lb}{\left[}
\newcommand{\rb}{\right]}
\newcommand{\cA}{\mathcal{A}}
\newcommand{\dgamma}{\dot{\gamma}}
\newcommand{\hM}{\hat{M}}
\newcommand{\Mat}{\mathop{\mathrm{Mat}}\nolimits}

\newcommand{\rJ}{\mathrm{J}}

\newcommand{\bu}{\mathbf{u}}
\newcommand{\bv}{\mathbf{v}}

\newcommand{\cC}{\mathcal{C}}
\begin{document}
{\bf Summary} The tangent space of differential manifold $M$ at a
point $x\in M$ is the vector space whose elements are velocities of
trajectories that pass through $x$. The standard notation for the
tangent space of $M$ at the point $x$ is $T_x M$. 


{\bf Definition (Standard).}  Let $M$ be a differential manifold and
$x$ a point of $M$. Let
$$\gamma_i:I_i\rightarrow M,\quad I_i\subset\reals,\quad i=1,2$$
be
two differentiable trajectories passing through $x$ at times $t_1\in
I_1, t_2\in I_2$, respectively.  We say that these trajectories are in
first order contact at $x$ if for all differentiable functions
$f:U\rightarrow\reals$ defined in some neighbourhood $U\subset M$ of
$x$, we have
$$(f\circ\gamma_1)'(t_1) = (f\circ\gamma_2)'(t_2).$$
First order
contact is an equivalence relation, and we define $T_x M$, the {\em
  tangent space} of $M$ at $x$, to be the set of corresponding
equivalence classes.  

Given a trajectory
$$\gamma:I\rightarrow M,\quad I\subset\reals$$
passing through $x$ at
time $t\in I$, we define $\dgamma(t)$ the {\em tangent vector} ,
a.k.a. the {\em velocity}, of $\gamma$ at time $t$, to be the
equivalence class of $\gamma$ modulo first order contact.  We endow
$T_x M$ with the structure of a real vector space by identifying it
with $\reals^n$ relative to a system of local coordinates.   These
identifications will differ from chart to chart, but they will all be
linearly compatible.

To describe this identification, consider a coordinate chart 
$$\alpha: U_\alpha\to \reals^n,\quad U_\alpha\subset M,\quad x\in U.$$
We call the real vector
$$(\alpha\circ\gamma)'(t)\in\reals^n$$
the representation of
$\dgamma(t)$ relative to the chart $\alpha$.  It is a simple exercise
to show that two trajectories are in first order contact at $x$ if and
only if their velocities have the same representation.  Another simple
exercise will show that for every $\bu\in \reals^n$ the trajectory
$$t\to\alpha^{-1}(\alpha(x) + t\bu)$$
has velocity $\bu$ relative to the chart $\alpha$.  Hence, every
element of $\reals^n$ represents some actual velocity, and therefore
the mapping  $T_x M\to \reals^n$ given by
$$[\gamma]\to (\alpha\circ\gamma)'(t),\quad \gamma(t)=x,$$
is a bijection.

Finally if $\beta:U_\beta\to \reals^n,\quad U_\beta\subset M,\quad
x\in U_\beta$ is another chart, then for all differentiable
trajectories $\gamma(t)=x$ we have
$$(\beta\circ\gamma)'(t) = J (\alpha\circ\gamma)'(t),$$
where $J$ is
the Jacobian matrix at $\alpha(x)$ of the suitably restricted mapping
$\beta\circ\alpha^{-1}:\alpha(U_\alpha\cap U_\beta)\to\reals^n$.  The
linearity of the above relation
implies that the vector space structure of $T_xM$ is independent of the
choice of coordinate chart.

{\bf Definition (Classical).}  Historically, tangent vectors were
specified as elements of $\reals^n$ relative to some system of
coordinates, a.k.a. a coordinate chart. This point of view naturally
leads to the definition of a tangent space as $\reals^n$ modulo changes
of coordinates.  

Let $M$ be a differential manifold represented as a collection of
parameterization domains
$$\{V_\alpha\subset\reals^n:\alpha\in \cA\}$$ indexed by labels
belonging to a set $\cA$, and 
transition function diffeomorphisms
$$\sigma_{\alpha\beta}:V_{\alpha\beta}\rightarrow
V_{\beta\alpha},\quad \alpha,\beta\in \cA,\quad V_{\alpha\beta}\subset
V_\alpha$$
Set 
$$\hM = \{ (\alpha,x)\in \cA\times \reals^n: x\in V_\alpha\},$$
and recall that a points of the manifold are represented by elements
of $\hM$ modulo an equivalence relation imposed by the transition functions
[see Manifold --- Definition (Classical)].
For a transition function $\sigma_{\alpha\beta}$,  let
$$\rJ\sigma_{\alpha\beta}:V_{\alpha\beta}\rightarrow
\Mat_{n,n}(\reals)$$
denote the corresponding Jacobian matrix of partial derivatives.   We
call a triple
$$(\alpha,x,\bu),\quad \alpha\in\cA,\; x\in V_\alpha,\; \bu\in
\reals^n$$
the representation of a tangent vector at $x$ relative to coordinate
system $\alpha$, and make the identification
$$(\alpha,x,\bu) \simeq
(\beta,\sigma_{\alpha\beta}(x),[\rJ\sigma_{\alpha\beta}](x)(\bu)),\quad
\alpha,\beta\in \cA,\; x\in V_{\alpha\beta},\; \bu \in \reals^n.$$
to arrive at the definition of a tangent vector at $x$.


{\bf Notes.} The notion of tangent space derives from the observation
that there is no natural way to relate and compare velocities at
different points of a manifold.  This is already evident when we
consider objects moving on a surface in 3-space, where the velocities
take their value in the tangent planes of the surface. On a general
surface, distinct points correspond to distinct tangent planes, and
therefore the velocities at distinct points are not commensurate.

The situation is even more complicated for an abstract manifold, where
absent an ambient Euclidean setting there is, apriori, no obvious
``tangent plane'' where the velocities can reside.  This point of view
leads to the definition of a velocity as some sort of equivalence
class.


{\bf See also:} tangent bundle, connection, parallel translation
\end{document}
