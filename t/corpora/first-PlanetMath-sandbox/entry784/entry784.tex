\documentclass{article}
\usepackage{ids}
\usepackage{planetmath-specials}
\usepackage{pmath}
\usepackage{amsmath}
\usepackage{amsfonts}
\usepackage{amssymb}


\newtheorem{proposition}{Proposition}

\newcommand{\reals}{\mathbb{R}}
\newcommand{\natnums}{\mathbb{N}}
\newcommand{\cnums}{\mathbb{C}}

\newcommand{\lp}{\left(}
\newcommand{\rp}{\right)}
\newcommand{\lb}{\left[}
\newcommand{\rb}{\right]}


\newcommand{\cA}{\mathcal{A}}
\newcommand{\cC}{\mathcal{C}}
\newcommand{\cT}{\mathcal{T}}
\begin{document}
\paragraph{Summary.} A {\em manifold} is a space that is
locally like $\reals^n$, however lacking a preferred system of
coordinates.  Furthermore, a manifold can have global topological
properties, such as non-contractible \PMlinkname{loops}{Curve}, that distinguish it from
the topologically trivial $\reals^n$.

\paragraph{Standard Definition.} 
An $n$-dimensional topological manifold $M$ is a second countable, Hausdorff
topological space\footnote{For connected manifolds, the  assumption that $M$ is
second-countable  is logically equivalent to $M$ being paracompact, or
equivalently to $M$ being metrizable. The topological hypotheses in the definition of a manifold are needed
to exclude certain counter-intuitive pathologies.  Standard
illustrations of these pathologies are given by the {\em long line}
(lack of paracompactness) and the {\em forked line} (points cannot be
separated). These pathologies are fully described in Spivak. 
See \PMlinkname{this page}{BibliographyForDifferentialGeometry}.} 
that is locally homeomorphic to open subsets of
$\reals^n$.

A differential manifold is a topological manifold with some additional
structure information.  A \emph{chart}, also known as a \emph{system
  of local coordinates}, is a mapping $\alpha: U \to \reals^n$, such that the domain $U\subset M$ is an open set, and such that $U$ is homeomorphic to the image $\alpha(U)$.  Let
$\alpha: U_\alpha \rightarrow \reals^n,$ and
$\beta:U_\beta\rightarrow\reals^n$ be two charts with overlapping
\PMlinkname{domains}{Function}. The continuous injection
$$\beta\circ\alpha^{-1}: \alpha(U_\alpha\cap
U_\beta)\rightarrow\reals^n$$
is called a \emph{transition function},
and also called a \emph{a change of coordinates}.  An {\em atlas}
$\cA$ is a collection of charts $\alpha:U_\alpha\rightarrow\reals^n$
whose domains cover $M$, i.e.
$$M = \bigcup_{\alpha} U_\alpha.$$
Note that each transition function
is really just $n$ real-valued functions of $n$ real variables, and so
we can ask whether these are continuously differentiable. The atlas
$\cA$ defines a differential structure on $M$, if every transition function 
is continuously differentiable.

More generally, for $k=1,2,\ldots,\infty,\omega$, the atlas $\cA$ is
said to define a $\cC^k$ differential structure, and $M$ is said to be
of class $\cC^k$, if all the transition functions are $k$-times
continuously differentiable, or real analytic in the case of
$\cC^\omega$.  Two differential structures of class $\cC^k$ on $M$ are
said to be isomorphic if the union of the corresponding atlases is
also a $\cC^k$ atlas, i.e. if all the new transition functions arising
from the merger of the two atlases remain of class $\cC^k$.  More
generally, two $\cC^k$ manifolds $M$ and $N$ are said to be
diffeomorphic, i.e.  have equivalent differential structure, if there
exists a homeomorphism $\phi:M\to N$ such that the atlas of $M$ is
equivalent to the atlas obtained as $\phi$-pullbacks of charts on $N$.

The atlas allows us to define differentiable mappings to and from a
manifold. Let
$$f:U\rightarrow\reals,\quad U\subset M$$
be a continuous function. For each $\alpha\in \cA$ we define
$$f_\alpha: V \rightarrow \reals,\quad V\subset\reals^n,$$
called the
representation of $f$ relative to chart $\alpha$, as the suitably
restricted composition
$$f_\alpha = f\circ \alpha^{-1}.$$
We judge $f$ to be differentiable if all
the representations $f_\alpha$ are differentiable. A path
$$\gamma: I\rightarrow M,\quad I\subset\reals$$
is judged to be differentiable, if for all differentiable
functions
$f$, the suitably restricted composition $f\circ\gamma$ is a
differentiable function from $\reals$ to $\reals$. Finally, given
manifolds $M, N$, we judge a continuous mapping $\phi:M\rightarrow
N$
between them to be differentiable if for all differentiable
functions
$f$ on $N$, the suitably restricted composition $f\circ\phi$ is a
differentiable function on $M$.
\end{document}
