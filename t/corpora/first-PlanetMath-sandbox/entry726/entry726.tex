\documentclass{article}
\usepackage{ids}
\usepackage{planetmath-specials}
\usepackage{pmath}
% this is the default PlanetMath preamble.  as your knowledge
% of TeX increases, you will probably want to edit this, but
% it should be fine as is for beginners.

% almost certainly you want these
\usepackage{amssymb}
\usepackage{amsmath}
\usepackage{amsfonts}

% used for TeXing text within eps files
%\usepackage{psfrag}
% need this for including graphics (\includegraphics)
%\usepackage{graphicx}
% for neatly defining theorems and propositions
%\usepackage{amsthm}
% making logically defined graphics
\usepackage{xypic} 

% there are many more packages, add them here as you need them

% define commands here
\begin{document}
The \emph{wheel graph} of $n$ vertices $W_n$ is a graph that contains a cycle of length $n-1$ plus a vertex $v$ (sometimes called the \emph{hub}) not in the cycle such that $v$ is connected to every other vertex.  The edges connecting $v$ to the rest of the graph are sometimes called \emph{spokes}.

$W_4$:

$$\xymatrix{
  & A \ar@{-}[d] \ar@{-}[ddl] \ar@{-}[ddr] & \\
  & D & \\
C \ar@{-}[ur] \ar@{-}[rr] &   & B \ar@{-}[ul]
}$$

$W_6$:
$$\xymatrix{
  &   & A \ar@{-}[d] \ar@{-}[drr] &   &   \\
E\ar@{-}[rr] \ar@{-}[urr] &   & F &   & B \ar@{-}[ll] \ar@{-}[dl] \\
  & D \ar@{-}[ur] \ar@{-}[ul] &   & C \ar@{-}[ul] \ar@{-}[ll]&   \\
}$$
\end{document}
