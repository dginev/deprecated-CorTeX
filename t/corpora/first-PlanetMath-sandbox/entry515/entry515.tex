\documentclass{article}
\usepackage{planetmath-specials}
\usepackage{pmath}
\usepackage{amssymb}
\usepackage{amsmath}
\usepackage{amsfonts}
\usepackage{graphicx}
\usepackage{xypic}
\begin{document}
The word orthogonal comes from the Greek \emph{orthe} and \emph{gonia}, or ``right angle.''  It was originally used as synonym of perpendicular. This is where the use of ``orthogonal'' in  orthogonal lines, orthogonal circles, and other geometric terms come from.

In the realm of linear algebra, two vectors are orthogonal when their dot product is zero, which gave rise a generalization of two vectors on some inner product space (not necessarily dot product) being orthogonal when their inner product is zero. 

There are also particular definitions on the following entries:
\begin{itemize}
 \item orthogonal matrices
 \item orthogonal polynomials
 \item orthogonal vectors
\end{itemize}

In a more broad sense, it can be said that two objects are orthogonal if they do not ``coincide'' in some way.
\end{document}
