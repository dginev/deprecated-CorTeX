\documentclass{article}
\usepackage{planetmath-specials}
\usepackage{pmath}
\usepackage{amsmath}
\usepackage{amsfonts}
\usepackage{amssymb}
\newcommand{\reals}{\mathbb{R}}
\newcommand{\natnums}{\mathbb{N}}
\newcommand{\cnums}{\mathbb{C}}

\newcommand{\cfty}{\mathcal{C}^\infty}
\newcommand{\ord}{\mathop{\mathrm{ord}}\nolimits}

\newtheorem{proposition}{Proposition}
\begin{document}
Roughly speaking, a {\em differential operator} is a mapping,
typically understood to be linear, that transforms a function into
another function by means of partial derivatives and multiplication by
other functions.  

On $\reals^n$, a differential operator is commonly understood to be a
linear transformation of $\cfty(\reals^n)$ having the form
$$  f \mapsto \sum_{I} a^I f_I,\quad f\in
\cfty(\reals^n),  
$$
where the sum is taken over a finite number of multi-indices
$I=(i^1,\ldots,i^n)\in \natnums^n$, where $a^I\in
\cfty(\reals^n)$, and where $f_I$ denotes a partial
derivative of $f$ taken $i_1$ times with respect to the first
variable, $i_2$ times with respect to the second variable, etc.
The {\em order} of the operator is the maximum number of derivatives
taken in the above formula, i.e. the maximum of $i_1+\ldots+i_n$
taken over all the $I$ involved in the above summation.

On a $\cfty$ manifold $M$, a differential operator is commonly
understood to be a linear transformation of $\cfty(M)$ having the
above form relative to some system of coordinates.  Alternatively, one
can equip $\cfty(M)$ with the limit-order topology, and define a
differential operator as a continuous transformation of $\cfty(M)$.

The order of a differential operator is a more subtle notion on a
manifold than on $\reals^n$.  There are two complications. First, one
would like a definition that is independent of any particular system
of coordinates.  Furthermore, the order of an operator is at best a local
concept: it can change from
point to point, and indeed be unbounded if the manifold is
non-compact.  To address these issues, for a differential operator $T$
and $x\in M$, 
we define $\ord_x(T)$ the order of $T$ at $x$, to be the smallest
$k\in\natnums$ such that
$$T[f^{k+1}](x) = 0$$
for all $f\in\cfty(M)$ such that $f(x)=0$.  For
a fixed differential operator $T$, the function $\ord(T):M\rightarrow
\natnums$ defined by
$$x\mapsto \ord_x(T)$$
is lower semi-continuous, meaning that
$$\ord_y(T)\geq \ord_x(T)$$
for all $y\in M$ sufficiently close to $x$.

The global order of $T$ is defined to be the maximum of $\ord_x(T)$
taken over all $x\in M$.  This maximum may not exist if $M$ is
non-compact, in which case one says that the order of $T$ is infinite.

Let us conclude by making two remarks. The notion of a differential
operator can be generalized even further by allowing the operator to
act on sections of a bundle.

A differential operator $T$ is a local operator, meaning that
$$T[f](x) = T[g](x),\quad f,g\in\cfty(M),\; x\in M,$$
if $f\equiv g$ in some neighborhood of $x$.  A theorem, proved by
Peetre states that the converse is also true, namely that every local
operator is necessarily a differential operator.

\bigskip
{\bf References}

\begin{enumerate}
\item Dieudonn\'e, J.A., {\em Foundations of modern analysis}
\item Peetre, J. , ``Une caract\'erisation abstraite des op\'erateurs
  diff\'erentiels'', {\em Math. Scand.}, v. 7, 1959, p. 211
\end{enumerate}
\end{document}
