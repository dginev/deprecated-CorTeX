\documentclass{article}
\usepackage{ids}
\usepackage{planetmath-specials}
\usepackage{pmath}
\usepackage{amssymb}
\usepackage{amsmath}
\usepackage{amsfonts}
\usepackage{graphicx}
\usepackage{xypic}
\begin{document}
Truly random numbers can only be generated by a physical process and cannot be generated via software. This makes it rather clumsy to use them in Monte Carlo calculations, since they must be first generated in a separate device and either sent to the computer or recorded (for example on removable storage media) for later use in calculations. Traditionally, tapes containing millions of random  numbers generated using radioactive decay were available from laboratories.  

Nowadays, standard digital computers often have provisions for obtaining truly random numbers, that is, numbers generated by a physical process.  Fir instance, Intel has provided a \PMlinkescapetext{function} since their $i810$ chipsets which utilizes noise in a particularly prone semiconductor as a source of randomness.  Often times it is possible to use other incidental physical noise as a source; for example, static on the input channel of a sound card.  In addition, peripheral devices (add-ons) to personal computers exist which provide truly random numbers, when the previous methods fail.

\begin{thebibliography}{3}

\bibitem (Derived from) The Data Analysis Briefbook.  \PMlinkexternal{http://rkb.home.cern.ch/rkb/titleA.html}{http://rkb.home.cern.ch/rkb/titleA.html}.

\end{thebibliography}
\end{document}
