\documentclass{article}
\usepackage{planetmath-specials}
\usepackage{pmath}
% this is the default PlanetMath preamble.  as your knowledge
% of TeX increases, you will probably want to edit this, but
% it should be fine as is for beginners.

% almost certainly you want these
\usepackage{amssymb}
\usepackage{amsmath}
\usepackage{amsfonts}

% used for TeXing text within eps files
%\usepackage{psfrag}
% need this for including graphics (\includegraphics)
%\usepackage{graphicx}
% for neatly defining theorems and propositions
%\usepackage{amsthm}
% making logically defined graphics
%\usepackage{xypic}

% there are many more packages, add them here as you need them

% define commands here

\newcommand{\sR}[0]{\mathbb{R}}
\newcommand{\sC}[0]{\mathbb{C}}
\newcommand{\sN}[0]{\mathbb{N}}
\newcommand{\sZ}[0]{\mathbb{Z}}
\begin{document}
\subsubsection*{Introduction}
The definiteness of a matrix is an important
property that has use in many areas of mathematics and \PMlinkescapetext{even} physics. 
Below are some examples:

\begin{enumerate}
\item In optimizing problems, the definiteness of the 
Hessian matrix determines the quality of an extremal value. 
The full details can be found on 
\PMlinkname{this page}{RelationsBetweenHessianMatrixAndLocalExtrema}.
\end{enumerate}


{\bf Definition} \cite{pease}
Suppose $A$ is an $n\times n$ square Hermitian matrix. 
If, for any non-zero vector $x$, we have that
 $$x^\ast Ax>0,$$
then $A$ a \emph{positive definite} matrix. (Here $x^\ast=\overline{x}^t$,
where $\overline{x}$ is the complex conjugate of $x$, and $x^t$ is
the transpose of $x$.)

One can show that  a Hermitian matrix is positive definite if 
and only if all its eigenvalues are positive \cite{pease}. 
Thus the determinant of a positive definite matrix
is positive, and 
 a positive definite matrix is always invertible.
The Cholesky decomposition provides an economical method for
solving linear equations involving a positive definite matrix. 
Further conditions and properties for positive definite matrices
are given in \cite{johnson:pdm}.
 
\begin{thebibliography}{9}
\bibitem {pease} M. C. Pease,
 \emph{Methods of Matrix Algebra},
 Academic Press, 1965
\bibitem{johnson:pdm} C.R. Johnson, \emph{Positive definite matrices},
  American Mathematical Monthly, Vol. 77, Issue 3 (March 1970) 259-264.
\end{thebibliography}
\end{document}
