\documentclass{article}
\usepackage{ids}
\usepackage{planetmath-specials}
\usepackage{pmath}
% this is the default PlanetMath preamble.  as your knowledge
% of TeX increases, you will probably want to edit this, but
% it should be fine as is for beginners.

% almost certainly you want these
\usepackage{amssymb}
\usepackage{amsmath}
\usepackage{amsfonts}

% used for TeXing text within eps files
%\usepackage{psfrag}
% need this for including graphics (\includegraphics)
%\usepackage{graphicx}
% for neatly defining theorems and propositions
%\usepackage{amsthm}
% making logically defined graphics
%\usepackage{xypic} 

% there are many more packages, add them here as you need them

% define commands here
\begin{document}
A \emph{discrete valuation ring} $R$ is a principal ideal domain with exactly one \textbf{nonzero} maximal ideal $M$. Any generator $t$ of $M$ is called a \emph{uniformizer} or \emph{uniformizing element} of $R$; in other words, a uniformizer of $R$ is an element $t \in R$ such that $t \in M$ but $t \notin M^2$.

Given a discrete valuation ring $R$ and a uniformizer $t \in R$, every element $z \in R$ can be written uniquely in the form $u \cdot t^n$ for some unit $u \in R$ and some nonnegative integer $n \in \mathbb{Z}$. The integer $n$ is called the \emph{order} of $z$, and its value is independent of the choice of uniformizing element $t \in R$.
\end{document}
