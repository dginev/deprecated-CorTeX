\documentclass{article}
\usepackage{planetmath-specials}
\usepackage{pmath}
\usepackage{amssymb}
\usepackage{amsmath}
\usepackage{amsfonts}
\begin{document}
\PMlinkescapeword{implies}
\PMlinkescapeword{property}

A relation $\mathcal{R}$ on a set $A$ is \emph{symmetric} if and only if
whenever $x\mathcal{R}y$ for some $x, y \in A$ then also $y\mathcal{R}x$.

An example of a symmetric relation on $\{a,b,c\}$
is $\{(a,a), (c,b), (b,c), (a,c), (c,a)\}$.
One relation that is not symmetric is
$\mathcal{R} = \{(b,b), (a,b), (b,a), (c,b) \} $,
because $(c,b) \in \mathcal{R}$ but $(b,c) \notin \mathcal{R}$.

On a finite set with $n$ elements there are $2^{n^2}$ relations,
of which $2^{\frac{n^2+n}{2}}$ are symmetric.

A relation $\mathcal{R}$ that is both symmetric and antisymmetric
has the property that $x\mathcal{R}y$ implies $x=y$.
On a finite set with $n$ elements there are only $2^n$ such relations.


\end{document}
