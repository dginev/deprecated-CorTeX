\documentclass{article}
\usepackage{ids}
\usepackage{planetmath-specials}
\usepackage{pmath}
\usepackage{amssymb}
\usepackage{amsmath}
\usepackage{amsfonts}

\begin{document}
\PMlinkescapeword{scalar}
\PMlinkescapeword{term}

Let $V$ be a vector space over a field $K$.
A \emph{linear functional} (or \emph{linear form}) on $V$
is a linear mapping $\phi\colon V\to K$,
where $K$ is thought of as a one-dimensional vector space over itself.

The collection of all linear functionals on $V$
can be made into a vector space
by defining addition and scalar multiplication pointwise;
this vector space is called the dual space of $V$.

The term {\it linear functional} derives from
the case where $V$ is a space of functions
(see the entry on \PMlinkname{functionals}{Functional}).
Some authors restrict the term to this case.
\end{document}
