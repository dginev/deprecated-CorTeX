\documentclass{article}
\usepackage{planetmath-specials}
\usepackage{pmath}
\usepackage{amssymb}
\usepackage{amsmath}
\usepackage{amsfonts}

% used for TeXing text within eps files
%\usepackage{psfrag}
% need this for including graphics (\includegraphics)
\usepackage{graphicx}
% for neatly defining theorems and propositions
%\usepackage{amsthm}
% making logically defined graphics
%\usepackage{xypic} 

% there are many more packages, add them here as you need them

% define commands here
\newcommand{\mv}[1]{\mathbf{#1}}	% matrix or vector
\newcommand{\mvt}[1]{\mv{#1}^{\mathrm{T}}}
\newcommand{\mvi}[1]{\mv{#1}^{-1}}
\newcommand{\mpderiv}[1]{\frac{\partial}{\partial {#1}}}
\newcommand{\borel}{\mathfrak{B}}
\newcommand{\reals}{\mathbb{R}}
\newcommand{\defined}{:=}
\newcommand{\var}{\mathrm{var}}
\newcommand{\cov}{\mathrm{cov}}
\newcommand{\corr}{\mathrm{corr}}
\newcommand{\set}[1]{\{#1\}}
\begin{document}
Let $f:\reals \to \reals$ be a function which is continuous on the interval $[a,b]$ and differentiable on $(a,b)$.  Then there exists a number $c: a < c < b$ such that

\begin{equation}
f'(c) = \frac{f(b) - f(a)}{b - a}.
\end{equation}
The geometrical meaning of this theorem is illustrated in the picture:\\[10pt]
\begin{center}
\includegraphics[scale=0.4]{mittelwertsatz.ps}
\end{center}
The dashed line connects the points $(a,f(a))$ and $(b,f(b))$. There is $c$ between $a$ and $b$ at which the tangent to $f$ has the same slope as the dashed line.

The mean-value theorem is often used in the integral context: There is a $c \in [a,b]$ such that

\begin{equation}
(b-a)f(c) = \int_{a}^{b} f(x) dx.
\end{equation}
\end{document}
