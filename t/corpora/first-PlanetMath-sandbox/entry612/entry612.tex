\documentclass{article}
\usepackage{planetmath-specials}
\usepackage{pmath}
% this is the default PlanetMath preamble.  as your knowledge
% of TeX increases, you will probably want to edit this, but
% it should be fine as is for beginners.

% almost certainly you want these
\usepackage{amssymb}
\usepackage{amsmath}
\usepackage{amsfonts}

% used for TeXing text within eps files
%\usepackage{psfrag}
% need this for including graphics (\includegraphics)
%\usepackage{graphicx}
% for neatly defining theorems and propositions
%\usepackage{amsthm}
% making logically defined graphics
%\usepackage{xypic} 

% there are many more packages, add them here as you need them

% define commands here
\begin{document}
Let $\{K_\alpha\}$, $\alpha \in J$, be a collection of subfields of a field $L$. The {\em composite field} of the collection is the smallest subfield of $L$ that contains all the fields $K_\alpha$.

The notation $K_1 K_2$ (resp., $K_1 K_2 \dots K_n$) is often used to denote the composite field of two (resp., finitely many) fields.
\end{document}
