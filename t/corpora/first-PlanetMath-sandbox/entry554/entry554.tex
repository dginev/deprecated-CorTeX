\documentclass{article}
\usepackage{planetmath-specials}
\usepackage{pmath}
\usepackage{amssymb}
\usepackage{amsmath}
\usepackage{amsfonts}
\usepackage{graphicx}
\usepackage{xypic}
\begin{document}
We are said to be using {\PMlinkescapetext {\it right function notation}}
if we write functions to the right of their arguments.
That is, if $\alpha : X \to Y$ is a function and $x \in X$,
then $x \alpha$ is the image of $x$ under $\alpha$.

Furthermore, if we have a function $\beta : Y \to Z$,
then we write the composition of the two functions
as $\alpha \beta : X \to Z$,
and the image of $x$ under the composition
as $x \alpha \beta = x (\alpha \beta) = (x \alpha) \beta$.

Compare this to left function notation.
\end{document}
