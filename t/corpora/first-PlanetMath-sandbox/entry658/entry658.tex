\documentclass{article}
\usepackage{ids}
\usepackage{planetmath-specials}
\usepackage{pmath}
\usepackage{amssymb}
\usepackage{amsmath}
\usepackage{amsfonts}

%\usepackage{psfrag}
%\usepackage{graphicx}
%\usepackage{xypic}
\begin{document}
The \emph{transitive property} of logic is 

$$ (a \Rightarrow b) \land (b \Rightarrow c) \Rightarrow (a \Rightarrow c) $$

Where $\Rightarrow$ is the conditional truth function.  From this we can derive that

$$ (a = b) \land (b = c) \Rightarrow (a = c) $$
\end{document}
