\documentclass{article}
\usepackage{planetmath-specials}
\usepackage{pmath}
\usepackage{amssymb}
\usepackage{amsmath}
\usepackage{amsfonts}
\usepackage{graphicx}
\usepackage{xypic}
\begin{document}
``Genus'' has number of distinct but compatible definitions.

In topology, if $S$ is an orientable surface, its genus $g(S)$ is the number of ``handles'' it has.
More precisely, from the classification of surfaces, we know that any orientable
surface is a sphere, or the connected sum of $n$ tori.  We say the sphere
has genus 0, and that the connected sum of $n$ tori has genus $n$ 
(alternatively, genus is additive with respect to connected sum, and the genus of a torus is 1).
Also, $g(S)=1-\chi(S)/2$ where $\chi(S)$ is the Euler characteristic of $S$. 

In algebraic geometry, the {genus} of a smooth projective curve $X$ over a field $k$ is the
dimension over $k$ of the vector space $\Omega^1(X)$ of global regular
differentials on $X$.  Recall that a smooth complex curve is also a Riemann surface,
and hence topologically a surface.  In this case, the two definitions of genus coincide.
\end{document}
