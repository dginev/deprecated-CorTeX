\documentclass{article}
\usepackage{planetmath-specials}
\usepackage{pmath}
\usepackage{amssymb}
\usepackage{amsmath}
\usepackage{amsfonts}
\begin{document}
\PMlinkescapeword{corollary}

A \emph{normal space} is a topological space $X$
such that whenever $A$ and $B$ are disjoint closed subsets of $X$,
then there are disjoint open subsets $U$ and $V$ of $X$
such that $A\subseteq U$ and $B\subseteq V$.

(Note that some authors include $\mathrm{T}_1$ in the definition,
which is equivalent to requiring the space to be Hausdorff.)

\emph{Urysohn's Lemma} states that $X$ is normal
if and only if
whenever $A$ and $B$ are disjoint closed subsets of $X$,
then there is a continuous function $f\colon X\to[0,1]$
such that $f(A)\subseteq\{0\}$ and $f(B)\subseteq\{1\}$.
(Any such function is called an \emph{Urysohn function}.)

A corollary of Urysohn's Lemma
is that normal \PMlinkname{$\mathrm{T}_1$}{T1Space} spaces are completely regular.
\end{document}
