\documentclass{article}
\usepackage{ids}
\usepackage{planetmath-specials}
\usepackage{pmath}
% this is the default PlanetMath preamble.  as your knowledge
% of TeX increases, you will probably want to edit this, but
% it should be fine as is for beginners.

% almost certainly you want these
\usepackage{amssymb}
\usepackage{amsmath}
\usepackage{amsfonts}

% used for TeXing text within eps files
%\usepackage{psfrag}
% need this for including graphics (\includegraphics)
%\usepackage{graphicx}
% for neatly defining theorems and propositions
%\usepackage{amsthm}
% making logically defined graphics

\usepackage{xypic} 

% there are many more packages, add them here as you need them

% define commands here
%\DeclareMathOperator{\det}{det}
\DeclareMathOperator{\GL}{GL}
\begin{document}
\PMlinkescapeword{theory}
\textbf{Definition}.
Let $\mathcal{C}$ and $\mathcal{D}$ be categories, and let 
$S,T:\mathcal{C}\to\mathcal{D}$ be covariant functors. Then suppose
that for every object $A$ in $\mathcal{C}$ one has a morphism 
$\eta_A :  S(A) \to T(A) $ in $\mathcal{D}$ such that for every morphism 
$\alpha: A \to B$ in $\mathcal{C}$ the following
$$
\xymatrix@+=4pc{S(A) \ar[d]_{S(\alpha)} \ar[r]^{\eta_A} & T(A) \ar[d]^{T(\alpha)} \\
S(B) \ar[r]^{\eta_{B}} & T(B)
}
$$

is \PMlinkname{commutative}{CommutativeDiagram}.  Then we variously write 
$$\eta: S \dot{\to} T \quad\mbox{ or }\quad \eta: S\Rightarrow T\quad \mbox{ or } \quad \eta:S\to T$$ and call $\eta$ a 
\emph{natural trasformation} from $S$ to $T$.

One may think of a natural transformation $\eta:S\to T$ as a `function' from the class of objects of $\mathcal{C}$ to the class of morphisms of $\mathcal{D}$.

As a first example, for every functor $S:\mathcal{C}\to \mathcal{D}$, we can associate the natural transformation $1_S: S\to S$ (the \emph{identity natural transformation} on $S$) that assigns every object $A$ of $\mathcal{C}$, the corresponding identity morphism $1_{S(A)}$.

Natural transformations are composed in a similar manner to morphisms, but they are nevertheless defined as correspondences between both objects and morphisms as shown in the square commutative diagram depicted above. 
%On the other hand, natural transformations can be regarded as functorial morphisms, because they do act as morphisms between functors in the 2-category of small categories that has been defined as a functor category.

 More precisely, given three functors $R,S,T:\mathcal{C}\to \mathcal{D}$, and two natural transformations, $\tau:R\to S$ and $\eta:S\to T$, we define the composition of $\tau$ with $\eta$, written $\eta \bullet \tau$, as a class of morphisms in $\mathcal{D}$ given by 
$$(\eta\bullet \tau)_A := \eta_A\circ \tau_A,$$ for every object $A$ in $\mathcal{C}$.
It is easy to see that $\eta\bullet \tau$ is a natural transformation, since we may ``compose'' two commutative squares and obtain a third one:
$$
\xymatrix@+=4pc{R(A) \ar[d]_{R(\alpha)} \ar[r]^{\tau_A}  & S(A) \ar[d]_{S(\alpha)} \ar[r]^{\eta_A} & T(A) \ar[d]^{T(\alpha)} \ar@{}[dr]|{=} &  
R(A) \ar[d]_{R(\alpha)} \ar[r]^{\eta_A \circ \tau_A} & T(A) \ar[d]^{T(\alpha)} 
\\
R(B) \ar[r]^{\tau_{B}} & S(B) \ar[r]^{\eta_{B}} & T(B) & 
R(B) \ar[r]^{\eta_B \circ \tau_B} & T(B)
}
$$
It is easy to see that the composition ``operation'' on natural transformations is associative:
$$(\zeta\bullet \eta)\bullet \tau = \zeta\bullet (\eta \bullet \tau)$$
for natural transformations $\tau:R\to S$, $\eta:S\to T$, and $\zeta:T\to U$.  In addition, any identity natural transformation acts as a compositional identity: if $\tau:R\to S$ and $\eta:S\to T$, then $$1_S\bullet \tau=\tau \qquad\mbox{ and }\qquad \eta \bullet 1_S = \eta.$$

%This replaces the previous \\
%\textbf{Definition 0.2}:
%A \emph{natural transformation} $\tau: S\dot{\to} T$ is a \emph{function}
%that carries each object $A$ of $\mathcal{C}$ to a morphism
%$\tau_A:S(A)\to T(A)$ of $\mathcal{D}$, and such that for any
%morphism $f:A\to B$ of $\mathcal{C}$, the diagram
%\[
%\xymatrix{& S(A) \ar[d]^{Sf} \ar[r]^{\tau_A} & T(A) \ar[d]^{Tf} \\
%& S(B) \ar[r]^{\tau_{B}} & T(B)
%}
%\]
%is \PMlinkname{commutative}{CommutativeDiagram}.  If every $\tau_A$ happens to be an isomorphism, then $\tau$ is %called a \emph{natural isomorphism}, a \emph{natural equivalence}, or an \emph{isomorphism of functors}.

%which was not either sufficiently general or explicit because it involved \emph{functions} rather than morphisms. \\

\textbf{Remarks}.  
\begin{itemize}
\item
Natural transformations arise frequently in mathematics.  Here's a ``concrete'' example of a natural transformation: the boundary map $H_n(X,A)\to H_{n-1}(A)$ in a homology theory. By definition, every morphism of presheaves is a natural transformation.
More prosaically, the determinant 
$\det:\GL_n\dot{\to}(\ )^*$
is natural.
%\item
%In the definition above, if $\eta_A$ is an \emph{isomorphism} for each $A$ in $\mathcal{C}$ then we call $\eta$ a \emph{natural equivalence} (or a functor isomorphism, or isomorphism of two functors), which has the obvious inverse denoted as $\eta^{-1}$.
%\item
%If $\mathcal{C}$ is a $\mathcal{U}$-category, then we can also define the \emph{functor category} $\mathcal{D}^\mathcal{C}$; the objects of $\mathcal{D}^\mathcal{C}$ are the functors $T:\mathcal{C}\to\mathcal{D}$, and the morphisms are the natural transformations $\tau:S\dot{\to} T$.  The composition of two composable functions which are natural transformations is again a natural transformation, and so $\mathcal{D}^\mathcal{C}$ is a category.
\item
Natural transformations are sometimes called also \emph{functorial morphisms} especially in applications related to the category theory development line pursued by Charles Ehresmann and the `Nicolas Bourbaki' group; this is also a natural translation of the same concept from French, viz. (ref. \cite{CE1965}).
\end{itemize}

\begin{thebibliography}{9}

\bibitem{Ha}A. Hatcher, \emph{Algebraic Topology}, Cambridge University Press, 2002.
\bibitem{Ma}S. Mac Lane, \emph{Categories for the Working Mathematician} (2nd edition), Springer-Verlag, 1997.

%\bibitem{EC1966}
%C. Ehresmann, Trends Toward Unity in Mathematics., \emph{Cahiers de Topologie et Geometrie Differentielle}
%\textbf{8}: 1-7, 1966.

\bibitem{CE1965}
C. Ehresmann, \emph{Cat\'egories et Structures}. Dunod: Paris , 1965.

\bibitem{Eh-quintettes}
C. Ehresmann, Cat\'egories doubles des quintettes: applications covariantes
, \emph{C.R.A.S. Paris}, \textbf{256}: 1891-1894, 1963.

%\bibitem{Eh-Oe}
%C. Ehresmann, \emph{Oeuvres compl\`etes et  comment\'ees:
%Amiens, 1980-84}, 1984 (\emph{edited and commented by Andr\'ee Ehresmann}).

%\bibitem{EML1}
%S. Eilenberg and S. Mac Lane.,  Natural Isomorphisms in Group Theory., \emph{American Mathematical Society 43}: %757-831, 1942.

\bibitem{EML45}
S. Eilenberg and S. Mac Lane,  The General Theory of Natural Equivalences, \emph{Transactions of the American Mathematical Society} \textbf{58}: 231-294, 1945.

%\bibitem{Gabriel1962}
%P. Gabriel, Des cat\'egories ab\'eliennes, \emph{Bull. Soc.Math. France} 
%\textbf{90}: 323-448, 1962.

%\bibitem{Alex3}
%A. Grothendieck, and J. Dieudon\'{e},  \emph{El\'{e}ments de geometrie alg\'{e}brique.}, \emph{Publ. Inst. des Hautes %Etudes de Science}, \textbf{4}, 1960.

\bibitem{MitchellB1965}
B. Mitchell., \emph{Theory of Categories}, Academic Press: New York and London.

%\bibitem{NP1973}
%N. Popescu, \emph{Abelian Categories with Applications to Rings and Modules.}, New York and London: Academic Press.,
%1973, 2nd edn. 1975, \emph{(English translation by I.C. Baianu)}.

\end{thebibliography}
\end{document}
