\documentclass{article}
\usepackage{planetmath-specials}
\usepackage{pmath}
% this is the default PlanetMath preamble.  as your knowledge
% of TeX increases, you will probably want to edit this, but
% it should be fine as is for beginners.

% almost certainly you want these
\usepackage{amssymb}
\usepackage{amsmath}
\usepackage{amsfonts}

% used for TeXing text within eps files
%\usepackage{psfrag}
% need this for including graphics (\includegraphics)
%\usepackage{graphicx}
% for neatly defining theorems and propositions
%\usepackage{amsthm}
% making logically defined graphics
%\usepackage{xypic}

% there are many more packages, add them here as you need them

% define commands here

\newcommand{\sR}[0]{\mathbb{R}}
\newcommand{\sC}[0]{\mathbb{C}}
\newcommand{\sN}[0]{\mathbb{N}}
\newcommand{\sZ}[0]{\mathbb{Z}}

% The below lines should work as the command
% \renewcommand{\bibname}{References}
% without creating havoc when rendering an entry in 
% the page-image mode.
\makeatletter
\@ifundefined{bibname}{}{\renewcommand{\bibname}{References}}
\makeatother

\newcommand*{\norm}[1]{\lVert #1 \rVert}
\newcommand*{\abs}[1]{| #1 |}
\begin{document}
\PMlinkescapeword{collection}
{\bf Definition} (\cite{kelley}, pp. 49)
Let $Y$ be a subset of a set $X$. A \textbf{cover} for $Y$ is a collection
of sets $\mathcal{U}=\{U_i\}_{i\in I}$ such that each $U_i$ 
is a subset of $X$, and 
$$ Y \subset \bigcup_{i\in I} U_i.$$
The collection of sets can be arbitrary, that is, $I$ can be 
finite, countable, or uncountable. The cover is  correspondingly called a 
\textbf{finite cover}, \textbf{countable cover}, or \textbf{uncountable cover}.

%Let $X$ be a set and let $\mathbb{P}(X)$ denote the power set of $X$.  A %collection $\mathcal{U}=\{ U_i\in\mathbb{P}(X) \colon i\in I\}$ of subsets of $X$ %is said to be a \emph{cover} of X if $$X\subseteq\bigcup_{i\in I}U_i$$

A \textbf{subcover} of $\mathcal{U}$ is a subset $\mathcal{U}'\subset\mathcal{U}$ such that $\mathcal{U}'$ is also a cover of $X$.

A \textbf{refinement} $\mathcal{V}$ of $\mathcal{U}$ is a cover of $X$ such that for every $V\in\mathcal{V}$ there is some $U\in\mathcal{U}$ such that $V\subset U$.  When $\mathcal{V}$ refines $\mathcal{U}$, it is usually written $\mathcal{V}\preceq \mathcal{U}$.  $\preceq$ is a preorder on the set of covers of any topological space $X$.

If $X$ is a topological space and the members of $\mathcal{U}$ are open sets,
then $\mathcal{U}$ is said to be an \emph{open cover}.
Open subcovers and open refinements are defined similarly.

{\bf Examples}
\begin{enumerate}
\item If $X$ is a set, then $\{X\}$ is a cover of $X$. 
\item The power set of a set $X$ is a cover of $X$.
\item A topology for a set is a cover of that set.
\end{enumerate}


\begin{thebibliography}{9}
 \bibitem{kelley} J.L. Kelley, \emph{General Topology},
 D. van Nostrand Company, Inc., 1955.
 \end{thebibliography}
\end{document}
