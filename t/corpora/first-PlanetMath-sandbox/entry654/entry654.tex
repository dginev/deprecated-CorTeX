\documentclass{article}
\usepackage{ids}
\usepackage{planetmath-specials}
\usepackage{pmath}
\usepackage{amssymb}
\usepackage{amsmath}
\usepackage{amsfonts}

\def\ip#1{{\langle #1\rangle}}
\begin{document}
\PMlinkescapeword{induced}
\PMlinkescapeword{norm}
\PMlinkescapeword{satisfies}

An \emph{inner product space} (or \emph{pre-Hilbert space}) is a vector space
(over $\mathbb{R}$ or $\mathbb{C}$)
with an inner product $\ip{\cdot,\cdot}$.

For example, $\mathbb{R}^n$ with the familiar dot product
forms an inner product space.

Every inner product space is also a normed vector space,
with the norm defined by $\Vert x \Vert := \sqrt{\ip{x,\,x}}$.
This norm satisfies the parallelogram law.

If the metric $\Vert{x-y}\Vert$
induced by the norm is \PMlinkname{complete}{Complete},
then the inner product space is called a Hilbert space.

The Cauchy--Schwarz inequality 
\begin{align}
  |\ip{x,\,y}| \le \Vert x\Vert \cdot\Vert y\Vert
\end{align}
holds in any inner product space.

According to (1), one can define the {\em angle between two non-zero vectors} $x$ and $y$:
\begin{align}
  \cos(x,\,y) := \frac{\ip{x,\,y}}{\Vert{x}\Vert\cdot\Vert{y}\Vert}.
\end{align}
This provides that the scalars are the real numbers.  In any case, the {\em perpendiculatity} of the vectors may be defined with the condition
 $$\ip{x,\,y} =0.$$
\end{document}
