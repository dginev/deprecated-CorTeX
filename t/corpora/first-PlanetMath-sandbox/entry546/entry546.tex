\documentclass{article}
\usepackage{planetmath-specials}
\usepackage{pmath}
\usepackage{amssymb}
\usepackage{amsmath}
\usepackage{amsfonts}
\usepackage{graphicx}
\usepackage{xypic}
\begin{document}
The coefficient of $x^{n-k}$ in the polynomial $(x+t_1) (x+t_2) \cdots (x+t_n)$ is called the $k^\mathrm{th}$ \emph{elementary symmetric polynomial} in the $n$ variables $t_1, \dots, t_n$. The elementary symmetric polynomials can also be constructed by taking the sum of all possible degree $k$ monomials in $t_1,\dots, t_n$ having distinct factors.

The first few examples are:
\begin{description}
\item[$n=1$:]
$
\begin{array}{l}
t_1
\end{array}
$
\item[$n=2$:]

$
\begin{array}{l}
 t_1 + t_2\\
 t_1 t_2
\end{array}
$
\item[$n=3$:]

$
\begin{array}{l}
 t_1 + t_2 + t_3\\
 t_1 t_2 + t_2 t_3 + t_1 t_3\\
 t_1 t_2 t_3
\end{array}
$
\end{description}
\end{document}
