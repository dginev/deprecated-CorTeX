\documentclass{article}
\usepackage{planetmath-specials}
\usepackage{pmath}
% this is the default PlanetMath preamble.  as your knowledge
% of TeX increases, you will probably want to edit this, but
% it should be fine as is for beginners.

% almost certainly you want these
\usepackage{amssymb}
\usepackage{amsmath}
\usepackage{amsfonts}

% used for TeXing text within eps files
%\usepackage{psfrag}
% need this for including graphics (\includegraphics)
%\usepackage{graphicx}
% for neatly defining theorems and propositions
%\usepackage{amsthm}
% making logically defined graphics
\usepackage{xypic} 
\usepackage{color}

% there are many more packages, add them here as you need them

% define commands here
\newcommand{\fm}[1]{{\it #1}}
\begin{document}
A \emph{complete k-partite graph} is a maximal \fm{k}-partite
graph.  In other words, a graph is complete \fm{k}-partite provided
that its vertex set admits a partition $V=V_1\sqcup\dots\sqcup V_k$
such that for any $u\in V_i$ and $v\in V_j$, \fm{uv} is an edge if and
only if $i\ne j$.

For any tuple $(a_1,\dots,a_k)$, there is up to graph isomorphism a
single complete \fm{k}-partite graph with maximal stable sets
$V_1,\dots, V_k$ such that for each $i$, there are exactly $a_i$
vertices in $V_i$.  This graph is denoted by $K_{a_1,\dots,a_k}$.

Below we display the 3-partite complete graph $K_{2,3,4}$:

\[\xymatrix{
  & {\color{red}A} \ar@{-}[dl] \ar@{-}[ddl] \ar@{-}[dddl] \ar@{-}[ddddl] \ar@{-}[rrrdd] \ar@{-}[rrrddd] & {\color{red}B} \ar@{-}[dll] \ar@{-}[ddll] \ar@{-}[dddll] \ar@{-}[ddddll] \ar@{-}[rrdd] \ar@{-}[rrddd]& {\color{red}C} \ar@{-}[dlll] \ar@{-}[ddlll] \ar@{-}[dddlll] \ar@{-}[ddddlll] \ar@{-}[rdd] \ar@{-}[rddd] &   \\
{\color{blue}D} \ar@{-}[rrrrd] \ar@{-}[rrrrdd] &   &   &   &   \\
{\color{blue}E} \ar@{-}[rrrr] \ar@{-}[rrrrd] &   &   &   & {\color{green}H} \\ 
{\color{blue}F} \ar@{-}[rrrru] \ar@{-}[rrrr] &   &   &   & {\color{green}I} \\
{\color{blue}G} \ar@{-}[rrrruu] \ar@{-}[rrrru] &   &   &   &
}\]

\PMlinkescapeword{complete}
\end{document}
