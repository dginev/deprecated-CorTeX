\documentclass{article}
\usepackage{planetmath-specials}
\usepackage{pmath}
\usepackage{amssymb}
\usepackage{amsmath}
\usepackage{amsfonts}
\begin{document}
Let $\Sigma$ be an alphabet. We then define the following using the
powers of an alphabet and infinite union, where $n \in \mathbb{Z}$.
\begin{eqnarray*}
\Sigma^+ & = & \bigcup_{n=1}^{\infty} \Sigma^n \\
\Sigma^* & = & \bigcup_{n=0}^{\infty} \Sigma^n = \Sigma^+ \cup \{\lambda\}
\end{eqnarray*}
where $\lambda$ is the element called \emph{empty string}.
A \emph{string} is an element of $\Sigma^*$, meaning that
it is a grouping of symbols from $\Sigma$ one after another (via concatenation). For example, $abbc$ is a
string, and $cbba$ is a different string.  A string is also commonly called a \emph{word}.  
$\Sigma^+$, like $\Sigma^*$, contains all finite strings except that
$\Sigma^+$ does not contain the empty string $\lambda$.
Given a string $s\in\Sigma^*$, a string $t$ is a \emph{substring} of $s$ if
$s = utv$ for some strings $u,v\in\Sigma^*$.  For example, $lp, al, ha, alpha$, and $\lambda$ (the empty string) are all substrings of the string $alpha$.

\textbf{Definition}.  A \emph{language} over an alphabet $\Sigma$ is a subset of $\Sigma^*$, meaning that
it is a \PMlinkname{set}{Set} of strings made from the symbols in the alphabet $\Sigma$.

Take for example an alphabet $\Sigma = \{\clubsuit, \wp, 63, a, A \}$.
The following are all languages over $\Sigma$:
\begin{itemize}
\item 
$\{aaa, \lambda, A\wp63, 63\clubsuit, AaAaA\}$, 
\item 
$\{\wp a,\wp aa, \wp aaa, \wp aaaa, \cdots\}$, 
\item The empty set $ \varnothing $.  In the context of languages, $\varnothing$ is called the \emph{empty language}.
\item $\lbrace 63 \rbrace$
\item $\lbrace a^{2n}\mid n\ge 0 \rbrace$
\end{itemize}
A language $L$ is said to be \emph{proper} if the empty string does not belong to it: $\lambda\notin L$.  A proper language is also said to be \emph{$\lambda$-free}.  Otherwise, it is \emph{improper}.  In the examples above, all but the first and the last examples are $\lambda$-free.  $L$ is a \emph{finite language} if $L$ is a finite set, and \emph{atomic} if it is a singleton subset of $\Sigma$, such as the fourth example above.  A language can be arbitrarily formed, or constructed via some set of rules called a formal grammar.

Given a language $L$ over $\Sigma$, the \emph{alphabet of $L$} is defined as the maximal subset $\Sigma(L)$ of $\Sigma$ such that every symbol in $\Sigma(L)$ occurs in some word in $L$.  Equivalently, define the alphabet of a word $w$ to be the set $\Sigma(w)$ of all symbols that occur in $w$, then $\Sigma(L)$ is the union of all $\Sigma(w)$, where $w$ ranges over $L$.

\textbf{Remark}.  A language can also be described in terms of ``infinite'' alphabets.  For example, in model theory, a language is built from a set of symbols, together with a set of variables.  These sets are often infinite.  Another way of generalizing the notion of a language is to allow the strings to have infinite lengths.  The way to do this is to think of a string as a partial function $f$ from some set $X$ to the alphabet $A$ such that $|\operatorname{dom}(f)|<|X|$.  Then the length of a string $f:X\to A$ is just $|\operatorname{dom}(f)|$.  This specializes to the finite case if we take $X$ to be the set of all non-negative integers.
\end{document}
