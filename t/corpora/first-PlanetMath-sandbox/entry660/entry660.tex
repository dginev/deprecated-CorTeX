\documentclass{article}
\usepackage{planetmath-specials}
\usepackage{pmath}
% this is the default PlanetMath preamble.  as your knowledge
% of TeX increases, you will probably want to edit this, but
% it should be fine as is for beginners.

% almost certainly you want these
\usepackage{amssymb}
\usepackage{amsmath}
\usepackage{amsfonts}

% used for TeXing text within eps files
%\usepackage{psfrag}
% need this for including graphics (\includegraphics)
%\usepackage{graphicx}
% for neatly defining theorems and propositions
%\usepackage{amsthm}
% making logically defined graphics
%\usepackage{xypic}

% there are many more packages, add them here as you need them

% define commands here
\begin{document}
A \emph{perfect ruler} of length $n$ is a ruler with a subset of the integer markings  $\{0,a_2,\ldots,n\}\subset\{0,1,2,\ldots,n\}$ that appear on a regular ruler.  The defining criterion of this subset is that there exists an $m$ such that any positive integer $k\leq m$ can be expresses uniquely as a difference $k=a_i-a_j$ for some $i,j$.  This is referred to as an \emph{$m$-perfect ruler}.

A 4-perfect ruler of length $7$ is given by $\{0,1,3,7\}$.  To verify this, we need to show that every number $1,2,\ldots,4$ can be expressed as a difference of two numbers in the above set:

\begin{align*}
1&=1-0\\
2&=3-1\\
3&=3-0\\
4&=7-3\\
\end{align*}

An optimal perfect ruler is one where for a fixed value of $n$ the value of $a_n$ is minimized.
\end{document}
