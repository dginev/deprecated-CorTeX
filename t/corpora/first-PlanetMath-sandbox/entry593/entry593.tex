\documentclass{article}
\usepackage{planetmath-specials}
\usepackage{pmath}
\usepackage{graphicx}
\usepackage{psfrag} 
\usepackage{bbm}
\newcommand{\Z}{\mathbbmss{Z}}
\newcommand{\C}{\mathbbmss{C}}
\newcommand{\R}{\mathbbmss{R}}
\newcommand{\Q}{\mathbbmss{Q}}
\newcommand{\mathbb}[1]{\mathbbmss{#1}}
\newcommand{\figura}[1]{\begin{center}\includegraphics{#1}\end{center}}
\newcommand{\figuraex}[2]{\begin{center}\includegraphics[#2]{#1}\end{center}}
\begin{document}
The \emph{incenter} of a geometrical shape is the center of the
incircle (if it has any).  The radius of the incircle is sometimes
called the \emph{inradius}.

On a triangle the incenter always exists and it is the intersection
point of the three internal angle bisectors. So in the next picture,
$AX,BY,CZ$ are angle bisectors, and $AB,BC,CA$ are tangent to the
circle.\psfrag{A}{$A$}\psfrag{B}{$B$}\psfrag{C}{$C$}\psfrag{X}{$X$}\psfrag{Y}{$Y$}\psfrag{Z}{$Z$}\psfrag{I}{$I$}
\figura{incentre}

\end{document}
