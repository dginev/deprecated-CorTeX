\documentclass{article}
\usepackage{planetmath-specials}
\usepackage{pmath}
\usepackage{amsmath}
\usepackage{amsfonts}
\usepackage{amssymb}
\usepackage{amsthm}

\newcommand{\rank}{\operatorname{rank}}
\newcommand{\Rset}{\mathbb{R}}
\begin{document}
\paragraph{Definition.}
Let $U,V,W$ be vector spaces over a field $K$.  A  \emph{bilinear map}
is a function $B: U
\times V \to W$ such that 
\begin{enumerate}
\item the map $x \mapsto B(x,y)$ from $U$ to $W$ is linear for each $y
  \in V$
\item the map $y \mapsto B(x,y)$ from $V$ to $W$ is linear for each $x
  \in U$.
\end{enumerate}
That is, $B$ is \emph{bilinear} if it is linear in each parameter,
taken separately.  

\paragraph{Bilinear forms.}
A \emph{bilinear form} is a bilinear map $B:V\times V\to K$.  A
$W$-valued bilinear form is a bilinear map $B: V\times V\to W$.  One
often encounters bilinear forms with additional assumptions.  A
bilinear form is called
\begin{itemize}
\item \emph{symmetric} if $B(x,y) = B(y,x)$, $x,y \in V$;
\item \emph{skew-symmetric} if $B(x,y) = -B(y,x)$, $x,y \in V$;
\item \emph{alternating} if $B(x,x) = 0$, $x \in V$.
\end{itemize}
By expanding $B(x+y,x+y) = 0$, we can show alternating implies
skew-symmetric. Further if $K$ is not of characteristic $2$, then
skew-symmetric implies alternating.

\paragraph{Left and Right Maps.}
Let $B:U\times V\to W$ be a bilinear map. We may identify $B$ with the
linear map $B_\otimes: U\otimes V\to W$ (see tensor product).  We may
also identify $B$ with the linear maps
\begin{align*}
  & B_L:U\to L(V,W),\qquad B_L(x)(y) = B(x,y),\; x\in U,\; y\in V;\\
  & B_R:V\to L(U,W),\qquad B_R(y)(x) = B(x,y),\; x\in U,\; y\in V.
\end{align*}
called the left and right map, respectively.  

Next, suppose that $B:V\times V\to K$ is a bilinear form.  Then both
$B_L$ and $B_R$ are linear maps from $V$ to $V^*$, the dual vector
space of $V$. We can therefore say that $B$ is symmetric if and only
if $B_L = B_R$ and that $B$ is anti-symmetric if and only if
$B_L=-B_R$.  If $V$ is finite-dimensional, we can identify $V$ and
$V^{**}$, and assert that $B_L=(B_R)^*$; the left and right maps are, in
fact, dual homomorphisms.


\paragraph{Rank.}
Let $B:U\times V\to K$ be a bilinear form, and suppose that $U,V$ are
finite dimensional. One can show that $\rank B_L = \rank B_R$. We call
this integer $\rank B$, the \emph{\PMlinkescapetext{rank}} of $B$.
Applying the rank-nullity theorem to both the left and right maps
gives the following results:
\begin{align*}
  \dim U &= \dim \ker {B_L} + \rank B\\
  \dim V &= \dim \ker {B_R} + \rank B
\end{align*}
We say that $B$ is
\emph{non-degenerate} if both the left and right map are
non-degenerate.  Note that in \PMlinkescapetext{order} for $B$ to be non-degenerate it is
necessary that $\dim U = \dim V$.  If this holds, then $B$ is
non-degenerate if and only if $\rank B$ is equal to $\dim U, \dim V$.

\paragraph{Orthogonal complements.}
Let $B:V\times V\to K$ be a bilinear form, and let $S \subset V$ be a
subspace.  The left and right orthogonal complements of $S$ are
subspaces ${^\perp}S, S^\perp \subset V$ defined as follows:
\begin{align*}
&{}^\perp S =  \{ u\in V \mid B(u,v) = 0 \; \text{for all } v \in S \},\\
&S^\perp = \{ v\in V \mid B(u,v) = 0 \; \text{for all } u \in S \} .
\end{align*}
We may also realize $S^\perp$ by considering the linear map $B_{R}':
V\to S^*$ obtained as the composition of  $B_R: 
V\to V^*$ and the dual homomorphism $V^*\to S^*$.
Indeed, $S^\perp = \ker B^\prime_R$. An analogous
statement can be made for ${}^\perp S$.  

Next, suppose that $B$ is non-degenerate.
By the rank-nullity
theorem  we have that
\begin{align*}
  \dim V &= \dim S + \dim  S^\perp\\
  &= \dim S + \dim {}^\perp S.   
\end{align*}
Therefore, if $B$ is non-degenerate, then 
\[\dim  S^\perp = \dim {}^\perp S. \]
Indeed, more can be said if $B$ is either symmetric or 
skew-symmetric.  In this case, we actually have
\[{}^\perp S = S^\perp.\]




We say that $S\subset V$ is a non-degenerate subspace relative to $B$
if the restriction of $B$ to $S\times S$ is non-degenerate.  Thus, $S$
is a non-degenerate subspace if and only if $S \cap S^\perp = \{0\}$,
and also $S\cap {}^\perp S = \{ 0\}$.  Hence, if $B$ is non-degenerate
and if $S$ is a non-degenerate
subspace, we have
\[ V = S \oplus S^\perp = S\oplus {}^\perp S.\] Finally, note that if
$B$ is positive-definite, then $B$ is necessarily non-degenerate and
that every subspace is non-degenerate.  In this way we arrive at the
following well-known result: if $V$ is positive-definite inner product
space, then
\[V = S \oplus S^\perp \] for every subspace $S\subset V$.

\paragraph{Adjoints.}
Let $B: V \times V \rightarrow K$ be a non-degenerate bilinear
form, and let  $T \in L(V,V)$ be a linear endomorphism.  We define the
right adjoint $T^\star \in 
L(V,V)$ to be the unique linear map such that
\[ B(Tu, v) = B(u, T^\star v) ,\quad u,v \in V.\]
Letting $T^\ast : V^\ast \to V^\ast$ denote the  dual homomorphism,
we also have
\[ T^\star = {B_R}^{-1} \circ T^\ast \circ B_R.\]
Similarly, we define the left adjoint ${}^\star T\in L(V,V)$ by
\[ {}^\star T = {B_L}^{-1} \circ T^\ast \circ B_L.\] We then have
\[ B(u, Tv) = B({}^\star  T u, v) ,\quad u,v \in V.\]
If $B$ is either symmetric or skew-symmetric, then ${}^\star T =
T^\star$, and we simply use $T^\star$ to refer to the adjoint homomorphism.

\paragraph{Additional remarks.}
\begin{enumerate}
\item 
if $B$ is a symmetric, non-degenerate bilinear form,
then the adjoint operation is represented, relative to an orthogonal
basis (if one exists), by the matrix transpose.

\item  If $B$ is a symmetric, non-degenerate bilinear form
then $T\in L(V,V)$ is then said to be a \emph{normal operator} (with
respect to $B$) if $T$ commutes with its adjoint $T^\star$.


\item An $n \times m$ matrix may be regarded as a bilinear form over
  $K^n\times K^m$. Two such matrices, $B$ and $C$, are said to be
  congruent if there exists an invertible $P$ such that $B = P^{T}CP$.

\item The identity matrix, $I_n$ on $\Rset^n\times \Rset^n$ gives the standard
Euclidean inner product on $\Rset^n$.
\end{enumerate}



\end{document}
