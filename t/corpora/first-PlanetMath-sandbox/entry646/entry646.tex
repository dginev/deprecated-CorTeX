\documentclass{article}
\usepackage{planetmath-specials}
\usepackage{pmath}
% this is the default PlanetMath preamble.  as your knowledge
% of TeX increases, you will probably want to edit this, but
% it should be fine as is for beginners.

% almost certainly you want these
\usepackage{amssymb}
\usepackage{amsmath}
\usepackage{amsfonts}

% used for TeXing text within eps files
%\usepackage{psfrag}
% need this for including graphics (\includegraphics)
%\usepackage{graphicx}
% for neatly defining theorems and propositions
%\usepackage{amsthm}
% making logically defined graphics
%\usepackage{xypic} 

% there are many more packages, add them here as you need them

% define commands here

\newcommand{\norm}[1]{\lVert #1 \rVert}
\begin{document}
\PMlinkescapeword{term}

A \emph{Banach space} $(X,\norm{\,\cdot\,})$ is a normed vector space such that $X$ is complete under the metric induced by the norm $\norm{\,\cdot\,}$.

Some authors use the term Banach space only in the case where $X$ is infinite-dimensional, although on Planetmath finite-dimensional spaces are also considered to be Banach spaces.

If $Y$ is a Banach space and $X$ is any normed vector space, then the set of continuous linear maps $f\colon X\to Y$ forms a Banach space, with norm given by the operator norm.  In particular, since $\mathbb{R}$ and $\mathbb{C}$ are complete, the continuous linear functionals on a normed vector space $\mathcal{B}$ form a Banach space, known as the \emph{dual space} of $\mathcal{B}$.

\emph{Examples:}
\begin{itemize}
\item \PMlinkname{Finite-dimensional normed vector spaces}{EveryFiniteDimensionalNormedVectorSpaceIsABanachSpace}.
\item \PMlinkname{$L^p$ spaces}{LpSpace} are by far the most common example of Banach spaces.
\item \PMlinkname{$\ell^p$ spaces}{Lp} are $L^p$ spaces for the counting measure on $\mathbb{N}$.
\item Continuous functions on a compact set under the supremum norm.
\item \PMlinkname{Finite}{FiniteMeasureSpace} signed measures on a \PMlinkname{$\sigma$-algebra}{SigmaAlgebra}.
\end{itemize}
\end{document}
