\documentclass{article}
\usepackage{ids}
\usepackage{planetmath-specials}
\usepackage{pmath}
\usepackage{amssymb}
\usepackage{amsmath}
\usepackage{amsfonts}
\begin{document}
\PMlinkescapeword{order}
\PMlinkescapeword{theorem}
\PMlinkescapeword{theorems}

If $G$ is a group (or ring, or module) and $H$ and $K$ are normal subgroups (or ideals, or submodules, respectively) of $G$, with $H\subseteq K$, then there is a natural isomorphism $(G/H)/(K/H)\cong G/K$.

This is usually known either as the Third Isomorphism Theorem, or as the Second Isomorphism Theorem (depending on the order in which the theorems are introduced). It is also occasionally called the Freshman Theorem.
\end{document}
