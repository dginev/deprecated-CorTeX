\documentclass{article}
\usepackage{ids}
\usepackage{planetmath-specials}
\usepackage{pmath}
% this is the default PlanetMath preamble.  as your knowledge
% of TeX increases, you will probably want to edit this, but
% it should be fine as is for beginners.

% almost certainly you want these
\usepackage{amssymb}
\usepackage{amsmath}
\usepackage{amsfonts}

% used for TeXing text within eps files
%\usepackage{psfrag}
% need this for including graphics (\includegraphics)
%\usepackage{graphicx}
% for neatly defining theorems and propositions
%\usepackage{amsthm}
% making logically defined graphics
%\usepackage{xypic} 

% there are many more packages, add them here as you need them

% define commands here
\newcommand{\GL}{\operatorname{GL}}
\begin{document}
Given a group $G$, the {\em regular representation} of $G$ over a field $K$ is the representation $\rho: G \longrightarrow \GL(K^G)$ whose underlying vector space $K^G$ is the $K$--vector space of formal linear combinations of elements of $G$, defined by
$$
\rho(g)\left(\sum_{i=1}^n k_i g_i\right) := \sum_{i=1}^n k_i (g g_i)
$$
for $k_i \in K$, $g, g_i \in G$.

Equivalently, the regular representation is the induced representation on $G$ of the trivial representation on the subgroup $\{1\}$ of $G$.
\end{document}
