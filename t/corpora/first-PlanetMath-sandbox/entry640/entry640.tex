\documentclass{article}
\usepackage{planetmath-specials}
\usepackage{pmath}
% this is the default PlanetMath preamble.  as your knowledge
% of TeX increases, you will probably want to edit this, but
% it should be fine as is for beginners.

% almost certainly you want these
\usepackage{amssymb}
\usepackage{amsmath}
\usepackage{amsfonts}

% used for TeXing text within eps files
%\usepackage{psfrag}
% need this for including graphics (\includegraphics)
%\usepackage{graphicx}
% for neatly defining theorems and propositions
%\usepackage{amsthm}
% making logically defined graphics
%\usepackage{xypic} 

% there are many more packages, add them here as you need them

% define commands here
\begin{document}
Let $G$ be a group, and let $V$ be a vector space. A \emph{representation} of $G$ in $V$ is a group homomorphism $\rho\colon G \to \operatorname{GL}(V)$ from $G$ to the general linear group $\operatorname{GL}(V)$ of invertible linear transformations of $V$.

Equivalently, a representation of $G$ is a vector space $V$ which is a $G$-module, that is, a (left) module over the group ring $\mathbb{Z}[G]$. The equivalence is achieved by assigning to each homomorphism $\rho\colon G \to \operatorname{GL}(V)$ the module structure whose scalar multiplication is defined by $g \cdot v := (\rho(g))(v)$, and extending linearly. Note that, although technically a group representation is a homomorphism such as $\rho$, most authors invariably denote a representation using the underlying vector space $V$, with the homomorphism being understood from context, in much the same way that vector spaces themselves are usually described as sets with the corresponding binary operations being understood from context.

\paragraph{Special kinds of representations}

(preserving all notation from above)

A representation is \emph{faithful} if either of the following equivalent conditions is satisfied:
\begin{itemize}
\item $\rho\colon G \to \operatorname{GL}(V)$ is injective,
\item $V$ is a faithful left $\mathbb{Z}[G]$--module.
\end{itemize}

A \emph{subrepresentation} of $V$ is a subspace $W$ of $V$ which is a left $\mathbb{Z}[G]$--submodule of $V$;
 such a subspace is sometimes called a $G$-invariant subspace of $V$. Equivalently, a subrepresentation of $V$ is a subspace $W$ of $V$ with the property that
$$
(\rho(g))(w) \in W \text{ for all } g \in G \text{ and } w \in W.
$$

A representation $V$ is called {\em irreducible} if it has no subrepresentations other than itself and the zero module.
\end{document}
