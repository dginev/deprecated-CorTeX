\documentclass{article}
\usepackage{planetmath-specials}
\usepackage{pmath}
\usepackage{amssymb}
\usepackage{amsmath}
\usepackage{amsfonts}

\def\R{\mathbb{R}}
\def\C{\mathbb{C}}

\begin{document}
\PMlinkescapeword{continuous}
\PMlinkescapeword{equivalent}
\PMlinkescapeword{independent}
\PMlinkescapeword{inverse}
\PMlinkescapeword{scalar}
\PMlinkescapeword{vector}

\section*{Definition}

A \emph{topological vector space} is a pair $(V,\mathcal{T})$,
where $V$ is a vector space over a topological field $K$,
and $\mathcal{T}$ is a topology on $V$ such that under $\mathcal{T}$
the scalar multiplication $(\lambda,v)\mapsto\lambda v$
is a continuous function $K\times V\to V$
and the vector addition $(v,w)\mapsto v+w$
is a continuous function $V\times V\to V$,
where $K\times V$ and $V\times V$ are given the respective product topologies.

We will also require that $\{0\}$ is closed
(which is equivalent to requiring the topology to be Hausdorff),
though some authors do not make this requirement.
Many authors require that $K$ be either $\R$ or $\C$
(with their usual topologies).

\section*{Topological vector spaces as topological groups}

A topological vector space is necessarily a topological group:
the definition ensures that the group operation (vector addition) is continuous,
and the inverse operation is the same as multiplication by $-1$,
and so is also continuous.

\section*{Finite-dimensional topological vector spaces}

A finite-dimensional vector space inherits a natural topology.
For if $V$ is a finite-dimensional vector space,
then $V$ is isomorphic to $K^n$ for some $n$;
then let $f\colon V\to K^n$ be such an isomorphism,
and suppose that $K^n$ has the product topology.
Give $V$ the topology where a subset $A$ of $V$ is open in $V$
if and only if $f(A)$ is open in $K^n$.
This topology is independent of the choice of isomorphism $f$,
and is the \PMlinkname{finest}{Coarser} topology on $V$
that makes it into a topological vector space.
\end{document}
