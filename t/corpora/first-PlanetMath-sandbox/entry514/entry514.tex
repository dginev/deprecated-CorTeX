\documentclass{article}
\usepackage{planetmath-specials}
\usepackage{pmath}
\def\ip#1{{\left\langle #1\right\rangle}}

\begin{document}
\PMlinkescapeword{bases}
\PMlinkescapeword{clear}
\PMlinkescapeword{columns}
\PMlinkescapeword{origin}
\PMlinkescapeword{orthogonal}
\PMlinkescapeword{rows}
\PMlinkescapeword{term}
\PMlinkescapeword{unit}

\section*{Definition}

An \emph{orthonormal set} is a subset $S$ of an inner product space,
such that $\ip{x,y}=\delta_{xy}$ for all $x,y\in S$.
Here $\ip{\cdot,\cdot}$ is the inner product,
and $\delta$ is the Kronecker delta.

More verbosely, we may say that an orthonormal set
is a subset $S$ of an inner product space
such that the following two conditions hold:
\begin{enumerate}
\item If $x,y \in S$ and $x\ne y$, then $x$ is \PMlinkname{orthogonal}{OrthogonalVector} to $y$.
\item If $x \in S$, then the norm of $x$ is $1$.
\end{enumerate}
Stated this way, the origin of the term is clear:
an orthonormal set of vectors is both orthogonal and normalized.

\section*{Notes}

Note that the empty set is orthonormal,
as is a set consisting of a single vector of unit norm
in an inner product space.

The columns (or rows) of a real orthogonal matrix form an orthonormal set.
In fact, this is an example of an orthonormal basis.

\section*{Applications}

A standard application is finding an orthonormal basis for a vector space,
such as by Gram-Schmidt orthonormalization.
Orthonormal bases are computationally simple to work with.

\end{document}
