\documentclass{article}
\usepackage{planetmath-specials}
\usepackage{pmath}
%\usepackage{amssymb}
%\usepackage{amsmath}
%\usepackage{amsfonts}
\begin{document}
\PMlinkescapeword{equivalent}

If $X$ is a partially ordered set
such that every chain in $X$ has an upper bound,
then $X$ has a maximal element.

Note that the empty chain in $X$ has an upper bound in $X$
if and only if $X$ is non-empty.
Because this case is rather different from the case of non-empty chains,
Zorn's Lemma is often stated in the following form:
If $X$ is a non-empty partially ordered set
such that every non-empty chain in $X$ has an upper bound,
then $X$ has a maximal element.
(In other words: Any non-empty inductively ordered set has a maximal element.)

In ZF, Zorn's Lemma is equivalent to the \PMlinkname{Axiom of Choice}{AxiomOfChoice}.

\end{document}
