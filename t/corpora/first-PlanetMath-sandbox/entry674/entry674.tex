\documentclass{article}
\usepackage{planetmath-specials}
\usepackage{pmath}
\usepackage{amsmath}
\usepackage{amsfonts}
\usepackage{amssymb}

\newcommand{\reals}{\mathbb{R}}
\begin{document}
Let $I\subset R$ be an interval and let $\gamma:I\to\reals^3$ be a
parameterized space curve, assumed to be
\PMlinkname{regular}{SpaceCurve} and free of points of inflection. We
interpret $\gamma(t)$ as the trajectory of a particle moving through
3-dimensional space. Let $T(t), N(t), B(t)$ denote the corresponding
moving trihedron. The speed of this particle is given by $\Vert
\gamma'(t) \Vert$.

In order for a moving particle to escape the osculating plane, it is
necessary for the particle to ``roll'' along the axis of its tangent
vector, thereby lifting the normal acceleration vector out of the
osculating plane. The ``rate of roll'', that is to say the rate at
which the osculating plane rotates about the tangent vector, is given
by $B(t)\cdot N'(t)$; it is a number that depends on the
speed of the particle. The rate of roll relative to the particle's
speed is the quantity
$$\tau(t) = \frac{B(t)\cdot N'(t)}{\Vert \gamma'(t)\Vert}= \frac{(
\gamma'(t)\times \gamma''(t)) \cdot \gamma'''(t)}{\Vert \gamma'(t)\times
\gamma''(t)\Vert^2 },$$
called the torsion of the curve, a quantity
that is invariant with respect to reparameterization. The torsion
$\tau(t)$ is, therefore, a measure of an intrinsic property of the
oriented space curve, another real number that can be covariantly
assigned to the point $\gamma(t)$.
\end{document}
