\documentclass{article}
\usepackage{planetmath-specials}
\usepackage{pmath}
\usepackage{amssymb}
\usepackage{amsmath}
\usepackage{amsfonts}
\usepackage{graphicx}
\usepackage{xypic}
\begin{document}
\section{QR Decomposition}

Orthogonal matrix triangularization (QR decomposition) reduces a real $m \times n$ matrix $A$ with $m \ge n$  and full rank to a much simpler form. It guarantees numerical stability by minimizing errors caused by machine roundoffs. A suitably chosen orthogonal matrix $Q$ will triangularize the given matrix:

$$ A = Q \begin{bmatrix} R \\ 0 \end{bmatrix} $$

with the $n \times n$ right triangular matrix $R$. One only has then to solve the triangular system $Rx = Pb$, where $P$ consists of the first $n$ rows of $Q$.

The least squares problem $Ax \approx b$ is easy to solve with $A = QR$ and $Q$ an orthogonal matrix (here and henceforth $R$ is the entire $ m \times n $ augmented matrix from above). The solution

$$ x = (A^TA)^{-1} A^Tb $$

becomes

$$ x = (R^TQ^TQR)^{-1}R^TQ^Tb = (R^TR)^{-1}R^TQ^Tb = R^{-1}Q^Tb $$

This is a matrix-vector multiplication $Q^Tb$, followed by the solution of the triangular system $Rx = Q^Tb$ by back-substitution. The QR factorization saves us the formation of $A^TA$ and the solution of the normal equations.

Many different methods exist for the QR decomposition, e.g. the Householder transformation, the Givens rotation, or the Gram-Schmidt decomposition.

\begin{thebibliography}{3}

\bibitem{DAB} The Data Analysis Briefbook. \PMlinkexternal{http://rkb.home.cern.ch/rkb/titleA.html}{http://rkb.home.cern.ch/rkb/titleA.html}

\end{thebibliography}
\end{document}
