\documentclass{article}
\usepackage{ids}
\usepackage{planetmath-specials}
\usepackage{pmath}
\usepackage{amssymb}
\usepackage{amsmath}
\usepackage{amsfonts}
\usepackage{graphicx}
\usepackage{xypic}
\begin{document}
Let $U \subset \mathbb{C}$ be a domain and let $a \in \mathbb{C}$. A function $f\colon U \to \mathbb{C}$ has a \emph{pole} at $a$ if it can be represented by a Laurent series centered about $a$ with only finitely many terms of negative exponent; that is,
\[
f(z) = \sum_{k=-n}^\infty c_k (z-a)^k
\]
in some nonempty deleted neighborhood of $a$, with $c_{-n} \neq 0$, for some $n \in \mathbb{N}$. The number $n$ is called the \emph{order} of the pole. A \emph{simple pole} is a pole of order 1.
\end{document}
