\documentclass{article}
\usepackage{planetmath-specials}
\usepackage{pmath}
\usepackage{amssymb}
\usepackage{amsmath}
\usepackage{amsfonts}


\newtheorem{proposition}{Proposition}
\newtheorem{definition}[proposition]{Definition}
\newtheorem{theorem}[proposition]{Theorem}
\newcommand{\rT}{{\scriptscriptstyle \mathrm{T}}}

\newcommand{\row}{\mathrm{row}}
\begin{document}
\begin{definition}
  Let $A$ be an $n\times m$ matrix.  An LU decomposition (sometimes also called an LU factorization) of $A$, if it
  exists, is an $n\times n$ unit lower triangular matrix $L$ and an
  $n\times m$ matrix U, in (upper) echelon form, such that
  \[ A = LU \]
\end{definition}
The LU factorization is closely related to the row reduction
algorithm.  In a very real sense, the factorization is a record of the
steps taken in row reducing a matrix to echelon form.  The  matrix
$L$ ``encodes'' the sequence of row replacement operations that row
reduce the given matrix $A$ to echelon form $U$.
\begin{proposition}
  Suppose that $A=LU$ is an LU factorization, and let $L_{ij}$ denote
  the entries of $L$.  Then, the row reduction
  $A\stackrel{\rho}{\Longrightarrow} U$ is accomplished by the
  following sequence $\rho$ of $\tfrac{1}{2}n(n-1)$ row replacement
  operations:
  \begin{align*}
    \hskip 2em & \row_j \mapsto  \row_j-L_{j1}\; \row_1,\quad
    j=2,\ldots, n;\\ 
     & \row_j \mapsto  \row_j-L_{j2}\; \row_2,\quad j=3,\ldots, n;\\
    & \hskip  3em \vdots \\
    & \row_j \mapsto  \row_j-L_{jk}\; \row_k,\quad j=k+1,\ldots,
    n,\quad k=1,\ldots,n-1 \\  
    & \hskip 3em \vdots 
  \end{align*}
\end{proposition}
Note: the first $n-1$ steps clear out column $1$, the next $n-2$ steps
clear out column $2$, etc.
\begin{proposition}
  Not every matrix admits an LU factorization.  Indeed, an LU
  factorization exists if and only if $A$ can be reduced to echelon
  without using row exchange operations.  However, if an LU
  factorization exists, then it is unique.
\end{proposition}


In the most general case, one has to employ row exchange operations.
\begin{proposition}
  Let $A$ be an $n\times m$ matrix.  Then, there exists an $n\times n$
  permutation matrix $P$ (indeed, many such) such that the matrix $PA$
  admits an LU factorization, i.e., there exist matrices $P, L, U$
  such that
  \[ PA=LU. \]
\end{proposition}
The key idea behind LU factorization is that one does not need to
employ row scalings to do row reduction until the second half (the
back-substitution phase) of the algorithm.  This has significant
implication for numerical stability of the algorithm.

The LU decomposition of a given matrix $A$ is useful for the solution
of the  systems of linear equations of the form $Ax = b$.  Indeed, it
suffices to first solve the linear system $Ly=b$, and second, to solve
the system $Ux=y$.  This two-step procedure is easy to implement,
because owing to the lower and upper-triangular nature of  the
matrices $L$ and $U$, the required row operations are determined, more
or less, directly from the entries of $L$ and $U$.  Indeed, the first
step, 
\[ [\, L\; b\, ] \stackrel{\rho_1}{\Longrightarrow}  [\, I \; y \,] \]
a sequence of row operations $\rho_1$, and the second step
\[ [\, U\; y\, ] \stackrel{\rho_2}{\Longrightarrow}  [\, E \; x_0 \,] \]
a sequence of row operations $\rho_2$, are exactly the same row operations 
one has to perform to row reduce $A$ directly to reduced echelon form $E$:
\[ 
[\, A\; b\, ] \stackrel{\rho_1}{\Longrightarrow} [\, U\; y \, ] 
\stackrel{\rho_2}{\Longrightarrow} [\, E\; x_0 \, ] .
\]
Note: $x_0$ is the particular solution of $Ax=b$  that sits in the
rightmost colulmn of the augmented matrix at the termination of the
row-reduction algorithm.
\end{document}
