\documentclass{article}
\usepackage{ids}
\usepackage{planetmath-specials}
\usepackage{pmath}
% this is the default PlanetMath preamble.  as your knowledge
% of TeX increases, you will probably want to edit this, but
% it should be fine as is for beginners.

% almost certainly you want these
\usepackage{amssymb}
\usepackage{amsmath}
\usepackage{amsfonts}

% used for TeXing text within eps files
%\usepackage{psfrag}
% need this for including graphics (\includegraphics)
%\usepackage{graphicx}
% for neatly defining theorems and propositions
%\usepackage{amsthm}
% making logically defined graphics
%\usepackage{xypic}

% there are many more packages, add them here as you need them

% define commands here
\begin{document}
For a matrix to be in \emph{reduced row echelon form} (or \emph{Hermite normal form}) it has to first satisfy the requirements to be in row echelon form and additionally satisfy the following requirements:

\begin{enumerate}
\item The first non-zero element in any row must be 1.
\item The first element of value 1 in any row must be the only non-zero value in its column.
\end{enumerate}
An example of a matrix in reduced row echelon form could be:
\begin{displaymath}
\left( \begin{array}{cccccccccc}
0 & 1 & 2 & 6 & 0 & 1 & 0 & 0 & 4 & 0\\
0 & 0 & 0 & 0 & 1 & 1 & 0 & 0 & 1 & 1\\
0 & 0 & 0 & 0 & 0 & 0 & 1 & 0 & 4 & 1\\
0 & 0 & 0 & 0 & 0 & 0 & 0 & 1 & 2 & 1\\
0 & 0 & 0 & 0 & 0 & 0 & 0 & 0 & 0 & 0\\
\end{array}
\right)
\end{displaymath}
\end{document}
