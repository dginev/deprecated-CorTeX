\documentclass{article}
\usepackage{planetmath-specials}
\usepackage{pmath}
% this is the default PlanetMath preamble.  as your 
% of TeX increases, you will probably want to edit this, but
% it should be fine as is for beginners.

% almost certainly you want these
\usepackage{amssymb}
\usepackage{amsmath}
\usepackage{amsthm}
\usepackage{amsfonts}

% used for TeXing text within eps files
%\usepackage{psfrag}
% need this for including graphics (\includegraphics)
%\usepackage{graphicx}
% for neatly defining theorems and propositions
%\usepackage{amsthm}
% making logically defined graphics
%\usepackage{xypic}

% there are many more packages, add them here as you need them

% define commands here

\newtheorem*{thm}{Theorem}
\newtheorem{defn}{Definition}
\newtheorem{prop}{Proposition}
\newtheorem{lemma}{Lemma}
\newtheorem*{cor}{Corollary}

\theoremstyle{definition}
\newtheorem{exa}{Example}

% Some sets
\newcommand{\Nats}{\mathbb{N}}
\newcommand{\Ints}{\mathbb{Z}}
\newcommand{\Reals}{\mathbb{R}}
\newcommand{\Complex}{\mathbb{C}}
\newcommand{\Rats}{\mathbb{Q}}
\newcommand{\Gal}{\operatorname{Gal}}
\newcommand{\Cl}{\operatorname{Cl}}
\begin{document}
\begin{defn}
Let $A$ and $B$ be integers such that $(A,B)=1$ with $AB\neq \pm 1$. A prime $p$ is called a \emph{primitive divisor} of $A^n-B^n$ if $p$ divides $A^n-B^n$ but $A^m-B^m$ is not divisible by $p$ for all positive integers $m$ that are less than $n$.
\end{defn}

Or, more generally:

\begin{defn}
Let $A$ and $B$ be algebraic integers in a number field $K$ such that $(A,B)=1$ and $A/B$ is not a root of unity. A prime ideal $\wp$ of $K$ is called a \emph{primitive divisor} of $A^n-B^n$ if $\wp|A^n-B^n$ but $\wp \nmid A^m-B^m$ for all positive integers $m$ that are less than $n$.
\end{defn}

The following theorem is due to A. Schinzel (see \cite{sch}):

\begin{thm}
Let $A$ and $B$ be as before. There is an effectively computable constant $n_0$, depending only on the degree of the algebraic number $A/B$, such that $A^n-B^n$ has a primitive divisor for all $n>n_0$.
\end{thm}

By putting $B=1$ we obtain the following corollary:

\begin{cor}
Let $A\neq 0,\pm 1$ be an integer. There exists a number $n_0$ such that $A^n-1$ has a primitive divisor for all $n>n_0$. In particular, for all but finitely many integers $n$, there is a prime $p$ such that the multiplicative order of $A$ modulo $p$ is exactly $n$.
\end{cor}

\begin{thebibliography}{8}
\bibitem{sch} A. Schinzel, {\em Primitive divisors of the expression $A^n-B^n$ in algebraic number fields}.
Collection of articles dedicated to Helmut Hasse on his seventy-fifth birthday, II. J. Reine Angew. Math. 268/269 (1974), 27--33.
\end{thebibliography}
\end{document}
