\documentclass{article}
\usepackage{ids}
\usepackage{planetmath-specials}
\usepackage{pmath}
\usepackage{graphicx}
%\usepackage{xypic} 
\usepackage{bbm}
\newcommand{\Z}{\mathbbmss{Z}}
\newcommand{\C}{\mathbbmss{C}}
\newcommand{\R}{\mathbbmss{R}}
\newcommand{\Q}{\mathbbmss{Q}}
\newcommand{\mathbb}[1]{\mathbbmss{#1}}
\newcommand{\figura}[1]{\begin{center}\includegraphics{#1}\end{center}}
\newcommand{\figuraex}[2]{\begin{center}\includegraphics[#2]{#1}\end{center}}
\newtheorem{dfn}{Definition}
\begin{document}
\textbf{Rectangle}.
A quadrilateral whose four angles are equal, that is, whose 4 angles are equal to $90^\circ$.

\figura{rectangulo}


\emph{\bf Any rectangle is a parallelogram.}\\
This follows from angles $\angle BAD $ and $\angle ADC$ adding up $180^\circ$.

Since parallelograms have their opposite sides equal, so do rectangles. In the picture, $AB=CD$ and $BC=DA$.

Rectangles are the only parallelograms to be also cyclic (since opposite angles add up $180^\circ$.

Notice that every square is also a rectangle, but there are rectangles that are not squares

Rectangles have their two diagonals equal (since triangles $ABC$ and $ABD$ are congruent), A nice result following from this is that the quadrilateral obtained by joining the midpoints of the sides is a rhombus.

\figura{rectangulo2}

Since $PQ$ joins midpoints of sides in triangle $ABC$, we have $PQ=CA/2$. Similarly we have $RS=CA/2$, $QR=BD/2$ and $SP=BD/2$ and thus the sides of quadrilateral $PQRS$ are all equal, in other words, $PQRS$ is a rhombus.
\end{document}
