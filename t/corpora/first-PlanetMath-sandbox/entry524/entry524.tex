\documentclass{article}
\usepackage{planetmath-specials}
\usepackage{pmath}
\usepackage{amssymb}
\usepackage{amsmath}
\usepackage{amsfonts}
\usepackage{graphicx}
\usepackage{xypic}
\begin{document}
An ordinal number is a well ordered set $S$ such that, for every $x \in S$,
$$
x = \{z \in S \mid z < x\}
$$
(where $<$ is the ordering relation on $S$).

It follows immediately from the definition that every ordinal is a transitive set.
Also note that if $a,b\in S$ then we have $a<b$ if and only if $a\in b$.

There is a theory of ordinal arithmetic which allows construction of various ordinals.
For example, all the numbers $0$, $1$, $2$, \ldots
have natural interpretations as ordinals,
as does the set of natural numbers (including $0$),
which in this context is often denoted by $\omega$,
and is the first infinite ordinal.
\end{document}
