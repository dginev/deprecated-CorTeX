\documentclass{article}
\usepackage{planetmath-specials}
\usepackage{pmath}
% this is the default PlanetMath preamble.  as your knowledge
% of TeX increases, you will probably want to edit this, but
% it should be fine as is for beginners.

% almost certainly you want these
\usepackage{amssymb}
\usepackage{amsmath}
\usepackage{amsfonts}

% used for TeXing text within eps files
%\usepackage{psfrag}
% need this for including graphics (\includegraphics)
%\usepackage{graphicx}
% for neatly defining theorems and propositions
%\usepackage{amsthm}
% making logically defined graphics
%\usepackage{xypic} 
\input xy
\xyoption{all}

% there are many more packages, add them here as you need them

% define commands here
\begin{document}
A \emph{directed graph} or \emph{digraph} is a pair $G=(V,E)$ where $V$ is a set of \emph{vertices} and $E$ is a subset of $V \times V$ called \emph{edges} or \emph{arcs}.

If $E$ is symmetric (i.e., $(u,v) \in E$ if and only if $(v,u) \in E$), then the digraph is isomorphic to an ordinary (that is, undirected) graph.

Digraphs are generally drawn in a similar manner to graphs with arrows on the edges to indicate a sense of direction.  For example, the digraph $$\left (\{a,b,c,d\}, \{(a,b),(b,d),(b,c),(c,b),(c,c),(c,d)\}\right )$$ may be drawn as

$$\xymatrix{
& a \ar[r] & b \ar[d] \ar[dl] \\
& c \ar[ur] \ar[r] \ar@(dr,dl)[] & d \\
& & \\
}$$

Since the graph is directed, one has the concept of the number of edges originating or terminating at a given vertex $v$. The out-degree, $d_{\textrm{out}}(v)$ of a vertex $v$ is the number of edges having $v$ as their originating vertex; similarly, the in-degree, $d_{\textrm{in}}(v)$ is the number of edges having $v$ as their terminating vertex.

If the graph has a finite number of vertices, say $v_1,\ldots,v_n$, then obviously
\[\sum_{i=1}^n d_{\textrm{in}}(v_i)=\sum_{i=1}^n d_{\textrm{out}}(v_i)\]

A \emph{directed path} in a digraph $G$ is a sequence of edges $e_1,\ldots,e_k$ such that the end vertex of $e_i$ is the start vertex of $e_{i+1}$ for $i=1,2,\ldots,k-1$. Such a path is called a \emph{directed circuit} if, in addition, the end vertex of $e_k$ is the start vertex of $e_1$.

A digraph is \emph{connected} (or, sometimes, \emph{strongly connected}) if for every pair of vertices $u$ and $v$ there is a directed path from $u$ to $v$. In addition, a digraph $G=(V,E)$ is said to have a \emph{root} $r\in V$ if every vertex $v\in V$ is reachable from $r$, i.e. if there is a directed path from $r$ to $v$.

A digraph is called a \emph{directed tree} if it has a root and if the underlying (undirected) graph is a tree. That is, it must morphologically look like a tree, and the structure imposed by the directional arrows must flow ``away'' from the root.

If $H$ is a subgraph of a digraph $G$, then $H$ is said to be a \emph{directed spanning tree} of $G$ if $H$ is a directed tree and $H$ contains all vertices of $G$. This is a direct analog of the concept of spanning trees for undirected graphs. Note that if $r$ is the root of $H$, then $r$ is clearly a root of $G$. Also, if $r$ is any root of $G$, it is possible to construct a directed spanning tree of $G$ with root $r$: construct $H$ edge by edge starting from $r$. At each vertex, add any edge from $G$ whose terminus is a vertex not yet in $H$. Since $r$ is a root of $G$, this process is guaranteed to include each vertex in $G$; since we are choosing at each step only vertices not yet visited, we are guaranteed to end up with a tree.
\end{document}
