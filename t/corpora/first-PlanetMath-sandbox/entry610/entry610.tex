\documentclass{article}
\usepackage{planetmath-specials}
\usepackage{pmath}
\usepackage{amssymb}
\usepackage{amsmath}
\usepackage{amsfonts}

%\usepackage{psfrag}
%\usepackage{graphicx}
%\usepackage{xypic}

\newcommand{\ccj}[1]{\overline{#1}}
\newcommand{\trace}[0]{\operatorname{trace}}
\begin{document}
\section{Definition}
\subsection{Scalar Complex Conjugate}

Let $z$ be a complex number with real part $a$ and imaginary part $b$,

$$ z = a+bi $$

Then the \emph{complex conjugate} of $z$ is 

$$ \bar{z} = a - bi $$

Complex conjugation represents a reflection about the real axis on the Argand diagram representing a complex number.

Sometimes a star ($*$) is used instead of an overline, e.g. in physics you might see

$$ \int_{-\infty}^{\infty}\Psi^{*}\Psi dx=1 $$

where $\Psi^*$ is the complex conjugate of a wave \PMlinkescapetext{function}.

\subsection{Matrix Complex Conjugate}

Let $A=(a_{ij})$ be a $n\times m$ matrix with complex
entries.  Then the \emph{complex conjugate} of $A$ is the matrix
$\ccj{A}=(\ccj{a_{ij}})$. In particular, if
$v=(v^1, \ldots, v^n)$ is a complex row/column vector, then
$\ccj{v}=(\ccj{v^1}, \ldots, \ccj{v^n})$.

Hence, the matrix complex conjugate is what we would expect: the same matrix with all of its scalar components conjugated.

\section{Properties of the Complex Conjugate}

\subsection{Scalar Properties}

If $u,v$ are complex numbers, then

\begin{enumerate}
 \item $\ccj{uv}= (\ccj{u})(\ccj{v})$
 \item $\ccj{u+v}= \ccj{u}+\ccj{v}$
 \item $\big(\ccj{u}\big)^{-1} = \ccj{u^{-1}}$
 \item $\ccj{(\ccj{u})} = u$
 \item If $v\neq 0$, then $\ccj{(\frac{u}{v})} = {\ccj{u}}/{\ccj{v}}$
 \item Let $u = a + bi$.  Then $\ccj{u} u = u \ccj{u} = a^2+b^2 \ge 0$ (the complex modulus).
 \item If $z$ is written in polar form as $z=r e^{i\phi}$, then
$\ccj{z}=re^{-i\phi}$.
\end{enumerate}

\subsection{Matrix and Vector Properties}

Let $A$ be a matrix with complex entries, and let $v$ be a complex row/column vector. 

Then
\begin{enumerate}
 \item $\ccj{A^T}=\big(\ccj{A}\big)^T$
 \item $\ccj{Av}=\ccj{A}\ccj{v}$, and $\ccj{vA}=\ccj{v}\ccj{A}$. (Here we assume that $A$ and $v$ are
compatible size.)
\end{enumerate}

Now assume further that $A$ is a complex square matrix, then

\begin{enumerate}
 \item $\trace \ccj{A} = \ccj{(\trace\ A)}$
 \item $\det \ccj{A} = \ccj{(\det A)}$
 \item $\big(\ccj{A}\big)^{-1} = \ccj{A^{-1}}$
\end{enumerate}
\end{document}
