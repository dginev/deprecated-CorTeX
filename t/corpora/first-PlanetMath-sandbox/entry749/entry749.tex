\documentclass{article}
\usepackage{planetmath-specials}
\usepackage{pmath}
% this is the default PlanetMath preamble.  as your knowledge
% of TeX increases, you will probably want to edit this, but
% it should be fine as is for beginners.

% almost certainly you want these
\usepackage{amssymb}
\usepackage{amsmath}
\usepackage{amsfonts}

% used for TeXing text within eps files
%\usepackage{psfrag}
% need this for including graphics (\includegraphics)
%\usepackage{graphicx}
% for neatly defining theorems and propositions
%\usepackage{amsthm}
% making logically defined graphics
%\usepackage{xypic} 

% there are many more packages, add them here as you need them

% define commands here
\begin{document}
A morphism $f \colon A \to B$ in a category is called a {\em monic} morphism, or \emph{monomorphism}, if it can be cancelled from the left --- for any object $C$ and any morphisms $g_1, g_2 \colon\ C\to A$ we have $f \circ g_1 = f \circ g_2$ if and only if $g_1 = g_2$.

A morphism $f : A \to B$ in a category is called a \emph{split monomorphism}
if there exists a morphism $g \colon B \to A$ such that $g \circ f = \operatorname{id}_A$.  Note that every split monomorphism is a monomorphism;
if $f$ is a split monomorphism and $f \circ h = f \circ k$, then one has
$g \circ (f \circ h) = g \circ (f \circ k)$.  By associativity, $(g \circ f)
\circ h = (g \circ f) \circ k$; by definition of split monomorphism,
$\operatorname{id}_a \circ h = \operatorname{id}_a \circ k$; by definition of
identity, $h = k$, so $f$ is a monomorphism.  Split monomorphisms are also
known as \emph{sections} and \emph{coretractions}.

The notion of epimorphism is dual to that of monomorphism.  An epimorphism of
a category is a monomorphism of the dual category and vice versa.

A monomorphism in the category of sets is simply a one-to-one function.
Moreover, in the category of sets all monomorphisms are split monomorphisms.
\end{document}
