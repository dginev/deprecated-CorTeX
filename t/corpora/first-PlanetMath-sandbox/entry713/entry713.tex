\documentclass{article}
\usepackage{planetmath-specials}
\usepackage{pmath}
\usepackage{amssymb}
\usepackage{amsmath}
\usepackage{amsfonts}
\begin{document}
\newcommand{\B}{\mathcal{B}}
\newcommand{\Bstar}{\mathcal{B}^\ast}
\newcommand{\Vstar}{V^{\ast}}
\newcommand{\Vstarstar}{V^{\ast\ast}}
\PMlinkescapeword{simple}
\PMlinkescapeword{pointwise}
\PMlinkescapeword{representation}

\textbf{Dual of a vector space; dual bases}

Let $V$ be a vector space over a field $k$. The \emph{dual} of $V$,
denoted by $\Vstar$, is the vector space of linear forms on $V$, i.e.
linear mappings $V\to k$.
The operations in $\Vstar$ are defined pointwise:
$$(\varphi + \psi )(v) = \varphi (v) + \psi (v) $$
$$(\lambda\varphi )(v) = \lambda\varphi (v)$$
for $\lambda\in K$, $v\in V$ and $\varphi,\psi\in\Vstar$.

$V$ is isomorphic to $\Vstar$ if and only if the dimension of
$V$ is finite. If not, then $\Vstar$ has a larger (infinite)
dimension than $V$; in other words, the cardinal of any basis
of $\Vstar$ is strictly greater than the cardinal of any basis of $V$.

Even when $V$ is finite-dimensional, there is no canonical or natural
isomorphism $V\to\Vstar$. But on the other hand, a basis
$\B$ of $V$ does define a basis $\Bstar$ of $\Vstar$, and moreover a
bijection $\B\to\Bstar$. For suppose
$\B=\{b_1,\dots,b_n\}$. For each $i$ from $1$ to $n$, define a mapping
$$\beta_i:V\to k$$
by
$$\beta_i(\sum_k x_k b_k)=x_i\;.$$
It is easy to see that the $\beta_i$ are nonzero elements of $\Vstar$
and are independent. Thus $\{\beta_1,\dots,\beta_n\}$ is a basis of
$\Vstar$, called the dual basis of $\B$.

The dual of $\Vstar$ is called the \emph{second dual} or \emph{bidual} of $V$.
There \emph{is} a very simple canonical injection $V\to\Vstarstar$,
and it is an isomorphism if the dimension of $V$ is finite.
To see it, let $x$ be any element of $V$ and define a mapping $x':\Vstar\to k$
simply by
$$x'(\phi)=\phi(x)\;.$$
$x'$ is linear by definition, and it is readily verified that the mapping
$x\mapsto x'$ from $V$ to $\Vstarstar$ is linear and injective.

\textbf{Dual of a topological vector space}

If $V$ is a topological vector space, the \emph{continuous dual}
$V^{\prime}$ of $V$ is the subspace of $\Vstar$ consisting of
the \emph{continuous} linear forms.

A \emph{normed} vector space $V$ is said to be \emph{reflexive} if the natural
embedding $V\to V^{\prime\prime}$ is an isomorphism. For example,
any finite dimensional space is reflexive, and any Hilbert space is
reflexive by the Riesz representation theorem.

\textbf{Remarks}

Linear forms are also known as linear functionals. 

Another way in which a linear mapping $V\to\Vstar$ can arise is via
a bilinear form $$V \times V \to k\;.$$
The notions of duality extend, in part, from vector spaces to modules,
especially free modules over commutative rings. A related notion is
the duality in projective spaces.
\end{document}
