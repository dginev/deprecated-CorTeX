\documentclass{article}
\usepackage{planetmath-specials}
\usepackage{pmath}

\begin{document}
A topological space $X$ is said to be \emph{weakly countably compact}
(or \emph{limit point compact})
if every infinite subset of $X$ has a limit point.

Every countably compact space is weakly countably compact.
The converse is true in \PMlinkname{$\mathrm{T}_1$ spaces}{T1Space}.

A metric space is weakly countably compact if and only if it is compact.

An easy example of a space $X$
that is not weakly countably compact
is any infinite set with the discrete topology.
A more interesting example is the countable complement topology
on an uncountable set.

\end{document}
