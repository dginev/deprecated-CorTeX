\documentclass{article}
\usepackage{planetmath-specials}
\usepackage{pmath}
\usepackage{amssymb}
\usepackage{amsmath}
\usepackage{amsfonts}

%\usepackage{psfrag}
%\usepackage{graphicx}
%\usepackage{xypic}
\begin{document}
\PMlinkescapeword{even}

\section{Triangular Matrix}

Let $n$ be a positive integer.

An \emph{upper triangular matrix} is of the form:

$$ \begin{bmatrix}
a_{11} & a_{12} & a_{13} & \cdots & a_{1n} \\
   0   & a_{22} & a_{23} & \cdots & a_{2n} \\
   0   &   0    & a_{33} & \cdots & a_{3n} \\
\vdots & \vdots & \vdots & \ddots & \vdots \\
   0   &   0    &    0   & \cdots & a_{nn}
\end{bmatrix} $$

An upper triangular matrix is sometimes also called \emph{right triangular}.

A \emph{lower triangular matrix} is of the form:

$$ \begin{bmatrix}
a_{11} &   0    &   0    & \cdots &    0   \\
a_{21} & a_{22} &   0    & \cdots &    0   \\
a_{31} & a_{32} & a_{33} & \cdots &    0   \\
\vdots & \vdots & \vdots & \ddots & \vdots \\
a_{n1} & a_{n2} & a_{n3} & \cdots & a_{nn} 
\end{bmatrix} $$

A lower triangular matrix is sometimes also called \emph{left triangular}.

Note that upper triangular matrices and lower triangular matrices must be square matrices.

A \emph{triangular matrix} is a matrix that is an upper triangular matrix or lower triangular matrix.  Note that some matrices, such as the identity matrix, are both upper and lower triangular.  A matrix is upper and lower triangular simultaneously if and only if it is a diagonal matrix.

Triangular matrices allow numerous algorithmic shortcuts in many situations. For example, if $A$ is an $n\times n$ triangular matrix, the equation $Ax=b$ can be solved for $x$ in at most $n^2$ operations.

In fact, triangular matrices are so useful that much computational linear algebra begins with factoring (or decomposing) a general matrix or matrices into triangular form.  Some matrix factorization methods are the Cholesky factorization and the LU-factorization. Even including the factorization step, enough later operations are typically avoided to yield an overall time savings.

\section{Properties}

Triangular matrices have the following properties (\PMlinkescapetext{prefix} ``triangular'' with either ``upper'' or ``lower'' uniformly):

\begin{itemize}
\item The inverse of a triangular matrix is a triangular matrix.
\item The product of two triangular matrices is a triangular matrix.
\item The determinant of a triangular matrix is the product of the diagonal elements.
\item The eigenvalues of a triangular matrix are the diagonal elements.
\end{itemize}

The last two properties follow easily from the cofactor expansion of the triangular matrix.
\end{document}
