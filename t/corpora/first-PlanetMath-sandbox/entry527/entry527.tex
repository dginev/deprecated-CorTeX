\documentclass{article}
\usepackage{planetmath-specials}
\usepackage{pmath}
\usepackage{amssymb}
\usepackage{amsmath}
\usepackage{amsfonts}
\usepackage{graphicx}
\usepackage{xypic}
\begin{document}
An irreducible polynomial $f \in F[x]$ with coefficients in a field $F$ is {\em separable} if $f$ factors into distinct linear factors over a splitting field $K$ of $f$.

A polynomial $g$ with coefficients in $F$ is {\em separable} if each irreducible factor of $g$ in $F[x]$ is a separable polynomial.

An algebraic field extension $K/F$ is {\em separable} if, for each $a \in K$, the minimal polynomial of $a$ over $F$ is separable. When $F$ has characteristic zero, every algebraic extension of $F$ is separable; examples of inseparable extensions include the quotient field $K(u)[t]/(t^p-u)$ over the field $K(u)$ of rational functions in one variable, where $K$ has characteristic $p > 0$.

More generally, an arbitrary field extension $K/F$ is defined to be {\em separable} if every finitely generated intermediate field extension $L/F$ has a transcendence basis $S \subset L$ such that $L$ is a separable algebraic extension of $F(S)$.
\end{document}
