\documentclass{article}
\usepackage{ids}
\usepackage{planetmath-specials}
\usepackage{pmath}
% this is the default PlanetMath preamble.  as your knowledge
% of TeX increases, you will probably want to edit this, but
% it should be fine as is for beginners.

% almost certainly you want these
\usepackage{amssymb}
\usepackage{amsmath}
\usepackage{amsfonts}

% used for TeXing text within eps files
%\usepackage{psfrag}
% need this for including graphics (\includegraphics)
%\usepackage{graphicx}
% for neatly defining theorems and propositions
%\usepackage{amsthm}
% making logically defined graphics
%\usepackage{xypic} 

% there are many more packages, add them here as you need them

% define commands here
\begin{document}
Let $f \colon \mathbb{C}\rightarrow\mathbb{C}$ be a polynomial, and suppose $f$ has no root in $\mathbb{C}$.  We will show $f$ is constant.

Let $g=\frac{1}{f}$.  Since $f$ is never zero, $g$ is defined and holomorphic on $\mathbb{C}$ (ie. it is entire).  Moreover, since $f$ is a polynomial, $|f(z)|\rightarrow\infty$ as $|z|\rightarrow\infty$, and so $|g(z)|\rightarrow 0$ as $|z|\rightarrow\infty$.  Then there is some $M>0$ such that $|g(z)|<1$ whenever $|z|>M$, and $g$ is continuous and so bounded on the compact set $\{ z\in\mathbb{C}:|z|\leq M\}$.

So $g$ is bounded and entire, and therefore by Liouville's theorem $g$ is constant.  So $f$ is constant as required.$\square$
\end{document}
