\documentclass{article}
\usepackage{planetmath-specials}
\usepackage{pmath}
\usepackage{amssymb}
\usepackage{amsmath}
\usepackage{amsfonts}
\usepackage{pstricks}
\begin{document}
The \emph{intersection} of two sets $A$ and $B$ is the set that contains all the elements $x$ such that
$x \in A$ and $x \in B$. The intersection of $A$ and $B$ is written as $A \cap B$.  The following Venn diagram illustrates the intersection of two sets $A$ and $B$:

\begin{center}
\begin{pspicture}(0,0)(6,4)
\pscircle[fillstyle=vlines,hatchcolor=red,hatchwidth=0.1\pslinewidth,hatchsep=1\pslinewidth](2,2){2}
\pscircle[fillstyle=vlines,hatchcolor=blue,hatchwidth=0.1\pslinewidth,hatchsep=1\pslinewidth](4,2){2}
\rput(1,2){$A$}
\rput(3,2){$A\cap B$}
\rput(5,2){$B$}
\rput(0,0){$.$}
\rput(6,4){$.$}
\end{pspicture}
\end{center}

\textbf{Example}. If $A=\{1,2,3,4,5\}$ and $B=\{1,3,5,7,9\}$ then $A\cap B=\{1,3,5\}$.

We can also define the intersection of an arbitrary number of sets. If $\{A_j\}_{j\in J}$ is a family of sets, we define the intersection of all them, denoted $\bigcap_{j\in J} A_j$, as the set consisting of those elements belonging to every set $A_j$:
\[ 
  \bigcap_{j\in J} A_ j = \{x: x\in A_j  \mbox{ for all } j\in J \}.
\]

A set $U$ \emph{intersects}, or \emph{meets}, a set $V$ if $U\cap V$ is non-empty.

Some elementary properties of $\cap$ are
\begin{itemize}
\item (idempotency) $A\cap A = A$,
\item (commutativity) $A\cap B=B\cap A$,
\item (associativity) $A\cap (B\cap C) = (A\cap B)\cap C$,
\item $A\cap A^\complement = \varnothing$, where $A^\complement$ is the complement of $A$ in some fixed universe $U$.
\end{itemize}

\textbf{Remark}.  What is $\bigcap_{j\in J} A_j$ when $J=\varnothing$?  In other words, what is the intersection of an empty family of sets?  First note that if $I\subseteq J$, then $$\bigcap_{j\in J} A_j \subseteq \bigcap_{i\in I} A_i.$$  This leads the conclusion that the intersection of an empty family of sets should be as large as possible.  How large should it be?  In addition, is this intersection a set?  The answer depends on what versions of set theory we are working in.  Some theories (for example, von Neumann-G\"odel-Bernays) say this is the class $V$ of all sets, while others do not define this notion at all.  However, if there is a fixed set $U$ in advance such that each $A_j\subseteq U$, then it is sometimes a matter of convenience to \emph{define} the intersection of an empty family of $A_j$ to be $U$.
\end{document}
