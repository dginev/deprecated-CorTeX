\documentclass{article}
\usepackage{planetmath-specials}
\usepackage{pmath}
\newtheorem{proposition}{Proposition}
\begin{document}
Let $V$ be a finite-dimensional linear space over a field $K$, and
$T:V\rightarrow V$ a linear transformation. To {\em diagonalize} $T$
is to find a basis of $V$ that consists of eigenvectors. The
transformation is called {\em diagonalizable} if such a basis exists.
The choice of terminology reflects the fact that
the matrix of a linear transformation relative to a given basis is  diagonal
if and only if that basis consists of
eigenvectors.

Next, we give necessary and sufficient conditions for $T$ to be
diagonalizable. For $\lambda\in K$ set
$$E_\lambda = \{ u\in V: Tu = \lambda u\}.$$
It isn't hard to show that $E_\lambda$ is a subspace of $V$, and that this subspace is
non-trivial if and only if $\lambda$ is an {\em eigenvalue} of $T$.  In that case, $E_\lambda$ is called the eigenspace
associated to $\lambda$. 
\begin{proposition}
A transformation is diagonalizable if and only if
$$\dim V = \sum_\lambda \dim E_\lambda,$$
where the sum is taken over all eigenvalues of the transformation.
\end{proposition}

\paragraph{The Matrix Approach.}
As was already mentioned, the  term  ``diagonalize'' comes from a matrix-based perspective.  Let
$M$ be a \PMlinkname{matrix representation}{matrix} of $T$ relative to some basis $B$.  Let
$$P=[v_1,\ldots,v_n],\quad n=\dim V,$$
be a matrix whose column vectors are eigenvectors expressed relative
to $B$.  Thus,
$$M v_i = \lambda_i v_i,\quad i=1,\ldots,n$$
where $\lambda_i$ is the eigenvalue associated to $v_i$.  The above
$n$ equations are more succinctly as the matrix equation
$$MP = PD,$$
where $D$ is the diagonal matrix with $\lambda_i$ in the $i$-th
position.  Now the eigenvectors in question form a basis, if and only
if $P$ is invertible.  In that case, we may write
\begin{equation}
  \label{eq:diag}
  M=PDP^{-1}.  
\end{equation}
Thus in the matrix-based approach, to ``diagonalize'' a matrix $M$ is
to find an invertible matrix $P$ and a diagonal matrix $D$ such that
equation (\ref{eq:diag}) is satisfied.  


\paragraph{Subtleties.}
There are two fundamental reasons why a transformation $T$ can fail to
be diagonalizable.
\begin{enumerate}
\item The characteristic polynomial of $T$ does not factor into linear
factors over $K$.
\item There exists an eigenvalue $\lambda$, such that the kernel of
$(T-\lambda I)^2$ is strictly greater than the kernel of $(T-\lambda
I)$. Equivalently, there exists an invariant subspace where $T$
acts as a nilpotent transformation plus some multiple of the
identity. Such subspaces manifest as non-trivial Jordan blocks in the Jordan canonical form of the transformation.\end{enumerate}
\end{document}
