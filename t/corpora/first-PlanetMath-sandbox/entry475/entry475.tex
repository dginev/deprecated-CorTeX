\documentclass{article}
\usepackage{ids}
\usepackage{planetmath-specials}
\usepackage{pmath}
\usepackage{amssymb}
\usepackage{amsmath}
\usepackage{amsfonts}
\usepackage{graphicx}
\usepackage{xypic}
\begin{document}
A \emph{geometric series} is a series of the form

\begin{align*}
\sum_{i=1}^n ar^{i-1}
\end{align*}

(with $a$ and $r$ real or complex numbers).  The partial sums of a geometric series are given by

\begin{align}
s_n=\sum_{i=1}^n ar^{i-1} =  \frac{a(1 -r^n)}{1-r}.
\end{align}

An \emph{infinite geometric series} is a geometric series, as above, with $n \rightarrow \infty$.  It is denoted by

\begin{align*}
\sum_{i=1}^\infty ar^{i-1}
\end{align*}

If $|r|\ge 1$, the infinite geometric series diverges.  Otherwise it converges to

\begin{align}
\sum_{i=1}^\infty ar^{i-1}  =  \frac{a}{1-r}
\end{align}

Taking the limit of $s_n$ as  $n \rightarrow \infty$, we see that $s_n$ diverges if $|r| \ge 1$.  However, if $|r| < 1$, $s_n$ approaches (2).

One way to prove (1) is to take 

\begin{align*}
s_n  =  a + ar + ar^2 + \cdots + ar^{n-1}
\end{align*}

and multiply by $r$, to get

\begin{align*}
 r s_n  =  ar + ar^2 + ar^3 + \cdots + ar^{n-1} + ar^{n}
\end{align*}

subtracting the two removes most of the terms:

\begin{align*}
 s_n - rs_n  =  a - ar^n
\end{align*}

factoring and dividing gives us

\begin{align*}
s_n = \frac{a(1 -r^n)}{1-r}
\end{align*}
$\square$
\end{document}
