\documentclass{article}
\usepackage{planetmath-specials}
\usepackage{pmath}
\usepackage{amssymb}
\usepackage{amsmath}
\usepackage{amsfonts}

\def\Iso{\operatorname{Iso}}
\def\Isom{\operatorname{Isom}}
\begin{document}
\PMlinkescapeword{isometries}
\PMlinkescapeword{isometry}
\PMlinkescapeword{term}

Let $(X_1,d_1)$ and $(X_2,d_2)$ be metric spaces.
A function $f\colon X_1\to X_2$ is said to be an \emph{isometric mapping}
(or \emph{isometric embedding}) if
\[
  d_1(x,y)=d_2(f(x),f(y))
\]
for all $x,y\in X_1$.

Every isometric mapping is injective,
for if $x,y\in X_1$ with $x\neq y$ then $d_1(x,y)>0$,
and so $d_2(f(x),f(y))>0$, and then $f(x)\neq f(y)$.
One can also easily show that every isometric mapping is continuous.

An isometric mapping that is surjective (and therefore bijective)
is called an \emph{isometry}. 
(Readers are warned, however,
that some authors do not require isometries to be surjective;
that is, they use the term {\it isometry}
for what we have called an isometric mapping.)
Every isometry is a homeomorphism.

If there is an isometry between the metric spaces $(X_1,d_1)$ and $(X_2,d_2)$,
then they are said to be \emph{isometric}.
Isometric spaces are essentially identical as metric spaces,
and in particular they are homeomorphic.

Given any metric space $(X,d)$,
the set of all isometries $X\to X$ forms a group under composition.
This group is called the \emph{isometry group}
(or \emph{group of isometries}) of $X$,
and may be denoted by $\Iso(X)$ or $\Isom(X)$.
In general, {\it an} (as opposed to {\it the}) isometry group 
(or group of isometries) of $X$ is any subgroup of $\Iso(X)$.
\end{document}
