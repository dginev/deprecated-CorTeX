\documentclass{article}
\usepackage{planetmath-specials}
\usepackage{pmath}
\usepackage{amssymb}
\usepackage{amsmath}
\usepackage{amsfonts}
\usepackage{graphicx}
\usepackage{xypic}
\begin{document}
Let $(X,\mathcal{T})$ be a topological space.  A subset $\mathcal{A}\subseteq\mathcal{T}$ is said to be a \emph{subbasis} if the collection $\mathcal{B}$ of intersections of finitely many elements of $\mathcal{A}$ is a \PMlinkname{basis}{BasisTopologicalSpace} for $\mathcal{T}$.

Conversely, given an arbitrary collection $\mathcal{A}$ of subsets of $X$, a topology can be formed by first taking the collection $\mathcal{B}$ of finite intersections of members of $\mathcal{A}$ and then taking the topology $\mathcal{T}$ generated by $\mathcal{B}$ as basis.  $\mathcal{T}$ will then be the smallest topology such that $\mathcal{A}\subseteq\mathcal{T}$.
\end{document}
