\documentclass{article}
\usepackage{ids}
\usepackage{planetmath-specials}
\usepackage{pmath}
\usepackage{amssymb}
\usepackage{amsmath}
\usepackage{amsfonts}
\usepackage{graphicx}
\usepackage{xypic}
\begin{document}
A \emph{rhombus} (plural \emph{rhombi}, or \emph{rhombuses})
is a quadrilateral with its four sides equal.
A square is a rhombus,
but a rhombus need not be a square,
since the angles need not all be equal.
Every rhombus is, however, a parallelogram.
\begin{center}
\includegraphics[scale=2]{rhombus}
\end{center}
In any rhombus, the diagonals are always perpendicular.
A nice result following from this,
is that joining the midpoints of the sides, always gives a rectangle.

If $D$ and $d$ are the lengths of the diagonals,
then the area of the rhombus is
\[
  \frac{Dd}{2}.
\]
\end{document}
