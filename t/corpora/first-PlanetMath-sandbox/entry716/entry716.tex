\documentclass{article}
\usepackage{planetmath-specials}
\usepackage{pmath}
% this is the default PlanetMath preamble.  as your knowledge
% of TeX increases, you will probably want to edit this, but
% it should be fine as is for beginners.

% almost certainly you want these
\usepackage{amssymb}
\usepackage{amsmath}
\usepackage{amsfonts}

% used for TeXing text within eps files
%\usepackage{psfrag}
% need this for including graphics (\includegraphics)
%\usepackage{graphicx}
% for neatly defining theorems and propositions
%\usepackage{amsthm}
% making logically defined graphics
\usepackage{xypic} 
\usepackage{color}

% there are many more packages, add them here as you need them

% define commands here
\begin{document}
The \emph{colouring problem} is to assign a \emph{colour} to every vertex of a graph such that no two adjacent vertices have the same colour.  These colours, of course, are not necessarily colours in the optic sense.

Consider the following graph:

$$\xymatrix{
A \ar@{-}[r] & B \ar@{-}[r] \ar@{-}[d] & C \ar@{-}[r] \ar@{-}[dl] & F \ar@{-}[dl] \\
& D \ar@{-}[r] & E & }$$

One potential colouring of this graph is:

$$\xymatrix{
{\color{red}A} \ar@{-}[r] & {\color{blue}B} \ar@{-}[r] \ar@{-}[d] & {\color{red}C} \ar@{-}[r] \ar@{-}[dl] & {\color{green}F} \ar@{-}[dl] \\
& {\color{green}D} \ar@{-}[r] & {\color{blue}E} & }$$

$A$ and $C$ have the same colour; $B$ and $E$ have a second colour; and $D$ and $F$ have another.

Graph colouring problems have many applications in such situations as scheduling and matching problems.
\end{document}
