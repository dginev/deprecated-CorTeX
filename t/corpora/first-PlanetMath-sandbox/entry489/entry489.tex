\documentclass{article}
\usepackage{planetmath-specials}
\usepackage{pmath}
\usepackage{amssymb}
\usepackage{amsmath}
\usepackage{amsfonts}
\usepackage{graphicx}
\usepackage{xypic}
\begin{document}
Horner's rule is a technique to reduce the work required for the computation of a polynomial at a particular value.  Its simplest form makes use of the repeated factorizations 

\begin{eqnarray*} 
 y & = & a_0 + a_1 x + a_2 x^2 + \cdots + a_n x^n \\
   & = & a_0 + x(a_1 + x(a_2 + x(a_3 + \cdots +xa_n)) \cdots ) 
\end{eqnarray*}

of the terms of the $n$th degree polynomial in $x$ in order to reduce the computation of the polynomial $y(a)$ (at some value $x = a$) to $n$ multiplications and $n$ additions.  

The rule can be generalized to a finite series

$$ y = a_0 p_0 + a_1 p_1 + \cdots + a_n p_n $$

of orthogonal polynomials $p_k = p_k(x)$.  Using the recurrence relation

$$ p_k = (A_k + B_k x) p_{k-1} + C_k p_{k-2} $$

for orthogonal polynomials, one obtains 

$$ y(a) = (a_0 + C_2 b_2) p_0(a) + b_1 p_1(a) $$

with 

\begin{eqnarray*}
 b_{n+1} & = & b_{n+2} = 0 , \\
 b_{k-1} & = &  (A_k + B_k\cdot a) b_k + C_{k+1} + b_{k+1} + a_{k-1} 
\end{eqnarray*}

for the evaluation of $y$ at some particular $a$.  This is a simpler calculation than the straightforward approach, since $a_0$ and $C_2$ are known, $p_0(a)$ and $p_1(a)$ are easy to compute (possibly themselves by Horner's rule), and $b_1$ and $b_2$ are given by a backwards-recurrence which is linear in $n$.

{\bf References}

\begin{itemize}
\item Originally from The Data Analysis Briefbook
(\PMlinkexternal{http://rkb.home.cern.ch/rkb/titleA.html}{http://rkb.home.cern.ch/rkb/titleA.html})
\end{itemize}
\end{document}
