\documentclass{article}
\usepackage{ids}
\usepackage{planetmath-specials}
\usepackage{pmath}
% this is the default PlanetMath preamble.  as your knowledge
% of TeX increases, you will probably want to edit this, but
% it should be fine as is for beginners.

% almost certainly you want these
\usepackage{amssymb}
\usepackage{amsmath}
\usepackage{amsfonts}

% used for TeXing text within eps files
%\usepackage{psfrag}
% need this for including graphics (\includegraphics)
%\usepackage{graphicx}
% for neatly defining theorems and propositions
%\usepackage{amsthm}
% making logically defined graphics
%\usepackage{xypic} 

% there are many more packages, add them here as you need them

% define commands here
\begin{document}
A morphism $f: A \longrightarrow B$ in a category is an \emph{isomorphism} if there exists a morphism $f^{-1}: B \longrightarrow A$ which is its inverse. The objects $A$ and $B$ are \emph{isomorphic} if there is an isomorphism between them.

A morphism which is both an isomorphism and an endomorphism is called an \emph{automorphism}. The set of automorphisms of an object $A$ is denoted $\operatorname{Aut}(A)$.

Examples:
\begin{itemize}
\item In the category of sets and functions, a function $f: A \longrightarrow B$ is an isomorphism if and only if it is bijective.
\item In the category of groups and group homomorphisms (or rings and ring homomorphisms), a homomorphism $\phi: G \longrightarrow H$ is an isomorphism if it has an inverse map $\phi^{-1}: H \longrightarrow G$ which is also a homomorphism.
\item In the category of vector spaces and linear transformations, a linear transformation is an isomorphism if and only if it is an invertible linear transformation.
\item In the category of topological spaces and continuous maps, a continuous map is an isomorphism if and only if it is a homeomorphism.
\end{itemize}
\end{document}
