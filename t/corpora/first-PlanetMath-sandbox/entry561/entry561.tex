\documentclass{article}
\usepackage{ids}
\usepackage{planetmath-specials}
\usepackage{pmath}
\usepackage{amssymb}
\usepackage{amsmath}
\usepackage{amsfonts}
\usepackage{graphicx}
\usepackage{xypic}
\begin{document}
A covariant functor $F$ is said to be {\it left exact} if whenever
$$0 \to A \buildrel \alpha \over \longrightarrow B \buildrel \beta \over \longrightarrow C$$
is an exact sequence, then 
$$0 \to FA \buildrel F\alpha \over \longrightarrow FB \buildrel F\beta \over \longrightarrow FC$$
is also an exact sequence.

A covariant functor $F$ is said to be {\it right exact} if whenever
$$A \buildrel \alpha \over \longrightarrow B \buildrel \beta \over \longrightarrow C \to 0$$
is an exact sequence, then
$$FA \buildrel F\alpha \over \longrightarrow FB \buildrel F\beta \over \longrightarrow FC \to 0$$
is also an exact sequence.

A contravariant functor $F$ is said to be {\it left exact} if whenever
$$A \buildrel \alpha \over \longrightarrow B \buildrel \beta \over \longrightarrow C \to 0$$
is an exact sequence, then
$$0 \to FC \buildrel F\beta \over \longrightarrow FB \buildrel F\alpha \over \longrightarrow FA$$
is also an exact sequence.

A contravariant functor $F$ is said to be {\it right exact} if whenever
$$0 \to A \buildrel \alpha \over \longrightarrow B \buildrel \beta \over \longrightarrow C$$
is an exact sequence, then
$$FC \buildrel F\beta \over \longrightarrow FB \buildrel F\alpha \over \longrightarrow FA \to 0$$
is also an exact sequence.

A (covariant or contravariant) functor is said to be {\it exact}
if it is both left exact and right exact.
\end{document}
