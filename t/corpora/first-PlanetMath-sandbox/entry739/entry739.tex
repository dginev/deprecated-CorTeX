\documentclass{article}
\usepackage{planetmath-specials}
\usepackage{pmath}
% this is the default PlanetMath preamble.  as your knowledge
% of TeX increases, you will probably want to edit this, but
% it should be fine as is for beginners.

% almost certainly you want these
\usepackage{amssymb}
\usepackage{amsmath}
\usepackage{amsfonts}

% used for TeXing text within eps files
%\usepackage{psfrag}
% need this for including graphics (\includegraphics)
%\usepackage{graphicx}
% for neatly defining theorems and propositions
%\usepackage{amsthm}
% making logically defined graphics
%\usepackage{xypic} 

% there are many more packages, add them here as you need them

% define commands here
\begin{document}
Two elements $g$ and $g'$ of a group $G$ are said to be {\em conjugate} if there exists $h \in G$ such that $g' = hgh^{-1}$. Conjugacy of elements is an equivalence relation, and the equivalence classes of $G$ are called {\em conjugacy classes}.

Two subsets $S$ and $T$ of $G$ are said to be {\em conjugate} if there exists $g \in G$ such that
$$
T = \{gsg^{-1} \mid s \in S\} \subset G.
$$
In this situation, it is common to write $gSg^{-1}$ for $T$ to denote the fact that everything in $T$ has the form $gsg^{-1}$ for some $s \in S$. We say that two subgroups of $G$ are conjugate if they are conjugate as subsets.
\end{document}
