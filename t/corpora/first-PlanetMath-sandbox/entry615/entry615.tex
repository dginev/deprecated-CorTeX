\documentclass{article}
\usepackage{planetmath-specials}
\usepackage{pmath}
% this is the default PlanetMath preamble.  as your knowledge
% of TeX increases, you will probably want to edit this, but
% it should be fine as is for beginners.

% almost certainly you want these
\usepackage{amssymb}
\usepackage{amsmath}
\usepackage{amsfonts}

% used for TeXing text within eps files
%\usepackage{psfrag}
% need this for including graphics (\includegraphics)
%\usepackage{graphicx}
% for neatly defining theorems and propositions
%\usepackage{amsthm}
% making logically defined graphics
%\usepackage{xypic} 

% there are many more packages, add them here as you need them

% define commands here
\begin{document}
Let $G$ be a group and let $X$ be a set. A left {\em group action} is a function $\cdot: G \times X \longrightarrow X$ such that:
\begin{enumerate}
\item $1_G \cdot x = x$ for all $x \in X$
\item $(g_1 g_2)\cdot x = g_1 \cdot (g_2 \cdot x)$ for all $g_1, g_2 \in G$ and $x \in X$
\end{enumerate}

A right {\em group action} is a function $\cdot: X \times G \longrightarrow X$ such that:
\begin{enumerate}
\item $x \cdot 1_G = x$ for all $x \in X$
\item $x \cdot (g_1 g_2) = (x \cdot g_1) \cdot g_2$ for all $g_1, g_2 \in G$ and $x \in X$
\end{enumerate}

There is a correspondence between left actions and right actions, given by associating the right action $x \cdot g$ with the left action $g \cdot x := x \cdot g^{-1}$. In many (but not all) contexts, it is useful to identify right actions with their corresponding left actions, and speak only of left actions.

{\bf Special types of group actions}

A left action is said to be {\em effective}, or {\em faithful}, if the function $x \mapsto g \cdot x$ is the identity function on $X$ only when $g = 1_G$.

A left action is said to be {\em transitive} if, for every $x_1,x_2 \in X$, there exists a group element $g \in G$ such that $g \cdot x_1 = x_2$.

A left action is {\em free} if, for every $x \in X$, the only element of $G$ that stabilizes $x$ is the identity; that is, $g \cdot x = x$ implies $g = 1_G$.

Faithful, transitive, and free right actions are defined similarly.
\end{document}
