\documentclass{article}
\usepackage{planetmath-specials}
\usepackage{pmath}
% this is the default PlanetMath preamble.  as your knowledge
% of TeX increases, you will probably want to edit this, but
% it should be fine as is for beginners.

% almost certainly you want these
\usepackage{amssymb}
\usepackage{amsmath}
\usepackage{amsfonts}

% used for TeXing text within eps files
%\usepackage{psfrag}
% need this for including graphics (\includegraphics)
%\usepackage{graphicx}
% for neatly defining theorems and propositions
%\usepackage{amsthm}
% making logically defined graphics
%\usepackage{xypic} 

% there are many more packages, add them here as you need them

% define commands here
\newcommand{\Ind}{\operatorname{Ind}}
\newcommand{\Res}{\operatorname{Res}}
\newcommand{\Hom}{\operatorname{Hom}}
\begin{document}
Let $V$ be a finite-dimensional representation of a finite group $G$, and let $W$ be a representation of a subgroup $H \subset G$. Then the characters of $V$ and $W$ satisfy the inner product relation
$$
(\chi_{\Ind(W)}, \chi_V) = (\chi_W, \chi_{\Res(V)})
$$
where $\Ind$ and $\Res$ denote the induced representation $\Ind_H^G$ and the restriction representation $\Res_H^G$.

The Frobenius reciprocity theorem is often given in the stronger form which states  that $\Res$ and $\Ind$ are adjoint functors between the category of $G$--modules and the category of $H$--modules:
$$
\Hom_H(W, \Res(V)) = \Hom_G(\Ind(W),V),
$$
or, equivalently
$$
V \otimes \Ind(W) = \Ind(\Res(V) \otimes W).
$$
\end{document}
