\documentclass{article}
\usepackage{ids}
\usepackage{planetmath-specials}
\usepackage{pmath}
\usepackage{amssymb}
\usepackage{amsmath}
\usepackage{amsfonts}

%\usepackage{psfrag}
%\usepackage{graphicx}
%\usepackage{xypic}
\begin{document}
\PMlinkescapeword{formula}

The \emph{double angle identities} are

\begin{eqnarray}
 \sin(2x) & = & 2\sin{x}\cos{x} \\
 \cos(2x) & = & \cos^2{x}-\sin^2{x} = 2\cos^2{x}-1 = 1-2\sin^2{x} \\
 \tan(2x) & = & \frac{2\tan{x}}{1-\tan^2{x}}
\end{eqnarray}

These are all derived from their respective trigonometric addition formulas.  For example,

\begin{eqnarray*}
 \sin(2x) & = & \sin(x+x) \\
          & = & \sin{x}\cos{x}+\cos{x}\sin{x} \\
          & = & 2\sin{x}\cos{x} 
\end{eqnarray*}

The formula for cosine follows similarly, and the formula tangent is derived by taking the ratio of sine to cosine, as always.

The double angle identities can also be derived from the de Moivre identity.
\end{document}
