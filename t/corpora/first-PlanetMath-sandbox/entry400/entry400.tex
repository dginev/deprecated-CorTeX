\documentclass{article}
\usepackage{planetmath-specials}
\usepackage{pmath}
% this is the default PlanetMath preamble.  as your knowledge
% of TeX increases, you will probably want to edit this, but
% it should be fine as is for beginners.

% almost certainly you want these
\usepackage{amssymb}
\usepackage{amsmath}
\usepackage{amsfonts}

% used for TeXing text within eps files
%\usepackage{psfrag}
% need this for including graphics (\includegraphics)
%\usepackage{graphicx}
% for neatly defining theorems and propositions
%\usepackage{amsthm}
% making logically defined graphics
%\usepackage{xypic} 

% there are many more packages, add them here as you need them

% define commands here

% The below lines should work as the command
% \renewcommand{\bibname}{References}
% without creating havoc when rendering an entry in
% the page-image mode.
\makeatletter
\@ifundefined{bibname}{}{\renewcommand{\bibname}{References}}
\makeatother
\begin{document}
\PMlinkescapeword{states}
\PMlinkescapeword{properties}
\PMlinkescapeword{addition}

\textbf{Definition:}\\ 
Let $(G,*)$ be a group and let $K$ be subset of $G$.  Then $K$ is a subgroup of $G$ defined under the same operation if $K$ is  a group by itself (with respect to $*$), that is:
\begin{itemize}
  \item $K$ is closed under the $*$ operation.
  \item There exists an identity element $e\in K$ such that for all $k \in K$, $k*e = k = e*k$.
  \item Let $k \in K$ then there exists an inverse $k^{-1} \in K$
  such that $k^{-1}*k = e = k*k^{-1}$.
\end{itemize}
The subgroup is denoted likewise $(K,*)$. We denote $K$ being a subgroup of $G$ by writing $K\leq G$.\\

In addition the notion of a subgroup of a semigroup can be defined in the following manner.  Let $(S,*)$ be a semigroup and $H$ be a subset of $S$. Then $H$ is a subgroup of $S$ if $H$ is a subsemigroup of $S$ and $H$ is a group.

\textbf{Properties:}
\begin{itemize}
  \item The set $\{e\}$ whose only element is the identity is a subgroup of any group.  It is called the trivial subgroup.
\item Every group is a subgroup of itself.
  \item The null set $\{ \}$ is never a subgroup (since the definition of group states that the set must be non-empty).
\end{itemize}

There is a very useful theorem that allows proving a given subset is a subgroup.

\textbf{Theorem:}\\
If $K$ is a nonempty subset of the group $G$.  Then $K$ is a \emph{subgroup} of $G$ if and only if $s,t \in K$ implies that $st^{-1} \in K$.

\textbf{Proof:}
First we need to show if $K$ is a subgroup of $G$ then $st^{-1} \in K$.  Since $s,t \in K$ then $st^{-1} \in K$, because $K$ is a group by itself.\\
Now, suppose that if for any $s,t\in K\subseteq G$ we have $st^{-1}\in K$. We want to show that $K$ is a subgroup, which we will accomplish by proving it holds the group axioms.

Since $tt^{-1}\in K$ by hypothesis, we conclude that the identity element is in $K$: $e\in K$. (Existence of identity)

Now that we know $e\in K$, for all $t$ in $K$ we have that $et^{-1}=t^{-1}\in K$ so the inverses of elements in $K $ are also in $K$. (Existence of inverses).

Let $s,t\in K$. Then we know that $t^{-1}\in K$ by last step. Applying hypothesis shows that
$$s(t^{-1})^{-1} = st\in K$$
so $K$ is closed under the operation. $QED$

\textbf{Example:}
\begin{itemize}
  \item Consider the group $(\mathbb{Z}, +)$.  Show that$(2\mathbb{Z}, +)$ is a subgroup. 

The subgroup is closed under addition since the  sum of even integers is even. 

The identity $0$ of $\mathbb{Z}$ is also on $2\mathbb{Z}$ since $2$ divides $0$.
For every $k \in 2\mathbb{Z}$ there is an $-k \in  2\mathbb{Z}$ which is the inverse under addition and satisfies $-k+k = 0 =
  k+(-k)$.  Therefore $(2\mathbb{Z}, +)$ is a subgroup of $(\mathbb{Z},  +)$.

 Another way to show $(2\mathbb{Z}, +)$ is a subgroup
  is by using the proposition stated above.  If $s,t \in
  2\mathbb{Z}$ then $s,t$ are even numbers and $s-t\in
  2\mathbb{Z}$ since the difference of even numbers is always an even number.
\end{itemize}

\textbf{See also:}
\begin{itemize}
\item Wikipedia, \PMlinkexternal{subgroup}{http://www.wikipedia.org/wiki/Subgroup}
\end{itemize}
\end{document}
