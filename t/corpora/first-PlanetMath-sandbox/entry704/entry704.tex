\documentclass{article}
\usepackage{ids}
\usepackage{planetmath-specials}
\usepackage{pmath}
% this is the default PlanetMath preamble.  as your knowledge
% of TeX increases, you will probably want to edit this, but
% it should be fine as is for beginners.

% almost certainly you want these
\usepackage{amssymb}
\usepackage{amsmath}
\usepackage{amsfonts}

% used for TeXing text within eps files
%\usepackage{psfrag}
% need this for including graphics (\includegraphics)
%\usepackage{graphicx}
% for neatly defining theorems and propositions
%\usepackage{amsthm}
% making logically defined graphics
%\usepackage{xypic} 

% there are many more packages, add them here as you need them

% define commands here
\newcommand{\SL}{\operatorname{SL}}
\newcommand{\Z}{{\mathbb{Z}}}
\renewcommand{\H}{{\mathbb{H}}}
\newcommand{\C}{{\mathbb{C}}}
\newcommand{\Q}{{\mathbb{Q}}}
\begin{document}
For any natural number $N \geq 1$, define the {\em modular group}
$\Gamma_0(N)$ to be the following subgroup of the group $\SL(2,\Z)$ of
integer coefficient matrices of determinant 1:
$$
\Gamma_0(N) := \left\{ \left. \begin{pmatrix} a & b \\ c & d
\end{pmatrix} \in \SL(2,\Z)\ \right|\ c \equiv 0 \pmod{N} \right\}.
$$
Let $\H^*$ be the subset of the Riemann sphere consisting of all
points in the upper half plane (i.e., complex numbers with strictly
positive imaginary part), together with the rational numbers and the
point at infinity. Then $\Gamma_0(N)$ acts on $\H^*$, with group
action given by the operation
$$
\begin{pmatrix} a & b \\ c & d \end{pmatrix} \cdot z := \frac{az+b}{cz+d}.
$$
Define $X_0(N)$ to be the quotient of $\H^*$ by the action of
$\Gamma_0(N)$. The quotient space $X_0(N)$ inherits a quotient
topology and holomorphic structure from $\C$ making it into a compact
Riemann surface. (Note: $\H^*$ itself is not a Riemann surface; only
the quotient $X_0(N)$ is.) By a general theorem in complex algebraic
geometry, every compact Riemann surface admits a unique realization as
a complex nonsingular projective curve; in particular, $X_0(N)$ has
such a realization, which by abuse of notation we will also denote
$X_0(N)$. This curve is defined over $\Q$, although the proof of this
fact is beyond the scope of this entry\footnote{Explicitly, the curve
$X_0(N)$ is the unique nonsingular projective curve which has function
field equal to $\C(j(z), j(Nz))$, where $j$ denotes the elliptic
modular $j$--function. The curve $X_0(N)$ is essentially the algebraic
curve defined by the polynomial equation $\Phi_N(X,Y) = 0$ where
$\Phi_N$ is the modular polynomial, with the caveat that this
procedure yields singularities which must be resolved manually. The
fact that $\Phi_N$ has integer coefficients provides one proof that
$X_0(N)$ is defined over $\Q$.}.

{\bf Taniyama-Shimura Theorem (weak form):} For any elliptic curve $E$ defined
over $\mathbb{Q}$, there exists a positive integer $N$ and a
surjective algebraic morphism $\phi: X_0(N) \to E$ defined over
$\mathbb{Q}$.

This theorem was first conjectured (in a much more precise, but equivalent formulation) by Taniyama, Shimura, and Weil in
the 1970's. It attracted considerable interest in the 1980's when
Frey~\cite{frey} proposed that the Taniyama-Shimura conjecture implies
Fermat's Last Theorem. In 1995, Andrew Wiles~\cite{wiles} proved a
special case of the Taniyama-Shimura theorem which was strong enough
to yield a proof of Fermat's Last Theorem. The full Taniyama-Shimura
theorem was finally proved in 1997 by a team of a half-dozen
mathematicians who, building on Wiles's work, incrementally chipped
away at the remaining cases until the full result was proved. As of this writing, the proof of the full theorem can still be found on \PMlinkexternal{Richard Taylor's preprints page}{http://abel.math.harvard.edu/~rtaylor/}.

\pagebreak

\begin{thebibliography}{9}
\bibitem{bc}{Breuil, Christophe; Conrad, Brian; Diamond, Fred; Taylor, Richard; {\em On the modularity of elliptic curves over $\mathbf{Q}$: wild 3-adic exercises}.  J. Amer. Math. Soc. {\bf 14}  (2001),  no. 4, 843--939}
\bibitem{frey}{Frey, G. {\em Links between stable elliptic curves and
certain Diophantine equations.} Ann. Univ. Sarav. {\bf 1} (1986), 1--40.}
\bibitem{wiles}{Wiles, A. {\em Modular elliptic curves and Fermat's Last
Theorem.} Annals of Math. {\bf 141} (1995), 443--551.}
\end{thebibliography}
\end{document}
