\documentclass{article}
\usepackage{planetmath-specials}
\usepackage{pmath}
\usepackage{amssymb}
\usepackage{amsmath}
\usepackage{amsfonts}
\usepackage{graphicx}
\usepackage{xypic}
\begin{document}
Let

$$ f=f(x) = f(x_1,\ldots,x_n) $$

be a function of $n$ variables, and let

$$ u = u(x) = (u_1(x),\ldots,u_n(x)) $$

be a function of $x$, where inversely $x$ can be expressed as a function of $u$,

$$ x = x(u) = (x_1(u), \ldots,x_n(u)) $$

The formula for a change of variable in an $n$-dimensional integral is then

$$ \int_\Omega f(x)d^nx = \int_{u(\Omega)} f(x(u))|\det(dx/du)|d^nu  $$

$\Omega$ is an integration region, and one integrates over all $x \in \Omega$, or equivalently, all $u \in u(\Omega)$. $dx/du = (du/dx)^{-1}$ is the Jacobi matrix and

$$ | \det(dx/du)| = |\det(du/dx)|^{-1} $$

is the absolute value of the \emph{Jacobi determinant} or \emph{Jacobian}.

As an example, take $n=2$ and

$$ \Omega = \{(x_1,x_2) | 0<x_1 \le 1, 0 < x_2 \le 1 \} $$

Define

$$ \begin{array}{cc}\rho = \sqrt{-2 \log(x_1)} & \varphi = 2\pi x_2 \\
    u_1 = \rho \cos \varphi & u_2 = \rho \sin \varphi \end{array} $$

Then by the chain rule and definition of the Jacobi matrix,

\begin{eqnarray*}
 du/dx & = & \partial(u_1,u_2) / \partial(x_1,x_2) \\
       & = & (\partial(u_1,u_2)/\partial(\rho,\varphi))(\partial(\rho,\varphi)/\partial(x_1,x_2)) \\
       & = & \begin{pmatrix}\cos \varphi & - \rho \sin \varphi \\ \sin \varphi & \rho \cos \varphi \end{pmatrix} \begin{pmatrix}-1 / \rho x_1 & 0 \\ 0 & 2 \phi \end{pmatrix} 
\end{eqnarray*}

The Jacobi determinant is

\begin{eqnarray*}
\det (du/dx) & = & \det \{ \partial(u_1,u_2)/\partial(\rho,\varphi)\} \det\{\partial(\rho,\varphi) / \partial(x_1,x_2)\} \\
 & = & \rho(-2 \pi /\rho x_1) = -2 \pi / x_i
\end{eqnarray*}

and

\begin{eqnarray*}
 d^2 x & = & | \det(dx/du) | d^2u = |\det(du/dx)|^{-1} d^2 u \\
  & = & (x_1/2\pi) = (1/2 \pi) \exp(-(u_1^2+u_2^2/2))d^2 u
\end{eqnarray*}

This shows that if $x_1$ and $x_2$ are independent random variables with uniform distributions between 0 and 1, then $u_1$ and $u_2$ as defined above are independent random variables with standard normal distributions. 

{\bf References}

\begin{itemize}
\item Originally from The Data Analysis Briefbook
(\PMlinkexternal{http://rkb.home.cern.ch/rkb/titleA.html}{http://rkb.home.cern.ch/rkb/titleA.html})
\end{itemize}
\end{document}
