\documentclass{article}
\usepackage{planetmath-specials}
\usepackage{pmath}
\usepackage{amssymb}
\usepackage{amsmath}
\usepackage{amsfonts}

%\usepackage{psfrag}
%\usepackage{graphicx}
%\usepackage{xypic}
\begin{document}
For a complex matrix $A$, let $A^\ast=\overline{A}^{T}$, where 
$A^T$ is the transpose, and $\bar{A}$ is the complex conjugate of $A$.

{\bf Definition}
A complex square matrix $A$ is \emph{Hermitian}, if 
$$ A = A^*. $$

\subsubsection*{Properties}
\begin{enumerate}
\item The eigenvalues of a Hermitian matrix are real.
\item The diagonal elements of a Hermitian matrix are real.
\item The complex conjugate of a Hermitian matrix is a Hermitian matrix.
\item If $A$ is a Hermitian matrix, and $B$ is a complex matrix
of same order as $A$, then $BAB^\ast$ is a Hermitian matrix.
\item A matrix is symmetric if and only if it is real and Hermitian.
\item Hermitian matrices are a vector subspace of the vector space of 
complex matrices. 
The real symmetric matrices are a subspace of the Hermitian matrices.
\item Hermitian matrices are also called \emph{self-adjoint} since if $A$ is
Hermitian, then in the usual
inner product of $\mathbb{C}^n$, we have 
$$ \langle u,Av \rangle = \langle Au,v\rangle$$
for all $u,v\in \mathbb{C}^n$.

\end{enumerate}

\subsubsection*{Example}
\begin{enumerate}
\item For any $n\times m$ matrix $A$, the $n\times n$ matrix $A A^\ast$ is
Hermitian. 
\item For any square matrix $A$, the Hermitian part of $A$,  
      $\frac{1}{2}(A+A^\ast)$ is Hermitian. 
      See \PMlinkname{this page}{DirectSumOfHermitianAndSkewHermitianMatrices}.
\item 
$$ \begin{bmatrix}
  1 & 1 + i & 1 + 2i & 1 + 3i \\
  1 - i & 2 & 2 + 2i & 2 + 3i \\
  1 - 2i & 2 - 2i & 3 & 3 + 3i \\
  1 - 3i & 2 - 3i & 3 - 3i & 4
\end{bmatrix} $$
\end{enumerate}
The first two examples are also examples of normal matrices.

\subsubsection*{Notes}
\begin{enumerate}
\item Hermitian matrices are named after Charles Hermite (1822-1901) \cite{hermite}, who proved  in 1855 that the 
eigenvalues of these matrices are always real \cite{eves}.
\item Hermitian, or self-adjoint operators on a Hilbert space play a fundamental
role in quantum theories as their eigenvalues are observable, or measurable; such
Hermitian operators can be represented by Hermitian matrices.
\end{enumerate}

\begin{thebibliography}{9}
 \bibitem{eves} H. Eves,
 \emph{Elementary Matrix Theory},
 Dover publications, 1980.
\bibitem{hermite}
 The MacTutor History of Mathematics archive, 
\PMlinkexternal{Charles Hermite}{http://www-gap.dcs.st-and.ac.uk/~history/Mathematicians/Hermite.html}
\end{thebibliography}
\end{document}
