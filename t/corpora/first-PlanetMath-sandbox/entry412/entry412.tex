\documentclass{article}
\usepackage{planetmath-specials}
\usepackage{pmath}
\usepackage{amssymb}
\usepackage{amsmath}
\usepackage{amsfonts}
\begin{document}
\PMlinkescapeword{independent}

\emph{Arclength} is the \PMlinkescapetext{length} of a section of a differentiable curve. Finding the length of an arc is useful in many applications, for the length of a curve can represent distance traveled, work, etc. It is commonly represented as $S$ or the differential $ds$ if one is differentiating or integrating with respect to change in arclength. 

If one knows the vector function or parametric equations of a curve, finding the arclength is \PMlinkescapetext{simple}, as it can be given by the sum of the lengths of the tangent vectors to the curve or 

$$ \int_a^b |\vec{F}'(t)| \; dt=S $$ 

Note that $t$ is an independent parameter. In Cartesian coordinates, arclength can be calculated by the formula 

$$ S=\int_a^b \sqrt{1+(f'(x))^2} \; dx $$ 

This formula is derived by viewing arclength as the Riemann sum 

$$ \lim_{\Delta x\rightarrow\infty}\sum_{i=1}^n \sqrt{1+f'(x_i)^2}\; \Delta x $$ 

The term being summed is the length of an approximating secant to the curve over the distance $\Delta x$. As $\Delta x$ vanishes, the sum approaches the arclength, as desired. Arclength can also be derived for polar coordinates from the general formula for vector functions given above. The result is
$$ L = \int_a^b \sqrt{r(\theta)^2 + (r'(\theta))^2}\; d\theta$$
\end{document}
