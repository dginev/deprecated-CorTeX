\documentclass{article}
\usepackage{ids}
\usepackage{planetmath-specials}
\usepackage{pmath}
% this is the default PlanetMath preamble.  as your knowledge
% of TeX increases, you will probably want to edit this, but
% it should be fine as is for beginners.

% almost certainly you want these
\usepackage{amssymb}
\usepackage{amsmath}
\usepackage{amsfonts}

% used for TeXing text within eps files
%\usepackage{psfrag}
% need this for including graphics (\includegraphics)
%\usepackage{graphicx}
% for neatly defining theorems and propositions
%\usepackage{amsthm}
% making logically defined graphics
%\usepackage{xypic} 

% there are many more packages, add them here as you need them

% define commands here
\begin{document}
Let $K/F$ be a Galois extension, and let $x \in K$. The {\em trace} $\operatorname{Tr}_F^K(x)$ of $x$ is defined to be the sum of all the elements of the orbit of $x$ under the group action of the Galois group $\operatorname{Gal}(K/F)$ on $K$; taken with multiplicities if $K/F$ is a finite extension.

In the case where $K/F$ is a finite extension,
$$
\operatorname{Tr}_F^K(x) := \sum_{\sigma \in \operatorname{Gal}(K/F)} \sigma(x)
$$

The trace of $x$ is always an element of $F$, since any element of $\operatorname{Gal}(K/F)$ permutes the orbit of $x$ and thus fixes $\operatorname{Tr}_F^K(x)$.

The name ``trace'' derives from the fact that, when $K/F$ is finite, the trace of $x$ is simply the trace of the linear transformation $T: K \longrightarrow K$ of vector spaces over $F$ defined by $T(v) := xv$.
\end{document}
