\documentclass{article}
\usepackage{planetmath-specials}
\usepackage{pmath}
\usepackage{amsmath}
\usepackage{amsfonts}
\usepackage{amssymb}
\newcommand{\reals}{\mathbb{R}}
\newcommand{\natnums}{\mathbb{N}}
\newcommand{\cnums}{\mathbb{C}}
\newcommand{\znums}{\mathbb{Z}}
\newcommand{\lp}{\left(}
\newcommand{\rp}{\right)}
\newcommand{\lb}{\left[}
\newcommand{\rb}{\right]}
\newcommand{\supth}{^{\text{th}}}
\newtheorem{proposition}{Proposition}
\newtheorem{definition}[proposition]{Definition}
\newcommand{\kf}{\mathbb{K}}
\newcommand{\KP}{{\mathbb{K}\mathrm{P}}}
\newcommand{\RP}{{\reals\mathrm{P}}}
\newcommand{\bx}{\mathbf{x}}
\newcommand{\nzkn}{\kf^{n+1}\backslash \{0\}}
\newcommand{\PSL}{{\mathrm{PSL}}}
\newcommand{\SL}{{\mathrm{SL}}}
\newcommand{\PGL}{{\mathrm{PGL}}}
\newcommand{\GL}{{\mathrm{GL}}}
\newcommand{\GammaL}{\operatorname{\Gamma L}}
\newcommand{\PGammaL}{{\operatorname{P\Gamma L}}}

\begin{document}
\paragraph{Projective space and homogeneous coordinates.}
Let $\kf$ be a field.  {\em Projective space} of dimension $n$ over
$\kf$, typically denoted by $\KP^n$, is the set of lines passing
through the origin in $\kf^{n+1}$.  More formally, consider the
equivalence relation $\sim$ on the set of non-zero points $\nzkn$
defined by
$$\bx \sim \lambda \bx,\quad \bx\in \nzkn,\quad \lambda \in
\kf\backslash\{0\}.$$
Projective space is defined to be the set of the
corresponding equivalence classes.

Every $\bx=(x_0,\ldots,x_n)\in\nzkn$ determines an element of
projective space, namely the line passing through $\bx$. Formally,
this line is the equivalence class $[\bx]$, or $[x_0:x_1:\ldots:x_n]$,
as it is commonly denoted. The numbers $x_0,\ldots,x_n$ are referred
to as \emph{homogeneous coordinates} of the line.  Homogeneous coordinates
differ from ordinary coordinate systems in that a given element of
projective space is labeled by multiple homogeneous ``coordinates''.

\paragraph{Affine coordinates.} Projective space also admits a more
conventional type of coordinate system, called affine coordinates.
Let $A_0\subset\KP^n$ be the subset of all elements
$p=[x_0:x_1:\ldots:x_n]\in\KP^n$ such that $x_0\neq 0$.  We then
define the functions
$$X_i:A_0\rightarrow \kf^n,\quad i=1,\ldots,n,$$
according to
$$X_i(p) = \frac{x_i}{x_0},$$
where $(x_0,x_1,\ldots,x_n)$ is {\em
  any} element of the equivalence class representing $p$. This
definition makes sense because other elements of the same equivalence
class have the form
$$(y_0,y_1,\ldots,y_n)=(\lambda x_0,\lambda x_1,\ldots,\lambda x_n)$$
for some non-zero $\lambda\in\kf$, and hence 
$$\frac{y_i}{y_0} = \frac{x_i}{x_0}.$$

The functions $X_1,\ldots,X_n$ are called \emph{affine coordinates} relative
to the hyperplane $$H_0=\{x_0=1\}\subset\kf^{n+1}.$$
Geometrically,
affine coordinates can be described by saying that the elements of
$A_0$ are lines in $\kf^{n+1}$ that are not parallel to $H_0$, and
that every such line intersects $H_0$ in one and exactly one point.
Conversely points of $H_0$ are represented by tuples
$(1,x_1,\ldots,x_n)$ with $(x_1,\ldots,x_n)\in\kf^n$, and each such
point uniquely labels a line $[1:x_1:\ldots:x_n]$ in $A_0$.  

It must be noted that a single system of affine coordinates does not
cover {\em all} of projective space.  However, it is possible to
define a system of affine coordinates relative to every hyperplane in
$\kf^{n+1}$ that does not contain the origin.  In particular, we get
$n+1$ different systems of affine coordinates corresponding to the
hyperplanes $\{ x_i = 1\},\; i=0,1,\ldots,n.$ Every element of
projective space is covered by at least one of these $n+1$ systems of
coordinates.

\paragraph{Projective automorphisms.}
A projective automorphism, also known as a projectivity, is a
bijective transformation of projective space that preserves all
incidence relations.  For $n\geq 2$, every automorphism of $\KP^n$ is
engendered by a semilinear invertible transformation of $\kf^{n+1}$.
Let $A:\kf^{n+1}\rightarrow\kf^{n+1}$ be an invertible semilinear
transformation.  The corresponding projectivity
$[A]:\KP^n\rightarrow\KP^n$ is the transformation $$[\bx] \mapsto
[A\bx],\quad \bx\in\kf^{n+1}.$$
For every non-zero $\lambda\in\kf$ the
transformation $\lambda A$ gives the same projective automorphism as
$A$.  For this reason, it is convenient we identify the group of
projective automorphisms with the quotient $$\PGammaL_{n+1}(\kf) =
\GammaL_{n+1}(\kf)/ \kf.$$
Here $\GammaL$ refers to the group of
invertible semi-linear transformations, while the quotienting $\kf$
refers to the subgroup of scalar multiplications.

A collineation is a special kind of projective automorphism, one that
is engendered by a strictly linear transformation.  The group of
projective collineations is therefore denoted by $\PGL_{n+1}(\kf)$
Note that for fields such as $\reals$ and $\cnums$, the group of
projective collineations is also described by the projectivizations
$\PSL_{n+1}(\reals), \PSL_{n+1}(\cnums)$, of the corresponding
unimodular group.

Also note that if a field, such as $\reals$, lacks non-trivial
automorphisms, then all semi-linear transformations are linear.  For
such fields all projective automorphisms are collineations.
Thus, 
$$\PGammaL_{n+1}(\reals)=\PSL_{n+1}(\reals)=\SL_{n+1}(\reals)/\{\pm\,
I_{n+1}\}.$$  By contrast, since $\cnums$
possesses non-trivial automorphisms, complex conjugation for example,
the group of automorphisms of complex projective space is larger than
$\PSL_{n+1}(\cnums)$, where the latter denotes the quotient of
$\SL_{n+1}(\cnums)$ by the subgroup of scalings by the $(n+1)$st roots
of unity.
\end{document}
