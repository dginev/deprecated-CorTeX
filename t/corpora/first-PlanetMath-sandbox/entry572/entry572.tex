\documentclass{article}
\usepackage{planetmath-specials}
\usepackage{pmath}
\usepackage{amssymb}
\usepackage{amsmath}
\usepackage{amsfonts}
\usepackage{graphicx}
\usepackage{xypic}
\begin{document}
Let $R$ be a commutative ring with unity.  A {\it free module} over $R$ 
is a (unital) module isomorphic to a direct sum of copies of $R$.  In particular, as every abelian group is a $\mathbb{Z}$-module, a {\it free abelian group} is a direct sum of copies of $\Bbb{Z}$.  This is equivalent to saying that the module has a {\it free basis},
i.e. a set of elements with the \PMlinkescapetext{property} that every element of the module can be uniquely expressed as an linear combination over $R$
of elements of the free basis.  In the case that a free module over $R$ is a sum of finitely many copies of $R$, then the number of copies is called the {\it rank} of the free module.

An alternative definition of a free module is via its universal property:  Given a set $X$, the free $R$-module $F(X)$ on the set $X$ is equipped with a function $i:X\rightarrow F(X)$ satisfying the property that for any other $R$-module $A$ and any function $f:X\rightarrow A$, there exists a unique $R$-module map $h:F(X)\rightarrow A$ such that $(h\circ i)=f$.
\end{document}
