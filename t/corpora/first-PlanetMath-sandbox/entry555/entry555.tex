\documentclass{article}
\usepackage{planetmath-specials}
\usepackage{pmath}
\usepackage{amssymb}
\usepackage{amsmath}
\usepackage{amsfonts}

\def\im{\operatorname{im}}
\def\ker{\operatorname{ker}}
\begin{document}
\PMlinkescapeword{adjacent}
\PMlinkescapeword{complex}
\PMlinkescapeword{equivalent}
\PMlinkescapeword{relation}
\PMlinkescapeword{satisfies}

Let $R$ be a ring.
A sequence of \PMlinkname{$R$-modules}{Module} and homomorphisms
\[
  \cdots \rightarrow
  A_{n+1} \buildrel {d_{n+1}} \over \longrightarrow
  A_n \buildrel {d_n} \over \longrightarrow
  A_{n-1} \rightarrow
  \cdots
\]
is said to be a \emph{chain complex}
(or \emph{$R$-complex}, or just \emph{complex})
if each pair of adjacent homomorphisms $(d_{n+1}, d_n)$
satisfies the relation $d_n\circ d_{n+1} = 0$.
This is equivalent to saying that 
$\im d_{n+1} \subseteq \ker d_n$.
We often denote such a complex by $({\bold A}, d)$, or simply ${\bold A}$.

Compare this to the notion of an exact sequence,
which requires $\im d_{n+1} = \ker d_n$.

The homomorphisms $d_n$ in the chain complex
are called \emph{boundary operators}, or \emph{boundary maps}.

\end{document}
