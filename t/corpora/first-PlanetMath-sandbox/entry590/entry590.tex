\documentclass{article}
\usepackage{planetmath-specials}
\usepackage{pmath}
\usepackage{graphicx}
\usepackage{psfrag} 
\usepackage{bbm}
\newcommand{\Z}{\mathbbmss{Z}}
\newcommand{\C}{\mathbbmss{C}}
\newcommand{\R}{\mathbbmss{R}}
\newcommand{\Q}{\mathbbmss{Q}}
\newcommand{\mathbb}[1]{\mathbbmss{#1}}
\newcommand{\figura}[1]{\begin{center}\includegraphics{#1}\end{center}}
\newcommand{\figuraex}[2]{\begin{center}\includegraphics[#2]{#1}\end{center}}
\begin{document}
The Lemoine point of a triangle, is the intersection point of its three symmedians. (That is, the isogonal conjugate of the centroid).

It is related with the Gergonne point by the following result: \\
On any triangle $ABC$, the Lemoine point of its Gergonne triangle is the Gergonne point of $ABC$.
\psfrag{L}{$L$}
\psfrag{I}{$I$}
\psfrag{G}{$G$}
\psfrag{A}{$A$}
\psfrag{B}{$B$}
\psfrag{C}{$C$}
\figura{lemoinep}
In the picture, the blue lines are the medians, intersecting an the centroid  $G$.
The green lines are anglee bisectors intersecting at the incentre $I$ and the red lines are symmedians. The symmedians intersect at Lemoine point $L$.
\end{document}
