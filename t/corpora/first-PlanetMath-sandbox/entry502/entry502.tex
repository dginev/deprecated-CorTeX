\documentclass{article}
\usepackage{planetmath-specials}
\usepackage{pmath}
\usepackage{amssymb}
\usepackage{amsmath}
\usepackage{amsfonts}
\usepackage{graphicx}
\usepackage{xypic}
\begin{document}
\section{Pivoting}

Pivoting is a process performed on a matrix in order to improve numerical stability.

\emph{Partial pivoting} of an $n \times n$ matrix is the sorting of the rows of the matrix so that row $i$ contains the maximum absolute column value for column $i$, among all rows $i, \ldots, n$. That is, we begin by swapping row 1 with the row that has the largest absolute value for the first column, then swap row 2 with the row that has the largest magnitude for the second column (among rows 2 and below), and so on.

\emph{Complete pivoting} is a reordering of both rows and columns, using the same method as above.  It is usually not necessary to ensure numerical stability.

Pivoting can be represented as multiplication by permutation matrices.

\begin{thebibliography}{3}
\bibitem{Golub} G. H. Golub, C. F. Loan, Matrix Computations, 3rd edition, Johns Hopkins, 1996.
\end{thebibliography}
\end{document}
