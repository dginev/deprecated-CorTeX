\documentclass{article}
\usepackage{ids}
\usepackage{planetmath-specials}
\usepackage{pmath}
\usepackage{amssymb}
\usepackage{amsmath}
\usepackage{amsfonts}
\usepackage{xypic}
\usepackage{color}
\begin{document}
The \emph{chromatic number} of a graph is the minimum number of colours required to colour it.

Consider the following graph:

$$\xymatrix{
{\color{red}A} \ar@{-}[r] & {\color{blue}B} \ar@{-}[r] \ar@{-}[d] & {\color{red}C} \ar@{-}[r] \ar@{-}[dl] & {\color{green}F} \ar@{-}[dl] \\
& {\color{green}D} \ar@{-}[r] & {\color{blue}E} & }$$

This graph has been coloured using $3$ colours.  Furthermore, it's clear that it cannot be coloured with fewer than $3$ colours, as well: it contains a subgraph ($BCD$) that is isomorphic to the complete graph of $3$ vertices.  As a result, the chromatic number of this graph is indeed $3$.

This example was easy to solve by inspection.  In general, however, finding the chromatic number of a large graph (and, similarly, an optimal colouring) is a very difficult (NP-hard) problem.
\end{document}
