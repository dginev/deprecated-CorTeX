\documentclass{article}
\usepackage{ids}
\usepackage{planetmath-specials}
\usepackage{pmath}
\usepackage{amssymb}
\usepackage{amsmath}
\usepackage{amsfonts}
\begin{document}
\PMlinkescapeword{fixed}
\PMlinkescapeword{name}
\PMlinkescapeword{equivalent}
\PMlinkescapeword{theory}
For positive integers $s$ and $n$, the complex number
$$c_s(n)=\sum_{\substack{0\le k<n \\ (k,n)=1}}e^{2\pi iks/n}$$
is referred to as a \emph{Ramanujan sum}, or a \emph{Ramanujan trigonometric sum}.
Since $e^{2\pi i}=1$, an equivalent definition is
$$c_s(n)=\sum_{k\in r(n)}e^{2\pi iks/n}$$
where $r(n)$ is some reduced residue system mod $n$, meaning any
subset of $\mathbb{Z}$ containing exactly one element of each
invertible residue class mod $n$.

Using a symmetry argument about roots of unity, one can show
$$
\sum_{d|s}c_s(n)=
\begin{cases}
s & \textrm{if $s|n$} \\ 0 & \text{otherwise.}
\end{cases}
$$
Applying M\"obius inversion, we get
$$c_s(n)=\sum_{\substack{d|n \\ d|s}}\mu(n/d)d
=\sum_{d|(n,s)}\mu(n/d)d$$
which shows that $c_s(n)$ is a real number, and indeed an integer. 
In particular $c_s(1)=\mu(s)$. More generally,
$$c_{st}(mn)=c_s(m)c_t(n)\text{ if }(m,t)=(n,s)=1\;.$$

Using the Chinese remainder theorem, it is not hard to show that
for any fixed $n$, the function $s\mapsto c_s(n)$ is multiplicative:
$$c_s(n)c_t(n)=c_{st}(n) \text{ if }(s,t)=1 \;.$$

If $m$ is invertible mod $n$, then the mapping $k\mapsto km$
is a permutation of the invertible residue classes mod $n$. Therefore
$$c_s(mn)=c_s(n)\text{ if }(m,s)=1\;.$$
\textbf{Remarks: } Trigonometric sums often make
convenient apparatus in number theory, since any
function on a quotient ring of $\mathbb{Z}$ defines
a periodic function on $\mathbb{Z}$ itself, and conversely. For
another example, see Landsberg-Schaar relation.

Some writers use different notation from ours, reversing the roles
of $s$ and $n$ in the expression $c_s(n)$.

The name ``Ramanujan sum'' was introduced by Hardy.
\end{document}
