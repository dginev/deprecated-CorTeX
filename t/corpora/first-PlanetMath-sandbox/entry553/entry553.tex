\documentclass{article}
\usepackage{planetmath-specials}
\usepackage{pmath}
\usepackage{amssymb}
\usepackage{amsmath}
\usepackage{amsfonts}
\usepackage{graphicx}
\usepackage{xypic}
\begin{document}
We are said to be using {\PMlinkescapetext {\it left function notation}}
if we write functions to the left of their arguments.
That is, if $\alpha : X \to Y$ is a function and $x \in X$,
then $\alpha x$ is the image of $x$ under $\alpha$.

Furthermore, if we have a function $\beta : Y \to Z$,
then we write the composition of the two functions
as $\beta \alpha : X \to Z$,
and the image of $x$ under the composition
as $\beta \alpha x = (\beta \alpha) x = \beta(\alpha x)$.

Compare this to right function notation.
\end{document}
