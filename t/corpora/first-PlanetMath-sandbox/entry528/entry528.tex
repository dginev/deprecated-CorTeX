\documentclass{article}
\usepackage{ids}
\usepackage{planetmath-specials}
\usepackage{pmath}
\usepackage{amssymb}
\usepackage{amsmath}
\usepackage{amsfonts}
\usepackage{graphicx}
\usepackage{xypic}
\begin{document}
A subgroup $H$ of a group $G$ is {\em normal} if $aH = Ha$ for all $a \in G$. Equivalently, $H \subset G$ is normal if and only if $aHa^{-1} = H$ for all $a \in G$, i.e., if and only if each conjugacy class of $G$ is either entirely inside $H$ or entirely outside $H$.

The notation $H \trianglelefteq G$ or $H \triangleleft G$ is often used to denote that $H$ is a normal subgroup of $G$.

The kernel $\ker(f)$ of any group homomorphism $f: G \longrightarrow G'$ is a normal subgroup of $G$. More surprisingly, the converse is also true: any normal subgroup $H \subset G$ is the kernel of some homomorphism (one of these being the projection map $\rho: G \longrightarrow G/H$, where $G/H$ is the quotient group).
\end{document}
