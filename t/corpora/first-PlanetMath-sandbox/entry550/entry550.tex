\documentclass{article}
\usepackage{ids}
\usepackage{planetmath-specials}
\usepackage{pmath}
% This is Cosmin's preamble version 1.01 (planetmath.org version)
 
% Packages
  \usepackage{amsmath}
  \usepackage{amssymb}
  \usepackage{amsfonts}
  \usepackage{amsthm}
  \usepackage{mathrsfs}
  %\usepackage[]{graphicx}
  %\usepackage{xypic}
  %\usepackage[]{babel}

% Theorem Environments
  \newtheorem*{thm}{Theorem}
  \newtheorem{thmn}{Theorem}
  \newtheorem*{lem}{Lemma}
  \newtheorem{lemn}{Lemma}
  \newtheorem*{cor}{Corollary}
  \newtheorem{corn}{Corollary}
  \newtheorem*{prop}{Proposition}
  \newtheorem{propn}{Proposition}

% New Commands
  % Other Commands
    \renewcommand{\geq}{\geqslant}
    \renewcommand{\leq}{\leqslant}
    \newcommand{\vect}[1]{\boldsymbol{#1}}
    \renewcommand{\div}{\!\mid\!}
  % Local
    %\DeclareMathOperator{}{}
\begin{document}
\PMlinkescapeword{inverse}
\PMlinkescapeword{pairing}
\PMlinkescapeword{satisfy}
\PMlinkescapeword{equation}

We first show that, if $p$ is a prime, then $(p-1)! \equiv -1 \pmod p.$ Since $p$ is prime, $\mathbb{Z}_p$ is a field and thus, pairing off each element with its inverse in the product $(p-1)! = \prod_{x=1}^{p-1}x,$ we are left with the elements which are their own inverses (i.e. which satisfy the equation $x^2 \equiv 1 \pmod p$), $1$ and $-1$, only. Consequently, $(p-1)! \equiv -1 \pmod p.$

To prove that the condition is necessary, suppose that $(p-1)! \equiv -1 \pmod p$ and that $p$ is not a prime. The case $p=1$ is trivial. Since $p$ is composite, it has a divisor $k$ such that $1 < k < p$, and we have $(p-1)! \equiv -1 \pmod k$. However, since $k\leq p-1$, it divides $(p-1)!$ and thus $(p-1)! \equiv 0 \pmod k,$ a contradiction.
\end{document}
