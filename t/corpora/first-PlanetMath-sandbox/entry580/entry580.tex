\documentclass{article}
\usepackage{ids}
\usepackage{planetmath-specials}
\usepackage{pmath}
\usepackage{amssymb}
\usepackage{amsmath}
\usepackage{amsfonts}
\usepackage{graphicx}
\usepackage{xypic}
\begin{document}
Let $(X,\leq)$ be a linearly ordered set.  The \emph{order topology} on $X$ is defined to be the topology $\mathcal{T}$ generated by the subbasis consisting of open rays, that is sets of the form $$(x,\infty)=\{ y\in X|y>x\}$$ $$(-\infty,x)=\{ y\in X|y<x\},$$
for some $x\in X$.

This is equivalent to saying that $\mathcal{T}$ is generated by the basis of open intervals; that is, the open rays as defined above, together with sets of the form $$(x,y)=\{ z\in X|x<z<y\}$$ for some $x,y\in X$.

The standard topologies on $\mathbb{R}$, $\mathbb{Q}$ and $\mathbb{N}$ are the same as the order topologies on these sets.

If $Y$ is a subset of $X$, then $Y$ is a linearly ordered set under the induced order from $X$.  Therefore, $Y$ has an order topology $\mathcal{S}$ defined by this ordering, the \emph{induced order topology}.  Moreover, $Y$ has a subspace topology $\mathcal{T}'$ which it inherits as a subspace of the topological space $X$.  The subspace topology is always finer than the induced order topology, but they are not in general the same.

For example, consider the subset $Y=\{ -1\}\cup\{ \frac{1}{n} \mid n\in\mathbb{N}\}\subseteq\mathbb{Q}$.  Under the subspace topology, the singleton set $\{ -1\}$ is open in $Y$, but under the order topology on $Y$, 
any open set containing $-1$ must contain all but finitely many members of the space.

A chain $X$ under the order topology is Hausdorff: pick any two distinct points $x, y \in X$; without loss of generality, say $x < y$.    If there is a $z$ such that $x < z < y$, then $(-\infty,z)$ and $(z,\infty)$ are disjoint open sets separating $x$ and $y$.  If no $z$ were between $x$ and $y$, then $(-\infty,y)$ and $(x,\infty)$ are disjoint open sets separating $x$ and $y$.
\end{document}
