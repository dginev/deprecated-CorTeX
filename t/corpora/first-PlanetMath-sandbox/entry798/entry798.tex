\documentclass{article}
\usepackage{planetmath-specials}
\usepackage{pmath}
\usepackage{amssymb}
\usepackage{amsmath}
\usepackage{amsfonts}

% used for TeXing text within eps files
%\usepackage{psfrag}
% need this for including graphics (\includegraphics)
%\usepackage{graphicx}
% for neatly defining theorems and propositions
%\usepackage{amsthm}
% making logically defined graphics
%\usepackage{xypic} 

% there are many more packages, add them here as you need them

% define commands here
\newcommand{\md}{d}
\newcommand{\mv}[1]{\mathbf{#1}}	% matrix or vector
\newcommand{\mvt}[1]{\mv{#1}^{\mathrm{T}}}
\newcommand{\mvi}[1]{\mv{#1}^{-1}}
\newcommand{\mderiv}[1]{\frac{\md}{\md {#1}}} %d/dx
\newcommand{\mnthderiv}[2]{\frac{\md^{#2}}{\md {#1}^{#2}}} %d^n/dx
\newcommand{\mpderiv}[1]{\frac{\partial}{\partial {#1}}} %partial d^n/dx
\newcommand{\mnthpderiv}[2]{\frac{\partial^{#2}}{\partial {#1}^{#2}}} %partial d^n/dx
\newcommand{\borel}{\mathfrak{B}}
\newcommand{\integers}{\mathbb{Z}}
\newcommand{\rationals}{\mathbb{Q}}
\newcommand{\reals}{\mathbb{R}}
\newcommand{\complexes}{\mathbb{C}}
\newcommand{\defined}{:=}
\newcommand{\var}{\mathrm{var}}
\newcommand{\cov}{\mathrm{cov}}
\newcommand{\corr}{\mathrm{corr}}
\newcommand{\set}[1]{\{#1\}}
\newcommand{\bra}[1]{\langle#1 \vert}
\newcommand{\ket}[1]{\vert \hspace{1pt}#1\rangle}
\newcommand{\braket}[2]{\langle #1 \ket{#2}}
\begin{document}
Let $q(t)$ be a function $\reals \to \reals$, $\dot{q} = \mderiv{t}{q}$, and $L = L(q, \dot{q}, t)$.  Begin with the time-relative Euler-Lagrange condition

\begin{equation}\label{el}
\mpderiv{q} L - \mderiv{t}\left(\mpderiv{\dot{q}} L\right) = 0.
\end{equation}

If $\mpderiv{t}L = 0$, then the Euler-Lagrange condition reduces to

\begin{equation}
L - \dot{q}{\mpderiv{\dot{q}} L} = C,
\end{equation}
which is the \emph{Beltrami identity}.  In the calculus of variations, the ability to use the Beltrami identity can vastly simplify problems, and as it happens, many physical problems have $\mpderiv{t}L = 0$.

In space-relative terms, with $q' \defined \mderiv{x}q$, we have 
\begin{equation}
\mpderiv{q} L - \mderiv{x}{\mpderiv{q'} L} = 0.
\end{equation}

If $\mpderiv{x}L = 0$, then the Euler-Lagrange condition reduces to

\begin{equation}
L - q' {\mpderiv{q'} L} = C.
\end{equation}

To derive the Beltrami identity, note that
\begin{equation}\label{step2}
\mderiv{t}\left(\dot{q}\mpderiv{\dot{q}}L\right) = \ddot{q}\mpderiv{\dot{q}}L + \dot{q}\mderiv{t}\left(\mpderiv{\dot{q}}L\right) 
\end{equation}
Multiplying (1) by $\dot{q}$, we have
\begin{equation}\label{step3}
\dot{q}\mpderiv{q} L - \dot{q}\mderiv{t}\left(\mpderiv{\dot{q}} L\right) = 0.
\end{equation}
Now, rearranging (5) and substituting in for the rightmost term of (6), we obtain
\begin{equation}\label{step4}
\dot{q}\mpderiv{q} L + \ddot{q}\mpderiv{\dot{q}}L - \mderiv{t}\left(\dot{q}\mpderiv{\dot{q}}L\right) = 0.
\end{equation}
Now consider the total derivative
\begin{equation}\label{step1}
\mderiv{t}L(q, \dot{q}, t) = \dot{q}\mpderiv{q}L + \ddot{q}\mpderiv{\dot{q}}L + \mpderiv{t}L.
\end{equation}
If $\mpderiv{t}L = 0$, then we can substitute in the left-hand side of (8) for the leading portion of (7) to get
\begin{equation}
\mderiv{t}L - \mderiv{t}\left(\dot{q}\mpderiv{\dot{q}}L\right) = 0.
\end{equation}
Integrating with respect to $t$, we arrive at
\begin{equation}
L - \dot{q}{\mpderiv{\dot{q}} L} = C,
\end{equation}
which is the Beltrami identity.
\end{document}
