\documentclass{article}
\usepackage{ids}
\usepackage{planetmath-specials}
\usepackage{pmath}
\usepackage{amssymb}
\usepackage{amsmath}
\usepackage{amsfonts}
\usepackage{graphicx}
\usepackage{xypic}
\begin{document}
Let $B$ be a ring with a subring $A$. The {\em integral closure} of $A$ in $B$ is the set $A' \subset B$ consisting of all elements of $B$ which are integral over $A$.

It is a theorem that the integral closure of $A$ in $B$ is itself a ring. In the special case where $A = \mathbb{Z}$, the integral closure $A'$ of $\mathbb{Z}$ is often called the {\em ring of integers} in $B$.
\end{document}
