\documentclass{article}
\usepackage{ids}
\usepackage{planetmath-specials}
\usepackage{pmath}
% this is the default PlanetMath preamble.  as your knowledge
% of TeX increases, you will probably want to edit this, but
% it should be fine as is for beginners.

% almost certainly you want these
\usepackage{amssymb}
\usepackage{amsmath}
\usepackage{amsfonts}

% used for TeXing text within eps files
%\usepackage{psfrag}
% need this for including graphics (\includegraphics)
%\usepackage{graphicx}
% for neatly defining theorems and propositions
%\usepackage{amsthm}
% making logically defined graphics
%\usepackage{xypic} 

% there are many more packages, add them here as you need them

% define commands here
\begin{document}
{\bf Definition} Suppose that $X$ and $Y$ are topological spaces and 
$f: X \to Y$ is a continuous map. 
If there exists a 
continuous map $g:Y \to X$ such that $f\circ g \simeq id_{Y}$
(i.e. $f\circ g$ is \PMlinkid{homotopic}{1584} to the identity 
mapping on $Y$), 
and $g \circ f \simeq id_{X}$, then 
$f$ is a \emph{homotopy equivalence}.
This homotopy equivalence is sometimes called
\emph{strong homotopy equivalence} to distinguish it from
weak homotopy equivalence.

If there exist a homotopy equivalence between the topological 
spaces $X$ and $Y$, we say that $X$ and $Y$ are  
\emph{homotopy equivalent}, or that 
$X$ and $Y$ are of the same \emph{homotopy type}. 
We then write  $X\simeq Y$. 

\subsubsection{Properties}
\begin{enumerate}
\item Any homeomorphism $f:X\to Y$ is obviously a homotopy equivalence with 
$g=f^{-1}$.
\item For topological spaces, homotopy equivalence is an 
equivalence relation.
\item A topological space $X$ is (by definition) contractible, 
if $X$ is homotopy equivalent to a point, i.e., $X\simeq \{x_0\}$.
\end{enumerate}

\begin{thebibliography}{9}
 \bibitem{hatcher_at} A. Hatcher, \emph{Algebraic Topology}, Cambridge University Press, 2002. Also available
 \PMlinkexternal{online}{http://www.math.cornell.edu/~hatcher/AT/ATpage.html}.
 \end{thebibliography}
\end{document}
