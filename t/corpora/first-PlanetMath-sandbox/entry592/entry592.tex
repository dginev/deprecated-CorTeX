\documentclass{article}
\usepackage{ids}
\usepackage{planetmath-specials}
\usepackage{pmath}
\usepackage{graphicx}
\usepackage{psfrag} 
\usepackage{bbm}
\newcommand{\Z}{\mathbbmss{Z}}
\newcommand{\C}{\mathbbmss{C}}
\newcommand{\R}{\mathbbmss{R}}
\newcommand{\Q}{\mathbbmss{Q}}
\newcommand{\mathbb}[1]{\mathbbmss{#1}}
\newcommand{\figura}[1]{\begin{center}\includegraphics{#1}\end{center}}
\newcommand{\figuraex}[2]{\begin{center}\includegraphics[#2]{#1}\end{center}}
\begin{document}
The \emph{incircle} or \emph{inscribed circle} of a triangle is a circle interior to the triangle and tangent to its three sides. 

Moreover, the incircle of a polygon is an interior circle tangent to all of the polygon's sides. Not every polygon has an inscribed circle, but triangles always do.\psfrag{A}{$A$}\psfrag{B}{$B$}\psfrag{C}{$C$}\psfrag{X}{$X$}\psfrag{Y}{$Y$}\psfrag{Z}{$Z$}\psfrag{I}{$I$}

The center of the incircle is called the incenter, and it's located at the point where the three angle bisectors intersect.
\figura{incentre}

If the sides of a triangle are $x$, $y$ and $z$, the area $A$ and the semiperimeter $p$, then the radius of incircle may be calculated from
$$r = \frac{2A}{x+y+z} = \frac{A}{p} = \sqrt\frac{(p-x)(p-y)(p-z)}{p}.$$
\end{document}
