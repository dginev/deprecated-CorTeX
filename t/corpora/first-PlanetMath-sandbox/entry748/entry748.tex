\documentclass{article}
\usepackage{ids}
\usepackage{planetmath-specials}
\usepackage{pmath}
% this is the default PlanetMath preamble.  as your knowledge
% of TeX increases, you will probably want to edit this, but
% it should be fine as is for beginners.

% almost certainly you want these
\usepackage{amssymb}
\usepackage{amsmath}
\usepackage{amsfonts}

% used for TeXing text within eps files
%\usepackage{psfrag}
% need this for including graphics (\includegraphics)
%\usepackage{graphicx}
% for neatly defining theorems and propositions
%\usepackage{amsthm}
% making logically defined graphics
%\usepackage{xypic} 

% there are many more packages, add them here as you need them

% define commands here
\begin{document}
The {\em category of sets} has, as its objects, all sets and, 
as its morphisms, functions between sets.  (This works if a 
category's objects are only required to be part of a class, 
as the class of all sets exists.)  The category of sets is often denoted by \textbf{Set}.

Alternately one can specify a universe, containing all sets 
of interest in the situation, and take the category to contain 
only sets in that universe and functions between those sets.

One of the most famous endofunctors associated with the category of sets 
is the \emph{powerset functor} $P$, which takes every set $A$ to its 
power set $P(A)$, and any function $f:A\to B$ to the function 
$P(f): P(A)\to P(B)$, given by 
\[
 P(f)(S) := f(S) = \lbrace b\in B\mid b = 
 f(a)\mbox{ for some }a\in S\rbrace.
\]
If $f \colon A\to B$ and $g \colon B\to C$ are functions, then $(P(g)\circ P(f))(S) = 
P(g)(P(f)(S)) = P(g)(f(S))=g(f(S))= (g\circ f)(S)=P(g\circ f)(S)$, 
so that $P$ is a covariant functor.  This functor may also be defined
in an ``arrow theoretic'' fashion as a ${\rm Hom}$ functor.  Let $T$
be a set with two elements, for instance $T = \{ \{\}, \{\{\}\}\}$.
(Since, by the definition of cardinality, all sets with the same 
number of elements are isomorphic in the category of sets, it does
not matter which set with two elements we pick as $T$.) 
Then define $P(A) = {\rm Hom} (A,T)$; likewise, given a function
$f \colon A \to B$, define $P(f) \colon P(B) \to P(A)$ by
$P(f)(g) = g \circ f$.
\end{document}
