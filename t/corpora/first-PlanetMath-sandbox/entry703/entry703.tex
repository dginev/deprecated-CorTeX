\documentclass{article}
\usepackage{ids}
\usepackage{planetmath-specials}
\usepackage{pmath}
% this is the default PlanetMath preamble.  as your knowledge
% of TeX increases, you will probably want to edit this, but
% it should be fine as is for beginners.

% almost certainly you want these
\usepackage{amssymb}
\usepackage{amsmath}
\usepackage{amsfonts}

% used for TeXing text within eps files
%\usepackage{psfrag}
% need this for including graphics (\includegraphics)
%\usepackage{graphicx}
% for neatly defining theorems and propositions
%\usepackage{amsthm}
% making logically defined graphics
\usepackage{xypic} 
%\input xy
%\xyoptions{all}

% there are many more packages, add them here as you need them

% define commands here
\begin{document}
A \emph{graph isomorphism} is a bijection between the vertices of two graphs $G$ and $H$:
$$f: V(G) \rightarrow V(H)$$
with the property that any two vertices $u$ and $v$ from $G$ are adjacent if and only if $f(u)$ and $f(v)$ are adjacent in $H$.

If an isomorphism can be constructed between two graphs, then we say those graphs are \emph{isomorphic}.

For example, consider these two graphs:

$$\xymatrix{
a \ar@{-}[r] \ar@{-}[rd] \ar@{-}[rdd] & g \\
b \ar@{-}[ru] \ar@{-}[r] \ar@{-}[rdd] & h \\
c \ar@{-}[ruu] \ar@{-}[r] \ar@{-}[rd] & i \\
d \ar@{-}[ruu] \ar@{-}[ru] \ar@{-}[r] & j}$$

$$\xymatrix{
1 \ar@{-}[rrr] \ar@{-}[rd] \ar@{-}[ddd] & & & 2 \\
& 5 & 6 \ar@{-}[ur] \ar@{-}[l] \ar@{-}[d] & \\
& 8 \ar@{-}[u] \ar@{-}[r] \ar@{-}[dl] & 7 & \\
4 & & & 3 \ar@{-}[uuu] \ar@{-}[lll] \ar@{-}[ul] }$$

Although these graphs look very different at first, they are in fact isomorphic; one isomorphism between them is
$$f(a) = 1$$
$$f(b) = 6$$
$$f(c) = 8$$
$$f(d) = 3$$
$$f(g) = 5$$
$$f(h) = 2$$
$$f(i) = 4$$
$$f(j) = 7$$
\end{document}
