\documentclass{article}
\usepackage{planetmath-specials}
\usepackage{pmath}
% this is the default PlanetMath preamble.  as your knowledge
% of TeX increases, you will probably want to edit this, but
% it should be fine as is for beginners.

% almost certainly you want these
\usepackage{amssymb}
\usepackage{amsmath}
\usepackage{amsfonts}

% used for TeXing text within eps files
%\usepackage{psfrag}
% need this for including graphics (\includegraphics)
%\usepackage{graphicx}
% for neatly defining theorems and propositions
%\usepackage{amsthm}
% making logically defined graphics
%\usepackage{xypic} 

% there are many more packages, add them here as you need them

% define commands here
\newcommand{\GL}{\operatorname{GL}}
\newcommand{\lra}{\longrightarrow}
\begin{document}
Let $G$ be a group, $H \subset G$ a subgroup, and $V$ a representation of $H$, considered as a $\mathbb{Z}[H]$--module. The {\em induced representation} of $\rho$ on $G$, denoted $\operatorname{Ind}_H^G(V)$, is the $\mathbb{Z}[G]$--module whose underlying vector space is the direct sum
$$
\bigoplus_{\sigma \in G/H} \sigma V
$$
of formal translates of $V$ by left cosets $\sigma$ in $G/H$, and whose multiplication operation is defined by choosing a set $\{g_\sigma\}_{\sigma \in G/H}$ of coset representatives and setting
$$
g(\sigma v) := \tau (h v)
$$
where $\tau$ is the unique left coset of $G/H$ containing $g \cdot g_\sigma$ (i.e., such that $g \cdot g_\sigma = g_\tau \cdot h$ for some $h \in H$).

One easily verifies that the representation $\operatorname{Ind}_H^G(V)$ is independent of the choice of coset representatives $\{g_\sigma\}$.
\end{document}
