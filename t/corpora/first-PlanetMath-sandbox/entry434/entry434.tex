\documentclass{article}
\usepackage{planetmath-specials}
\usepackage{pmath}
\usepackage{amssymb}
\usepackage{amsmath}
\usepackage{amsfonts}

\begin{document}
A variety over an algebraically closed field $k$
is \emph{nonsingular} at a point $x$
if the local ring $\mathcal{O}_x$ is a regular local ring.
Equivalently, if around the point one has an open affine neighborhood
wherein the variety is cut out
by certain polynomials $F_1, \ldots, F_n$ of $m$ variables $x_1, \ldots, x_m$,
then it is nonsingular at $x$ if the Jacobian has maximal rank at that point.
Otherwise, $x$ is a \emph{singular point}.

A variety is \emph{nonsingular} if it is nonsingular at each point.

Over the real or complex numbers, nonsingularity corresponds to ``smoothness'':
at nonsingular points, varieties are locally real or complex manifolds
(this is simply the implicit function theorem).
Singular points generally have ``corners'' or self intersections.
Typical examples are the curves $x^2=y^3$,
which has a cusp at $(0,0)$ and is nonsingular everywhere else,
and $x^2(x+1)=y^2$,
which has a self-intersection at $(0,0)$ and is nonsingular everywhere else.
\end{document}
