\documentclass{article}
\usepackage{planetmath-specials}
\usepackage{pmath}
\usepackage{amssymb}
\usepackage{amsmath}
\usepackage{amsfonts}
\usepackage{graphicx}
\usepackage{xypic}
\begin{document}
In a probability space, we say that the random events $A_1,\dots,A_n$ are 
\emph{independent} if
$$ P(A_{i_1}\cap A_{i_2}\cap\dots\cap A_{i_k}) = P(A_{i_1})\dots P(A_{i_k}) $$
for all $i_1,\dots,i_k$ such that $1\leq i_1<i_2<\cdots<i_k\leq n$.

An arbitrary family of random events is independent if every finite subfamily is independent.

The random variables $X_1,\dots,X_n$ are independent if, given any Borel sets $B_1,\dots,B_n$, the random events $[X_1\in B_1],\dots,[X_n\in B_n]$ are independent. This is equivalent to saying that 

\[F_{X_1,\dots,X_n} = F_{X_1}\dots F_{X_n}\]

where $F_{X_1},\dots, F_{X_n}$ are the distribution functions of $X_1,\dots, X_n$, respectively, and $F_{X_1,\dots,X_n}$ is the joint distribution function. When the density functions $f_{X_1},\dots,f_{X_n}$ and $f_{X_1,\dots,X_n}$ exist, an equivalent condition for independence is that

\[f_{X_1,\dots,X_n} = f_{X_1}\dots f_{X_n}.\]

An arbitrary family of random variables is independent if every finite subfamily is independent.
\end{document}
