\documentclass{article}
\usepackage{planetmath-specials}
\usepackage{pmath}
\usepackage{graphicx}
\usepackage{bbm}
\newcommand{\Z}{\mathbbmss{Z}}
\newcommand{\C}{\mathbbmss{C}}
\newcommand{\R}{\mathbbmss{R}}
\newcommand{\Q}{\mathbbmss{Q}}
\newcommand{\mathbb}[1]{\mathbbmss{#1}}
\begin{document}
If an hexagon $ADBFCE$ (not necessarily convex) is inscribed into a conic (in particular into a circle), then the points of intersections of opposite sides
($AD$ with $FC$, $DB$with $CE$ and $BF$ with $EA$) are collinear. This line is called the \emph{Pascal line} of the hexagon.

A very special case happens when the conic degenerates into two lines, however the theorem still holds although this particular case is usually called Pappus theorem.
\begin{center}
\includegraphics{hexpasc}
\end{center}
\end{document}
