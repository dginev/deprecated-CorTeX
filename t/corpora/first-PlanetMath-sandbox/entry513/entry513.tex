\documentclass{article}
\usepackage{planetmath-specials}
\usepackage{pmath}
\usepackage{amssymb}
\usepackage{amsmath}
\usepackage{amsfonts}
\usepackage{graphicx}
\usepackage{xypic}

\begin{document}
The inverse $A^{-1}$ of a matrix $A$ exists only if $A$ is square and has full rank. In this case, $Ax = b$ has the solution $x = A^{-1}b$.

The \emph{pseudoinverse} $A^+$ (beware, it is often denoted otherwise) is a generalization of the inverse, and exists for any $m \times n$ matrix. We assume $m > n$. If $A$ has full rank ($n$) we define:

$$ A^+ = (A^T A)^{-1} A^T $$

and the solution of $Ax = b$ is $x = A^+b$.

More accurately, the above is called the \emph{Moore-Penrose pseudoinverse}.

\section{Calculation}

The best way to compute $A^+$ is to use singular value decomposition. With $A=USV^T$ , where $U$ and $V$ (both $n \times n$) orthogonal and $S$ ($m \times n$) is diagonal with real, non-negative singular values $\sigma_i$, $i=1,\ldots,n$.  We find 

$$ A^+ = V(S^TS)^{-1}S^TU^T $$

If the rank $r$ of $A$ is smaller than $n$, the inverse of $S^TS$ does not exist, and one uses only the first $r$ singular values; $S$ then becomes an $r \times r$ matrix and $U$,$V$ shrink accordingly. see also Linear Equations.

\section{Generalization}

The term ``pseudoinverse'' is actually used for any operator $\operatorname{pinv}$ satisfying

$$ M \operatorname{pinv}(M) M = M $$

for a $m \times n$ matrix $M$.  Beyond this, pseudoinverses can be 
defined on any reasonable matrix identity.

{\bf References}

\begin{itemize}
\item Originally from The Data Analysis Briefbook
(\PMlinkexternal{http://rkb.home.cern.ch/rkb/titleA.html}{http://rkb.home.cern.ch/rkb/titleA.html})
\end{itemize}
\end{document}
