\documentclass{article}
\usepackage{planetmath-specials}
\usepackage{pmath}
% this is the default PlanetMath preamble.  as your knowledge
% of TeX increases, you will probably want to edit this, but
% it should be fine as is for beginners.

% almost certainly you want these
\usepackage{amssymb}
\usepackage{amsmath}
\usepackage{amsfonts}

% used for TeXing text within eps files
%\usepackage{psfrag}
% need this for including graphics (\includegraphics)
%\usepackage{graphicx}
% for neatly defining theorems and propositions
%\usepackage{amsthm}
% making logically defined graphics
%\usepackage{xypic} 

% there are many more packages, add them here as you need them

% define commands here
\begin{document}
Let $F,G: \mathcal{C}\to \mathcal{D}$ be a pair of functors from the category $\mathcal{C}$ to the category $\mathcal{D}$.  A natural transformation between functors $\tau : F\to G$ is called a {\em natural equivalence} (or a {\em natural isomorphism}) if there is a natural transformation $\sigma : G\to F$ such that $\tau\bullet \sigma = {\rm id}_G$ and $\sigma\bullet \tau = {\rm id}_F$ where ${\rm id}_F$ is the identity natural transformation on $F$, and composition $\bullet$ is the usual (vertical) composition on natural transformations.

Equivalently, one can define a natural equivalence from functors $F$ to $G$ to be a natural transformation $\tau$ such that for each object $A$ in $\mathcal{C}$, the morphism $\tau_A : F(A)\to G(A)$ is an isomorphism in $\mathcal{D}$.
\end{document}
