\documentclass{article}
\usepackage{ids}
\usepackage{planetmath-specials}
\usepackage{pmath}
%\usepackage{graphicx}
%\usepackage{xypic} 
\usepackage{bbm}
\newcommand{\Z}{\mathbbmss{Z}}
\newcommand{\C}{\mathbbmss{C}}
\newcommand{\R}{\mathbbmss{R}}
\newcommand{\Q}{\mathbbmss{Q}}
\newcommand{\mathbb}[1]{\mathbbmss{#1}}


\usepackage{amsmath}
\usepackage{amsthm}

\newtheorem{thm}{Theorem}
\newtheorem{defn}{Definition}
\newtheorem{prop}{Proposition}
\newtheorem{lemma}{Lemma}
\newtheorem{cor}{Corollary}
\begin{document}
\begin{defn}
A topological space is a \emph{regular space} if
 it is both a \PMlinkname{$T_0$ space}{T0Space} and a 
 \PMlinkname{$T_3$ space}{T3Space}.
\end{defn}

{\bf Example.}
Consider the set $\R$ with the topology $\sigma$ generated  by the basis $$\beta=\{ U=V-C: V \mbox{ is open with the standard topology and\ } C \mbox{ is (infinite) numerable}\}.$$

Since $\Q$ is numerable and $\R$ open, the set of irrational numbers $\R-\Q$ is open and therefore $\Q$ is closed.
It can be shown that $\R-\Q$ is an open set with this topology and $\Q$ is closed. 

Take any irrational number $x$. Any open set $V$  containing all $\Q$ must contain also $x$, so the regular space property cannot be satisfied. Therefore, $(\R,\sigma)$ is not a regular space.

\subsubsection*{Note}
In topology, the terminology for separation axioms is not
standard. Therefore there are also other meanings of regular. 
In some references (e.g. \cite{kelley})
the meanings of regular and $T_3$ is exchanged. That is, 
$T_3$ is a stronger property than regular. 

 \begin{thebibliography}{9}
 \bibitem{steen} L.A. Steen, J.A.Seebach, Jr.,
 \emph{Counterexamples in topology},
 Holt, Rinehart and Winston, Inc., 1970.
\bibitem{kelley}
J.L. Kelley, \emph{General Topology}, D. van Nostrand Company, Inc., 1955.

 \end{thebibliography}
\end{document}
