\documentclass{article}
\usepackage{planetmath-specials}
\usepackage{pmath}
% this is the default PlanetMath preamble.  as your knowledge
% of TeX increases, you will probably want to edit this, but
% it should be fine as is for beginners.

% almost certainly you want these
\usepackage{amssymb}
\usepackage{amsmath}
\usepackage{amsfonts}

% used for TeXing text within eps files
%\usepackage{psfrag}
% need this for including graphics (\includegraphics)
%\usepackage{graphicx}
% for neatly defining theorems and propositions
%\usepackage{amsthm}
% making logically defined graphics
%\usepackage{xypic}

% there are many more packages, add them here as you need them

% define commands here
\begin{document}
A matrix is said to be in row echelon form if each non-zero row has more leading zeros than the previous row.   Row-echelon form is the key idea underlying the Gaussian elimination algorithm and LU factorization.

Let us give the precise definition.  Let $(M_{ij})$ be an $n\times m$
matrix. For each row $i=1,\ldots,n$ define the pivot position $P_i$ to
be either the minimum value of $j=1,\ldots, m$ for which $M_{ij}\neq
0$, or $\infty$ if the row consists entirely of zeros.  A matrix is in
echelon form if for all $i>1$, either $P_i=\infty$ or $P_{i-1}<P_i$.

Examples of matrices in row echelon form include,

\begin{displaymath}
\left( \begin{array}{ccc}
 0 & 2 & 1 \\
 0 & 0 & 1 \\
 0 & 0 & 0 \\
\end{array}
\right)
,\left( \begin{array}{ccccc}
 5 & 0 & 1 & 3 & 2 \\
 0 & 0 & 4 & 1 & 0 \\
 0 & 0 & 0 & 0 & 7 \\
\end{array} \right)
\end{displaymath}

Note that if a matrix is an echelon form, then necessarily rows which
are composed completely of zeros will be grouped at the bottom of the
matrix.  Also note that if several rows have the same number of
leading zeros then the matrix is not in row echelon form unless the
rows in question are composed entirely of zeros.
\end{document}
