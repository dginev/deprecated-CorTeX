\documentclass{article}
\usepackage{planetmath-specials}
\usepackage{pmath}
% This is Cosmin's preamble.

% Packages
  \usepackage{amsmath}
  \usepackage{amssymb}
  \usepackage{amsfonts}
  \usepackage{amsthm}
  \usepackage{mathrsfs}
  %\usepackage{graphicx}
  %\usepackage{xypic}
  %\usepackage{babel}
  \usepackage{pstricks}

% Theorem Environments
  \newtheorem*{thm}{Theorem}
  \newtheorem{thmn}{Theorem}
  \newtheorem*{lem}{Lemma}
  \newtheorem{lemn}{Lemma}
  \newtheorem*{cor}{Corollary}
  \newtheorem{corn}{Corollary}
  \newtheorem*{prop}{Proposition}
  \newtheorem{propn}{Proposition}
  
% Other Commands
    \renewcommand{\geq}{\geqslant}
    \renewcommand{\leq}{\leqslant}
    \newcommand{\vect}[1]{\boldsymbol{#1}}
    \newcommand{\mat}[1]{\mathsf{#1}}
    \renewcommand{\div}{\!\mid\!}
\newcommand{\Ints}{\mathbb{Z}}
\newcommand{\Rats}{\mathbb{Q}}
\newcommand{\Reals}{\mathbb{R}}
\newcommand{\suchthat}{\ \mid\ }
\begin{document}
\PMlinkescapeword{entire}
\PMlinkescapeword{area}
The \emph{union} of two sets $A$ and $B$ is the set which contains all $x \in A$ and all $x \in B$, denoted $A \cup B$. In the Venn diagram below, $A\cup B$ is the entire area shaded in blue.
\begin{center}
\begin{pspicture*}(0,0)(6,4)
\pscircle[fillstyle=vlines,hatchcolor=blue,hatchwidth=0.1\pslinewidth,hatchsep=1\pslinewidth](2,2){2}
\pscircle[fillstyle=vlines,hatchcolor=blue,hatchwidth=0.1\pslinewidth,hatchsep=1\pslinewidth](4,2){2}
\rput(1,2){$A$}
\rput(5,2){$B$}
\rput(-1,0){$.$}
\rput(7,4){$.$}
\end{pspicture*}
\end{center}

We can extend this to any (finite or infinite) family $(A_i)_{i\in I}$, writing $\bigcup_{i\in I}A_i$ for the union of this family. Formally, for a family $(A_i)_{i\in I}$ of sets:
\[ x \in \bigcup_{i\in I}A_i\; \Leftrightarrow \;\bigvee_{i\in I}\, (x\in A_i) \]
Alternatively, and equivalently,
\[x \in \bigcup_{i\in I}A_i\; \Leftrightarrow \;\exists i\in I\text{ such that } x\in A_i\]
This characterization makes it much clearer that if $I$ is itself the empty set (that is, if we are taking the union of an empty family), then the union is empty; that is,
\[\bigcup_{i\in\emptyset}A_i=\emptyset\]

Often elements of sets are taken from some universe $U$ of elements under consideration (for example, the real numbers $\Reals$, or living things on the planet, or words in a particular book). When this is the case, it is meaningful to discuss the \emph{complement} of a set: if $A$ is a set of elements from some universe $U$, then the complement of $A$ is the set
\[A^C = U\backslash A= \{x\in U\suchthat x\notin A\}\]

From an axiomatic point of view, the existence of the union is guaranteed by the axiom of union.

Note that the sets $A_i$ may be, but need not be, disjoint. Unions satisfy some basic properties that are obvious from the definitions:
\begin{itemize}
\item Idempotency: $A \cup A = A$
\item $A \cup A^C = U$ where $U$ is the \emph{universe} of $A$
\item Commutativity: $A \cup B = B \cup A$
\item Associativity: $(A \cup B) \cup C = A \cup (B \cup C)$
\end{itemize}

Here are some examples of set unions:
\begin{gather*}
\{1,2\}\cup\{3,4\} = \{1,2,3,4\}\\
\{blue, green\}\cup\emptyset = \{blue, green\}\\
\{x\in\Ints\ \mid\ x\geq 1\}\cup\{x\in\Ints\ \mid\ x\leq -1\} = \{x\in\Ints\ \mid\ x\neq 0\}\\
\{1,2\}\cup\{1,4\} = \{1,2,4\}\\
\{x\in\Reals\ \mid\ x\geq 2\}\cup\{x\in\Reals\ \mid\ x\leq 2\} = \Reals\\
\bigcup_{\substack{n\in\Ints\\n>0}} \left\{\frac{p}{n}\,\mid\,p\in\Ints\right\} = \Rats
\end{gather*}
The first three of these are the union of disjoint sets, while the latter three are not - in those cases, the sets overlap each other.
\end{document}
