\documentclass{article}
\usepackage{ids}
\usepackage{planetmath-specials}
\usepackage{pmath}
% this is the default PlanetMath preamble.  as your knowledge
% of TeX increases, you will probably want to edit this, but
% it should be fine as is for beginners.

% almost certainly you want these
\usepackage{amssymb}
\usepackage{amsmath}
\usepackage{amsfonts}

% used for TeXing text within eps files
%\usepackage{psfrag}
% need this for including graphics (\includegraphics)
%\usepackage{graphicx}
% for neatly defining theorems and propositions
%\usepackage{amsthm}
% making logically defined graphics
%\usepackage{xypic} 
\usepackage{pstricks}

% there are many more packages, add them here as you need them

% define commands here
\begin{document}
\section{Definition}

Let $A$ be a subset of $X$. The \emph{complement} of $A$ in $X$ (denoted $A^\complement$ when the larger set $X$ is clear from context) is the set difference $X \setminus A$.

The Venn diagram below illustrates the complement of $A$ in red.

\begin{center}
\begin{pspicture}(0,0)(8,6)
\pspolygon[fillstyle=vlines,hatchcolor=red,hatchwidth=0.1\pslinewidth,hatchsep=1\pslinewidth](0,0)(8,0)(8,6)(0,6)
\pscircle[fillstyle=solid,hatchcolor=white,hatchwidth=0.1\pslinewidth,hatchsep=1\pslinewidth](4,3){2}
\rput(4,3){$A$}
\rput(1,5){$X\setminus A$}
\rput(8.5,5.75){$X$}
\rput(0,0){$.$}
\rput(8,6){$.$}
\end{pspicture}
\end{center}

\section{Properties}

\begin{itemize}
\item $(A^{\complement})^\complement=A$
\item $\emptyset^\complement = X$
\item $X^\complement = \emptyset$
\item If $A$ and $B$ are subsets of $X$, then
  $A\setminus B = A\cap B^\complement$, where the complement is taken in $X$.
\end{itemize}

\section{de Morgan's laws}

Let $X$ be a set with subsets $A_i \subset X$ for $i\in I$, where
$I$ is an arbitrary index-set. In other words, $I$ can be finite,
countable, or uncountable. Then

\begin{eqnarray*}
\left( \bigcup_{i\in I} A_i \right)^\complement &=& \bigcap_{i\in I} A_i^\complement, \\
\left( \bigcap_{i\in I} A_i \right)^\complement &=& \bigcup_{i\in I} A_i^\complement.
\end{eqnarray*}
\end{document}
