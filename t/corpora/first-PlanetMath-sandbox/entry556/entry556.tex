\documentclass{article}
\usepackage{ids}
\usepackage{planetmath-specials}
\usepackage{pmath}
\usepackage{amssymb}
\usepackage{amsmath}
\usepackage{amsfonts}
\usepackage{graphicx}
\usepackage{xypic}
\begin{document}
If we have two homomorphisms $f : A \to B$ and $g : B \to C$
in some category of modules,
then we say that $f$ and $g$ are {\it exact} at $B$
if the image of $f$ is equal to the kernel of $g$.

A sequence of homomorphisms
$$
  \cdots \rightarrow
  A_{n+1} \buildrel {f_{n+1}} \over \longrightarrow
  A_n \buildrel {f_n} \over \longrightarrow
  A_{n-1} \rightarrow
  \cdots
$$
is said to be {\it exact} if each pair
of adjacent homomorphisms $(f_{n+1}, f_n)$ is exact -- 
in other words if ${\rm im} f_{n+1} = {\rm ker} f_n$ for all $n$.

Compare this to the notion of a chain complex.

\textbf{Remark}.  The notion of exact sequences can be generalized to any abelian category $\mathcal{A}$, where $A_i$ and $f_i$ above are objects and morphisms in $\mathcal{A}$.
\end{document}
