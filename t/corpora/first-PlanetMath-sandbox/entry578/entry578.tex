\documentclass{article}
\usepackage{planetmath-specials}
\usepackage{pmath}
\usepackage{amssymb}
\usepackage{amsmath}
\usepackage{amsfonts}
\usepackage{graphicx}
\usepackage{xypic}
\newcommand{\C}{\mathbb{C}}
\newcommand{\N}{\mathbb{N}}
\newcommand{\Z}{\mathbb{Z}}
\newcommand{\Q}{\mathbb{Q}}
\newcommand{\barQ}{\overline{\Q}}
\newcommand{\R}{\mathbb{R}}
\newcommand{\F}{\mathbb{F}}
\renewcommand{\S}{\mathcal{S}}
\renewcommand{\a}{{\mathfrak{a}}}
\renewcommand{\b}{{\mathfrak{b}}}
\newcommand{\m}{{\mathfrak{m}}}
\newcommand{\p}{{\mathfrak{p}}}
\renewcommand{\P}{{\mathfrak{P}}}
\newcommand{\q}{{\mathfrak{q}}}
\renewcommand{\c}{{\bf{c}}}
\renewcommand{\d}{{\bf{d}}}
\newcommand{\x}{{\bf{x}}}
\renewcommand{\Re}{\operatorname{Re}}
\newcommand{\0}{\bf{0}}
\newcommand{\category}[1]{\mbox{\boldmath $\mathsf{{#1}}$}}
\newcommand{\qr}[2]{{\mbox{$\left(\frac{{#1}}{{#2}}\right)$}}}
\newcommand{\s}[1]{\EuScript{{#1}}}
\newcommand{\conj}[1]{{\overline{{#1}}}}

\newlength{\algcwidth} \newlength{\algcheight}
\newcommand{\algc}[1]{%
\settowidth{\algcwidth}{#1} \settoheight{\algcheight}{#1}%
\raisebox{1.2\algcheight}[0pt]{%
\makebox[0pt][l]{\hspace{.4\algcwidth}%
\rule{.6\algcwidth}{0.05\algcheight}}}%
{{#1}}}

\newcommand{\jacobi}[2]{{\left(\frac{#1}{#2}\right)}}
\renewcommand{\H}{\category{H}}
\newcommand{\A}{\category{A}}
\newcommand{\B}{\category{B}}
\renewcommand{\O}{\mathcal{O}}
\newcommand{\M}{\mathfrak{M}}
\newcommand{\<}{\langle}
\renewcommand{\>}{\rangle}
\newcommand{\lra}{\longrightarrow}
\newcommand{\ra}{\rightarrow}
\newcommand{\iso}{\cong}
\newcommand{\dsum}{\oplus}
\newcommand{\bigdsum}{\bigoplus}
\newcommand{\conv}{\circ}
\newcommand{\comp}{\circ}
%\newcommand{\st}{\text{\ s.t.\ }}
\newcommand{\st}{\mid}
\renewcommand{\div}{\mid}
\newcommand{\ndiv}{\nmid}
\newcommand{\intersect}{\cap}
\newcommand{\union}{\cup}
\newcommand{\bigintersect}{\bigcap}
\newcommand{\bigunion}{\bigcup}
\renewcommand{\Im}{\operatorname{Im}}
\newcommand{\cross}{\times}
\newcommand{\tensor}{\otimes}
\newcommand{\Aut}{\operatorname{Aut}}
\newcommand{\Sym}{\operatorname{Sym}^2}
\newcommand{\Alt}{\operatorname{Alt}^2}
\newcommand{\Tr}{\operatorname{Tr}}
\newcommand{\Res}{\operatorname{Res}}
\newcommand{\Spec}{\operatorname{Spec}}
\newcommand{\Nil}{\operatorname{Nil}}
\newcommand{\Ann}{\operatorname{Ann}}
\newcommand{\Ass}{\operatorname{Ass}}
\newcommand{\Hom}{\operatorname{Hom}}
\newcommand{\End}{\operatorname{End}}
\newcommand{\Co}{\operatorname{Co}}
\newcommand{\Supp}{\operatorname{Supp}}
\newcommand{\cl}{\operatorname{cl}}
\newcommand{\coker}{\operatorname{coker}}
\newcommand{\disc}{\operatorname{disc}}
\newcommand{\Gal}{\operatorname{Gal}}
\newcommand{\lcm}{\operatorname{lcm}}
\newcommand{\sign}{\operatorname{sign}}
\newcommand{\SL}{\operatorname{SL}}
\newcommand{\GL}{\operatorname{GL}}
\newcommand{\semidirect}{\times}
\newcommand{\bd}{\partial}
\newcommand{\dual}{\vee}
\newcommand{\pth}{{\mbox{$p^{\text{th}}$}}}
\newcommand{\qth}{{\mbox{$q^{\text{th}}$}}}
\begin{document}
To solve the cubic polynomial equation $x^3 + ax^2 + bx + c = 0$ for $x$, the first step is to apply the Tchirnhaus transformation $x = y-\frac{a}{3}$. This reduces the equation to $y^3 + py + q = 0$, where
\begin{eqnarray*}
p & = & b - \frac{a^2}{3} \\
q & = & c - \frac{ab}{3} + \frac{2a^3}{27}
\end{eqnarray*}
The next step is to substitute $y = u-v$, to obtain
\begin{equation}\label{original}
(u-v)^3 + p(u-v) + q = 0
\end{equation}
or, with the terms collected,
\begin{equation}\label{rearrange}
(q - (v^3 - u^3)) + (u-v)(p - 3 u v) = 0
\end{equation}
From equation \eqref{rearrange}, we see that if $u$ and $v$ are chosen so that $q = v^3-u^3$ and $p = 3uv$, then $y = u-v$ will satisfy equation \eqref{original}, and the cubic equation will be solved!

There remains the matter of solving $q = v^3-u^3$ and $p = 3uv$ for $u$ and $v$. From the second equation, we get $v = p/(3u)$, and substituting this $v$ into the first equation yields
$$
q = \frac{p^3}{(3u)^3} - u^3
$$
which is a quadratic equation in $u^3$. Solving for $u^3$ using the quadratic formula, we get
\begin{eqnarray*}
u^3 & = & \frac{-27q + \sqrt{108 p^3 + 729 q^2}}{54} = \frac{-9q + \sqrt{12 p^3 + 81 q^2}}{18}\\
v^3 & = & \frac{27q + \sqrt{108 p^3 + 729 q^2}}{54} = \frac{9q + \sqrt{12 p^3 + 81 q^2}}{18}
\end{eqnarray*}
Using these values for $u$ and $v$, you can back--substitute $y=u-v$, $p=b-a^2/3$, $q=c-ab/3+2a^3/27$, and $x = y-a/3$ to get the expression for the first root $r_1$ in the cubic formula. The second and third roots $r_2$ and $r_3$ are obtained by performing synthetic division using $r_1$, and using the quadratic formula on the remaining quadratic factor.
\end{document}
