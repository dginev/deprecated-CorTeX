\documentclass{article}
\usepackage{ids}
\usepackage{planetmath-specials}
\usepackage{pmath}
\usepackage{amssymb}
\usepackage{amsmath}
\usepackage{amsfonts}

%\usepackage{psfrag}
%\usepackage{graphicx}
%\usepackage{xypic}
\begin{document}
The \emph{city-block metric}, defined on $\mathbb{R}^n$, is

$$ d(a,b) = \sum_{i=1}^n |b_i-a_i| $$

where $a$ and $b$ are vectors in $\mathbb{R}^n$ with $a=(a_1,\ldots,a_n)$ and $b=(b_1,\ldots,b_n)$.

In two dimensions and with discrete-valued vectors, when we can picture the set of points in $\mathbb{Z} \times \mathbb{Z}$ as a grid, this is simply the number of edges between points that must be traversed to get from $a$ to $b$ within the grid.  This is the same problem as getting from corner $a$ to $b$ in a rectilinear downtown area, hence the name ``city-block metric.''
\end{document}
