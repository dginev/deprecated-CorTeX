\documentclass{article}
\usepackage{planetmath-specials}
\usepackage{pmath}
% this is the default PlanetMath preamble.  as your knowledge
% of TeX increases, you will probably want to edit this, but
% it should be fine as is for beginners.

% almost certainly you want these
\usepackage{amssymb}
\usepackage{amsmath}
\usepackage{amsfonts}

% used for TeXing text within eps files
%\usepackage{psfrag}
% need this for including graphics (\includegraphics)
%\usepackage{graphicx}
% for neatly defining theorems and propositions
%\usepackage{amsthm}
% making logically defined graphics
%\usepackage{xypic} 

% there are many more packages, add them here as you need them

% define commands here
\begin{document}
Let $L/K$ be a finite Galois extension with Galois group $G = \operatorname{Gal}(L/K)$. The modern formulation of Hilbert's Theorem 90 states that the first Galois cohomology group $H^1(G, L^*)$ is 0.

The original statement of Hilbert's Theorem 90 differs somewhat from the modern formulation given above, and is nowadays regarded as a corollary of the above fact. In its original form, Hilbert's Theorem 90 says that if $G$ is cyclic with generator $\sigma$, then an element $x \in L$ has norm 1 if and only if
$$
x = y/\sigma(y)
$$
for some $y \in L$. Note that elements of the form $y/\sigma(y)$ are obviously contained within the kernel of the norm map; it is the converse that forms the content of the theorem.
\end{document}
