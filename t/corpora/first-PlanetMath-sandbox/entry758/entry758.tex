\documentclass{article}
\usepackage{ids}
\usepackage{planetmath-specials}
\usepackage{pmath}
% this is the default PlanetMath preamble.  as your knowledge
% of TeX increases, you will probably want to edit this, but
% it should be fine as is for beginners.

% almost certainly you want these
\usepackage{amssymb}
\usepackage{amsmath}
\usepackage{amsfonts}

% used for TeXing text within eps files
%\usepackage{psfrag}
% need this for including graphics (\includegraphics)
%\usepackage{graphicx}
% for neatly defining theorems and propositions
%\usepackage{amsthm}
% making logically defined graphics
%\usepackage{xypic} 

% there are many more packages, add them here as you need them

% define commands here
\begin{document}
A \emph{Hilbert space} is an inner product space which is \PMlinkid{complete}{603} under the \PMlinkescapetext{induced} metric.

In particular, a Hilbert space is a Banach space in the norm \PMlinkescapetext{induced} by the inner product, since the norm and the inner product both induce the same metric.  Any finite-dimensional inner product space is a Hilbert space, but it is worth mentioning that some authors require the space to be infinite dimensional for it to be called a Hilbert space.
\end{document}
