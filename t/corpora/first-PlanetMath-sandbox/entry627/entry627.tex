\documentclass{article}
\usepackage{planetmath-specials}
\usepackage{pmath}
% this is the default PlanetMath preamble.  as your knowledge
% of TeX increases, you will probably want to edit this, but
% it should be fine as is for beginners.

% almost certainly you want these
\usepackage{amssymb}
\usepackage{amsmath}
\usepackage{amsfonts}

% used for TeXing text within eps files
%\usepackage{psfrag}
% need this for including graphics (\includegraphics)
%\usepackage{graphicx}
% for neatly defining theorems and propositions
%\usepackage{amsthm}
% making logically defined graphics
%\usepackage{xypic} 

% there are many more packages, add them here as you need them

% define commands here
\begin{document}
\PMlinkescapeword{inverse}
\PMlinkescapeword{operation}

{\it This entry is obsolete, having been superseded by \PMlinkname{a new entry}{TopologicalGroup2}. It is being retained for a short while because of the attached thread.}

A \emph{topological group} is a triple $(G,\cdot,\mathcal{T})$ where $(G,\cdot)$ is a group and $\mathcal{T}$ is a topology on $G$ such that under $\mathcal{T}$, the group operation $(x,y)\mapsto x\cdot y$ is continuous with respect to the product topology on $G\times G$ and the inverse map $x\mapsto x^{-1}$ is continuous on $G$.

Many authors require that the topology be Hausdorff.
\end{document}
