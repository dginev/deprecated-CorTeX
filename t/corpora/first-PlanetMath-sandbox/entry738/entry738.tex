\documentclass{article}
\usepackage{planetmath-specials}
\usepackage{pmath}
% this is the default PlanetMath preamble.  as your knowledge
% of TeX increases, you will probably want to edit this, but
% it should be fine as is for beginners.

% almost certainly you want these
\usepackage{amssymb}
\usepackage{amsmath}
\usepackage{amsfonts}

% used for TeXing text within eps files
%\usepackage{psfrag}
% need this for including graphics (\includegraphics)
%\usepackage{graphicx}
% for neatly defining theorems and propositions
%\usepackage{amsthm}
% making logically defined graphics
%\usepackage{xypic} 

% there are many more packages, add them here as you need them

% define commands here
\newcommand{\C}{\mathbb{C}}
\begin{document}
Given a field $K$, a $K$--valued {\em class function} on a group $G$ is a function $f: G \longrightarrow K$ such that $f(g) = f(h)$ whenever $g$ and $h$ are elements of the same conjugacy class of $G$.

An important example of a class function is the character of a group representation. Over the complex numbers, the set of characters of the irreducible representations of $G$ form a basis for the vector space of all $\C$--valued class functions, when $G$ is a compact Lie group.

\paragraph{Relation to the convolution algebra}

Class functions are also known as central functions, because they correspond to functions $f$ in the convolution algebra $C^*(G)$ that have the property $f*g = g*f$ for all $g \in C^*(G)$ (i.e., they commute with everything under the convolution operation).  More precisely, the set of measurable complex valued class functions $f$ is equal to the set of central elements of the convolution algebra $C^*(G)$, for $G$ a locally compact group admitting a Haar measure.
\end{document}
