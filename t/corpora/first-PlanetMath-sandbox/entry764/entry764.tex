\documentclass{article}
\usepackage{ids}
\usepackage{planetmath-specials}
\usepackage{pmath}
% this is the default PlanetMath preamble.  as your knowledge
% of TeX increases, you will probably want to edit this, but
% it should be fine as is for beginners.

% almost certainly you want these
\usepackage{amssymb}
\usepackage{amsmath}
\usepackage{amsfonts}

% used for TeXing text within eps files
%\usepackage{psfrag}
% need this for including graphics (\includegraphics)
%\usepackage{graphicx}
% for neatly defining theorems and propositions
%\usepackage{amsthm}
% making logically defined graphics
%\usepackage{xypic} 

% there are many more packages, add them here as you need them

% define commands here
\newcommand{\mv}[1]{\mathbf{#1}}	% matrix or vector
\newcommand{\cov}{\mathrm{cov}}
\newcommand{\mvt}[1]{\mv{#1}^{\mathrm{T}}}
\newcommand{\mvi}[1]{\mv{#1}^{-1}}
\newcommand{\mpderiv}[1]{\frac{\partial}{\partial {#1}}}
\begin{document}
\newtheorem{thm}{Theorem}
\begin{thm}[]
Let $f:\mathbb{R}^n \to \mathbb{R}$ be a continuous probability
density function. Let $X_1, \ldots, X_n$ be random variables with density $f$
and with covariance matrix $\mv{K}$, $K_{ij} = \cov(X_i, X_j)$.  Let $\phi$ be the distribution of the 
\PMlinkname{multidimensional Gaussian}{JointNormalDistribution}  with mean $\mv{0}$ and covariance matrix $\mv{K}$.  Then the  Gaussian distribution maximizes the differential entropy for a given covariance matrix $\mv{K}$.  That is, $h(\phi) \ge h(f)$.
\end{thm}
\end{document}
