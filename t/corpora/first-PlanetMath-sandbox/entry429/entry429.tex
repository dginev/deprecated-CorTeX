\documentclass{article}
\usepackage{planetmath-specials}
\usepackage{pmath}
\usepackage{amsmath}
\usepackage{amsfonts}
\usepackage{amssymb}

\newcommand{\reals}{\mathbb{R}}
\newcommand{\natnums}{\mathbb{N}}
\newcommand{\cnums}{\mathbb{C}}
\newcommand{\znums}{\mathbb{Z}}

\newcommand{\lp}{\left(}
\newcommand{\rp}{\right)}
\newcommand{\lb}{\left[}
\newcommand{\rb}{\right]}

\newcommand{\supth}{^{\text{th}}}

\newcommand{\fg}{\mathfrak{g}}
\newcommand{\lact}{\lambda}

\newtheorem{proposition}{Proposition}
\newtheorem{definition}[proposition]{Definition}
\begin{document}
A Lie group is a group endowed with a compatible \PMlinkname{analytic structure}{ComplexAnalyticManifold}.
To be more precise, Lie group structure consists of two kinds of
data
\begin{itemize}
\item a finite-dimensional, real-analytic manifold $G$, and
\item  two analytic maps, one for multiplication $G\times
  G\rightarrow G$ and one for inversion $G\rightarrow G$, which obey
  the appropriate
  group axioms.
\end{itemize}
Thus, a homomorphism in the category of Lie groups is a group
homomorphism that is simultaneously an analytic mapping between two
real-analytic manifolds.

Next, we describe a natural construction that associates a certain
Lie algebra $\fg$ to every Lie group $G$.  Let $e\in G$ denote the
identity element of $G$.
% and let $\fg$ denote $T_e G$, the tangent space of $G$ at the
% identity.
For $g\in G$ let $\lact_g \colon G \to G$ denote the
diffeomorphisms corresponding to left multiplication by $g$.  

\begin{definition}
  A vector field $V$ on $G$ is called left-invariant if $V$ is
  invariant with respect to all left multiplications.  To be more
  precise, $V$ is left-invariant if and only if
  $$(\lact_g)_*(V) = V$$
  (see push-forward of a vector-field) for all
  $g\in G$.   
\end{definition}
\begin{proposition}
  The Lie bracket of two left-invariant vector fields is
  again, a left-invariant vector field.
\end{proposition}
{\em Proof.}
Let $V_1, V_2$ be left-invariant vector fields, and let $g\in G$.
The bracket operation is covariant with respect to diffeomorphism, and
in particular
$$(\lact_g)_*[V_1,V_2] = [(\lact_g)_*V_1,(\lact_g)_*V_2] =
[V_1,V_2].$$
Q.E.D.

\begin{definition}
  The Lie algebra of $G$, denoted hereafter by $\fg$, is the vector space
  of all left-invariant vector fields equipped with the vector-field
  bracket.
\end{definition}

Now a right multiplication is invariant with respect to all
left multiplications, and it turns out that we can characterize a
left-invariant vector field as being an infinitesimal right multiplication.
\begin{proposition}
  Let $a\in T_eG$ and let $V$ be a left-invariant vector-field such that
  $V_e = a$.  Then for all $g\in G$ we have
  $$V_g = (\lact_g)_*(a).$$
\end{proposition}
The intuition here is that $a$
gives an infinitesimal displacement from the identity element and that
$V_g$ gives a corresponding infinitesimal right displacement away from $g$.
Indeed consider a curve 
$$\gamma \colon (-\epsilon,\epsilon) \to G$$
passing through the identity element with velocity $a$; i.e.
$$\gamma(0) = e,\quad \gamma'(0) = a.$$
The above proposition is then
saying that the curve 
$$t\mapsto g\gamma(t),\quad t\in(-\epsilon,\epsilon)$$
passes through
$g$ at $t=0$ with velocity $V_g$.

Thus we see that a left-invariant vector field is completely
determined by the value it takes at $e$, and that therefore $\fg$ is
isomorphic, as a vector space to $T_eG$.

Of course, we can also consider the Lie algebra of right-invariant
vector fields.  The resulting Lie-algebra is anti-isomorphic (the
order in the bracket is reversed) to the Lie algebra of left-invariant
vector fields.  Now it is a general principle that the group inverse
operation gives an anti-isomorphism between left and right group
actions.  So, as one may well expect, the anti-isomorphism between the
Lie algebras of left and right-invariant vector fields can be realized
by considering the linear action of the inverse operation on $T_e G$.

Finally, let us remark that one can induce the Lie algebra structure
directly on $T_e G$ by considering adjoint action of $G$ on $T_eG$. 

\paragraph{History and motivation.}

\paragraph{Examples.}



\paragraph{Notes.}
\begin{enumerate}
\item 
No generality is lost in assuming that a Lie group has analytic,
rather than $C^\infty$ or even $C^k,\; k=1,2,\ldots$ structure.
Indeed, given a $C^1$ differential manifold with a $C^1$
multiplication rule, one can show that the exponential mapping endows
this manifold with a compatible real-analytic structure.  

Indeed, one can go even further and show that even $C^0$ suffices.  In
other words, a topological group that is also a finite-dimensional
topological manifold possesses a compatible analytic structure.  This
result was formulated by Hilbert as his \PMlinkexternal{fifth problem}{http://www.reed.edu/~wieting/essays/LieHilbert.pdf}, and
proved in the 50's by Montgomery and Zippin.


\item One can also speak of a complex Lie group, in which case $G$ and the
multiplication mapping are both complex-analytic.  The theory of
complex Lie groups requires the notion of a holomorphic vector-field.
Not withstanding this complication, most of the essential features of
the real theory carry over to the complex case.

\item The name ``Lie group'' honours the Norwegian mathematician
  Sophus Lie who pioneered and developed the theory of continuous
  transformation groups and the corresponding theory of Lie algebras
  of vector fields (the group's infinitesimal generators, as Lie
  termed them).  Lie's original impetus was the study of continuous
  symmetry of geometric objects and differential equations.  
  
  The scope of the theory has grown enormously in the 100+ years of
  its existence.  The contributions of Elie Cartan and Claude
  Chevalley figure prominently in this evolution. Cartan is
  responsible for the celebrated ADE classification of simple Lie
  algebras, as well as for charting the essential role played by Lie
  groups in differential geometry and mathematical physics.  Chevalley
  made key foundational contributions to the analytic theory, and did
  much to pioneer the related theory of algebraic groups.  Armand
  Borel's book ``Essays in the History of Lie groups and algebraic
  groups'' is the definitive source on the evolution of the Lie group
  concept.  Sophus Lie's contributions are the subject of a number of
  excellent articles by T. Hawkins.

\end{enumerate}
\end{document}
