\documentclass{article}
\usepackage{planetmath-specials}
\usepackage{pmath}
\usepackage{amsmath}
\usepackage{amsfonts}
\usepackage{amssymb}
\newcommand{\reals}{\mathbb{R}}
\begin{document}
Let $I\subset \reals$ be an interval and let $\gamma:I\to\reals^3$ be a
parameterized space curve, assumed to be
\PMlinkname{regular}{SpaceCurve} and free of points of inflection. We
interpret $\gamma(t)$ as the trajectory of a particle moving through
3-dimensional space.  The moving trihedron (also known as the Frenet
frame, the Frenet trihedron, the rep\`ere mobile, and the moving
frame) is an orthonormal basis of 3-vectors $T(t),N(t),B(t),$ defined
and named as follows:
\begin{align*}
  T(t) &= \displaystyle \frac{\gamma'(t)}{\Vert \gamma'(t) \Vert}\, ,
  &&
  \text{the unit  tangent;}\\
  N(t) &= \displaystyle \frac{T'(t)}{\Vert T'(t) \Vert} \, ,&&
  \text{the unit normal;}\\ \\
  B(t) &= T(t)\times N(t)  \, ,&& \text{the unit binormal.}\\
\end{align*}
A straightforward application of the chain rule shows that these
definitions are covariant with respect to reparameterizations.  Hence,
the above three vectors should be conceived as being attached to the
point $\gamma(t)$ of the oriented space curve, rather than being
functions of the parameter $t$.

Corresponding to the above vectors are 3 planes, passing through each
point $\gamma(t)$ of the space curve.  The \emph{osculating plane} at
the point $\gamma(t)$ is the plane spanned by $T(t)$ and $N(t)$; the
\emph{normal plane} at $\gamma(t)$ is the plane spanned by $N(t)$ and
$B(t)$; the rectifying plane at $\gamma(t)$ is the plane spanned by
$T(t)$ and $B(t)$.
\end{document}
