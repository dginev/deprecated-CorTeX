\documentclass{article}
\usepackage{ids}
\usepackage{planetmath-specials}
\usepackage{pmath}
\usepackage{amssymb}
\usepackage{amsmath}
\usepackage{amsfonts}
\begin{document}
\PMlinkescapeword{joins}
\PMlinkescapeword{contain}
In graph theory, a \emph{loop} is an edge which joins a vertex
to itself, rather than to some other vertex. By definition,
a graph cannot contain a loop; a pseudograph, however, may contain
both multiple edges and multiple loops. Note that by some definitions,
a multigraph may contain multiple edges and no loops, while other texts
define a multigraph as a graph
allowing multiple edges and multiple loops.

In algebra, a \emph{loop} is a quasigroup which contains an identity element.
\end{document}
