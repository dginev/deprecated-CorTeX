\documentclass{article}
\usepackage{planetmath-specials}
\usepackage{pmath}
% this is the default PlanetMath preamble.  as your knowledge
% of TeX increases, you will probably want to edit this, but
% it should be fine as is for beginners.

% almost certainly you want these
\usepackage{amssymb,amscd}
\usepackage{amsmath}
\usepackage{amsfonts}

% used for TeXing text within eps files
%\usepackage{psfrag}
% need this for including graphics (\includegraphics)
%\usepackage{graphicx}
% for neatly defining theorems and propositions
%\usepackage{amsthm}
% making logically defined graphics
%\usepackage{xypic}

% there are many more packages, add them here as you need them

% define commands here
\begin{document}
Let $X$ be a topological space.  A subset $A$ of $X$ is called a
\emph{regular open} set if $A$ is equal to the interior of the closure of itself:
$$A=\operatorname{int}(\overline{A}).$$
Clearly, every regular open set is open, and every clopen set is regular open.

\textbf{Examples}.  Let $\mathbb{R}$ be the real line with the usual
topology (generated by open intervals).
\begin{itemize}
\item $(a,b)$ is regular open whenever $-\infty<a\leq b<\infty$.
\item $(a,b)\cup(b,c)$ is not regular open for $-\infty<a\leq b
\leq c<\infty$ and $a\neq c$.  The interior of the closure of
$(a,b)\cup(b,c)$ is $(a,c)$.
\end{itemize}

If we examine the structure of $\operatorname{int}(\overline{A})$ a
little more closely, we see that if we define
$$A^{\bot}:=X-\overline{A},$$ then $$A^{\bot\bot}=
\operatorname{int}(\overline{A}).$$  So an alternative definition of
a regular open set is an open set $A$ such that $A^{\bot\bot}=A$.

\textbf{Remarks}.
\begin{itemize}
\item For any $A\subseteq X$, $A^{\bot}$ is always open.
\item $\varnothing^{\bot}=X$ and $X^{\bot}=\varnothing$.
\item $A\cap A^{\bot}=\varnothing$ and $A\cup A^{\bot}$ is dense in
$X$.
\item $A^{\bot}\cup B^{\bot}\subseteq(A\cap
B)^{\bot}$ and $A^{\bot}\cap B^{\bot}=(A\cup B)^{\bot}$.
\item It can be shown that if $A$ is open, then $A^{\bot}$ is
regular open.  As a result, following from the first property, $\operatorname{int}(\overline{A})$, being $A^{\bot\bot}$, is regular open for any subset $A$ of $X$.
\item In addition, if both $A$ and $B$ are regular open, then $A\cap
B$ is regular open.
\item It is not true, however, that the union of two regular open
sets is regular open, as illustrated by the second example above.
\item It can also be shown that the set of all regular open sets of
a topological space $X$ forms a Boolean algebra under the following
set of operations:
\begin{enumerate}
\item $1=X$ and $0=\varnothing$,
\item $a\land b=a\cap b$,
\item $a\lor b=(a\cup b)^{\bot\bot}$, and
\item $a^{\prime}=a^{\bot}$.
\end{enumerate}
This is an example of a Boolean algebra coming from a collection of
subsets of a set that is not formed by the standard set operations
union $\cup$, intersection $\cap$, and complementation $^{\prime}$.
\end{itemize}

The definition of a regular open set can be dualized.  A closed set $A$ in a topological space is called a \emph{regular closed set} if $A=\overline{\operatorname{int}(A)}$.

\begin{thebibliography}{[AHU]}
\bibitem{halmos} P. Halmos (1970).  {\em{Lectures on Boolean Algebras}}, Springer.
\bibitem{willard} S. Willard (1970). \emph{General Topology}, Addison-Wesley Publishing Company.
\end{thebibliography}
\end{document}
