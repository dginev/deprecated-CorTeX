\documentclass{article}
\usepackage{planetmath-specials}
\usepackage{pmath}
\usepackage{amsmath}
\usepackage{amsfonts}
\usepackage{amssymb}
\newcommand{\reals}{\mathbb{R}}

\newcommand{\SO}{\operatorname{SO}}
\begin{document}
Let $I\subset\reals$ be an interval, and let $\gamma:I\to\reals^3$ be
an arclength parameterization of an oriented space curve, assumed to
be \PMlinkname{regular}{Curve}, and free of points of inflection.  Let $T(s)$, $N(s)$,
$B(s)$ denote the corresponding moving trihedron, and $\kappa(s),
\tau(s)$ the corresponding \PMlinkname{curvature}{CurvatureOfACurve}
and \PMlinkname{torsion functions}{Torsion}.  The following
differential relations, called the Serret-Frenet equations, hold
between these three vectors.
\begin{eqnarray}
\label{eq:dT}
T'(s) &=& \kappa(s) N(s);\\
\label{eq:dN}
N'(s) &=& -\kappa(s) T(s) + \tau(s)B(s); \\
\label{eq:dB}
B'(s) &=& -\tau(s) N(s).
\end{eqnarray}

Equation \eqref{eq:dT} follows directly from the \PMlinkname{definition of the
normal}{MovingFrame} $N(s)$ and from the \PMlinkname{definition of the
curvature}{CurvatureAndTorsion}, $\kappa(s)$. Taking the derivative of
the relation 
$$N(s)\cdot T(s) = 0,$$
gives
$$N'(s)\cdot T(s)  = - T'(s) \cdot N(s) = -\kappa(s).$$
Taking the derivative of the relation
$$N(s)\cdot N(s) = 1,$$
gives
$$N'(s) \cdot N(s) = 0.$$
By the \PMlinkname{definition of torsion}{CurvatureAndTorsion}, we have
$$N'(s)\cdot B(s) = \tau(s).$$
This proves equation \eqref{eq:dN}.
Finally,
taking derivatives of the relations
\begin{gather*}
T(s)\cdot B(s) = 0,\\
N(s)\cdot B(s) = 0,\\
B(s)\cdot B(s) =1,  
\end{gather*}
and making use of \eqref{eq:dT} and \eqref{eq:dN}
gives
\begin{gather*}
B'(s) \cdot T(s) = -T'(s)\cdot B(s) = 0,\\
B'(s) \cdot N(s) = -N'(s)\cdot B(s) = -\tau(s),\\
B'(s)\cdot B(s) = 0.  
\end{gather*}
This proves equation \eqref{eq:dB}.

It is also convenient to describe the Serret-Frenet equations by using
matrix notation.  Let $F:I \to \SO(3)$ (see - special orthogonal
group), the mapping defined by
$$F(s) = (T(s),N(s),B(s)),\quad s\in I$$
represent the Frenet frame as a $3\times 3$ orthonormal
matrix.  Equations \eqref{eq:dT} \eqref{eq:dN} \eqref{eq:dB} can be
succinctly given as
$$F(s)^{-1} F'(s) = 
\begin{pmatrix}
  0 & \kappa(s) & 0 \\
  -\kappa(s)  & 0 & \tau(s) \\
  0 & -\tau(s) & 0
\end{pmatrix}
$$
In this formulation, the above relation is also known as the structure
equations of an oriented space curve.
\end{document}
