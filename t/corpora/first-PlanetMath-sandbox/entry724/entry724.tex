\documentclass{article}
\usepackage{ids}
\usepackage{planetmath-specials}
\usepackage{pmath}
% this is the default PlanetMath preamble.  as your knowledge
% of TeX increases, you will probably want to edit this, but
% it should be fine as is for beginners.

% almost certainly you want these
\usepackage{amssymb}
\usepackage{amsmath}
\usepackage{amsfonts}

% used for TeXing text within eps files
%\usepackage{psfrag}
% need this for including graphics (\includegraphics)
%\usepackage{graphicx}
% for neatly defining theorems and propositions
%\usepackage{amsthm}
% making logically defined graphics
\usepackage{xypic} 

% there are many more packages, add them here as you need them

% define commands here
\begin{document}
A \emph{cycle} in a graph, digraph, or multigraph, is a simple path from a vertex to itself (i.e., a path where the first vertex is the same as the last vertex and no edge is repeated).

For example, consider this graph:
$$\xymatrix{
A \ar@{-}[r] \ar@{-}[d] & B \ar@{-}[dl] \ar@{-}[d] \\
D \ar@{-}[r] & C
}$$

$ABCDA$ and $BDAB$ are two of the cycles in this graph.  $ABA$ is not a cycle, however, since it uses the edge connecting $A$ and $B$ twice.  $ABCD$ is not a cycle because it begins on $A$ but ends on $D$.

A cycle of length $n$ is sometimes denoted $C_n$ and may be referred to as a polygon of $n$ sides: that is, $C_3$ is a triangle, $C_4$ is a quadrilateral, $C_5$ is a pentagon, etc.

An \emph{even} cycle is one of even length; similarly, an \emph{odd} cycle is one of odd length.
\end{document}
