\documentclass{article}
\usepackage{planetmath-specials}
\usepackage{pmath}
\usepackage{amssymb}
\usepackage{amsmath}
\usepackage{amsfonts}

%\usepackage{psfrag}
%\usepackage{graphicx}
%\usepackage{xypic}
\begin{document}
The Dirac delta ``function'' $\delta(x)$, or \emph{distribution} is not a true function because it is not uniquely defined for all values of the argument $x$. Similar to the Kronecker delta symbol, the notation $\delta(x)$ stands for

$$ \delta(x) = 0 \;\text{for}\; x \ne 0, \;\text{and}\; \int_{-\infty}^\infty \delta(x) dx = 1 $$

For any continuous function $F$:

$$ \int_{-\infty}^\infty \delta(x) F(x)dx = F(0) $$

or in $n$ dimensions:

$$\int_{\mathbb{R}^n} \delta(x - s)f(s) \, d^ns = f(x)$$

$\delta(x)$ can also be defined as a normalized Gaussian function (normal distribution) in the limit of zero width.

\textbf{Notes:}
However, the limit of the normalized Gaussian function is still meaningless as a function, but some people still write such a limit as being equal to the Dirac distribution considered above in the first paragraph. \\
An example of how the Dirac distribution arises in a physical, classical context is available 
\PMlinkexternal{on line.}{http://www.rose-hulman.edu/~rickert/Classes/ma222/Wint0102/dirac.pdf}

{\bf Remarks:}
Distributions play important roles in Dirac's formulation of quantum mechanics.

\begin{thebibliography}{9}
\bibitem{rudin_fap}
W. Rudin, {\em Functional Analysis},
McGraw-Hill Book Company, 1973.
\bibitem{hormander}
L. H\"ormander, {\em The Analysis of Linear Partial Differential Operators I,
(Distribution theory and Fourier Analysis)}, 2nd ed, Springer-Verlag, 1990.
\bibitem{TDAB} 
Originally from The Data Analysis Briefbook
(\PMlinkexternal{http://rkb.home.cern.ch/rkb/titleA.html}{http://rkb.home.cern.ch/rkb/titleA.html})
\end{thebibliography}
\end{document}
